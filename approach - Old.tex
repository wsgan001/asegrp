\section{Approach}
Our approach consists of two major phases: Static Analysis phase and Dynamic Test Generation Phase.
The static analysis phase first determines all the changed parts in the program and then finds all the branches in the program that cannot help in either of  
P, I, E of PIE model. These branches are later used in making the Dynamic Tests Generation Phase effective.
\subsection{Static Analysis Phase}
The static analysis phase consists of the following components:
\\ \textbf{Difference Finder.} The \emph{Difference Finder} component takes as input two versions of an assembly 
	file and the name of class under test. The \emph{Difference Finder} component implements the differencing 
	algorithm for finding differences between the two versions of class under test. 
	The component gives as output the list of changed lines between the two versions of the source code. 
	In addition, for each changed line, the component gives the kind of change ( addition, deletion, or modification). 
	This information is used by the \CodeIn{Graph Generator} and \CodeIn{Graph Traverser} component.
	\\ \textbf{Graph Generator.} The \emph{Graph Generator} component constructs the control flow graph from the source code of the modified version of class under test. The component also uses the information about changes in the source code passed by the \emph{Difference Finder} component. The nodes (in the Control Flow Graph) corresponding to these changes are marked as changed. The information about these changes is also stored at these changed nodes.
	\\ \textbf{Graph Traverser.} The \emph{Graph Traverser} component, takes as input the graph constructed by the \emph{Graph Generator} component and traverses the graph to find all the branching nodes in the control flow graph that do not have a path leading to any of the nodes marked as changed in the CFG.    
	\\ \textbf{Instrumenter.} The \emph{Instrumenter} component instuments the source code, so that it is convenient for us to perform DSE on the source code. In particular, the \emph{Instrumenter} component combines the two versions of software system under test so that only one instance of DSE can be performed instead of performing DSE on both the versions separately. In addition, this component also synthesizes a class containing Parametrized unit tests (PUT) (one for each changed method). These  PUT's serve as test drivers for Pex. 
	
	\begin{figure}[t]
\begin{CodeOut}
\begin{alltt}
  \textbf{public} TestClass\{
  \hspace{0.5cm}\textbf{public boolean} testMe(\textbf{int }x, \textbf{int[] } y)\{
1 \hspace{1.0cm} \textbf{int} j=0, k=0;
2 \hspace{1.0cm} \textbf{if}(x==90)\{
3 \hspace{1.5cm} \textbf{for}(\textbf{int} i=0; i< y.Length; i++)\{
4 \hspace{2.0cm} \textbf{if}(y[i] == 15)
5 \hspace{2.5cm} x++;
6 \hspace{2.0cm} \textbf{else if}(y[i] == 25)
7 \hspace{2.5cm} \textbf{return} x;
8 \hspace{2.0cm} \textbf{if}(x >= 110 && x<= 150)
9 \hspace{2.5cm} j =1;
10\hspace{2.0cm} \textbf{else if}(x>150)
11\hspace{2.5cm} j =2;
12\hspace{2.0cm} \textbf{for}(i=200; i<= x; i++)
13\hspace{2.5cm} k++;
14\hspace{2.0cm} \textbf{if}(j > 0) // \textbf{if}(j >= 0)
15\hspace{2.5cm} x = j+2; //x = 2*j+1
16\hspace{2.0cm} \textbf{for}(int i=0; i< k+1; i++) 
17\hspace{2.5cm} x = x*k;
18\hspace{1.5cm} \}
19\hspace{1.0cm} \}
20\hspace{1.0cm} \textbf{return} x;
  \hspace{0.5cm}\}
  \hspace{0.5cm}...
  \hspace{0.5cm}A (\textbf{a})...
  \hspace{0.5cm}B (\textbf{b})...
  \hspace{0.5cm}C (\textbf{c})...
  \}  
\end{alltt}
\end{CodeOut}
\vspace{-0.15 in}
\caption{An Example Program}
\label{fig:example}
\end{figure}

	Consider the example in Figure~\ref{fig:example}. Figure~\ref{fig:example} contains a class \CodeIn{TestClass}. The class \CodeIn{TestClass} contains a method \CodeIn{testMe} that has been changed in the new version. The class also contains 3 other methods \CodeIn{A}, \CodeIn{B}, and \CodeIn{C} that have not been changed in the new version. Suppose the statement at Line 15 of the method \CodeIn{testMe} has been modified to the one shown in the comment at Line 15.
	Figure~\ref{fig:Changed} shows the changed parts between the original and synthesized class. Our approach inserts an argument \CodeIn{isOriginalVersion} of type \CodeIn{boolean} in the method containing the changed statement. Our approach also inserts new branches at the location of changed statement(s). The \CodeIn{true} branch contains the original statement while the \CodeIn{false} branch contains the modified statement. The \CodeIn{true} branch is executed when \CodeIn{isOriginalVersion} is \CodeIn{true} and vice versa. The Line 15 in Figure~\ref{fig:example} has been changed to Lines 15-18 in Figure~\ref{fig:Changed}. The \CodeIn{true} branch of the \CodeIn{if} statement contains the original statement, while the false branch contains the statement in modified version. The synthesized class \CodeIn{TestClassSynthesized} helps us in concurrently running the two versions without having to execute two instances of DSE on the two versions. After every run of DSE, we flip the \CodeIn{if(isOriginalVersion)} branch (if the branch is executed in that particular run) to check the program behavior in the modified version. In the rest of this paper we will refer to the program in Figure~\ref{fig:example}. Whenever we refer to comparison of program states after execution of the two versions under test, it means we flip the branch at Line 14 to execute the statement in the other version and compare the program states before and after flipping the branch.
	
	\begin{figure}[t]
\begin{CodeOut}
\begin{alltt}
  \textbf{public} TestClassSynthesized\{
  \hspace{0.5cm}\textbf{public boolean} testMe(\textbf{int }x, \textbf{int[] } y, 
  \hspace{3cm}\textbf{boolean} isOriginalVersion)\{
  \hspace{2.0cm} ...
14\hspace{2.0cm} \textbf{if}(j > 0)
15\hspace{2.5cm} \textbf{if}(isOriginalVersion)	 
16\hspace{3.0cm} x = j+2;
17\hspace{2.5cm} \textbf{else}
18\hspace{3.0cm} x = 2*j+1
  \hspace{2.0cm} ...
  \}
\end{alltt}
\end{CodeOut}
\vspace{-0.15 in}
\caption{Program synthesized for the example program}
\label{fig:Changed}
\end{figure}

\begin{figure}[t]
\begin{CodeOut}
\begin{alltt}
  [PexFactoryClass]
  \textbf{public class} FactoryClass\{
  \hspace{0.5cm}[PexFactoryMethod(typeof(\textbf{TestClassSynthesized}))]
  \hspace{0.5cm}\textbf{public static TestClassSynthesized} create(int a, 
  \hspace{1.5cm}int b, int c, int n)\{
  \hspace{1.0cm}\textbf{TestClassSynthesized} obj = 
  \hspace{1.5cm}\textbf{new} TestClassSynthesized(); 
  \hspace{1.0cm} \textbf{for}(int i=0; i< methodSeuenceLength; i++)\{
  \hspace{1.5cm} \textbf{switch}(n)
  \hspace{2.0cm} \textbf{case} 0: \textbf{obj}.A(a);\textbf{break};
  \hspace{2.0cm} \textbf{case} 1: \textbf{obj}.B(b);\textbf{break};
  \hspace{2.0cm} \textbf{case} 2: \textbf{obj}.C(c);\textbf{break};
  \hspace{2.0cm} \textbf{case} 3: \textbf{obj}.testMe();\textbf{break};
  \hspace{2.0cm} \textbf{default}: \textbf{break};
  \hspace{1.0cm}\}
  \hspace{1.5cm} \textbf{return} obj;
  \hspace{0.5cm}\}
  \}
\end{alltt}
\end{CodeOut}
\vspace{-0.15 in}
\caption{Factory Class to generate an object of \CodeIn{TestClassSynthesized}}
\label{fig:factory}
\end{figure}

\begin{figure}[t]
\begin{CodeOut}
\begin{alltt}
  [TestClass]
  \textbf{public class} PUTClass\{
  \hspace{0.5cm}[PexMethod]
  \hspace{0.5cm}\textbf{public boolean} testMePUT(\textbf{TestClassSynthesized} obj, int x, 
  \hspace{1.5cm}int[] y, boolean version)\{
 1\hspace{1.0cm}PexAssume.IsNotNull(obj);
 2\hspace{1.0cm}obj.testMe(x, y, version);
  \hspace{0.5cm}\}
  \}
\end{alltt}
\end{CodeOut}
\vspace{-0.15 in}
\caption{Parameterized test for Method \CodeIn{testMe}}
\label{fig:PUT}
\end{figure}
	
	The \emph{Instrumenter} component also synthesizes a Parametrized Unit Test in addition to a factory method for the class under test for Pex to generate tests. Figure~\ref{fig:PUT} shows a Parameterized Unit Test for Pex to generate tests for the method \CodeIn{testMe} in the class \CodeIn{TestClassSynthesized}. Figure~\ref{fig:factory} shows the factory method used for synthesizing an object for \CodeIn{TestClassSynthesized}. The factory method instantiates a new instance of the class \CodeIn{TestClassSynthesized}, generates a sequence of method calls containing all combinations \CodeIn{A}, \CodeIn{B}, \CodeIn{C}, and testMe. The length of the sequence is \CodeIn{methodSeuenceLength} which is set to total number of methods in the class \CodeIn{TestClassSynthesized} (4).
	
	

\subsection{Dynamic Test Generation Phase}
In the \emph{Dynamic Test Generation} phase, our approach uses Dynamic Symbolic Execution (DSE)~\cite{dart, cute, exe} for 
generating regression tests for the two given versions of a class. DSE explores all the feasible 
paths in the program to find inputs for covering a specific target. However, there are exponentially 
many paths with respect to the number of branches in the program.
 Many of these branches cannot help us in detecting behavioral differences. 
 In other words, covering these branches do not help in achieving either of E, I, or P in the PIE model described earlier. 
 We do not explore these paths in our search strategies for finding test inputs that expose 
 behavioral differences between the two given versions of a program.
\\ \textbf{Paths not leading to a changed statement.} The paths that are not reachable to $\delta$ need not be 
explored in order for $\delta$ to get executed. For example, consider the method \CodeIn{testMe} in Figure~\ref{fig:example}.
 If the changed statement is the one shown at Line 15 ($\delta$), then while searching for a path 
 to cover $\delta$, we do not need to explore paths containing the \CodeIn{true} branch of the condition at Line 6.
\\ \textbf{Paths after execution of changed statement.} Suppose we cover the target statement 
in Figure~\ref{fig:example} using input \CodeIn{x=90}, and the array \CodeIn{y} of length 20, 
with each element of \CodeIn{y} having a value 15. The execution will take the path $P$ 
executing the loop 20 times, assigning the variable \CodeIn{x} to 110 and eventually 
covering the $\delta$ at line 15 . However, the program state is not infected since 
after execution of $\delta$, the value of \CodeIn{x} is 3 in both versions. Now our 
search strategy need not flip the branch at Line 16 in search of a path that infects 
the program state just after the execution of $\delta$.
\\ \textbf{Paths from changed statement to observation point.} If we want to propagate 
a state infection to some observation point, we do not need to explore paths passing 
through $\delta$ that do not lead to an observation point. Note that our approach focuses 
on propagating the infection to one observation point at a time.



The \emph{Dynamic Test Generation} Phase gets the branches in each of the categories from the \emph{Graph Traveler} component of the \emph{Static Analysis} phase. Our approach then prevents the DSE from exploring these branches.


 
 