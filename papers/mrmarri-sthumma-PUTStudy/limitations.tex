\section{Discussion} 
\label{sec:limitations}

One of the major limitations of PUTs is that PUTs requires
more effort from developers than writing conventional unit tests requires. Although PUTs reduce the complexity of writing multiple conventional unit tests with various concrete test inputs, developers
need additional expertise in writing such PUTs as PUTs are more generic compared to conventional unit tests. For example, writing PUTs requires developers to prescribe a test oracle that can deal with the generality of test inputs. We show that to reduce the complexity of writing PUTs, we adopted a methodology of writing PUTs in two phases. In Phase 1, we generalized the existing conventional unit tests to PUTs using suggested test patterns. In Phase 2, we used supporting techniques to write more PUTs to assist Pex in generating high-covering tests. 

In our study, we observed that we took longer time to write PUTs in Phase 2 compared to the time we took in Phase 1. In Phase 1, we transformed the existing conventional unit tests to PUTs and the existing unit tests assist in writing PUTs as shown in Section~\ref{sec:example} with an example. In Phase 2, we discovered the un-covered code portions and wrote more PUTs to assist Pex to generate tests to cover the un-covered code portions. Based on our experience, we believe that to enjoy the test effectiveness achieved by writing PUTs and to reduce the cost involved in writing PUTs, a practical solution could be retrofitting PUTs by transforming these conventional unit tests to PUTs. We expect that writing a single conventional unit test to test a method under test and then transforming it to a PUT can help developers in writing the PUTs. Such a single conventional unit test can act as an \textit{example} unit test representing the intention of ``what'' needs to be tested. We expect that this methodology can both ease the process of writing PUT and still achieve high test effectiveness in unit testing. 

As shown in our study, although the usage of the suggested test patterns and the supporting techniques can reduce effort of developers in writing PUTs and allow the test generation tool to generate high-covering tests, the complexity lies in being able to use them. In general, developers might consider it a tougher job to learn the supporting techniques and use them to write PUTs than writing multiple possible conventional unit tests. Nevertheless, our study shows that PUTs are more effective than conventional unit tests in detecting defects and also in achieving high code coverage. Therefore, we believe that the choice of writing PUTs is a trade-off between cost and benefit.