\section{Open Source Project Under Test} 
\label{sec:opensource}

NUnit is a widely used open source unit-testing framework for all .NET languages, a counterpart of JUnit for Java~\cite{nunit}. NUnit is written in C\# and uses attribute-based programming model~\cite{TDD} 
through a variety of attributes such as \CodeIn{[TestFixture]} and \CodeIn{[Test]}.
The rationale behind choosing NUnit for our test generalization is the large number of manually written unit tests available with the project. These unit tests also provide information about the runtime behavior of the system. The source code of the entire project includes 560 files and about 53 KLOC. The test code includes 264 source files with 25 KLOC.
This significant amount of test code makes this project a desirable subject
for our empirical study. For the purpose of the study, 
we chose the Util package (\CodeIn{nunit.util.dll}), which is one of the core components of the framework.
\setlength{\tabcolsep}{3pt}
\begin{table}[t]
\begin{center}
\centering
\begin{tabular}{|l|r|}
\hline
\textbf{Attribute} & \textbf{Value} \\
\hline
\#Total Files & 560\\
\hline
\#Files in NUnit.Util & 72\\
\hline
Total LOC & 53K\\
\hline
NUnit.Util LOC & 7.2K\\
\hline
\# Test Files of NUnit.Util & 32\\
\hline
\end{tabular}
\end{center}
%\end{SmallOut}
\caption{Code metrics of project used in the study (NUnit)\label{tab:utilmetrics}}
\end{table}

The Util package includes 7.2 KLOC with 72 files and 326 methods. 
The numbers of test files, test methods, and test LOC account to 32, 335, and 3.4 KLOC, respectively. 
The reason for choosing the \CodeIn{Util} namespace\footnote{A namespace in C\# is equivalent to a package in Java.} for the study is two-fold (1) its significance in probably being one of the first modules being developed for the framework (2) its being an independent module reduces the amount of work required to facilitate unit testing by reducing dependent modules of the module used in the study.

%but also the existence of non-trivial code fragments such as the example
%shown in Figure~\ref{fig:nunitexample}, that enable us to study the 
%test generalization process.

%\begin{figure}[t]
%\begin{CodeOut}
%\begin{alltt}
%01:public static string Canonicalize( string path )
%02:\{
%03:\hspace*{0.1in}ArrayList parts = new ArrayList(path.Split(
%04:\hspace*{0.3in}DirectorySeparatorChar,AltDirectorySeparatorChar));
%05:\hspace*{0.1in}for( int index = 0; index < parts.Count; )
%06:\hspace*{0.1in}\{
%07:\hspace*{0.3in}string part = (string)parts[index];
%08:\hspace*{0.3in}switch( part )
%09:\hspace*{0.3in}\{
%10:\hspace*{0.3in}	case ".":
%11:\hspace*{0.5in}	parts.RemoveAt( index );
%12:\hspace*{0.5in}	break;
%13:\hspace*{0.3in}	case "..":
%14:\hspace*{0.5in}	parts.RemoveAt( index );
%15:\hspace*{0.5in}	if ( index > 0 )
%16:\hspace*{0.7in} parts.RemoveAt( --index );
%17:\hspace*{0.5in}	break;
%18:\hspace*{0.3in}	default:
%19:\hspace*{0.5in}	index++;
%20:\hspace*{0.5in}	break;
%21:\hspace*{0.3in}\}
%22:\hspace*{0.1in}\}
%23:\hspace*{0.1in}return String.Join(DirectorySeparatorChar.ToString(),
%24:\hspace*{0.3in}(string[])parts.ToArray( typeof( string ) ) );
%25:\}
%\end{alltt}\end{CodeOut}
%	\caption{\label{fig:nunitexample}A non-trivial code fragment from the Util package of the NUnit framework.}
%\end{figure}


%\begin{tabular}{l|r|r|r|r|r|r}
%\hline
%\textbf{Language} & \textbf{files} & \textbf{blank} & \textbf{comment} & \textbf{code} &\textbf{scale} &\textbf{3rg gen. equiv} \\
%\hline
%C\#  & 58  &  1301 &  1371   &  5585 x  & 1.36 =    &    7595.60\\
%\hline
%MSBuild scripts  &    2      &   1  &   0  &     602 x  &  1.90 =    &    1143.80\\
%\hline
%NAnt scripts    &      1   &      3    &     0    &    85 x  &  1.90 =  &       161.50\\
%\hline
%SUM:    &   61  &    1305   &   1371   &   6272 x  & 1.42 =  &      8900.90 \\
%\hline
%\end{tabular}
%-------------------------------------------------------------------------------
%Language          files     blank   comment      code    scale   3rd gen. equiv
%-------------------------------------------------------------------------------
%C#                   58      1301      1371      5585 x   1.36 =        7595.60
%MSBuild scripts       2         1         0       602 x   1.90 =        1143.80
%NAnt scripts          1         3         0        85 x   1.90 =         161.50
%-------------------------------------------------------------------------------
%SUM:                 61      1305      1371      6272 x   1.42 =        8900.90
%-------------------------------------------------------------------------------
