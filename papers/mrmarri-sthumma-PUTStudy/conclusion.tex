\section{Conclusion}
\label{sec:conclusion}
\vspace*{2ex}
We conducted an empirical study to investigate the utility of PUTs in unit testing. We first generalize the existing conventional tests by transforming conventional
unit tests into PUTs, and then write new PUTs to increase code coverage. In Phase 1 of our study, we generalized $57$ conventional unit tests in the NUnit framework to write $49$ PUTs. We identified benefits of test generalization such as increase in the block coverage by $9.68$\% (on average) with a maximum increase of $45.26\%$ for one class under test and detection of $7$ new defects. In Phase 2 of our study, we wrote $21$ new PUTs and achieved an increase in the block coverage of $17.41$\% over the conventional unit tests. We also identified types of conventional tests that are not amenable for test generalization and proposed new PUT patterns. We present details on the utility of suggested test patterns and supporting techniques that help in writing PUTs. We further discuss the limitations of retrofitting unit tests for PUTs in terms of effort required in writing these PUTs. 