\section{Conclusion}
\label{sec:conclusion}

Recent advances in software testing introduced Parameterized Unit Tests (PUT), which are a generalized form of conventional unit tests (CUTs). With PUTs, developers do not need to provide test data (in PUTs), which is generated automatically using the dynamic-symbolic-execution approach. Although PUTs are more beneficial than CUTs in assuring the quality of the code under test, PUTs are still not widely adopted in software industry. In practice, many developers still rely on CUTs as a primary means for ensuring the quality of their developed software. These CUTs can be transformed to PUTs, referred to as test generalization, to exploit the benefits of PUTs. In this paper, we proposed the first systematic procedure that assists developers in performing test generalization. In our study, we also showed that test generalization reduces $407$ CUTs to $224$ PUTs (45\%), thereby reducing the effort of maintaining test code. Along with achieving higher branch coverage (a maximum increase of 52\% for one class under test and 10\% for one application under analysis), test generalization detected $16$ new defects that are not detected by existing CUTs. A few of these defects are quite complex and are hard to be detected using CUTs. Given these benefits of test generalization, we expect that developers in the software industry can use our systematic procedure and transform their CUTs to PUTs to exploit the benefits of PUTs, and thereby increase the quality of their developed software.

%We conducted an empirical study to investigate the utility of PUTs in unit testing. We first generalize the existing conventional tests by transforming conventional
%unit tests into PUTs, and then write new PUTs to increase code coverage. In Phase 1 of our study, we generalized $57$ conventional unit tests in the NUnit framework to write $49$ PUTs. We identified benefits of test generalization such as increase in the block coverage by $9.68$\% (on average) with a maximum increase of $45.26\%$ for one class under test and detection of $7$ new defects. In Phase 2 of our study, we wrote $21$ new PUTs and achieved an increase in the block coverage of $17.41$\% over the conventional unit tests. We also identified types of conventional tests that are not amenable for test generalization and proposed new PUT patterns. We present details on the utility of suggested test patterns and supporting techniques that help in writing PUTs. We further discuss the limitations of retrofitting unit tests for PUTs in terms of effort required in writing these PUTs.