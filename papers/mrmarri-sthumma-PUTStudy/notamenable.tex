\section{Conventional Unit Tests Not Am- enable to Test Generalization} 
\label{sec:notamenable}

In our study, we found that $12.75$\% of conventional unit tests are not amenable to test generalization. There are two common cases under which we found that conventional unit tests are not amenable to test generalization. 

\textbf{Default checks.} We found that there are multiple conventional unit
tests that verify default values, which are often static values. We suspect
that the purpose of these tests could be to make sure that the developers do not
change the static values accidentally. We suspect that these tests are
not amenable for test generalization as values verified by these tests
are constants.

\textbf{Missing test oracles.} We found that test generalization
may cause the loss of test oracles in some cases. These cases can often occur when generalizing conventional unit tests that belong to the PUT patterns \emph{Roundtrip} and \emph{Commutative diagram}.
We explain this issue using the illustrative example shown below.

\begin{CodeOut}
\begin{alltt}
public void Canonicalize() \{
\hspace*{0.1in}PexAssert.AreEqual(@"C:/folder1/file.tmp",
\hspace*{0.3in}PathUtils.Canonicalize(@"C:/folder1/./folder2/
\hspace*{0.3in}../file.tmp")); 
\}
\end{alltt}
\end{CodeOut}
 
The \CodeIn{Canonicalize} method in \CodeIn{PathUtils} accepts
a \CodeIn{string} parameter and uses a complex procedure to transform the input \CodeIn{string} into a standard form. It is easy to identify the expected output for concrete strings
such as ``\CodeIn{C:/folder1/.../folder2/.. ./file.tmp}''. However, when the
conventional unit test is generalized with a parameter for the input string, it is challenging to 
identify the expected output. Developers need to implement a logic to assert the behavior of the method under test. The amount of effort required in such cases could be higher than the effort required to write the implementation of the actual method under test and therefore such conventional unit tests are not amenable to test generalization.
