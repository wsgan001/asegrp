\begin{table*}[t]%
\centering
\begin{minipage}{\textwidth}
\centering
\begin{tabular}{|l|r|r|r|r|r|r|r|r|}
%----------------- HEADER ------------------------
\hline
Subject 		& Downloads 	& \multicolumn{4}{|c|}{Code Under Test} 					&	\multicolumn{3}{|c|}{Existing Test Code} \\ \cline{3-9}
						& 						&	\#Classes	&	\#Methods	& \#LOC	& Avg. Complexity	&	\#Classes	&	\#Methods &	\#LOC		\\ \hline\hline
%----------------- END HEADER ------------------------ 
NUnit				&		$193,563$\footnote{Anonymous read count in the year 2009}
													&						&						&				&									&						&						&					\\ \hline
DSA					&		$2241$		&						&						&				&									&						&						&					\\ \hline		
QuickGraph	&		$7969$ 		&						&						&				&									&						&						&					\\ \hline
\end{tabular}
%\footnotetext[1]{Anonymous read count in the year 2009}
\end{minipage}
\caption{Details of the subject applications}
\label{tab:subjectmetrics}
\end{table*}

\section{Empirical Study}

We conducted an empirical study using three real world applications to study the benefits of Parameterized Unit Tests (PUT) over Conventional Unit Tests (CUT) in specific and to show the applicability of the proposed systematic procedure in general. In our empirical study, we show the benefits of PUTs over existing CUTs in terms of three metrics: code coverage, defects, and test code maintenance. In particular, we address the following three research questions through our empirical study:

\begin{itemize}
	\item \textbf{RQ1: Coverage.} How much higher percentage of \emph{code coverage} is achieved by PUTs compared to existing CUTs? Since PUTs are a generalized form of CUTs, this research question helps to address whether PUTs can achieve additional coverage compared to CUTs.
	\item \textbf{RQ2: Defects.} How many new \emph{defects} (that are not detected by CUTs) are detected by PUTs and vice-versa? This research question helps to address whether PUTs have more fault-detection capabilities compared to CUTs.
	\item \textbf{RQ3: Test code maintenance.} How many number of tests are reduced by generalizing CUTs to PUTs? This research question helps to address whether PUTs require lesser efforts for maintaining test code compared to CUTs.
\end{itemize}

To address these research questions, we execute the existings CUTs of each application and execute the unit tests generated by Pex with the help of the transformed PUTs and measure the three metrics. Initially, we execute the existing CUTs and meticulously record the code coverage achieved and note the failing test cases as defects. In addition to that, we measure the number of CUTs and lines of code of each of the CUTs. We then generalize these CUTs to PUTs based on our systematic procedure. Consequently, we apply Pex using the generalized PUTs and record the coverage achieved, the number of failing test cases, the number of PUTs, and the number of lines of these PUTs. Furthermore, we manually analyze to ensure that the failing test cases did not fail due to invalid PUTs, such as missing assumptions on the parameters. We thus confirm that the test cases failed due to a defect in the code under test. 

\noindent\textbf{Metrics}

\subsection{Subject Applications}

We chose three open source applications to carry out our study, NUnit~\cite{nunit}, DSA~\cite{dsa}, and Quickgraph~\cite{quickgraph}. The criteria for choosing these projects are (1) large usage of the application in terms of the download records shown by the hosting site, (2) Existing conventional unit test suite size, (3) the size of the application sources, and (4) the code under test should contain non-trivial logic to validate the feasibility of test generalization. Table~\ref{tab:subjectmetrics} shows the characteristics of the three subject applications. Column ``Subject'' gives the name of the subject applications and Column ``Downloads'' gives the number of downloads of the applications as listed on the respective hosting websites. Column ``Code Under Test'' gives details of the code under test in the corresponding application in terms of the number of classes (``\#Classes''), number of methods (``\#Methods''), number lines of code (``\#LOC''), and the average complexity of the code under test. Column ``Existing Test Code'' provides details on the existing test code corresponding to the code under test. Subcolumns ``\#Classes'', ``\#Methods'', and ``\#LOC'' give the corresponding test code's metrics. In the following sections, we provide details on each of the subject applications.

\subsubsection{NUnit}
\label{sec:nunit}
We use an open source application, called NUnit~\cite{nunit}, as one of the subject applications for our study. NUnit, a counterpart of JUnit for Java~\cite{JUnit}, is a widely used open source unit-testing framework for all .NET languages. NUnit is written in C\# and uses attribute-based programming model~\cite{TDD} through a variety of attributes such as \CodeIn{[TestFixture]} and \CodeIn{[Test]}. The rationale behind choosing NUnit for test generalization is its popularity in terms of the number of downloads ($450$ anonymous reads in the year 2002 and $193,563$ in the year $2009$) of the application as listed by its hosting website SourceForge\footnote{http://sourceforge.net/}. Furthermore, the large number of manually written unit tests available with the project makes NUnit a good subject for test generalization. For the purpose of the study, we used $9$ classes from the \CodeIn{util} package (\CodeIn{nunit.util.dll}), which is one of the core components of the framework. We chose the \CodeIn{util} package in the study for two reasons: (1) it is one of the basic modules to be developed for the framework (2) it is an independent module and is not dependent on the other modules of the NUnit framework. We chose the $9$ classes as the code under test since the logic in these classes is non-trivial; the average complexity is $0$ as shown in Table~\ref{tab:subjectmetrics}. For our study, we used NUnit release version $2.4.8$. 

\subsubsection{DSA}

Data Structures and Algorithms (DSA)~\cite{dsa} is a library that contains implementation of data structures and algorithms, a few of which are not available in the .NET 3.5 framework. There are two major packages in the library's source code, \CodeIn{Dsa.DataStructures} and \CodeIn{Dsa.Algorithms}. The library containt $0$ classes and $0$ methods. The existing test suite available with the source code release of the library contains a total of $0$ test classes and $0$ test cases. The rationale behind choosing DSA is the size of its test suite available with the release version $0.6$. The existing test suite containted test cases that test most of the source code. Furthermore, another reason for choosing DSA for our study is the influence in terms of the large number of downloads, a total of $2236$ downloads as shown by its hosting site, Codeplex\footnote{http://codeplex.com/}. 

\subsubsection{QuickGraph}

QuickGraph~\cite{quickgraph} is a C\# graph library that provides various directed/undirected graph data structures. QuickGraph also provides algorithms such as depth-first search, breadth-first search, and A* search. QuickGraph includes $165$ classes and interfaces with $5$ KLOC. The existing test suite available with the source code release of the library contains a total of $0$ test classes and $0$ test cases. The existing test suite containted test cases that majorly test two search algorithms, one sorting algorithm, and graph concepts. We generalized only these existing test cases again for this subject, i.e., the code under test contains $0$ classes and $0$ methods. Since QuickGraph includes graph search algorithms, the code under test includes non-trivial logic and is therefore a good subject for our test generalization. QuickGraph is also popularly used reflected by the number of downloads, $7969$ as shown by CodePlex.


% Tao, in the following, could you add something so that the following policy

% coverage definitions are related to those abstract ones defined in section 3?

\section{Policy Coverage in XACML}
\label{sec:coverage}

In XACML languages, we can see there are three major entities:
policies, rules for each policy, and a condition for each rule. We
define policy coverage as follows:

\begin{itemize}

\item \Intro{Policy hit percentage}. A policy is hit by a request
if the policy is applicable to the request; in other words, all the
conditions in the policy's target are satisfied by the request.
Policy hit percentage is the number of hit policies divided by the
number of total policies.

\item \Intro{Rule hit percentage.} A rule for a policy is hit by a
request if the rule is also applicable to the request; in other
words, the policy is applicable to the request and all the
conditions in the rule's target are satisfied by the request. Rule
hit percentage is the number of hit rules divided by the number of
total rules.

\item \Intro{Condition hit percentage.} The evaluation of the
condition for a rule has two outcomes: true and false, which are
called as the true condition and false condition, respectively. A
true condition for a rule is hit by a request if the rule is
applicable to the request and the condition is evaluated to be true.
A false condition for a rule is hit by a request if the rule is
applicable to the request and the condition is evaluated to be
false. Condition hit percentage is the number of hit true conditions
and hit false conditions divided by twice of the number of total
conditions.

\end{itemize}

Note that a policy has at least one rule but a rule can have no
condition, indicating an implicit condition \CodeIn{true}, which is
always satisfied when the rule is applicable. Therefore, when there
are no conditions defined within the policies under consideration,
the condition hit percentage is always the same as the rule hit
percentage. Normally a policy tester shall be able to generate
requests to achieve 100\% for all three types of policy coverage. In
other words, all the to-be-covered entities defined in the policy
coverage are feasible to be covered in principle; otherwise, those
infeasible parts of policy specifications could be removed like dead
code in programs.

To automate the measurement of policy coverage, we have developed a
measurement tool by instrumenting Sun's open source XACML
implementation~\cite{sun05:xacml}. Sun's implementation facilitates
the construction of a PDP. We instrument several methods throughout
their implementation that collect policy, rule, and condition
information when a policy is loaded into the PDP. Then coverage
information is collected and stored in a singleton as requests are
evaluated by a PDP against the policy under test.

\Comment{ To automate the measurement of policy coverage, we have
developed a measurement tool based on Sun's open source XACML
implementation~\cite{sun05:xacml}, written in Java. Based on Sun's
XACML implementation, we first built a Policy Decision Point (PDP),
which receives an access request and returns an access decision. We
then developed several public methods in a Java class for collecting
runtime coverage and insert some call sites to these methods in
several places in the code of Sun's XACML implementation. When PDP
loads given policies, we insert a method call to collect all the
policies, rules, and conditions in the given policies. Every time
PDP determines that all the conditions in a policy's target are
satisfied, we insert a method call to collect policy hit
information. Every time PDP determines that all the conditions in a
rule's target are satisfied, we insert a method call to collect rule
hit information. Every time PDP determines that a condition for a
rule is evaluated to be true or false, we insert a method call to
collect condition hit information.}

After the PDP returns the decision, we output the coverage
information into a text file, whose name is determined by the names
of given policies; if a text file with the same name exists, the
coverage information in the text file is updated by incorporating
the new coverage information. Therefore, when PDP receives several
requests separately against the same set of policies, the aggregated
coverage information achieved by these requests is collected.
Besides the basic coverage information, we also output the details
of covered entities and their covering requests as well as the
details of uncovered entities. The extra information can help
developers or external tools in generating or selecting requests for
achieving higher policy coverage.

\subsection{RQ2: Defects}

To address RQ2, we identify the number of defects detected by PUTs. Since we are using the CUTs that are available with the sources, we did not find any failing CUTs, i.e., no defects are reported by the existing unit tests for any of the subjects. Therefore, any failing tests among the unit tests generated from PUTs are considered as new defects that are not detected by the CUTs. However, before confirming whether a failing unit test implies a defect in the code under test, we manually verify that the generated test data is valid and the defect is not a false positive due to an ineffective PUT. 

Through test generalization of the three subjects in our study, we found $13$ new defects in the DSA application and $3$ new defects in the NUnit application. Since we were sure that the resulting failing tests were due to the defects in the code under test, we reported the failing test scenarios on the hosting website and waiting for the confirmation from the developers\footnote{Reported bugs can be found at the DSA CodePlex website with defect IDs from $8846$ to $8858$ and the NUnit SourceForge website with defect IDs from $0$ to $0$.}.

We next explain an example defect detected in the \CodeIn{Heap} class for the DSA application by our test generalization. The \CodeIn{Heap} class is a heap data structure implementation in the \CodeIn{DataStructure} namespace. This class includes methods to add, remove, and to heapify the elements in the heap. The \CodeIn{Remove} method of the class takes an item to be removed as a parameter and returns \CodeIn{True} when the item to be removed is in the heap, and returns \CodeIn{False} otherwise. Figure~\ref{fig:defectCUT} shows the existing CUT that tests whether the \CodeIn{Remove} method returns \CodeIn{False} when an item that is not in the heap is passed as the parameter. On executing, this CUT passes indicating no defect in the code under test since there are no other CUTs in the test suite that test the behavior of the method. However, from our generalized PUT shown in Figure~\ref{fig:defectPUT}, a few of the generated unit tests failed indicating the possibility of a defect in the \CodeIn{Remove} method. The test data for the failing unit tests had the following common properties: the heap size is less than $4$ (the \CodeIn{input} parameter of the PUT is of size less than $4$), the item to be removed is $0$ (the \CodeIn{item} parameter of the PUT), and the item $0$ was not already added to the heap (generated value for \CodeIn{input} did not contain the item $0$). When we manually analyzed the cause of the failing unit tests, we found that in the constructor of the \CodeIn{Heap} class, a default array of size $4$ (of type \CodeIn{int}) is created to store the items. In C\# an integer array is by default assigned values zero to the elements of the array. Therefore, there is always an item $0$ in the heap unless an input list of size greater than or equal to $4$ is passed as parameter. Therefore, on calling the \CodeIn{Remove} method to remove the item $0$, even when there is no such item in the heap, the method returns \CodeIn{True} indicating that the item has been successfully removed and causing the assertion statement to fail (statement $6$ of the PUT). However, this defect was not detected by the CUT shown in Figure~\ref{fig:defectCUT} as the unit test assigns the heap with $5$ elements (statement $2$) and therefore the defect-exposing scenario of heap size $< 4$ cannot be encountered. 

\begin{figure}[t]
\begin{CodeOut}
\begin{alltt}
//To test Remove item not present
01: public void RemoveCUT()\{
02: \hspace*{0.07in}Heap<int> actual = new Heap<int> \{2, 78, 1, 0, 56\};
03: \hspace*{0.07in}Assert.IsFalse(actual.Remove(99));
04: \hspace*{0.02in}\}
\end{alltt}
\end{CodeOut}
\caption{Existing CUT to test the \CodeIn{Remove} method of the \CodeIn{Heap} class}
\label{fig:defectCUT}
\end{figure}

\begin{figure}[t]
\begin{CodeOut}
\begin{alltt}
01: public void RemoveItemPUT(
\hspace*{0.7in} [PAUT]List<int> input, int item) \{
02: \hspace*{0.07in}Heap<int> actual = new Heap<int> (input);
03: \hspace*{0.07in}if (input.Contains(item)) \{
04: \hspace*{0.2in}..... \}
05: \hspace*{0.07in}else \{
06: \hspace*{0.2in}PexAssert.IsFalse(actual.Remove(randomPick));
07: \hspace*{0.2in}PexAssert.AreEqual(input.Count, actual.Count);
08: \hspace*{0.2in}CollectionAssert.AreEquivalent(actual, input);\}
09: \hspace*{0.02in}\}
\end{alltt}
\end{CodeOut}
\caption{Generalized PUT of the existing CUT shown in Figure~\ref{fig:defectCUT}.}
\label{fig:defectPUT} 
\end{figure}

This example defect that was detected in our study displays the strength of generalized unit tests such as PUTs. The $16$ defects reported in our study that were not detected by the existing unit tests show that PUTs can assist in an unbiased test generation and are an effective means for rigorous testing of the code under test.  
%------------------------------------------------------------------------------------------------------------------------------------------
%We conduct second evaluation based on mutation testing to further show the effectiveness of PUTs in detecting defects compared to CUTs. The reason for the second evaluation is that the existing CUTs do not detect any defects in the code under test. In this evaluation, we seed defects in the code under test using a mutation testing tool and verify
%how many mutants are killed by existing CUTs and PUTs. We consider that a mutant is killed if any previously passing test fails after executed with the seeded fault. 
%
%We use the following five basic mutation operators, recommended by Offutt, for seeding defects in the code under test.
%
%\begin{itemize}
%\item ABS: Forces each arithmetic expression to take values zero, positive and negative values.
%\item AOR: Replaces each arithmetic operator with every syntactically legal operator.
%\item LCR: Replaces logical connector (AND and OR) with other kinds of logical operators.
%\item ROR: Replaces each relational operator with other relational operators.
%\item UOI: Inserts unary operators in front of expressions.
%\end{itemize}
\subsection{Maintenance of Test Code}

We next address the third research question of whether PUTs reduce the efforts for the
maintenance of test code. We address this question by counting the number of CUTs and
the number of PUTs generalized from CUTs. We also calculate the lines of code of CUTs
and PUTs. 