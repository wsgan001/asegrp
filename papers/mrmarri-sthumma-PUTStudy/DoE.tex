\begin{table*}[t]%
\centering
\begin{minipage}{\textwidth}
\centering
\begin{tabular}{|l|r|r|r|r|r|r|r|r|r|}
%----------------- HEADER ------------------------
\hline
Subject 		& Downloads 	& \multicolumn{5}{|c|}{Code Under Test} 															&	\multicolumn{3}{|c|}{Existing Test Code} 	\\ \cline{3-10}
						& 						&	\#Classes	&	\#Methods	& \#KLOC	& Avg. Complexity & Max. Complexity	&	\#Classes	&	\#CUTs	 	&	\#KLOC						\\ \hline\hline
%----------------- END HEADER ------------------------ 
NUnit				&		$193,563$\footnote{Number of reads as shown in sourceforge for the year 2009.}
													&	9					&	87  			&		1.4		&				1.48 			& 14 							&			9			&			49		&		0.9							\\ \hline
DSA					&		$2241$		&	27  			&	259 			&		2.4		&				2.09 			& 16							&			20		&			337		&		2.5							\\ \hline		
QuickGraph	&		$7969$ 		&	56				&	463				&		6.2		&				1.79			& 16							&			9			&			21		&		1.2							\\ \hline
\end{tabular}
%\footnotetext[1]{Anonymous read count in the year 2009}
\end{minipage} \vspace*{-3ex}
\caption{Details of the subject applications.} \vspace*{-3ex}
\label{tab:subjectmetrics}
\end{table*} 

\section{Empirical Study}
\label{sec:study}

We conducted an empirical study using three real-world applications to show the benefits of generalizing CUTs to PUTs in specific and also to show the applicability of our systematic procedure in general. In our empirical study, we show the benefits of PUTs over existing CUTs in terms of three metrics: branch coverage, the number of detected defects, and the number of tests (their LOC) being reduced by test generalization. In particular, we address the following three research questions in our empirical study:

\begin{itemize}
	\item \textbf{RQ1: Branch Coverage.} How much higher percentage of \emph{branch coverage} is achieved by PUTs compared to existing CUTs? Since PUTs are a generalized form of CUTs, this research question helps to address whether PUTs can achieve additional branch code coverage compared to CUTs.
	\item \textbf{RQ2: Defect Detection.} How many new \emph{defects} (that are not detected by CUTs) are detected by PUTs and vice-versa? This research question helps to address whether PUTs have higher fault-detection capabilities compared to CUTs.
	\item \textbf{RQ3: Test-code maintenance.} How many tests are reduced by generalizing CUTs to PUTs? This research question helps to address whether PUTs require less effort in maintaining test code compared to CUTs.
\end{itemize}

To address these research questions, we execute the existing CUTs (of each application) and the unit tests generated by Pex with the help of the transformed PUTs, and measure the preceding three metrics. Initially, we execute the existing CUTs and meticulously record the achieved branch coverage and also the failing tests as defects. In addition, we measure the number of CUTs and lines of code (LOC) of CUTs for all applications. We then generalize these CUTs to PUTs based on our systematic procedure and use Pex to generate unit tests. We then execute the new tests generated from PUTs and record achieved branch coverage, the number of failing tests, the number of PUTs, and the number of LOC of PUTs. Furthermore, we manually analyze all failing tests to ensure that the failing tests are not due to invalid PUTs, such as missing assumptions on the parameters. We thus confirm that the finally recorded failing tests are due to defects in the code under test.

For measuring the branch coverage of the code under test, we use a coverage measurement tool, called NCover\footnote{http://www.ncover.com/}. For measuring the code metrics such as LOC and cyclomatic complexity, we use the Count Lines of Code\footnote{http://cloc.sourceforge.net/} (CLOC) tool.

\subsection{Subject Applications}

We use three open source applications in our study: NUnit~\cite{nunit}, DSA~\cite{dsa}, and Quickgraph~\cite{quickgraph}. The criteria for choosing these applications are (1) popular usage of these applications in industry as shown by their download records in their hosting web sites, (2) the existing CUTs, (3) the number of lines of the code under test of these applications, and (4) the code under test itself including non-trivial logic to validate the feasibility of our systematic procedure. Table~\ref{tab:subjectmetrics} shows the characteristics of the three subject applications. Column ``Subject'' gives the name of the subject application. Column ``Downloads'' gives the number of downloads of the application as listed in its hosting web site. Column ``Code Under Test'' gives details of the code under test (of the application) in terms of the number of classes (``\#Classes''), number of methods (``\#Methods''), number of lines of code (``\#LOC''), and the average complexity of the code under test. Column ``Existing Test Code'' provides details on the existing test code similar to the details of the code under test. Subcolumns ``\#Classes'', ``\#Methods'', and ``\#LOC'' give the corresponding metrics of the test code. We next provide details of all the subject applications.

\subsubsection{NUnit}
\label{sec:nunit}
We use an open source application, called NUnit~\cite{nunit}, as one of the subject applications for our study. NUnit, a counterpart of JUnit for Java~\cite{JUnit}, is a widely used open source unit-testing framework for all .NET languages. NUnit is written in C\# and uses attribute-based programming model~\cite{TDD} through a variety of attributes such as \CodeIn{[TestFixture]} and \CodeIn{[Test]}. The rationale behind choosing NUnit for test generalization is its popularity  (increase in the usage from $450$ reads in the year 2002 to $193,563$ in the year $2009$) as listed by its hosting website SourceForge\footnote{http://sourceforge.net/}. Furthermore, the large number of manually written unit tests available with the application makes NUnit a suitable subject for test generalization. For the purpose of the study, we used nine classes from the \CodeIn{util} package (\CodeIn{nunit.util.dll}), which is one of the core components of the framework. We chose the \CodeIn{util} package with two reasons: (1) it is one of the basic modules developed for the framework and (2) it is an independent module and is not dependent on the other modules of the NUnit framework. We chose the nine classes as the code under test since the logic in these classes is non-trivial; the average complexity is $1.48$ (maximum of 14) as shown in Table~\ref{tab:subjectmetrics}. For our study, we used NUnit release version $2.4.8$. 

\subsubsection{DSA}

Data Structures and Algorithms (DSA)~\cite{dsa} is a library that contains implementation of data structures and algorithms, a few of which are not available in the .NET 3.5 framework. There are two major packages in the library's source code, \CodeIn{Dsa.DataStructures} and \CodeIn{Dsa.Algorithms}. The DSA library includes $27$ classes and $259$ methods. The existing test suite available with the DSA library includes $20$ test classes and $337$ CUTs. The rationale behind choosing DSA is the size of its test suite available with the release version $0.6$. The existing test suite includes CUTs that test all the classes of the DSA source code. Furthermore, another reason for choosing DSA for our study is the influence of DSA in the software industry as shown by the number of downloads, a total of $2,236$ downloads as shown by its hosting site, Codeplex\footnote{http://codeplex.com/}. 

\subsubsection{QuickGraph}

QuickGraph~\cite{quickgraph} is a C\# graph library that provides various directed/undirected graph data structures. QuickGraph also provides algorithms such as depth-first search, breadth-first search, and A* search. The source code of QuickGraph (version $1.0$) used in our study includes $56$ classes and interfaces with $6.2$ KLOC. The existing test suite available with the source code release of the library includes $9$ test classes and $21$ CUTs. The existing test suite includes CUTs that primarily test two search algorithms, one sorting algorithm, and other graph concepts. We generalized only these existing CUTs of this subject application. Since QuickGraph includes graph search algorithms, the code under test includes non-trivial logic and is therefore a suitable subject for test generalization. QuickGraph is also popularly used as reflected by $7,969$ downloads (shown by CodePlex).

\subsection{RQ1: Branch Coverage}

We next describe our empirical results for addressing RQ1. We execute the existing test suite (CUTs) and measure the branch coverage achieved by these CUTs. We then measure the branch coverage achieved by the unit tests generated by Pex from the given PUTs. Consequently, we compare the branch coverage achieved by both the CUTs and PUTs for each class or namespace.
%We use NCover~\cite{} coverage tool to measure the dynamic coverage achieved by both CUTs and PUTs.  
\begin{table}[t]%
\begin{tabular}{|l|l|r|r|c|}
\hline
%------Header --------------
\textbf{Subject} & \textbf{Namespace}/										& \multicolumn{2}{|c|}
																														{\textbf{Branch Coverage}}  &  \multicolumn{1}{|c|}{\textbf{Coverage}}\\
								 & 	\textbf{Class}												&	\multicolumn{2}{|c|}{}&  \multicolumn{1}{|c|}{\textbf{Increase}}\\
\hline
								 &																	 						&	CUTs							&	PUTs						&	     \\
%------end ------------------
\hline
\hline
NUnit 			& \multicolumn{3}{|c|}{}																					& 10\%\\ \hline
 						& MemorySettings 					&										& 									&			\\
 						&								Storage 	& 100.00\% 					& 100.00\% 					& 		\\ \cline{2-4}
 						& NunitProject						&  \textbf{76.00\%}	& \textbf{77.00\%}	& 		\\ \cline{2-4}
						& NunitRegistry						&  85.00\%					&  85.00\% 					& 		\\ \cline{2-4}
						&	PathUtils								&  79.00\% 					&  79.00\% 					& 		\\ \cline{2-4}
						& RegistrySettings				&										&										&			\\
						& 								Storage	&	\textbf{48.00\%} 	& \textbf{86.00\%}	& 		\\ \cline{2-4}
						& RemoteTestAgent					& 100.00\% 					&	100.00\% 					& 		\\ \cline{2-4}
						& ServerUtilities					&  91.00\% 					&  91.00\%					& 		\\ \cline{2-4}
						& SettingsGroup						& \textbf{39.00\%}	& \textbf{91.00\%}	& 		\\ \cline{2-4}
						& TestAgency 							&  86.00\%					&  86.00\%					& 		\\ \cline{2-4}
\hline
\hline
DSA 				& \multicolumn{3}{|c|}{} 																					& 1\% \\ \hline
						& Algorithms							& \textbf{93.00\%}	&  \textbf{94.00\%}	& 		\\ \cline{2-4}
						& DataStructures					& \textbf{99.00\%}	& \textbf{100.00\%}	&			\\ \cline{2-4}
						& Properties							& 96.00\%						&  96.00\%					& 		\\ \cline{2-4}
						& Utility 								& 78.00\% 					&  78.00\%					&			\\ \cline{2-4}
\hline
\hline
Quick				& \multicolumn{3}{|c|}{} 														  						&			\\ 
Graph				& \multicolumn{3}{|c|}{} 																					&	2\%	\\ \hline
						& Default									& 87.00\%						&  87.00\%					& 		\\ \cline{2-4}
						& Algorithms							& \textbf{90.00\%}	&  \textbf{92.00\%}	&			\\ \cline{2-4}
						& Collections							& 97.00\%						&  97.00\%					& 		\\ \cline{2-4}
						& Concepts 								& \textbf{72.00\%} 	&  \textbf{83.00\%}	&			\\ \cline{2-4}
						& Exceptions 							&100.00\% 					& 100.00\%					&			\\ \cline{2-4}
						& Predicates 							& 83.00\% 					&  83.00\%					&			\\ \cline{2-4}
						& Representations					& \textbf{82.00\%}	&  \textbf{84.00\%} &			\\ \cline{2-4}
\hline
\hline
\end{tabular}
\caption{Coverage of the existing CUTs and the unit tests generated by Pex using the generalized PUTs. A highlighted row indicates the class/namspace that had an increase in coverage.} \vspace*{-3ex}
\label{tab:coverage}
\end{table}

\begin{figure}[t]
\begin{CodeOut}
\begin{alltt}
01: public void RemoveSetting(string settingName) \{
02: \hspace*{0.05in}int dot = settingName.IndexOf( '.' );
03: \hspace*{0.15in}if (dot < 0)
04: \hspace*{0.2in}storageKey.DeleteValue(settingName, false);
05: \hspace*{0.15in}else \{
06: \hspace*{0.2in}using(RegistryKey subKey = storageKey.OpenSubKey(
\hspace*{0.8in}settingName.Substring(0,dot),true))\{
07: \hspace*{0.3in}if (subKey != null)
08: \hspace*{0.5in}subKey.DeleteValue(
\hspace*{1.0in}settingName.Substring(dot + 1)); \} \}
09: \hspace*{0.02in}\} 
\end{alltt}
\end{CodeOut} \vspace*{-3ex}
\caption{\CodeIn{RemoveSetting} method whose coverage is increased by $60\%$ due to test generalization.} %\vspace*{-2ex}
\label{fig:excoverage}
\end{figure}

Table~\ref{tab:coverage} shows the branch coverage achieved by executing the existing CUTs and the unit tests generated by Pex using the transformed PUTs. Column ``Coverage Increase'' shows the overall increase in the branch coverage from using the existing CUTs and the generalized PUTs. We use NCover to measure these reported branch coverages. For the subjects DSA and QuickGraph, since we generalized all unit tests in the existing test suite, we report the coverage for each namespace. For NUnit, we generalized nine test classes and therefore, we report branch coverage individually for these nine classes under test. For NUnit, we excluded branch coverage for the classes that were covered by the unit tests but were not directly a part of our target code under test. However, the branch coverage for those classes (excluded from the table) were the same for both CUTs and PUTs.

\begin{figure}[t]
\begin{CodeOut}
\begin{alltt}
01: public void CUT() \{
02: \hspace*{0.07in}storage.SaveSetting("X",5);
03: \hspace*{0.07in}storage.SaveSetting("NAME","Charlie");
04: \hspace*{0.07in}storage.RemoveSetting("X");
05: \hspace*{0.07in}Assert.IsNull(storage.GetSetting("X"),
\hspace*{1.0in}"X not removed");
06: \hspace*{0.07in}Assert.AreEqual("Charlie", 
\hspace{1.0in}storage.GetSetting("NAME"));
07: \hspace*{0.07in}storage.RemoveSetting("NAME");
08: \hspace*{0.07in}Assert.IsNull(storage.GetSetting("NAME"), 
\hspace*{1.8in}"NAME not removed"); 
09: \hspace*{0.02in}\}
\end{alltt}
\end{CodeOut}  \vspace*{-3ex}
\caption{Existing CUT to test the \CodeIn{RemoveSetting} method.} \vspace*{-2ex}
\label{fig:excoveragetest}%
\end{figure}

Since generalized tests often help cover more paths in the code under test, we found that test generalization helped to achieve effective increase in the branch coverage. For example, for \CodeIn{Registry SettingsStorage}, Table~\ref{tab:coverage} shows increase in branch coverage by $38\%$. We next present an illustrative example to show how test generalization helps increase branch coverage of the code under test. Figure~\ref{fig:excoverage} shows the \CodeIn{RemoveSetting} method and Figure~\ref{fig:excoveragetest} shows the existing CUT for \CodeIn{RemoveSetting}. On executing the test and analyzing the code portions that are not covered in the \CodeIn{RegistrySettingsStorage} class, we observed that the code inside the \CodeIn{else} block (Statements $5$-$8$) was not covered by the unit test, i.e., the \CodeIn{false} branch was not covered. Figure~\ref{fig:excoveragePUT} shows the PUT that was generalized from the existing CUT. On executing Pex with this PUT, we observed that the code portions not covered by the CUT, i.e., Statements $5$-$8$ of the \CodeIn{RemoveSetting} method were covered by the new generated unit tests. This example shows that Pex was able to generate unit tests that covered both the \CodeIn{true} and \CodeIn{false} branches of the branching condition shown in Statement $3$. Using the generalized PUT, two types of test inputs for the ``\CodeIn{settingName}'' parameter were generated, those that contained `.' and those that did not contain `.'. Therefore, in contrast to the $20\%$ branch coverage achieved by executing the CUT, $80\%$ branch coverage was achieved by executing the unit tests generated by Pex using the PUT. 

\begin{figure}[t]
\begin{CodeOut}
\begin{alltt}
01: public void PUT([PAUT]String[] name, 
\hspace*{1.7in}[PAUT]Object[] value) \{
02: \hspace*{0.07in}............
03: \hspace*{0.07in}for (int i = 0; i < name.Length; i++) \{
04: \hspace*{0.22in}storage.SaveSetting(name[i], value[i]); \}
05: \hspace*{0.07in}for (int i = 0; i < name.Length; i++) \{
06: \hspace*{0.17in}if (storage.GetSetting(name[i]) != null) \{
07: \hspace*{0.3in}storage.RemoveSetting(name[i]);
08: \hspace*{0.3in}PexAssert.IsNull(storage.GetSetting(name[i]), 
\hspace*{1.5in}name[i] + " not removed"); 
08: \hspace*{0.1in}\}\}\}
\end{alltt}
\end{CodeOut} \vspace*{-4ex}
\caption{Transformed PUT of the CUT shown in Figure~\ref{fig:excoveragetest}.} \vspace*{-3ex}
\label{fig:excoveragePUT}%
\end{figure}

For NUnit, the branch coverage of three classes is increased by $1\%$, $38\%$, and $52\%$ respectively. For QuickGraph, there is an increase in branch coverage for three namespaces, one with an increase of $11\%$ and the other two namespaces have an increase of $2\%$. In DSA, for two namespaces, the branch coverage is increased  by $1\%$. One major reason for not achieving a significant increase in the coverage for DSA is that the existing CUTs already achieved a high branch coverage and PUTs help achieve a little higher coverage than the existing CUTs. In summary, for the three subjects, NUnit, DSA, and QuickGraph, generalizing the existing CUTs resulted in an average increase in coverage by $10\%$, $1\%$, and $2\%$, respectively. For all of the three subjects, there is no decrease in the branch coverage due to test generalization. The reason is that we used the existing CUTs as test scenarios to write the transformed PUTs. Therefore, when the transformed PUTs are provided with necessary assumptions on the test data, the generated tests achieve at least as much coverage as the existing CUTs. Furthermore, through our systematic procedure we provide additional assistance to Pex to meet the same test objective as the existing CUT.


%-----------------------------------------------------------------------------------------------------------------------------------------
%\begin{table}%
%\begin{center}
%\begin{tabular}{|l|r|r|r|r|r|}
%\multicolumn{6}{c}{\textbf{A. DSA}}\\
%\hline
%&Overall 	& Algorithms & DataStructures &	Properties	& Utility\\
%\hline
%Existing &	93.25\% &	99.00\%	& 96.00\% &	78.00\%	& 100.00\% \\			
%\hline
%PexGenerated &	93.50\%	& 100.00\% &	96.00\% &	78.00\%	& 100.00\% \\
%\hline
%\hline
%\end{tabular}
%\vspace{2ex}
%\begin{tabular}{|l|r|r|r|r|r|r|r|r|}
%
%\multicolumn{9}{c}{\textbf{B. QuickGraph}} \\
%\hline							
%&Overall	& Default &	Algorithms	& Collections	& Concepts &	Exceptions & Predicates &	Representations \\
%\hline
%Existing	& 87.29\% &	87.00\% &	90.00\% &	97.00\% &	72.00\% &	100.00\% &	83.00\% &	82.00\% \\
%\hline
%PexGenerated &	87.71\% &	87.00\%	& 93.00\% &	97.00\% &	70.00\% &	100.00\% &	83.00\%	& 84.00\% \\
%\hline
%\hline
%\end{tabular}
%
%\vspace{2ex}
%
%\begin{tabular}{|l|r|r|r|r|r|r|r|r|r|r|}
%\multicolumn{11}{c}{\textbf{C. NUnit}} \\
%\hline	
%&	Overall &	MSS &	Project &	Registry & PU &	RSS &	RTA	& SU &	SG	& TA \\
%\hline
%Existing &	78.22\%	& 100.00\% &	76.00\%	& 85.00\% &	79.00\% &	48.00\% &	100.00\% & 91.00\% & 39.00\%	& 86.00\% \\
%\hline
%PexGenerated	 & 88.33\% &	100.00\% &	76.00\%	& 85.00\% &	79.00\%	& 86.00\%	& 100.00\% &	91.00\%	& 92.00\%	& 86.00\% \\
%\hline
%\hline
%\end{tabular}
%
%\end{center}
%\caption{Coverage of the existing CUTs and the unit tests generated by Pex for the given PUTs.}
%\label{tab:coverage}
%\end{table}
\subsection{RQ2: Defects}

To address RQ2, we identify the number of defects detected by PUTs. Since we are using the CUTs that are available with the sources, we did not find any failing CUTs, i.e., no defects are reported by the existing unit tests for any of the subjects. Therefore, any failing tests among the unit tests generated from PUTs are considered as new defects that are not detected by the CUTs. However, before confirming whether a failing unit test implies a defect in the code under test, we manually verify that the generated test data is valid and the defect is not a false positive due to an ineffective PUT. 

Through test generalization of the three subjects in our study, we found $13$ new defects in the DSA application and $3$ new defects in the NUnit application. Since we were sure that the resulting failing tests were due to the defects in the code under test, we reported the failing test scenarios on the hosting website and waiting for the confirmation from the developers\footnote{Reported bugs can be found at the DSA CodePlex website with defect IDs from $8846$ to $8858$ and the NUnit SourceForge website with defect IDs from $0$ to $0$.}.

We next explain an example defect detected in the \CodeIn{Heap} class for the DSA application by our test generalization. The \CodeIn{Heap} class is a heap data structure implementation in the \CodeIn{DataStructure} namespace. This class includes methods to add, remove, and to heapify the elements in the heap. The \CodeIn{Remove} method of the class takes an item to be removed as a parameter and returns \CodeIn{True} when the item to be removed is in the heap, and returns \CodeIn{False} otherwise. Figure~\ref{fig:defectCUT} shows the existing CUT that tests whether the \CodeIn{Remove} method returns \CodeIn{False} when an item that is not in the heap is passed as the parameter. On executing, this CUT passes indicating no defect in the code under test since there are no other CUTs in the test suite that test the behavior of the method. However, from our generalized PUT shown in Figure~\ref{fig:defectPUT}, a few of the generated unit tests failed indicating the possibility of a defect in the \CodeIn{Remove} method. The test data for the failing unit tests had the following common properties: the heap size is less than $4$ (the \CodeIn{input} parameter of the PUT is of size less than $4$), the item to be removed is $0$ (the \CodeIn{item} parameter of the PUT), and the item $0$ was not already added to the heap (generated value for \CodeIn{input} did not contain the item $0$). When we manually analyzed the cause of the failing unit tests, we found that in the constructor of the \CodeIn{Heap} class, a default array of size $4$ (of type \CodeIn{int}) is created to store the items. In C\# an integer array is by default assigned values zero to the elements of the array. Therefore, there is always an item $0$ in the heap unless an input list of size greater than or equal to $4$ is passed as parameter. Therefore, on calling the \CodeIn{Remove} method to remove the item $0$, even when there is no such item in the heap, the method returns \CodeIn{True} indicating that the item has been successfully removed and causing the assertion statement to fail (statement $6$ of the PUT). However, this defect was not detected by the CUT shown in Figure~\ref{fig:defectCUT} as the unit test assigns the heap with $5$ elements (statement $2$) and therefore the defect-exposing scenario of heap size $< 4$ cannot be encountered. 

\begin{figure}[t]
\begin{CodeOut}
\begin{alltt}
//To test Remove item not present
01: public void RemoveCUT()\{
02: \hspace*{0.07in}Heap<int> actual = new Heap<int> \{2, 78, 1, 0, 56\};
03: \hspace*{0.07in}Assert.IsFalse(actual.Remove(99));
04: \hspace*{0.02in}\}
\end{alltt}
\end{CodeOut} \vspace*{-3ex}
\caption{Existing CUT to test the \CodeIn{Remove} method of \CodeIn{Heap}.}
\label{fig:defectCUT}

\begin{CodeOut}
\begin{alltt}
01: public void RemoveItemPUT(
\hspace*{0.7in} [PAUT]List<int> input, int item) \{
02: \hspace*{0.07in}Heap<int> actual = new Heap<int> (input);
03: \hspace*{0.07in}if (input.Contains(item)) \{
04: \hspace*{0.2in}..... \}
05: \hspace*{0.07in}else \{
06: \hspace*{0.2in}PexAssert.IsFalse(actual.Remove(randomPick));
07: \hspace*{0.2in}PexAssert.AreEqual(input.Count, actual.Count);
08: \hspace*{0.2in}CollectionAssert.AreEquivalent(actual, input);\}
09: \hspace*{0.02in}\}
\end{alltt}
\end{CodeOut} \vspace*{-3ex}
\caption{Transformed PUT of the CUT shown in Figure~\ref{fig:defectCUT}.} \vspace*{-1ex}
\label{fig:defectPUT} 
\end{figure}

This example defect that was detected in our study displays the strength of generalized unit tests such as PUTs. The $16$ defects reported in our study that were not detected by the existing unit tests show that PUTs can assist in an unbiased test generation and are an effective means for rigorous testing of the code under test.  
%------------------------------------------------------------------------------------------------------------------------------------------
%We conduct second evaluation based on mutation testing to further show the effectiveness of PUTs in detecting defects compared to CUTs. The reason for the second evaluation is that the existing CUTs do not detect any defects in the code under test. In this evaluation, we seed defects in the code under test using a mutation testing tool and verify
%how many mutants are killed by existing CUTs and PUTs. We consider that a mutant is killed if any previously passing test fails after executed with the seeded fault. 
%
%We use the following five basic mutation operators, recommended by Offutt, for seeding defects in the code under test.
%
%\begin{itemize}
%\item ABS: Forces each arithmetic expression to take values zero, positive and negative values.
%\item AOR: Replaces each arithmetic operator with every syntactically legal operator.
%\item LCR: Replaces logical connector (AND and OR) with other kinds of logical operators.
%\item ROR: Replaces each relational operator with other relational operators.
%\item UOI: Inserts unary operators in front of expressions.
%\end{itemize}
\subsection{RQ3: Test Code Maintenance}

We next address RQ3 of whether test generalization can reduce the efforts in maintaining test code. We use two metrics to address this research question. First, we compare the number of CUTs and the number of PUTs. The higher the difference between the number of CUTs and the number of PUTs, the lesser are the efforts required in maintaining the test code.
The reason is that whenever the code under test is modified, all failing tests need to be modified based on the new expected behavior of the code under test. Therefore, lower number of PUTs can significantly reduce the efforts in maintaining the test code since only a few PUTs need to be modified. Second, we compare the Lines of Code (LOC) of CUTs and PUTs. The reason for the second metric is that lower number of PUTs with a high amount of LOC does not help in reducing the efforts required in maintaining the test code.

\begin{figure}[t]
\centering
\includegraphics[scale=0.45,clip]{charts/CUTs_PUTs_1.eps}\vspace*{-1ex}
\caption{\label{fig:cutsnputs}Comparison of the number of CUTs and PUTs}
\end{figure}

\begin{table}[t]
\begin{CodeOut}
\begin{center}
\centering \caption {\label{tab:cutputmapping} Mapping of the number of CUTs and their transformed PUTs.}
\begin {tabular} {|c|c|c|}
\hline \textbf{CUT} & \textbf{PUT} & \textbf{\# of occurrences}\\
\hline
\hline 1   & 1   & 129\\
\hline 1   & 2   & 2\\
\hline 1   & 3   & 2\\
\hline 2   & 1   & 29\\
\hline 3   & 1   & 14\\
\hline 4   & 1   & 15\\
\hline 5   & 1   & 4\\
\hline 6   & 1   & 4\\
\hline 7   & 1   & 1\\
\hline 8   & 1   & 2\\
\hline 9   & 1   & 1\\
\hline 15   & 1   & 1\\
\hline
\end{tabular}\vspace*{-3ex}
\end{center}
\end{CodeOut}
\end{table}

Figure~\ref{fig:cutsnputs} shows the comparison of the number of CUTs with PUTs for all subject applications. The x-axis shows the subject application and y-axis shows the number of CUTs or PUTs. In total, we generalized $407$ CUTs to $224$ PUTs that achieved a higher code coverage than CUTs and also detected new defects that are not detected by the CUTs. The figure shows that there is a significant reduction in the number of tests for the subjects DSA and NUnit. This reduction in the number of tests is because multiple CUTs are generalized to a single PUT. Table~\ref{tab:cutputmapping} shows the mappings between the number of CUTs and their transformed PUTs. For example, Row 1 shows that one CUT is transformed to one PUT in 129 occurrences. Similarly, the last row shows that 15 CUTs are transformed into a single PUT in one occurrence. Rows 2 and 3 show exceptional cases where a CUT is generalized to more than one PUT. Section~\ref{sec:limitations} discusses more about these exceptional cases. 

Figure~\ref{fig:transformedPUT} shows an example PUT, which is a result of generalizing four CUTs of the \CodeIn{SinglyLinkedList} class of DSA. The objective of all four CUTs is to test the \CodeIn{AddFirst} method (method under test) that adds an element to a list object of type \CodeIn{SinglyLinkedList}. These four CUTs verify different behaviors by adding one element or two elements to a list and by verifying whether the \CodeIn{head} and \CodeIn{tail} values of the list are updated correctly. We generalized all these four CUTs into a single PUT shown in Figure~\ref{fig:transformedPUT}. The CUTs verify the behavior of the \CodeIn{AddTest} method by adding only a fixed number of elements (with fixed values) to the list. In contrast, our PUT can verify the behavior of the \CodeIn{AddFirst} method with variable number of elements in the list. 

\begin{figure}[t]
\begin{CodeOut}
\begin{alltt}
public void AddFirstTest([PAUT]SinglyLinkedList<int> sll, 
\hspace*{0.6in}[PAUT]int[] ne) \{            
\hspace*{0.2in}PexAssume.IsTrue(ne.Length > 1);
\hspace*{0.2in}PexAssume.IsTrue(sll.Count == 0);
\hspace*{0.2in}for (int i = 0; i < ne.Length; i++)
\hspace*{0.4in}sll.AddFirst(ne[i]);
\hspace*{0.2in}PexAssert.AreEqual(ne[ne.Length - 1], sll.Head.Value);            
\hspace*{0.2in}PexAssert.AreEqual(ne[0], sll.Tail.Value);
\hspace*{0.2in}PexAssert.AreEqual(ne.Length, sll.Count);
\}
\end{alltt}
\end{CodeOut}
\Caption{\label{fig:transformedPUT} The \CodeIn{AddFirstTest} PUT for the \CodeIn{AddFirst} method under test.}
\end{figure}

Figure~\ref{fig:loccomp} shows the results of comparing LOC of CUTs and PUTs. For DSA, LOC of PUTs is less than the LOC of CUTs, whereas for the other two subjects, LOC of PUTs is slightly more than the LOC of CUTs. We identify that among new LOC written for PUTs, many statements are related to the additional \CodeIn{using} statements or new annotations added during our test generalization. On an average, LOC of PUTs is almost the same as the LOC of CUTs. Therefore, our results show that test generalization can help reduce the number of tests and thereby reduce the efforts in maintaining test code.

\begin{figure}[t]
\centering
\includegraphics[scale=0.45,clip]{charts/LOC_1.eps}\vspace*{-1ex}
\caption{\label{fig:loccomp}Comparison of Lines of Code of CUTs and PUTs}
\end{figure}
