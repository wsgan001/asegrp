\section{Introduction}

\label{sec:introduction}

Access control is one of the most fundamental and widely used
security mechanisms. It controls which principals (users, processes,
etc.) have access to which resources in a system. To better manage
access control, systems often explicitly specify access control
policies using policy languages such as XACML~\cite{oasis05:xacml}
and Ponder \cite{damianou01:ponder}. Whenever a principal requests
access to a resource, that request is passed to a software component
called a Policy Decision Point (PDP). PDP evaluates the request
against access control policies, and grants or denies the request
accordingly.

The specification of access control policies is often a challenging
problem. It is common that a system's security is compromised due to
the misconfiguration of access control policies instead of the
failure of cryptographic primitives or protocols. This problem
becomes increasingly severe as software systems become more and more
complex, and are deployed to manage a large amount of sensitive
information and resources that are organized into sophisticated
structures.

Formal verification is an important means to ensuring the correct
specification of access control
policies\Comment{~\cite{jajodia97:logical, sandhu99:arbac}}.
Recently, several tools have been developed to verify XACML access
control policies against user-specified
properties~\cite{hughes04:automated,fisler05:verification,zhang05:evaluating}.
However, it is often beyond the capabilities of these tools to
verify complex access control policies in large-scale information
systems. Furthermore, user-specified properties are often not
available~\cite{fisler05:verification}.

Like in software development, errors in access control policies may
also be discovered through testing. In fact, once access control
policies are specified, they are often tested with some access
requests so that security officers may manually check the PDP's
responses against expected ones~\cite{anderson02:xacml}. However,
current policy testing practice tends to be ad hoc. Although there
exist various coverage criteria~\cite{zhu97:software} for software
programs, there are no criteria or good heuristics to guide
systematic generation of high-quality policy test suites. With an ad
hoc policy testing, it is questionable that high confidence could be
gained on the correctness of access control policies.

This paper presents a first step toward systematic policy testing.
We propose the concept of \Intro{policy coverage} to measure the
quality of policy test suites, which are sets of request-response
pairs. Intuitively, the more policy rules (as well as their
components such as subjects, resources, and conditions) are involved
when evaluating a test suite, the more likely it is to discover
errors in access control policies. We have developed a
coverage-measurement tool to measure the coverage of XACML policies
achieved by a set of access requests. We have also developed a
request-generation tool that randomly generates policy test suites
for a given set of policies.

Although the randomly generated test suites can achieve high policy
coverage, and are effective in detecting a variety of policy
specification errors, it may potentially include a huge number of
requests, which makes it difficult to efficiently inspect and verify
the correctness of responses from the PDP. To mitigate this problem,
we further propose a request reduction technique to significantly
reduce the size of a test suite while maintaining its policy
coverage.

Previous experiments~\cite{rothermel98:empirical} showed that test
reduction based on program code coverage can severely compromise the
fault-detection capabilities of the original test suite. To evaluate
the impact of the proposed request reduction technique on the
quality of policy testing, we conduct an experiment on a set of real
policies with mutation testing~\cite{demillo78:hints}, which is a
specific form of fault injection that consists of creating faulty
versions of a policy by making small syntactic changes. In the
experiment, we compare the fault-detection capabilities of the
reduced set and original set of requests. Our experimental results
show that our coverage-based request reduction technique can
substantially reduce the size of generated requests but incur only
relatively low loss in fault detection capabilities. We also conduct
a study that measures the policy coverage of an XACML conformance
test suite as well as a conference reviewing system's policy. Our
results show that the measurement of policy coverage can effectively
identify uncovered parts of policies. Such results can be used to
guide the development of further test cases, significantly improving
the quality of policy testing.

\Comment{ This paper makes the following main contributions:
\begin{itemize}
\item We propose the concept of policy coverage based on
a general access control model. We further instantiate this
concept in the context of XACML, a widely used and standardized
meta policy language
for expressing domain-specific access control requirements.

\item We develop a coverage-measurement tool for automatically
measuring the coverage for XACML policies achieved by a given set
of requests.

\item We develop a request-generation tool for randomly generating
requests for XACML policies and a request-reduction tool for
greedily selecting a minimal set of requests for achieving the
same policy coverage as the original set of requests.

\item We develop initial mutation operators for XACML policies to
conduct mutation testing. We conduct an experiment on a set of
real policies to compare the effectiveness of the minimal set of
requests with the original set of requests in terms of fault
detection. We also conduct a study on policy coverage of existing
manually generated requests.

\end{itemize}
}

The rest of the paper is organized as follows. Section
\ref{sec:xacml} presents background information on XACML, a widely
used and standardized meta policy language for expressing
domain-specific access control requirements.
Section~\ref{sec:model} proposes the concept of policy testing and
policy coverage based on a general access control model. In
Section~\ref{sec:coverage}, we instantiate the concept of policy
coverage in the context of XACML. We also present the design of a
coverage measurement tool. Sections~\ref{sec:reqgen}
and~\ref{sec:reqreduce} describe the request-generation tool and
our request reduction technique, respectively.
Section~\ref{sec:mutation} presents a set of initial mutation
operators developed for policies. Section~\ref{sec:experiment}
presents the experiment conducted to assess request reduction and
its effect on fault detection capabilities.
Section~\ref{sec:experimentalresults} illustrates the study of
measuring the policy coverage achieved by manually generated
requests. Section~\ref{sec:related} discusses related work and
Section~\ref{sec:conclusion} concludes the paper with future
directions.

















\sloppy
