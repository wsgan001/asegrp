\section{XACML}
\label{sec:xacml}

%%%%%%%%%%%%%%%%%%%%%%% added by Ting: start

\Comment{In this section, we focus our discussion of policy testing
on XACML (eXtensible Access Control Markup Language).} XACML
(eXtensible Access Control Markup Language) is a language
specification standard designed by OASIS. It can be used to express
domain-specific access control policy languages as well as access
request languages. Besides offering a large set of built-in
functions, data types, and combining logic, XACML also provides
standard extension interfaces for defining application-specific
features. Since it was proposed, XACML has received much attention
from both the academia and the industry. Many domain-specific access
control languages have been developed using XACML
\cite{moses03:xacml,lorch03:xacml}. Open source XACML
implementations are also available for different platforms (e.g.,
Sun's XACML implementation~\cite{sun05:xacml} and
XACML.NET~\cite{net05:xacml}). Therefore, XACML provides an ideal
platform for the development of policy testing techniques that can
be easily applied to multiple domains and applications.\Comment{ In
this section, we first give a brief introduction of XACML, and then
discuss the policy coverage criteria for XACML.}


%%%%%%%%%%%%%%%%%%%%%% added by ting: end

The basic concepts of access control in XACML include
\Intro{policies}, \Intro{rules}, \Intro{targets}, and
\Intro{conditions}. A single access control policy is represented by
a policy element, which includes a target element and one or more
rule elements. A target element contains a set of constraints on the
subject (e.g., the subject's role is equal to faculty), resources
(e.g., the resource name is grade), and actions (e.g., the action
name is assign)\footnote{Conditions of ``AnySubject'',
``AnyResource'', and ``AnyAction'' can be satisfied by any subject,
resource, or action, respectively.}. A target specifies to what
kinds of requests a policy can be applied. If a request cannot
satisfy the constraints in the target, then the whole policy element
can be skipped without further examining its rules.

%. To help determine which policies that apply
%to a given request, policy writers can define a \Intro{target} for
%a policy. A target defines a set of conditions for the
%\Intro{subject} (e.g., faculty), \Intro{resource} (e.g., grades),
%and \Intro{action} (e.g., assign) that should be
%satisfied\footnote{Conditions of ``AnySubject'', ``AnyResource'',
%and ``AnyAction'' can be satisfied by any subject, resource, or
%action, respectively.} before applying a policy. If all the
%conditions of a policy's target are satisfied by a request, the
%policy is applied to the request.

We next describe how a policy is applied to a request in details. A
policy element contains a sequence of rule elements. Each rule also
has its own target, which is used to determine whether the rule is
applicable to a request. If a rule is applicable, a
\Intro{condition} (a boolean function) associated with the rule is
evaluated. If the condition is evaluated to be true, the rule's
\Intro{effect} (\CodeIn{Permit} or \CodeIn{Deny}) is returned as a
\Intro{decision}; otherwise, \CodeIn{NotApplicable} is returned as a
decision. If an error occurs when a request is applied against
policies or their rules, \CodeIn{Indeterminate} is returned as a
decision.

More than one rule in a policy may be applicable to a given request.
To resolve conflicting decisions from different rules, a \Intro{rule
combining algorithm} can be specified to combine multiple rule
decisions into a single decision. For example, a deny overrides
algorithm determines to return \CodeIn{Deny} if any rule evaluation
returns \CodeIn{Deny} or no rule is applicable. A first applicable
algorithm determines to return what the evaluation of the first
applicable rule returns. In general, an XACML policy specification
may also include multiple policies, which are included with a
container element called \Intro{PolicySet}. When a request can also
be applied to multiple policies, a \Intro{policy combining
algorithm} can also be specified in a similar way.

%\mbskip\mbskip\sbskip
%\begin{figure}
%\begin{alltt} \small
\begin{figure}[t]
\begin{CodeOut}
\begin{alltt}
 1<Policy PolicyId="demo" RuleCombinationAlgId="first-applicable">
 2 <Target>
 3   <Subjects> <AnySubjects/> </Subjects>
 4   <Resources>
 5    <Resource>
 6     <ResourceMatch MatchId="equal">
 7      <AttributeValue>demo:5</AttributeValue>
 8      <ResourceAttributeDesignator AttributeId="objectid"/>
 9     </ResourceMatch>
10    </Resource>
11   </Resources>
12   <Actions> <AnyAction/></Actions>
13 </Target>
14 <Rule RuleId="1" Effect="Deny">
15  <Target> <Subjects><AnySubject/></Subjects>
16   <Resources> <AnyResource/> </Resources>
17   <Actions>
18    <Action>
19     <ActionMatch MatchId="equal">
20      <AttributeValue>Dissemination</AttributeValue>
21      <ActionAttributeDesignator AttributeId="actionid"/>
22     </ActionMatch>
23    </Action>
24   </Actions>
25  </Target>
26  <Condition FunctionId="not">
27   <Apply FunctionId="at-least-one-member-of">
28    <SubjectAttributeDesignator AttributeId="loginid"/>
29    <Apply FunctionId="string-bag">
30     <AttributeValue>testuser1</AttributeValue>
31     <AttributeValue>testuser2</AttributeValue>
32     <AttributeValue>fedoraAdmin</AttributeValue>
33    </Apply>
34   </Apply>
35  </Condition>
36 </Rule>
37 <Rule RuleId="2" Effect="Permit"/>
38</policy>
\end{alltt}
\end{CodeOut}
\caption{An example XACML policy}
 \label{fig:example}
\end{figure}

Figure \ref{fig:example} shows an example XACML policy, which is
revised and simplified from a sample
Fedora\footnote{\CodeIn{http://www.fedora.info}} policy (to be used
in our experiment described in Section~\ref{sec:experiment}). This
policy has one policy element which in turn contains two rules. The
rule composition function is ``first-applicable'', whose meaning has
been explained earlier. Lines 2-13 define the target of the policy,
which indicates that this policy applies only to those access
requests of an object ``demo:5''. The target of Rule 1 (Lines 15-25)
further narrows the scope of applicable requests to those asking to
perform a ``Dissemination'' action on object ``demo:5''. Its
condition (Lines 26-35) indicates that if the subject's ``loginId''
is ``testuser1'', ``testuser2'', or ``fedoraAdmin'', then the
request should be denied. Otherwise, according to Rule 2 (Line 37)
and the rule composition function of the policy (Line 1), a request
applicable to the policy should be permitted.
