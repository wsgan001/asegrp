\section{Empirical Study of Manually Generated Requests' Policy Coverage}
\label{sec:experimentalresults}

We have applied the coverage-measurement tool on the whole set of
the XACML committee specification conformance test
suite~\cite{anderson02:xacml} and a conference paper review
system's policy and its requests developed by Zhang et
al.~\cite{zhang04:synthesis}.

\begin{table}[t]
\caption {\label{table:results} Policy coverage of the XACML
conformance test suite}
\begin{small}
\begin{center}
\centering
\begin {tabular} {|l|r|r|r|r|r|} \hline
type&\CenterCell{100\%}&\CenterCell{50\%}&\CenterCell{non-0\%}&\CenterCell{0\%}&\CenterCell{total}\\
&\CenterCell{all}&\CenterCell{cond}&\CenterCell{rule/cond}&\CenterCell{rule/cond}&\\
\hline policies&24&172&24&14&234\\
\hline
\hline Permit&31&144&6&&181\\
\hline Deny&&&6&&6\\
\hline NotApp&13&28&6&10&57\\
\hline Indet&1&2&6&4&13\\
\hline
\end{tabular}
\end{center}
\end{small}
\vspace*{-4.0ex}
\end{table}

The XACML conformance test suite includes 337 distinct
policies\footnote{In the XACML conformance test suite, there are
374 policies, each of which receives a single request. We have
reduced those policies with the same policy content into a single
policy, which can then receive multiple requests.}, 374 requests,
their expected responses from the application of the policies.
Among these 337 distinct policies, we show the results of 234
policies in this section because for the requests of the remaining
103 policies, Sun's XACML implementation~\cite{sun05:xacml}
responded different decisions than the ones specified in their
expected responses. Applying the requests on these 103 policies
failed to conform with expected responses because Sun's XACML
implementation does not support some optional features of XACML
specifications.

The conference paper review system's policy specified by Zhang et
al.~\cite{zhang04:synthesis} has 11 requests and 15 rules, which
have 10 conditions. These 10 conditions involve the execution of SQL
statements that access an external database. Because it is not
trivial to adapt Sun's XACML implementation to support these
conditions, we simply remove these 10 conditions as well as some
attributes that are not parsed by Sun's XACML implementation, in
order to allow us to focus on the measurement of rule hit
percentage.

We fed 374 requests in the XACML conformance test suite to the
coverage-measurement tool. Table~\ref{table:results} shows the
reported statistics of policy coverage. Note that all policies in
the conformance test suite are hit by the requests, achieving 100\%
policy hit percentage. Column 1 shows the type of data and Columns
2-5 show the data for different types of coverage. Row 2 shows the
number of policies. Rows 3-6 show the number of requests whose
returned decisions are \CodeIn{Permit}, \CodeIn{Deny},
\CodeIn{NotApplicable}, and \CodeIn{Indeterminate}, respectively.
When a data entry has a zero value, we do not show the zero value
but leave the entry empty.

Column 2 shows the data for policies whose policy, rule, and
condition hit percentages reach 100\%. These policies have achieved
the optimal policy coverage. Column 3 shows the data for policies
whose policy and rule hit percentages reach 100\% but condition hit
percentage reaches 50\%.\Comment{ These policies achieve
almost-optimal policy coverage because sometimes it is not very
essential to cause a condition of a rule to be evaluated to be
false.} Column 4 shows the data for policies whose rule or condition
hit percentage is less than 100\% but not equal to 0\% (but we do
not include the cases shown in Column 3 here). The coverage of these
policies needs to be improved. Column 5 shows the data for policies
whose rule or condition hit percentage is equal to 0\%. These
policies are especially in need for improvement. The last column
shows the sum of all the data in Columns 2-5.

From the results shown in Table~\ref{table:results}, we observed
that a majority of policies fell into the category of Column 3,
where policy and rule hit percentages reach 100\% but condition
hit percentage reaches 50\%. Many polices in the XACML conformance
test suite contain single rules each of which has a condition.
Often each of these policies receives only one request, which
basically cover the policy's rule and the rule's true condition.

We took a close look at the details of 14 policies in Column 5. Two
of them had 100\% for rule hit coverage but 0\% for condition hit
percentage. Their coverage results were against our expectation
because if their conditions were applicable, we expected that either
a true or false condition would be hit. We inspected their requests
and found that a subject's age was specified twice and their
conditions access the subject's age. When evaluating the conditions,
PDP encountered an error and returned a decision of
\CodeIn{Indeterminate}; therefore, neither true or false condition
was hit.

Note that the XACML conformance test suite was not specifically
constructed to achieve high coverage of policies but the
measurement results still give us some insights of the common
coverage distribution, reflecting policy portions that are
commonly hit by manually created requests.

After we fed to the coverage-measurement tool 11 requests for the
conference paper review system's policy~\cite{zhang04:synthesis},
73\% rule hit percentage was achieved: 4 out of 15 rules were not
hit. These four uncovered rules included the case of permitting a PC
chair to read papers and no request matched this case. Interestingly
one of these uncovered four rules was the last rule, which has the
effect of \CodeIn{Deny} and this rule's target can be matched by any
request. This rule is often used for the \CodeIn{permit-overrides}
rule combination algorithm~\cite{oasis05:xacml}. Given the
measurement results of the coverage-measurement tool, we could
construct new requests without much difficulty to cover these
uncovered rules in the policy of the conference paper review system
as well as those uncovered rules or conditions in many policies of
the XACML conformance test suite.
