\section{Request Generation} \label{sec:reqgen}
\Comment{To generate requests automatically for achieving policy
coverage, we have developed a request-generation tool for inspecting
the policy under test and constructing a request factory that
provides requests on demand. There are various algorithms that can
be devised to generate requests.  At present we have implemented
three techniques for request generation called the
\CodeIn{AllComboReqFactory} and the \CodeIn{RandomReqFactory}. The
former attempts to generate requests for all possible combinations
of attribute id-value pairs found in the policy while the latter
randomly selects requests from the set of all combinations.  This is
achieved by representing a particular request as a vector of bits.
The length of this vector is equal to the number of different
attribute values found in the policy targets, rule targets, and rule
conditions of the policy under test. Each attribute value appears in
the request if its corresponding bit in the vector is $1$;
otherwise, the value is not present.

To generate all possible combinations we increment an integer $i$
from 0 to $2^n$ where $n$ is the number of attribute values found in
the policy.  To construct a request from the integer $i$ we first
convert $i$ to binary and use the $n$ least significant bits as the
vector of bits that indicate the presence or absence of the possible
attribute values.  This approach guarantees that all possible
combinations of the available attribute values are generated.
However this is a simplistic approach and not realistic for larger
policies since the number of possible requests increases
exponentially with the number of possible attribute values.  In
addition to this shortcoming, there are instances in which the set
of attribute values is not finite, such as cases with integer data
types and greater than or less than condition functions.  Note that
this algorithm considers only the attribute values found in the
policy rather than all possible attribute values that can occur in
requests.\Comment{ Such instances make the use of the
\CodeIn{AllComboReqFactory} impractical or even impossible.}

The generation of random requests is done in a similar fashion.
First, the policy is inspected and the $n$ possible attribute values
are determined (including attribute values found in the policy and
optionally including other sample attribute values supplied by the
user). Each request is generated by setting each bit in the vector
to $0$ or $1$ with probability $0.5$. The number of randomly
generated requests can be configured by the user and the configured
number can be considerably smaller than the total number of
combinations. To help achieve adequate coverage with a small set of
random requests, we further modified this algorithm to ensure that
each bit was set to $1$ and $0$ at least once. We accomplish this by
explicitly setting the $i^{th}$ bit to $1$ for the first $n$
generated requests where $i=1,2,...n$. Similarly, for the next $n$
requests, we explicitly set the $(i-n)^{th}$ bit to $0$. This
approach guarantees that each attribute value is present and absent
at least once as long as the number of randomly generated requests
is greater than $2n$.}

%%%%%%%%%%%%%%%%%%%%%%%%%
Because manually generating requests for testing policies is
tedious, we have developed a technique for randomly generating
requests given only the policy under test. The random request
generator analyzes the policy under test and generates requests on
demand by randomly selecting requests from the set of all
combinations of attribute id-value pairs found in the policy. A
particular request is represented as a vector of bits. The length of
this vector is equal to the number of different attribute values
found in the policy set targets, policy targets, rule targets, and
rule conditions of the policy under test. Each attribute value
appears in the request if its corresponding bit in the vector is
$1$; otherwise, the value is not present.

More specifically, all possible combinations can be represented by
integers from $0$ to $2^n$ where $n$ is the number of attribute
values found in the policy. Each request is generated by setting
each bit in the vector to $0$ or $1$ with probability $0.5$. The
number of randomly generated requests can be configured by the user
and the configured number can be considerably smaller than the total
number of combinations. To help achieve adequate coverage with a
small set of random requests, we modified the random test generation
algorithm to ensure that each bit was set to $1$ and $0$ at least
once. In particular, we explicitly set the $i^{th}$ bit to $1$ for
the first $n$ generated requests where $i=1,2,...n$. Similarly, for
the next $n$ requests, we explicitly set the $(i-n)^{th}$ bit to $0$
where $i=n+1,n+2,...2n$. This improved algorithm guarantees that
each attribute value is present and absent at least once as long as
the number of randomly generated requests is greater than $2n$.

\Comment{
\subsection{Test Generation via Change-Impact Analysis}
\label{sec:cirgreqgen}

To automatically generate high-quality test suites for access
control policies, we have developed a framework based on
change-impact analysis~\cite{martin06:cirg}. The framework receives
a set of policies under test and outputs a set of tests in the form
of request-response pairs for developers to inspect for correctness.
The framework consists of three major components: version synthesis,
change-impact analysis, and request generation. The key notion of
the framework is to synthesize two versions of the given policies in
such a way that test coverage targets (e.g., certain policies,
rules, and conditions) are encoded as the differences of the two
synthesized versions. Then a change-impact analysis tool can be
leveraged to generate counterexamples to witness these differences,
thus covering the test coverage targets. The framework generates
tests (in the form of requests) based on the generated
counterexamples. We implemented this framework in a tool called Cirg
that leverages a change-impact analysis tool called
Margrave~\cite{xxx} to automatically generate test suites with high
structural coverage~\cite{martin06:cirg}.}
