\section{Request Reduction}
\label{sec:reqreduce}

The request reduction problem can be stated similar to the test
minimization problem for program
testing~\cite{harrold93:methodology}:

\vspace{5mm}\noindent
 {\it Given}: request set QS, a set of requirements $r_1$, $r_2$, ..., $r_n$
 that must be satisfied to provide the desired test coverage of the
 policies, and subsets of QS, $Q_1$, $Q_2$,..., $Q_n$, one associated
 with each of the $r_i$s such that any one of the request $q_j$ belonging
 to $Q_i$ can be used to test $r_i$.

\noindent{\it Problem}: Find a representative set of requests from
QS that satisfies all
 of $r_i$s.

\vspace{5mm}

In the problem statement, the $r_i$s can represent policy coverage
requirements, such as covering a certain policy, a certain rule,
and a certain condition. In a representative set of requests that
satisfies all of the $r_i$s, at least one request satisfies each
$r_i$. We call a representative set is \Intro{minimal} if removing
any request from the set causes the set not to be a representative
set. Given a request set $QS$, there can be several minimal
representative sets $QS' \subseteq QS$. Among the minimal
representative request sets, we could find a request set that has
the smallest possible number of requests. Finding such request
tests reduces to optimization problems called ``minimum set
cover'' and ``minimum exact cover'', respectively; these problems
are known to be NP complete, and in practice approximation
algorithms are used~\cite{johnson74:approximation}.

In our implementation of coverage-based request reduction, we use a
greedy algorithm for selecting requests as they are generated by the
random request factory if and only if the generated request
increases any of the coverage metrics described in
Section~\ref{sec:coverage}. More specifically, we iteratively
generate a random request and add it to the large set. We then
evaluate that request against the policy in order to both compute
the response and measure the coverage. If the coverage increases due
to the evaluation of the request, then that request is added to the
reduced request set.

We note that this greedy algorithm may not produce a minimal
representative set. In practice, it does, however, often produce a
representative set whose size is near the size of a minimal
representative set. We call our reduced set as a \Intro{nearly
minimal} representative set.

\Comment{ and this may be the subject of future work. It does,
however, guarantee that the size of the reduced request set will
be no larger than its larger counterpart and both request sets
will have identical coverage.}
