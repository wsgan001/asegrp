\section{Measuring Fault-Detection Capability}
\label{sec:mutation}

In order to investigate the effect of request reduction on
fault-detection capabilities, we can inject faults into the original
policy thereby creating faulty policies. Since fault detection is
the central focus of any testing process, it provides an external
measure of the effectiveness of that process. We aim to demonstrate
that reduced request sets based on coverage can detect a large
percentage of the faults detected by the original request set. We
use mutation testing~\cite{demillo78:hints} as a mechanism to
compare request sets in terms of fault detection.\Comment{ We first
introduce the general concept of mutation testing and describe
mutation testing for access control policies. We then present how to
detect equivalent mutants among generated mutants. Finally, we
present a set of mutation operators used by our automated policy
mutator.}

\Comment{ In mutation testing, the policy under test is mutated to
introduce an error and a request set is evaluated against the mutant
policy. If the request set produces any response that differs from
the corresponding response produced by the original policy, then the
mutant is killed.

There are many studies concerned with the types and effectiveness of
mutating general purpose programming
languages~\cite{ma02:interclass, hennessy05:analysis}, however many
of these do not directly apply to mutating policies.  We describe
the types of mutation operators employed in our experiments in
Table~\ref{table:mutation_types}.}

%\subsection{Mutation Testing}

Mutation testing~\cite{demillo78:hints} has historically been
applied to general-purpose programming languages. The program under
test is iteratively mutated to produce numerous mutants, each
containing one fault. A test input is independently executed on the
original program and each mutant program. If the output of a test
executed on a mutant differs from the output of the same test
executed on the original program, then the fault is detected and the
mutant is said to be killed. The fundamental premise of mutation
testing as stated by Geist et al.~\cite{geist92:estimation} is that,
in practice, if the software contains a fault, there will usually be
a set of mutants that can only be killed by a test that also detects
that fault. In other words, the ability to detect small, minor
faults such as mutants implies the ability to detect complex faults.
Because fault detection is the central focus of any testing process,
mutation testing provides an external measure of the effectiveness
of that process. The higher the percentage of killed mutants, the
more effective the test set is at fault detection.

\begin{table}[t]
    \begin{small}
        \begin{center}
        \caption{\label{table:mutationops}Index of mutation operators.}
            \begin{tabular}{|l|l|}
                \hline ID  & \CenterCell{Description}\\
                \hline
                \hline PSTT & Policy Set Target True. The policy set is applied to all requests.\\
                \hline PSTF & Policy Set Target False. The policy set is not applied to any requests.\\
                \hline PTT & Policy Target True. The policy is applied to all requests.\\
                \hline PTF & Policy Target False. The policy is not applied to any requests.\\
                \hline RTT & Rule Target True. The rule is applied to all requests.\\
                \hline RTF & Rule Target False. The rule is not applied to any requests.\\
                \hline RCT & Rule Condition True. The condition always evaluates to true.\\
                \hline RCF & Rule Condition False. The condition always evaluated to false.\\
                \hline CPC & Change Policy Combining Algorithm. Each policy combining algorithm is tried in turn.\\
                \hline CRC & Change Rule Combining Algorithm. Each rule combining algorithm is tried in turn.\\
                %\hline CPO & Change Policy Order\\
                %\hline CRO & Change Rule Order\\
                \hline CRE & Change Rule Effect. The rule effect is inverted (e.g. permit for deny).\\
                \hline
            \end{tabular}
        \end{center}
    \end{small}
\end{table}

In policy mutation testing, the program under test, test inputs, and
test outputs correspond to the policy, requests, and responses,
respectively. We first define a set of mutation operators shown in
Table~\ref{table:mutationops}. Given a policy and a set of mutation
operators, a mutator generates a number of mutant policies. Given a
request set, we evaluate each request in the request set on both the
original policy and a mutant policy. The request evaluation produces
two responses for the request based on the original policy and the
mutant policy, respectively. If these two responses are different,
then we determine that the mutant policy is killed by the request;
otherwise, the mutant policy is not killed.

Unfortunately, there are various expenses and barriers associated
with mutation testing. The first and foremost is the generation and
execution of a large number of mutants.  For general-purpose
programming languages, the number of mutants is proportional to the
product of the number of data references and the number of data
objects in the program~\cite{offutt00:mutation}.  For XACML
policies, the number of mutants is proportional to the number of
policy elements, namely policy sets, policies, targets, rules,
conditions, and their associated attributes.

\Comment{
\subsection{Equivalent-Mutant Detection}

Cost of mutation testing also includes detection of equivalent
mutants~\cite{offutt00:mutation}. Although there are syntactic
differences between an equivalent mutant and the program under test,
the mutant is semantically identical to the original one. In other
words, the mutant will produce the same result as the original one
for all test inputs and thus provides no benefit. Equivalent-mutant
detection provides a mechanism to better evaluate mutation operators
and more efficiently perform mutation testing because computational
resources will not be wasted in evaluating test inputs or comparing
test outputs for equivalent mutants. Detecting such mutants in
software is generally intractable~\cite{deMillo91:constraint} and
historically has been done by hand~\cite{offutt00:mutation}. But
using a change-impact analysis tool such as
Margrave~\cite{fisler05:verification} allows us to detect equivalent
mutants among generated mutants. We originally believed
equivalent-mutant detection to be an important efficiency
improvement though we found in practice that evaluating requests and
comparing responses to be computationally cheaper than performing
change-impact analysis with Margrave. Furthermore, limitations of
Margrave prevented the detection of equivalent mutants for mutation
operators on conditions and some combining algorithms. }

\Comment{
\subsection{Mutation Operators} \label{sec:operators}

Previous studies~\cite{ma02:interclass, hennessy05:analysis} have
been conducted to investigate the types and effectiveness of various
mutation operators for general-purpose programming languages;
however, these mutation operators often do not directly apply to
mutating policies. This section describes the chosen mutation
operators for XACML policies. An index of the mutation operators is
listed in Table~\ref{table:mutationops} and their details are
described as below.

\begin{table}[t]
    \begin{small}
        \begin{center}
        \caption{\label{table:mutationops}Index of mutation operators.}
            \begin{tabular}{|l|l|}
                \hline ID  & \CenterCell{Description}\\
                \hline
                \hline PSTT & Policy Set Target True\\
                \hline PSTF & Policy Set Target False\\
                \hline PTT & Policy Target True\\
                \hline PTF & Policy Target False\\
                \hline RTT & Rule Target True\\
                \hline RTF & Rule Target False\\
                \hline RCT & Rule Condition True\\
                \hline RCF & Rule Condition False\\
                \hline CPC & Change Policy Combining Algorithm\\
                \hline CRC & Change Rule Combining Algorithm\\
                %\hline CPO & Change Policy Order\\
                %\hline CRO & Change Rule Order\\
                \hline CRE & Change Rule Effect\\
                \hline
            \end{tabular}
        \end{center}
    \end{small}
\end{table}

\begin{enumerate}

\item \Intro {Policy Set Target True (PSTT)}. Ensure that the
policy set is applied to all requests by removing the
\CodeIn{<Target>} tag of each \CodeIn{PolicySet} element. The number
of mutants created by this operator is equal to the number of
\CodeIn{PolicySet} elements with a \CodeIn{<Target>} tag. The number
of equivalent mutants created depends on the specific policy under
test.

\item \Intro {Policy Set Target False (PSTF)}. Ensure that the
policy set is never applied to a request by modifying the
\CodeIn{<Target>} tag of each \CodeIn{PolicySet} element. The number
of mutants created by this operator is equal to the number of
\CodeIn{PolicySet} elements. The number of equivalent mutants
created depends on the specific policy under test.

\item \Intro{Policy Target True (PTT)}. Ensure that the policy is
applied to all requests simply by removing the \CodeIn{<Target>} tag
of each \CodeIn{Policy} element. The number of mutants created by
this operator is equal to the number of \CodeIn{Policy} elements
with a \CodeIn{<Target>} tag. The number of equivalent mutants
created depends on the specific policy under test.

\item \Intro{Policy Target False (PTF)}. Ensure that the policy is
never applied to a request by modifying the \CodeIn{<Target>} tag of
each \CodeIn{Policy} element. The number of mutants created by this
operator is equal to the number of \CodeIn{Policy} elements. The
number of equivalent mutants created depends on the specific policy
under test.

\item \Intro{Rule Target True (RTT)}. Ensure that the rule is
applied to all requests simply by removing the \CodeIn{<Target>} tag
of each \CodeIn{Rule} element. The number of mutants created by this
operator is equal to the number of \CodeIn{Rule} elements with a
\CodeIn{<Target>} tag. The number of equivalent mutants created
depends on the specific policy under test.

\item \Intro{Rule Target False (RTF)}. Ensure that the rule is
never applied to a request by modifying the \CodeIn{<Target>} tag of
each \CodeIn{Rule} element. The number of mutants created by this
operator is equal to the number of \CodeIn{Rule} elements. The
number of equivalent mutants created depends on the specific policy
under test.

\item \Intro{Rule Condition True (RCT)}. Ensure that the condition
always evaluates to \CodeIn{True} simply by removing the condition
of each \CodeIn{Rule} element. The number of mutants created by this
operator is equal to the number of \CodeIn{Rule} elements with a
\CodeIn{<Condition>} tag. The number of equivalent mutants created
depends on the specific policy under test.

\item \Intro{Rule Condition False (RCF)}. Ensure that the
condition always evaluates to \CodeIn{False} by manipulating the
condition value or the condition function. The number of mutants
created by this operator is equal to the number of \CodeIn{Rule}
elements. The number of equivalent mutants created depends on the
specific policy under test.

%    \Comment{ \item \Intro{Change Policy Order (CPO)}. Try all permutations of policy orderings.  This mutation operator is only meaningful if there is more than one policy and the policy combining algorithm is order sensitive. For instance, the permit-overrides combining algorithm is not order sensitive whereas the first-applicable algorithm is. The number of mutants created by this operator is zero when the combining algorithm is not order sensitive. The number of mutants created by this operator is $P!-1$, where $P$ is the number of \CodeIn{Policy} elements, when the combining algorithm is order sensitive. The number of equivalent mutants created depends on the specific policy under test.}

%   \Comment{ \item \Intro{Change Rule Order (CRO)}. Try all permutations of rule orderings.  As for CPO, this mutation operator is only meaningful if there is more than one rule and the rule combining algorithm is order sensitive. The number of mutants created by this operator is zero when the combining algorithm is not order sensitive. The number of mutants created by this operator is $R!-1$, where $R$ is the number of \CodeIn{Rule} elements, when the rule combining algorithm is order sensitive. The number of equivalent mutants created depends on the specific policy under test.}

\item \Intro{Change Policy Combining Algorithm (CPC)}. Try all
possible policy combining algorithms. This high-level mutation may
change the way that various policies interact. This operator is only
meaningful if there is more than one \CodeIn{Policy} element in the
policy under test. Currently there are six policy combining
algorithms implemented in Sun's XACML
implementation~\cite{sun05:xacml} but four of these algorithms
semantically reduce to two, leaving only four policy combining
algorithms, namely deny-overrides, permit-overrides,
first-applicable, and only-one-applicable. The number of mutants
created by this operator for policies with more than one
\CodeIn{Policy} element is three and zero otherwise. The number of
equivalent mutants created depends on the specific policy under
test.

\item \Intro{Change Rule Combining Algorithm (CRC)}. Try all
possible rule combining algorithms. This high-level mutation may
change the way that various rules interact. This operator is only
meaningful if there is more than one \CodeIn{Rule} element in the
policy under test. Currently there are five rule combining
algorithms implemented in Sun's XACML
implementation~\cite{sun05:xacml} but four of these algorithms
semantically reduce to two, leaving only three rule combining
algorithms, namely deny-overrides, permit-overrides, and
first-applicable. The number of mutants created by this operator for
policies with more than one \CodeIn{Rule} element is two and zero
otherwise. The number of equivalent mutants created depends on the
specific policy under test.

\item \Intro{Change Rule Effect (CRE)}. Invert each rule's
\CodeIn{Effect} by changing \CodeIn{Permit} to \CodeIn{Deny} or
\CodeIn{Deny} to \CodeIn{Permit}. The number of mutants created by
this operator is equal to the number of rules in the policy. This
operator should never create equivalent mutants unless a rule is
unreachable, in which case the rule should probably be removed.

\end{enumerate}
}
