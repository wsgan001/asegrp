\section{Conclusion}
\label{sec:conclusion}

In this paper, we have developed a first step toward systematic
policy testing by defining and measuring policy coverage. We have
proposed the concept of policy testing and policy coverage based
on a general access control model. We further defined three levels
of specific policy coverage for XACML policies: policy hit
percentage, rule hit percentage, and condition hit percentage. To
support systematic policy testing based on policy coverage
automatically, we have developed a coverage-measurement tool, a
request-generation tool, and a request-reduction tool. By using
mutation testing, we have conducted an experiment that assesses
the coverage-based request reduction and its effect on
fault-detection capabilities. The experimental results showed that
the coverage-based request reduction substantially reduce the size
of the request set but incur only relatively low loss of
fault-detection capabilities. We also conducted a study on the
policy coverage achieved by manually generated requests for
policies in a conformance test suite for XACML
specifications~\cite{anderson02:xacml} and a conference reviewing
system~\cite{zhang04:synthesis}. Our results showed that our
measurement results can pinpoint uncovered areas of policies and
guide the development of new requests to achieve higher policy
coverage.

In future work, we plan to develop a comprehensive suite of
techniques and tools for systematic policy testing. In particular,
we plan to extend our policy coverage to consider cases that
reflect the interactions of different rules or different policies,
which are not focused by our existing policy coverage. We also
plan to conduct experiments on a larger scope of policies.

\Comment{
Our policy-testing tools and experimental results are available at:\\
\url{http://www.csc.ncsu.edu/faculty/xie/poco/}. }
