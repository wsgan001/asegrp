\section{Access Control Policies and Policy Coverage}
\label{sec:model}

Besides XACML, a generic policy language, many access control policy
languages have been proposed for different application domains.
Policies in these languages are usually composed of a set of rules,
which specify under what conditions a subject is allowed or denied
access to certain objects in a system. To discuss policy coverage
criteria in general, we model access requests and policies in this
paper as follows.

Let $\mathcal{S}$, $\mathcal{O}$ and $\mathcal{A}$ denote
respectively the set of all the subjects, objects and actions in an
access control system. Each subject, object, or action is associated
with a set of attributes that may be used for access control
decisions. For example, a subject's attributes may include a user's
role, rank, and security clearance. An object's attributes may
include a file's type, a document's security class, and a printer's
location.

An access request $q$ is a tuple $(s, o, a)$, where $s \in
\mathcal{S}$, $o \in \mathcal{O}$ and $a \in \mathcal{A}$.
A request $(s, o, a)$ means that
subject $s$ requests to take action $a$ on object $o$.

An access control policy $P$ is a sequence of rules, each of which
is of the form $(Cond_s, Cond_o, Cond_a, decision, Cond_g)$.
$Cond_s$, $Cond_o$ and $Cond_a$ are constraints over the attributes
of a subject, object, and action, respectively. $Cond_g$ is a
general constraint that may potentially be over all the attributes
of subjects, objects, actions, and other properties of a system
(e.g., the current time and the load of a system). A $decision$ is
either $deny$ or $permit$. Given a request $(s,o,a)$, if
$Cond_s(s)$, $Cond_o(o)$, $Cond_a(a)$, and $Cond_g$ are all
evaluated to be $true$, then the request is either permitted or
denied, according to $decision$ in the rule.

One may wonder that since $Cond_g$ can be a general constraint over
the attributes of subjects, objects, and actions as well as other
properties of a system, why do we still need $Cond_s$, $Cond_o$, and
$Cond_a$ in a rule? The reason is that, although conceptually those
conditions can be merged with the general condition $Cond_g$, by
separating them, it makes it easy to quickly locate relevant rules
to a request. For example, given a request $(s,o,a)$, if one of
$Cond_s$, $Cond_o$ and $Cond_a$ is evaluated to be false, then we do
not need to further evaluate $Cond_g$ that sometimes may be much
more complex than the former three. Such a form of access control
rules is commonly supported in access control policy languages. If a
request satisfies $Cond_s$, $Cond_o$ and $Cond_a$ of a rule, then we
say the rule is {\em applicable} to the request.

A policy may have multiple rules that are applicable to a request.
These rules may in fact offer conflicting decisions. The final
decision regarding the request depends on application-specific
conflict resolution functions. Commonly used conflict resolution
functions include denial overriding permission (where a request is
denied if it is denied by at least one rule), permission overriding
denial (where a request is permitted if it is permitted by at least
one rule) and first applicable (where the final decision is the same
as that of the first applicable rule in a sequence of rules whose
condition $Cond_g$ is evaluated true). We use $PDP$ (Policy Decision
Point) to denote the component of a system where final decisions are
made according to the decision of each rule and a specific
confliction resolution function. Conceptually, given a policy $P$
and a request $q$, a PDP returns the access control decision of $q$.
%One basic module of PDP is
%$RuleEval$, which
%takes a rule and a request, and returns the decision the rule's access control
%decision on the request.
%Another basic module is $CondEval$, which evaluates
%whether a condition is true.

Since we are interested in capturing potential errors in policy
specifications, we assume that PDP is correctly implemented in the
rest of the paper. In practice, generic PDP implementations are
often available, which have been scrutinized by the public.

We next start our discussion on policy testing based on the
preceding model. The basic idea of policy testing is simple. Like
software testing, given a policy, we would like to generate a set of
requests, and check whether the access control decisions on these
requests are expected. Any unexpected decision indicates potential
errors in the specification of the policy.

If no requests are evaluated against a rule during testing, then
potential errors in that rule cannot be discovered. Thus, it is
important to generate requests so that a large portion of rules are
involved in the evaluation of at least one of the requests. In other
words, we are interested in requests that cause a rule's conditions
to be evaluated to be true. Recall that if a request satisfies
$Cond_s$, $Cond_o$, and $Cond_a$ of a rule, then we say the rule is
{\em applicable} to the request.

\begin{definition}
Given a request $q$ and a rule $m$ in a policy $P$, we say $q$ {\em covers}
$m$ if $m$ is applicable to $q$.
% and $RuleEval(m, r)$ is invoked by the
%PDP when determining the final decision on $q$.
Given a set of requests $\mathcal{Q}$, the
rule coverage of $P$ by $\mathcal{Q}$ is the ratio between
the number of rules covered by at least one request in $\mathcal{Q}$
and the total number of rules in $P$.
\end{definition}

Intuitively, the higher the rule coverage of a set of requests, the
better chance specification errors may be discovered. Like software
testing, it is often infeasible to have exhausted policy testing
when the space of possible requests is large. Therefore, policy
specification errors may still exist even after testing with
requests that cover all the rules.

To improve the quality of policy testing, it helps to further examine potential errors
in the specification of conditions in each rule, which can also be tested by requests.

\begin{definition}
Given a request $q$ and a rule $m(Cond_s, Cond_o, Cond_a,
decision, Cond_g)$, we say $Cond_g$ is {\em positively
(negatively) covered} by $q$ if $m$ is covered by $q$ and $Cond_g$
is evaluated to be true (false). Given a set of requests
$\mathcal{Q}$, the condition coverage of $P$ by $\mathcal{Q}$ is
the ratio between the numbers of general conditions positively or
negatively covered by at least one request in $\mathcal{Q}$ and
two times of the total number of rules in $P$.
\end{definition}

\Comment{
\begin{definition}
Given a request $q$ and a rule $m(Cond_s, Cond_o, Cond_a, decision, Cond_g)$, we
say $Cond_g$ is {\em positively (negatively) covered} by $q$ if $m$ is covered by $q$ and $Cond_g$ is evaluated
to be true (false). Given a set of requests $\mathcal{Q}$, the
positive (negative) condition coverage of $P$
by $\mathcal{Q}$ is the ratio between the numbers of general conditions positively (negatively)
covered by at least one request in $\mathcal{Q}$ and the total number of rules in $P$.
The overall condition coverage is the average of the positive and negative condition coverage.
\end{definition}
}

The intuition behinds the above definition is as follows.
An error in the condition of a rule may have two types of impacts on a request. Suppose
$Cond_g'$ is the condition when an error is introduced to the original condition $Cond_g$.
Given a request $q$, $Cond_g'(q)$ may be evaluated to be true while $Cond_g(q)$ is false,
or vice versa. That is why we concern with both
 positive and negative coverage of a condition in the preceding definition.

Our definition of condition coverage corresponds to clause coverage
or condition coverage~\cite{myers79:art} in program testing. Note
that there exist more complicated coverage criteria for logical
expressions. For example, in program testing, predicate coverage
(also called decision coverage or branch
coverage)~\cite{myers79:art} requires to cover both true and false
of compound conditions in a logical expression. In policy testing,
predicate coverage requires that the whole compound condition for a
rule needs to be evaluated to be true and false, respectively. In
program testing, combinatorial coverage (also called multiple
condition coverage)~\cite{myers79:art} requires to cover each
possible combination of outcomes of each condition in a logical
expression. In policy testing, combinatorial coverage requires to
cover each possible combination of outcomes of each condition for a
rule. In our existing approach, we use basic, simple criteria for
conditions in rules; in future work, we plan to investigate these
more complicated alternatives in terms of their effects on
fault-detection capability.
