\section {Evaluation}
\label{sec:case study}

\New{In our evaluation, we address the following research questions.
(1) What is the percentage of hotspots and coldspots among the classes of a given framework?
This question helps to signify the approximate efforts required for a programmer unfamiliar
with the framework to start reusing that framework.
(2) How close the detected hotspots align with the instantiation code? In this evaluation,
an application using a given framework is used a test application. Hotspots
of the framework are detected using a code search engine, by excluding 
that specific test application. The detected hotspots
are later verified against the test application.
(3) Does any relationship exists between changes made in source files among several
versions and hotspots? This question is to confirm our claim whether
changes made among versions often include changes among hotspots?}

We evaluated SpotWeb with JUnit and Log4j frameworks to show
that our approach can detect hotspots described in their respective documentations.
We also explain how SpotWeb results can be used by a framework user.
The primary reason for selecting Log4j\footnote{\url{http://logging.apache.org/log4j/docs/manual.html}}
and JUnit\footnote{\url{http://junit.sourceforge.net/doc/cookstour/cookstour.htm}}
for analysis is the availability of their documentation that can
help validate the detected hotspots.

\setlength{\tabcolsep}{1pt}
\begin{table}[t]
\begin{CodeOut}
\begin{center}
\begin {tabular} {|l|c|c|c|c|c|c|}
\hline
Subject&\# Classes&\multicolumn{5}{|c|}{Hotspots}\\
\cline{3-7}
&&\# Classes&\%&\# Templ&\# Hooks&\# Depend\\
\hline JUnit            &56&23&41.07&11&5&11\\
\hline Log4j        &207&74&35.74&49&13&44\\
\hline
\end{tabular}\vspace*{-3ex}
\centering \caption {\label{tab:hotresults} Evaluation results with JUnit and Log4j}
\end{center}
\end{CodeOut}\vspace*{-8ex}
\end{table}

Table~\ref{tab:hotresults} shows the number of hotspots detected in each framework.
The table also shows the number of templates, hooks, and the number of dependencies among 
templates and hooks. SpotWeb detected all hotspots described in documentations
of JUnit and Log4j resulting in a recall of 100\%. SpotWeb detected a few other hotspots that are not 
described in documentation, resulting in a precision of 26.08\% for JUnit and
16.21\% for Log4j. \Comment{However, the real hotspots could be beyond the starting points 
described in the documentation and our real precision could be much higher 
than the one calculated based on the documentation. }

We next explain hotspots detected in the JUnit framework.

We next describe the detected hotspots in the Log4j framework.
Log4j provides several features such as Appenders and Layouts, and
for each such feature, Log4j provides several classes.
Among those several classes, a few classes are much more often 
used than other classes. SpotWeb correctly detected the classes that
are often used for each such feature. For example, SpotWeb detected that
\CodeIn{ConsoleAppender} and \CodeIn{FileAppender} are the commonly used
classes for the Appender feature. SpotWeb also captured the dependency 
information that is described in the documentation.
For example, the appender classes of the Log4j library require layout classes.
SpotWeb correctly identified a \CodeIn{TEMPLATE\_TEMPLATE} dependency between appenders
and layouts. The dependency describes that the user needs an instance of
layouts such as \CodeIn{PatternLayout} or \CodeIn{SimpleLayout} to create an instance of appenders
such as \CodeIn{ConsoleAppender} or \CodeIn{FileAppender}.
