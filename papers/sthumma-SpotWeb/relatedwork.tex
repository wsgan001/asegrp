\section {Related Work}
\label{sec:related}
We used CSEs for gathering related code examples in our 
previous approaches called MAPO~\cite{mapo:xie} and PARSEWeb~\cite{thummalapenta07:parseweb}.
With MAPO and PARSEWeb, SpotWeb shares \emph{only} the code downloader component that is used for gathering
related code examples. As each approach targets
a different problem, each approach has different techniques 
in analyzing gathered code examples to address the
challenges of that problem. MAPO and PARSEWeb focus on mining
API usage patterns to assist programmers to write effective API client code.
Given an API, MAPO identifies the frequent usage patterns of that API from
gathered code examples. PARSEWeb analyzes each gathered code example and builds
a CFG, which includes only method-invocation nodes, and captures sequences
that serve as solutions for the queries of the form ``Source $\rightarrow$ Destination''.
The SpotWeb approach is significantly different from these approaches as SpotWeb targets at capturing
\emph{UsageMetrics} of classes and methods of a given input framework and uses
these metrics to detect hotspots and coldspots.

An approach by Viljamaa~\cite{viljamaa:reverse} also recovers the hotspots of a framework
by using the source code of the input framework and a set of available example applications. Their approach
uses concept analysis~\cite{ganter:concept} for uncovering hotspots of the framework. One major
problem with their approach is that applying concept analysis to the entire input source code
can result in a huge pattern that is not useful in practice. To
address the preceding problem, their approach 
suggests to select only those program elements 
that are relevant to the hotspot \emph{h} at hand. Therefore, their
approach requires the users to have some initial knowledge of the structure
and hotspots of the framework under analysis. In contrast,
our approach uses simple statistical analysis and can handle an entire input
framework. Furthermore, SpotWeb does not require
the users to have any knowledge of the input framework. Moreover,
our approach performs better than their approach as shown in our evaluation.

Mendonca et al.~\cite{mendoca:instantiation} proposed an approach to
assist framework instantiation and to understand the intricate
details surrounding the framework design.
However, their approach requires framework developers to
manually specify the framework design in a specific process
language, called Reuse Definition Language, proposed by their approach.

Holmes and Walker~\cite{holmes:apiusage} proposed an approach 
that quantitatively determines how existing APIs are used. Their approach
gathers a few applications that already reuse those existing APIs
and computes metrics to detect how existing APIs are used by those applications.
SpotWeb differs from their approach in three main aspects. First, their approach 
expects the framework users to have knowledge 
of the APIs of the framework. Therfore, their approach is mainly useful to users who are already
familiar with those framework APIs. Second, their approach presents only the
number of times that the APIs are reused. Third, their approach computes metrics
from a limited data scope. In contrast, SpotWeb does not require the users
to have the knowledge of APIs of the input framework and presents information
in a more comprehensive form through templates and hooks.

Baxter et al.~\cite{baxter:shape} proposed an approach to discover
the structure of Java programs and the way that the classes relate
to each other through inheritance and composition. Their study is
useful for the framework developers who can evaluate the structural
features of their own programming practice and optimize their
performance. In contrast, SpotWeb is useful for the framework users
in effectively reusing the APIs of the framework.