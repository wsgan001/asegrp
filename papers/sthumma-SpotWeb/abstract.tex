\begin{abstract}

Software developers often face
challenges in reusing open source frameworks due to several factors
such as the framework complexity and lack of proper
documentation. In this paper, we propose a code-search-engine-based
approach that detects \Intro{hotspots} in a given framework
by mining code examples gathered from open source repositories available on the web; 
these hotspots are API classes and methods that are frequently reused. 
Hotspots can serve as starting points for developers
in understanding and reusing the given framework. 
Our approach also detects \Intro{coldspots}, which are API classes and methods that are rarely used.
Coldspots serve as caveats for developers as there can 
be difficulties in finding relevant code examples and are generally less exercised
compared to hotspots. We developed a tool, called SpotWeb, for
frameworks or libraries written in Java and used our tool
to detect hotspots and coldspots of eight widely used open source
frameworks. We show the utility of our detected hotspots 
by comparing these hotspots with the API classes reused by a real application
and compare our results with the results of a previous related approach.
\end{abstract}
