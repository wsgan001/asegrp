\section {Implementation}
\label {sec:impl}

We developed an Eclipse plugin tool for our approach, which
identifies hotspots and coldspots in libraries developed in the Java
programming language. Our tool takes the input library in the form
of an Eclipse project. Our tool requires the input library to be
compilable, i.e., all dependent jars have to be specified in
classpath of the input library. Our tool can be invoked by selecting
the menu item provided as a pop up menu over the Eclipse project.
The computed \emph{UsageMetrics} as well as the identified hotspots
and coldspots are presented in the form of text files. In our implementation, 
GCSE~\cite{GCSE} is used as an underlying CSE. As GCSE provides code
examples in the HTML format, the code downloader invokes a
postprocessor written in the Perl programming language to transform
HTML files into Java source code examples.

The library reader and the code analyzer components use Eclipse JDT
Compiler~\cite{EclipseJDT} for building ASTs from the downloaded
Java files. We used a Directed Acyclic Graph (DAG) for representing
the intermediate form used for analysis, because DAG provides an
easy mechanism for traversing different paths that are used to
identify groups of APIs of the given library. We used the
Jung~\cite{Jung} library for building the DAG for each code sample.
\Comment{
Mining Maximal Frequent Itemsets (MAFIA)~\cite{burick:mafia} is used
as a frequent item set miner for mining frequent usage scenarios
from the entire set of gathered usage scenarios. MAFIA performs a
depth-first traversal of the item set lattice and incorporates
effective pruning mechanisms. The primary reason for choosing MAFIA
among available frequent item set miners is that MAFIA performs
better than other efficient frequent item set miners such as DepthFirst
by a factor of three to five on average. We used a support threshold
value of 0.01, which is low enough to identify a good number of 
frequent usage scenarios. The threshold value is derived based on our
evaluation results with JUnit and Log4j libraries. The mined frequent 
usage scenarios are sorted with respect to their corresponding support values.}
