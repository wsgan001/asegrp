\Comment{\section {Discussion and Future Work}
\label{sec:discussion}

In our current implementation, our tool interacts with the Google
Code Search Engine~\cite{GCSE} for gathering relevant code examples
for all classes and interfaces of the given input library.
Therefore, our current results are dependent on code examples
returned by the GCSE. In future work, we plan to extend our tool to
collect code examples from other CSEs such as Koders~\cite{KODERS}
and Krugle~\cite{KRUGLE}, and analyze the results to compare the
effectiveness of different CSEs.}

\section{Threats to Validity}
\label{sec:threats}
The threats to external validity primarily include the degree to
which the subject programs and used CSE are representative of true
practice. The current subjects range from small-scale applications
such as Grappa to large-scale applications such as BCEL and JUNG. In
SpotWeb, we used only one CSE, i.e., Google code search. We plan to
reduce these threats by conducting more experiments on wider types
of subjects and by using other CSEs in future work. The threats to
internal validity are instrumentation effects that can bias our
results. Faults in our SpotWeb prototype might cause such effects
while detecting hotspots and coldspots. To reduce these threats, we
inspected some code examples gathered from the CSE to double check
the metrics computed by SpotWeb for these code examples.

