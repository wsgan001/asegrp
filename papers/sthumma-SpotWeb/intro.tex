\section {Introduction}

Reuse of existing open source frameworks or libraries (referred as frameworks)
has become a common practice in the
current software development process due to several factors such as low
cost and high efficiency. However, existing frameworks often offer
complex procedures that pose challenges to developers for effective reuse.
This complexity also makes the documentation of the framework a
vital resource. However, the documentation is often missing for many
existing frameworks and even if such documentation exists, it
is often outdated~\cite{document:leth}.

In general, frameworks expose certain areas (API classes and methods) of
flexibility that are intended for reuse by their users. Software
developers who reuse classes and methods of these frameworks must be aware of
these flexible areas for effective reuse of frameworks. These
areas of flexibility are often referred as \Intro{hotspots}.
As described by Pree~\cite{pree:metapatterns}, hotspots depict a framework's
flexibility and proneness to reuse. The foundations of hotspots are
built upon the Open-Closed principle by Martin~\cite{martin:open}.
The Open-Closed principle encompasses two main definitions: the
``open'' and the ``closed'' parts. The ``open'' parts (referred as \Intro{hooks}) represent
areas that are flexible and variant, whereas the ``closed'' parts (referred as \Intro{templates}) 
represent areas that are immutable in the given framework. 
A \Intro{hotspot} is defined as a combination of templates and hooks.

Hotspots are useful to both users and developers of the framework
in several ways. First, new users can browse and inspect hotspots to understand commonly
reused classes and find out the classes that the users want to reuse.
Second, users may have more confidence or tendencies in
reusing hotspots because generally bugs in these hotspots may be
fewer (or more easily exposed previously) than the ones in
non-hotspots; we can view the application code that reuses framework hotspots to be a special
type of test code that can help expose bugs in
hotspots. Third, developers or maintainers of these frameworks can choose to
invest their improvement efforts (e.g., performance improvement or bug fixing) 
on these hotspots because the resulting returns on
investment may be substantial.

In contrast to hotspots, we call a framework's areas that
are rarely used by users as \Intro{coldspots}.
The concept of coldspots is introduced by our approach and these
coldspots can serve as caveats to users of the given framework.
As coldspots represent the rarely used classes and methods, there can be difficulties
in identifying relevant code examples that can help users in reusing
those classes and methods. Moreover, coldspots are generally less tested compared to hotspots
with regards to the ``testing'' conducted by API client code as
test code.

Detecting hotspots and coldspots of an input framework requires domain knowledge of how the API classes
and methods of the input framework are reused by applications, referred as client
applications. Various open source projects that reuse classes of a given input framework
are available on the web and these open source projects
can serve as a basis for gathering the information of how classes of the
input framework are reused, and hence can help in detecting hotspots and coldspots.
Therefore, our approach, called SpotWeb, leverages a code search engine (CSE) to gather
relevant code examples of classes of the input framework from these open source
projects. Given a query, a CSE can extract code examples
with usages of the query from open source projects available on the web. Our approach analyzes
gathered code examples statically and detects hotspots and coldspots of the
given framework. Our approach tackles
the problems related to the quality of code examples gathered
from a CSE by capturing the most common usages of classes through mining.

The paper\footnote{An earlier version of this work is described in a 4-page position paper
presented at MSR 2008~\cite{thummalapenta08:spotweb}.}makes the following main contributions:
\vspace{2ex}
\begin{Itemize}

\Item An approach for detecting hotspots of a given framework by analyzing relevant
code examples gathered from a CSE.
\vspace{3ex}
\Item An approach for detecting coldspots of a given framework.
\vspace{2ex}
\Item An Eclipse plugin implemented for the proposed approach 
and several evaluations to assess the effectiveness of the tool. In our evaluation,
SpotWeb detects hotspots and coldspots of eight widely used open source frameworks
by analyzing a total of $7.8$ million lines of code. We show the utility of 
detected hotspots by comparing detected hotspots
of a framework with a real application reusing that framework. We also compare our
results with the results of a previous related approach by Viljamaa~\cite{viljamaa:reverse}.
\end{Itemize}
\vspace{2ex}

The rest of the paper is organized as follows. 
Section~\ref{sec:example} explains our approach through an illustrative example.
Section~\ref{sec:specapproach} describes key aspects of the approach.
Section~\ref{sec:eval} discusses evaluation results.
Section~\ref{sec:threats} presents threats to validity.
\Comment{Section~\ref{sec:discussion} presents limitations and future work.}
Section~\ref{sec:related} presents related work.
Finally, Section~\ref{sec:conclusion} concludes.


