\section{Problem Definition}

In this section, we present formal definitions of terms that we use throughout this paper.
In addition, we formally define the regression test generation problem that 
we address in this paper. 

We use concrete-state representation from our previous work~\cite{xie04:rostra} 
to represent the state of an object. The formal definitions of 
 Object State and Object State Equivalence  can be found 
in our previous work~\cite{xie04:rostra}. 

\Comment{
We also call each part a \Intro{heap}
and view it as a graph: nodes represent objects, and edges represent
fields. Let $P$ be the set consisting of all primitive values,
including \CodeIn{null}, integers, etc. Let $O$ be a set of objects
whose fields form a set $F$. (Each object has a field that
represents its class, and array elements are considered
index-labeled fields of the array objects.)
\define{Heap.}
{
%When a variable (such as the return or receiver of a method
%invocation) is a non-primitive-type object, we use concrete-state
%representation from our previous work~\cite{xie04:rostra} to
%represent the variable's value or state. 
%A program executes on the
%program state that includes a program heap. 
%The concrete-state representation of an object considers only parts of the heap that
%are reachable from the object. 
%We also call each part a \Intro{heap}
%and view it as a graph: nodes represent objects, and edges represent
%fields. Let $P$ be the set consisting of all primitive values,
%including \CodeIn{null}, integers, etc. Let $O$ be a set of objects
%whose fields form a set $F$. (Each object has a field that
%represents its class, and array elements are considered
%index-labelled fields of the array objects.)\\\\
A \emph{heap} is an edge-labeled graph $\Pair{O}{E}$, where
$E=\SetSuch{\Triple{o}{f}{o'}}{o \in O, f \in F, o' \in O \cup
P}$.
}
Heap isomorphism is defined as graph isomorphism based on node
bijection~\cite{boyapata02-korat}.
%\vspace*{-0.5ex}
\define{Heap Isomorphism.}{
Two heaps $\Pair{O_1}{E_1}$ and $\Pair{O_2}{E_2}$ are
\emph{isomorphic} iff there is a bijection $\rho: O_1 \to O_2$
such that:
\begin{eqnarray}
E_2 & = & \SetSuch{\Triple{\rho(o)}{f}{\rho(o')}}{\Triple{o}{f}{o'} \in E_1, o' \in O_1} \cup \nonumber\\
    &   & \SetSuch{\Triple{\rho(o)}{f}{o'}}{\Triple{o}{f}{o'} \in E_1, o' \in P}. \nonumber
\end{eqnarray}
}
%\vspace*{-1ex}
 The definition allows only object identities to vary:
two isomorphic heaps have the same fields for all objects and the
same values for all primitive fields.
\define{Object State.}
{
The state of an object $r$ is represented with a \emph{rooted} heap:
%\vspace*{-0.5ex}
A \emph{rooted heap} is a pair $\Pair{r}{h}$ of the root object $r$ and a
heap $h$ whose all nodes are reachable from $r$.
}

\define{Object State Equivalence.}
{
Two object states $s_1=\Pair{r_1}{h_1}$ and $s_2=\Pair{r_2}{h_2}$ are equivalent ($s_1 \equiv s_2$)
iff heap $h_1$ is isomorphic with heap $h_2$.
}
}

\vspace{-0.3cm}
\define {Program Execution Point.}{
Let $S= ( s_1, s_2,...,s_n)$ be the sequence of statements executed during a 
program execution $V(I)$ (i.e., when program $V$ is executed with a list of argument values $I$). 
Since a statement $s$ can be executed multiple times, multiple instances of $s$ 
can be present in $S$. A program execution point (in short an execution point) for a program execution $V(I)$
is a statement $s_i \in S$ such that statements $(s_1, s_2..., s_{i})$ have already been executed
but statements $s_{i+1}, s_{i+1},.. s_{n}$ are yet to be executed.
}
\vspace{-0.6cm}
\define{Program State.}
{
Let $M = \{m_1, m_2, m_3,..., m_n\}$ be the set of methods in 
the program call stack at an execution point $e$. 
Let $R = \{o_1, o_2, o_3,..., o_l\}$ be the set of objects such that $o_i$ is the 
receiver object of $m_i$, $A = \{ a_1, a_2,..., a_m\}$ be the set of all the arguments 
(primitive or non primitive) to methods in $M$, and 
$L= \{l_1, l_2, l_3,..., l_x\}$ be the set of all the in-scope local and global variables (primitive or non-primitive) 
at the statements  
$P = \{p_1, p_2, p_3,..., p_y, s_e\}$, where $p_i$ is the statement at which method $m_i$ is invoked and the 
$s_e$ is the statement corresponding to the execution point $e$.
The program state at an execution point $e$ is the set of object states $S_R \cup S_A \cup S_L$, where 
 $S_R = \{S_{o1}, S_{o2}, S_{o3}..., S_{ol}\}$ ($S_{oi}$
is the object state for object $o_i$ at $e$), 
 $S_A = \{S_{a1}, S_{a2}, S_{a3}..., S_{an}\}$ 
($S_{ai}$ is the object state for $a_i$ if $a_i$ is non-primitive otherwise 
$S_{ai}$ is the value of $a_i$), and $S_L = \{S_{l1}, S_{l2}, S_{l3}..., S_{lx}\}$ 
($S_{li}$ is the object state for $l_i$ at $s_e$ if $l_i$ is non-primitive otherwise 
$S_{li}$ is the value of $l_i$). 
}

%An object oriented program $P$ consists of a set 
%of classes $C = \{c_1, c_2, ..., c_l\}$. Each class $c_i$ contains a set of methods 
%$M_i = \{m_{i1}, m_{i2},..., m_{im}\}$ and a set of fields $F_i = \{f_{i1}, f_{i2},..., f_{in}\}$. 
%A method $m$ takes a set of arguments $I= \{i_1, i_2,...,\ i_o\}$ and the receiver object state $S_{R0}$ and produces as output $\Omega$ and a resulting receiver object state $S_{R1}$.
%An input $i \in I$ can be of a primitive type 

\Comment{
\define{Object State.}
{
State $S_o$ of an object $O$ (instantiated from class $C$) at a point $p$ in 
method $M$ in class $C$ is
defined by the values of fields $F$ in $C$ and the set of in-scope local variables $L$ at point $p$.
%\begin{center}
\\ \\
$S_o = \{<V(f_1), V(f_2),..., V(f_n), V(l_1), V(l_2), ..., V(l_m)>|f_i\in F,$ $V(x)$ is the 
value of $x,$ $l_i\in L\}$ 
%\end{center}
 \\ \\ 
Two program states $S_{o1}$ and $S_{o2}$ of the object $O$ at point $p$  
are equivalent ($S_{o1} \equiv S_{o2}$) if all fields 
$f_{i}\in F$ and all in-scope variables $L_j \in L$ 
of Program $P$ have equal values\footnote{
If a field $f_i$ (or an in-scope variable $l_i$) is of non-primitive type $V_1(f_i)$ (or $V_1(l_i)$) should be a deep clone of $V_2(f_i)$ (or $V_2(l_i)$).}
, i.e., $V_1(f_i) = V_2(f_i)$ (and $V_1(l_i) = V_2(l_i)$), 
where $V_1(x)$ is the value of field (or variable) x in $S_{o1}$ and 
$V_2(x)$ is the value of field (or variable) x in $S_{o2}$. 
%\footnote{Value of a non primitive type field is define by graph}.   
}

\define{Program State.}
{
Let $M = \{m_1, m_2, m_3,..., m_n\}$ be the set of methods in 
the program call stack at an execution point $e$. 
Let $R = \{o_1, o_2, o_3,..., o_n\}$ be the set of objects such that $o_i$ is the 
receiver object of $m_i$. 
Program state at an execution point $e$ is the set of object states of 
all the receiver objects $S_P = \{S_{o1}, S_{o2}, S_{o3}..., S_{on}\}$, where $S_{oi}$
is the object state for object $o_i$ at point of invocation of method $m_i$.  
}
}
\vspace{-0.2cm}
Let a program version $V$ be modified to incorporate a set of changes 
$\Delta$.
We assume that signatures of methods in $V$ are not changed and no new fields are added, modified, or deleted in $V$. 
$\Delta$ also includes the new statements added to $V$ and statements deleted from $V$.
The program version $V'$ is the resulting program version after   
$\Delta$ is applied to the program version $V$. A method $M$ in $V$
corresponds to a method $M'$ (denoted as $M \leftrightarrow M'$) in $V'$ if $M$ and $M'$ have the same signatures.
Assume that some dummy statements
are added to 
the two program versions for matching added/deleted statements between the two versions (as done by Santelices et al.~\cite{santelices}). 
These dummy statements make sure that 
a statement $s$ in $V$ can be mapped to a corresponding statement $s'$ in $V'$. 

\define{Execution Index~\cite{xin-pldi08}.}
{
An execution Index $E_{V(I)}$ of a program execution $V(I)$ is the function of execution points in 
$V(I)$ that satisfies the following property:
for any two execution points $x \in V(I)$ and $y \in V(I)$ $:$ if $x \neq y \Rightarrow E_{V(I)}(x) \neq E_{V(I)}(y) $
}
\vspace{-0.3cm}
\define{Corresponding Execution Points~\cite{xin-pldi08}.}
{
An execution point $e$ in the execution $V(I)$ corresponds to an
execution point $e'$ (denoted as $e \leftrightarrow e'$) in the execution $V'(I)$ 
($V'$ is a modified version of $V$) when $E_{V(I)}(e) = E_{V'(I)}(e')$.
For an execution point $e$ in the execution $V(I)$, there may or may not be a
corresponding execution point $e'$ in the execution $V'(I)$.
}
\vspace{-0.3cm}
\define{Parameterized Unit Test (PUT).}
{
In our formal model, we assume that a program is invoked through a 
Parametrized Unit Test (PUT)~\cite{tillmann05:parameterized}. 
A PUT $P_V =  <A, S, \lambda> $ is a method with a list of arguments $A$
and a list of statements $S=(s_1, s_2, ..s_n)$
such that $s_i \in S$ is a statement other than an assertion statement and $\lambda$ is a list of assertion statements. 
}
\vspace{-0.3cm}
\define{Corresponding PUTs.}
{
In our formal model, we assume that a program is invoked through a 
Parametrized Unit Test (PUT)~\cite{tillmann05:parameterized}. 
A PUT $P_V = <A, S, \lambda>$ in a program version $V$, corresponds to a PUT $P_{V'} = <A', S', \lambda'>$ (denoted as $P_V \leftrightarrow P_V'$), in program version $V'$ if
$V$ and $V'$ are different versions of a same program, $A = A'$, $S = S'$, and $\lambda = \lambda'$.
}


%\define{Parameterized Unit Test (PUT).}
%{
%In our formal model, we assume that a program is invoked through a 
%Parametrized Unit Test (PUT)~\cite{tillmann05:parameterized}. 
%A PUT $P_V = <V, I, M, \lambda> $ takes as input a list of arguments $A =(a_1, a_2,..., a_k)$, where $a_i$
%is a primitive or a non-primitive argument. 
%$P_V$ invokes the list of methods $M = (m_1, m_2, m_3,..., m_n)$ 
%in program version $V$  
%%with list of arguments $I_M = (I_1, I_2, I_3,..., I_n)$ such that if $i\in I_i \Rightarrow i \in I$ 
%and 
%asserts the behavior of $V$ using a list of assertions $\lambda$. We consider $I$ as inputs to the program version $V$.
%}
%\vspace{-0.5cm}
%\define{Corresponding PUTs.}
%{
%Two PUTs $P_V = <V, I, M, \lambda>$ and $P_{V'} = <V', I, M', \lambda>$ are corresponding (denoted as $P_V \leftrightarrow P_V'$) if
%$V$ and $V'$ are two versions of the same program, all the methods 
%$M = \{m_1, m_2, m_3,..., m_n\}$ invoked in $P_V$ are replaced with invocations to corresponding methods
% $M' = \{m'_1, m'_2, m'_3,..., m'_n\}$ in $V'$, respectively. 
%}
\vspace{-0.5cm}
%\define{Program State Infection.}
%{
%Two corresponding PUTs $P_V$ and $P_{V'}$ are invoked with a list of arguments (input) $I =(a_1, a_2,..., a_k)$.
%The input $I$ causes program state infection if the program state at some execution point 
%$e \in V(I)$ is $S$ and the program state at the corresponding program point
%$e' \in V'(I)$ is $S'$ such that $\neg (S \equiv S')$. If a corresponding point $e'$ does not exist,
% program state at $e \in P_V(I)$ is considered to be infected.
%}
\define{Infected Program State.}
{
The program state  of a program version $V'$ (invoked from PUT $P_{V'}$) is infected with respect to a 
program version $V$ (invoked by PUT $P_{V}$: $P_V \leftrightarrow P_V'$) 
at an execution point $e'\in V'(I)$, if 
$\exists e \in V(I): e \leftrightarrow e' $, program state of $V'$ at $e'$ is $S'$ and program state of $V$ at
$e \in V(I)$ is $S$, and $S$ is not equivalent to $S'$ (definition of equivalent states can be found in our previous work~\cite{xie04:rostra}).
}

\vspace{-0.5cm}
\define{Behavioral Difference.} 
{
Two corresponding PUTs $P_V$ and $P_{V'}$ are invoked with a list of argument values (input) $I =(a_1, a_2,..., a_k)$.
There is a behavioral difference between $V$ and $V'$ if there exists an input $I$ such that 
when $P_V$ and $P_{V'}$ are invoked with input $I$, $\exists \lambda_i \in \lambda$ such that $\lambda_i$ passes
for $P_V$ but fails for $P_V'$ or $\lambda_i$ fails
for $P_V$ but passes for $P_V'$. $I$ is referred to as an input detecting behavioral difference.
}
\vspace{-0.5cm}
\define{Problem Definition.} 
{
Regression test generation for a PUT $P_{V'}$ in program version $V'$
is the problem of generating an input $I$ for $P_{V'}$  
that detects behavioral difference 
between $V$ and $V'$.
This paper focuses on efficiently generating inputs detecting behavioral differences.
}
