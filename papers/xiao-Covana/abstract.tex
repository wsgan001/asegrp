\begin{abstract}
High structural coverage of the code under test is often used as an indicator of the thoroughness and the confidence level of testing. Dynamic symbolic execution is a testing technique which explores feasible paths of the program under test by executing it with different generated test inputs to achieve high structural coverage. It collects the symbolic constraints along the path explored and negates one of the constraints to obtain a new path. However, due to the difficulty of method sequence generation, long run loop and testability issues, it may not be able to generate the test inputs for every feasible path. These problems could be solved by involving developers' help to assist the generation of inputs for solving the constraints. To help the developers figure out the problems, reportig every issue encountered is not enough since browsing through a long list of reported issues and picking up the most related one for the problem is not a easy task as well. In this paper, we propose an approach for carrying out the casual analysis of residual structural coverage in dynamic symbolic execution, which collects the reported issues and coverage information, filter out the unrelated ones and report the non-covered branches with the associated issues. We conducted the evaluation on a set of open source projects and the result shows that our approach reported ?\% related issues(may be 100\% without false negative) and ?\% less issues than the issues reported by Pex, an automated structural testing tool developed at Microsoft Research for .NET programs.
\end{abstract}

