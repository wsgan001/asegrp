\begin{abstract}
Dynamic symbolic execution is a testing technique which exercises different execution paths of the program under test by executing it with generated test inputs. By writing parameterized unit tests, which takes inputs and states assumption and assertions, a tool using dynamic symbolic execution could create a test suite with high code coverage by genrating test inputs to satisfy different path conditions. However, 
solving some of path conditions require desirable method sequences to create and modify objects, which is often challenging due to the large search space of possible sequences. The other path conditions may not be easily satisfied due to testability issues as well, such as solving constraints which depends on the result of method from external library or running out of time before exit the loop. Failing to solve these issues would result in the uncoverage of the corresponding feasible paths.
In this paper, we propose an approach for carrying out the Casual Analysis of Residual Structural Coverage in Dynamic Symbolic Execution, which could provide the information of the causal effect chain for the specific uncovered branch and help developers to make it covered. For an uncovered branch, the reason why it could not be covered by adopting dynamic symbolic execution could be just one testability problem or the combination of them, like requiring method sequences which involves methods from external library as well. Based on the information collected by running the tool which uses dynamic symbolic execution, our approach could narrow down the issues reported and point out which one are relevant for helping achieve residual structural coverage. Using this information, developers could assist the tools to cover the branch much easier.
Implementation and experiment evaluation comes here \ldots
\end{abstract}

