\begin{abstract}
High structural coverage of the code under test is often used as an indicator of the thoroughness and the confidence level of testing. Dynamic symbolic execution is a testing technique which explores feasible paths of the program under test by executing it with different generated test inputs to achieve high structural coverage. However, due to the difficulty of method sequence generation, loop with uncertainemo and testability issues, some of the constraints encountered during executions could not be easily solved, which would result in the uncoverage of the corresponding feasible paths. These problems could be solved by involving developers' help to assist the generation of inputs for solving the constraints. To help the developers figure out the problems, reportig every issue encountered is not enough as browsing through a list of reported issues and picking up the most related one for the problem is not a easy task as well. In this paper, we propose an approach for carrying out the casual analysis of residual structural coverage in dynamic symbolic execution, which would classify the reported issues to different unsolved constraints, filter out the unrelated ones and rank the constraints based on the different factors, such as number of dependent blocks and number of issues related. We also provide another view of encountered issues by ranking them with the number of constraints associated. We conducted the evaluation on a set of open source projects and the result shows that our approach reported ?\% less issues than the issues reported by Pex, an automated structural testing tool developed at Microsoft Research for .NET programs.
\end{abstract}

