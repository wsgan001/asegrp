\section{Example}
\label{sec:example}
% A simple example
% - Include: where it comes from; a figure listing source code; brief description
% - Throughout the paper, it is important to have illustrating examples for those places that contain "dry" descriptions of your research
% - If you use several examples throughout the paper, you may not need a separate Example section
%
% Optional/important part of the section: high level description of applying your approach on the example
% - describe inputs/outputs of your approach without getting into too much detail
% - very important if the later approach description involves heavy hard-to-understand formalisms

We next illustrates how our techniques helps performance analyst to navigate and do drill down into performance problems. As an example, we use a trace file\footnote{The schema language for the trace can be found at http://www.eclipse.org/tptp/platform/documents/\\
resources/profilingspec/trace.dtd.html} collected via TPTP.
Figure ~\ref{fig:example} shows the relevant parts of the trace file.

\begin{figure}[t]
\begin{CodeOut}
\begin{alltt}
<?xml version="1.0"?>
<\textbf{TRACE}>
<\textbf{node} nodeId="..." hostname=""
 ipaddress="" timezone="" time="..."/>
<\textbf{processCreate} processId="..." pid="..."
 nodeIdRef="" time="..."/>
<\textbf{agentCreate} agentId="..." version="..."
 processIdRef="..." agentName="Java Profiling Agent"
 agentType="Profiler" agentParameters="server-enabled"
 time=""/>
<\textbf{traceStart} traceId="..." agentIdRef="..."
 time="..."/>
<\textbf{filter} pattern="..." mode="..." .../>
<\textbf{option} key="..." value=""/>
<\textbf{threadStart} threadId="..." time="..."
 threadName="Thread-38" groupName="main"
 parentName="system" objIdRef="0"/>
<\textbf{threadEnd} threadIdRef="..." time="..." />
<\textbf{classDef} threadIdRef="..." name="..." classId="\underline{26359}"/>
...
<\textbf{methodDef} name="\underline{doGet}" signature="" startLineNumber="..."
 endLineNumber="..." methodId="\underline{26342}" classIdRef="\underline{26359}"/>
<\textbf{methodEntry} thredIdRef="" time="" classIdRef="\underline{26359}"
 methodIdRef="\underline{26342}"/>
<\textbf{classDef} threadIdRef="..." name="..." classId="\underline{26361}"/>
<\textbf{methodDef} name="create" signature="" startLineNumber="..."
 endLineNumber="..." methodId="\underline{26345}" classIdRef="\underline{26361}"/>
<\textbf{methodEntry} thredIdRef="" time="" classIdRef="\underline{26361}"
 methodIdRef="\underline{26345}"/>
<\textbf{methodExit} threadIdRef="" time="".../>
...
<\textbf{/TRACE}>
\end{alltt}
\end{CodeOut}\vspace*{-2ex}
\Caption{\label{fig:example}Example Trace file}\vspace*{-3ex}
\end{figure}

The trace file consists of several elements.
\CodeIn{TRACE} is the root element for a valid document in XML. The \CodeIn{node} element contains the information that is associated with a specific node or host. The \CodeIn{processCreate} element...
The \CodeIn{classDef} defines a class and \CodeIn{methodDef} defines a method, respectively.
The \CodeIn{methodEntry} contains the information that is associated with a method entry. Especially, it has \CodeIn{methodIdRef} and \CodeIn{classIdRef} attributes. If the value of the attribute \CodeIn{methodIdRef} and \CodeIn{classIdRef} are the same as \CodeIn{classId} of the previous \CodeIn{classDef} and \CodeIn{methodId} of the previous \CodeIn{methodDef}, we identify the \CodeIn{methodEntry} elements as the child of the transaction.

For J2EE-based web applications, when a url request arrives at the server side, the web application invokes \CodeIn{doGet} or \CodeIn{doPost} method to process the url request.
So, we treat the methods as the beginning of a new transaction.
For example, there is a \CodeIn{classDef} that its \CodeIn{classId} is \CodeIn{26359}. The \CodeIn{classIdRef} of the \CodeIn{methodDef} named \CodeIn{doGet} is \CodeIn{26342} that is the same as the previous class definition. So, the method is the starting point of a new transaction. Because \CodeIn{classIdRef} and \CodeIn{methodIDRef} of \CodeIn{methodEntry}
 are \CodeIn{26359} and \CodeIn{26342} respectively, we identify the element as a child of the transaction. 