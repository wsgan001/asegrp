\begin{abstract}
% Abstract writing pitfall:
% - Don't put unexplained or undefined terms whose meanings are not well known
% - Solutions: explain them; rephrase then using plain words; not get into too much detail (without mentioning them)
% (1) Short Introduction
Software performance testing helps testers to validate and verify software quality in terms of response time, reliability, and scalability. Many performance-testing tools can measure client-side response time and correlated it with collected application transaction monitoring data. 
% (2) Short motivation (problem)
Such correlation can break down client-side response time into its constituent parts on the various tiers of the server(s) under test. 
% (3) Proposed Solution
To avoid high overhead, some other performance tools for Java applications perform runtime profiling by using JVM profiling with statistical sampling. However, such an imprecise profiling mechanism poses challenges for performance analysis. It is challenging to identify the correlation between a transaction or time-range on the client side and the profiling information at a server's JVM. Therefore, it is difficult to provide user-friendly drill-down navigation from client-side response time or time-range to determine problem spots and bottlenecks in the JVM(s) on the server sides because of a lack of correlation between the transactions or time-range driven by a load testing tool at the client side and the runtime sample data collected by the JVM profiling tools at the server sides.
% 4. Evaluation; Evaluation Results
In the paper, we have developed a new technique and tool to help to address the preceding issues in deep root-cause performance analysis. We evaluate the accuracy, efficiency, and effectiveness of our approach by conducting experiments with several real-world web applications.
\end{abstract}