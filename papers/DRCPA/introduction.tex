\section{Introduction}
\label{sec:introduction}

% For each section of the paper, consider writing a mini-introduction that says what its organization is,
% what is in each part, and how the parts relate to one another. For the whole paper, this is probably a paragraph.
% For a section or sub-section, it can be as short as a sentence. This may feel redundant to you (the author),
% but readers haven't spent as much time with the paper's structure as you have,
% so they will truly appreciate these signposts that help them orient themselves within your text.

% Introductory paragraph: Very briefly, what is the problem and why is it relevant to the audience attending *THIS CONFERENCE*?
% Moreover, why is the problem hard, and what is your solution?
Performance testing and analysis tools help testers to identify and diagnose the presence and cause of application or system bottlenecks. 
In performance testing, testers create loads to expose problems and observe failures. 
In the analysis and diagnosis against the testing results, they try to get the location of the root causes of the observed failures. 
Usually, deep root-cause performance analysis requires correlation of test results from the client side and traces/events information from the server side.

% Summary of Motivation
% Long motivation, problem to be solved, why existing solutions are not sufficient (sometimes examples help)
Many performance-testing tools \cite{RPT} \cite{LoadRunner} can measure client-side response time and correlated it with collected application transaction monitoring data from the server side. Application Response Measurement (ARM) \cite{ARM} used by the tools provides a common way to get profiling information and allows to measure application availability, application performance, and end-to-end transaction response time.

For non ARM-instrumented servers running Java applications, 
Java Virtual Machine Tool Interface (JVMTI) \cite{JVMTI} feature of Eclipse Test and Performance Tools Platform (TPTP) \cite{TPTP} allows performance-testing tool to collect full traces from the non ARM-instrumented servers. One problem is here is that the collection of full traces is expensive and slow. To reduce the cost of the collection of full traces from the servers, a selective instrumentation should be conducted. However, such an imprecise profiling mechanism poses challenges for performance analysis because it is difficult to identify the correlation between a transaction (or time-range) on the client and the profiling information at a server's JVM.

% Summary of Approach and Implementation
% Proposed solution and brief summary
% Some reviewers don't like you to claim your own approach to be "novel"
In our approach, we show that a technique to recover information that is available in full traces and necessary for performance analysis from sampled traces. 
With the sampled traces, we show a robust technique for matching the sampled method calls with the start and end points identified

% identifies which methods change object states or not,
%and to recommend assertions based on the side-effect analysis.
We have developed a new tool, an Eclipse plug-in, for reading in performance tests and traces in the XML format and identifying the start and end point of a transaction or a request among the method execution on the server side. The tool helps testers to perform deep root-cause analysis by integrating with a load-testing tool and navigating the performance results.

% Summary of Evaluation
In the evaluation, we compared the precision and recall of the set of we compared the runtime cost of the full traces with the cost of the sampled traces. {The result is.}

% Summary of Contribution
% Don't overclaim
% Similarly don't over-criticize other's work
% If you want to claim unjustified points, it is better to put them in conclusion or discussion section
% Even if so, be careful on wording
% What is the increment over earlier work by the same authors? by other authors? Is this a sufficient increment, given the usual standards of subdiscipline?

Contributions of our work are the followings:
\begin{itemize}
\item We provide an approach for supporting performance analysis in the absence of ARM. \vspace*{-1ex}
\item We suggest a technique for supporting performance analysis with sampled traces rather than fully collected traces.\vspace*{-1ex}
\end{itemize}

% Structure layout of the paper
The remainder of this paper is organized as follows.
Section~\ref{sec:approach} presents our approach.

%In the remainder of this paper, we present an example 