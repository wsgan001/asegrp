\section{Background}
In this section, we present a background on DSE based tool Pex~\cite{Pex} that we use in our approach. Pex starts the program execution with some default inputs. Pex then collects constraints on program inputs from the predicates at the branching statements executed in the program. We refer to each constraint at a branching statement as a branch condition. 
The conjunction of all the branch condition in the path followed by the input is referred to as a path constraint. Pex keeps track of the previous executions to build a dynamic execution tree. Pex, in the next run, chooses one of the unexplored branch of the execution tree (dynamically discovered thus far). Pex flips the chosen branching node in the dynamic execution tree (discovered thus far) to generate a new input that follows a new execution path. Pex chooses a branching node from the execution tree of the program using various search search strategies attempting to achieve high statement coverage fast. Pex combines all search strategies it uses into a meta-strategy that performs a fair choice between the strategies.
 
