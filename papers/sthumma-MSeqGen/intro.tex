\section{Introduction}
\label{sec:intro}

A primary objective of unit testing is to achieve high structural coverage such as branch coverage. Achieving high structural coverage with passing tests gives high confidence in the quality of the code under test. To achieve high structural coverage of object-oriented code, unit testing requires desirable method-call sequences (in short as \emph{sequences})  that create and mutate objects. These sequences help generate target object states (in short as \emph{target states}) of the receiver or arguments of the method under test (MUT). As a real example for a target state, the \CodeIn{Compute} MUT is shown in Figure~\ref{fig:mut}. The MUT performs a depth-first search on an undirected graph. A target state for reaching Statement 8, 14, or 22 requires that a graph object used in test execution has a non-empty set of vertices and edges. 

\begin{figure}[t]
\begin{CodeOut}
\begin{alltt}
//UndirectedDFS: Short form of UndirectedDepthFirstSearch
00:class UndirectedDFS \{
01:\hspace*{0.1in}IVertexAndEdgeListGraph VisitedGraph; ...
02:\hspace*{0.1in}public UndirectedDFS(IVertexAndEdgeListGraph g) \{ 
03:\hspace*{0.2in}... 
04:\hspace*{0.1in}\}
05:\hspace*{0.1in}public void Compute(IVertex s) \{
06:\hspace*{0.2in}//init vertices
07:\hspace*{0.2in}foreach(IVertex u in VisitedGraph.Vertices) \{
08:\hspace*{0.3in}\emph{Colors[u]=GraphColor.White;}
09:\hspace*{0.3in}if (InitializeVertex != null)
10:\hspace*{0.3in}InitializeVertex(this, new VertexEventArgs(u));
11:\hspace*{0.2in}\}
12:\hspace*{0.2in}//init edges
13:\hspace*{0.2in}foreach(IEdge e in VisitedGraph.Edges) \{
14:\hspace*{0.3in}\emph{EdgeColors[e]=GraphColor.White;} \}
15:\hspace*{0.2in}//use start vertex			
16:\hspace*{0.2in}if (s != null) \{
17:\hspace*{0.3in}if (StartVertex != null)
18:\hspace*{0.4in}StartVertex(this,new VertexEventArgs(s));
19:\hspace*{0.3in}Visit(s); \}
20:\hspace*{0.2in}// visit vertices
21:\hspace*{0.2in}foreach(IVertex v in VisitedGraph.Vertices) \{
22:\hspace*{0.3in}\emph{if (Colors[v] == GraphColor.White)} \{
23:\hspace*{0.4in}if (StartVertex != null)
24:\hspace*{0.5in}StartVertex(this,new VertexEventArgs(v));
25:\hspace*{0.4in}Visit(v); \}
26:\hspace*{0.2in}\}
27:\hspace*{0.1in}\}
28:\}
\end{alltt}
\end{CodeOut} \vspace*{-4ex}
\Caption{\label{fig:mut} A method under test from the QuickGraph library~\cite{QUICKGRAPH}.} \vspace*{-5ex}
\end{figure}

To generate target states, there exist four major categories of sequence-generation approaches: bounded-exhaustive~\cite{khurshid:symbolic, xie:rostra}, evolutionary~\cite{tonella:etoc,  inkumsah08:improving},  random~\cite{csallner:jcrasher, JTEST, pacheco:feedback}, and heuristic approaches~\cite{tillman:pexwhite}. Bounded-exhaustive approaches generate sequences exhaustively up to a small bound of sequence length. However, generating target states often requires longer sequences beyond the small bound that can be handled by the bounded-exhaustive approaches. On the other hand, evolutionary approaches accept an initial set of sequences and evolve those sequences to produce new sequences that can generate target states. These approaches use \emph{fitness}~\cite{Xiyang:fitness}, a metric computed toward reaching a target state, as a guidance for producing new sequences. However, the generation is still a random process and shares the same characteristics as random approaches discussed next. 

Random approaches use a random mechanism to combine individual method calls to generate sequences. Although random approaches are shown as effective as systematic approaches theoretically~\cite{random:duran}, random approaches still face challenges in practice to generate sequences for achieving target states. The reason is that there is often a low probability of generating required sequences at random for achieving target states. To illustrate the challenges faced by random approaches, we applied a state-of-the-art random approach, called Randoop~\cite{pacheco:feedback}, on the MUT shown in Figure~\ref{fig:mut}. Randoop achieved branch coverage of 31.8\% (7 out of 22 branches). The reason for low coverage is that the random mechanism of Randoop is not able to generate a graph with vertices and edges. 

A heuristic approach for automatic sequence generation is used by a recent approach based on dynamic symbolic execution (DSE)~\cite{king:symex, Clarke:symbolic, godefroid:dart, koushik:cute}, called Pex~\cite{tillman:pexwhite}. Initially, Pex identifies constructors and other methods of a class under test that set values to different fields, hopefully helping generate desirable target state. Using the identified constructors and methods, Pex generates method-call-sequence skeletons (in short as \emph{skeletons}), which are basically method-call sequences with symbolic values for primitive types. These skeletons can be considered as a general form of sequences, where symbolic values are used instead of constant values for primitive types. Pex computes concrete values for the symbolic values in skeletons based on constraints collected from branch statements in the code under test. Based on our experience of applying Pex on industrial code bases, we observed that many branches in the code under test are not covered due to lack of proper skeletons. For example, Pex achieved branch coverage of 45.5\% (10 out of 22 branches) on the MUT shown in Figure~\ref{fig:mut}. The reason for low coverage is that Pex is also not able to generate a graph with vertices and edges.

A common characteristic among these previous approaches for generating sequences is that these approaches generate sequences either randomly or based on implementation informations of method calls used in a sequence. As these existing approaches are not effective in generating sequences, in this paper, we propose a novel approach called $\smoot$. $\smoot$ addresses this significant problem of sequence generation from a novel perspective of how these method calls are used together in practice. More specifically, using information of how the method calls are used in practice helps generate sequences that can achieve target states. To gather such usage information of method calls, our $\smoot$ approach mines code bases that are already using the object types such as receiver or argument object types of the MUT. For a MUT, these code bases include source code of the application that the MUT belongs to and test code for that application. In addition, code bases also include other applications using the receiver or argument object types of the MUT available in both proprietary and open source code (available on the web). 

To the best of our knowledge, our approach is the first one that addresses the significant problem of sequence generation by leveraging the information of how method calls are used in practice. Our approach mines code bases to extract sequences related to receiver or argument object types of a MUT. Our approach uses extracted sequences to assist both random and DSE-based approaches in achieving higher structural coverage. More specifically, our approach addresses three major issues in extracting sequences from code bases. First, code bases are often large and complete analysis of these code bases can be prohibitively expensive. To address this issue, our approach searches for code portions relevant to receiver or argument object types of the MUT and analyzes those relevant code portions only. Second, constant values in sequences extracted from code bases can be different from values required to achieve target states. To address this issue, our approach converts extracted sequences into skeletons by replacing constant values for primitive types with symbolic values. We refer to this process of converting sequences into skeletons as \emph{sequence generalization}. Third, extracted sequences individually may not be useful in achieving target states. Our approach addresses this issue by combining extracted sequences randomly to generate new sequences that may produce target states.
\Comment{
In this paper, we propose a novel approach, called $\smoot$, that mines existing code bases and automatically extracts method-call sequences from these code bases. Our approach uses these method-call sequences to assist test generation approaches such as random approaches in generating desirable method-call sequences for achieving the $\theta$ state. Our approach further generalizes these sequences as method-call sequence skeletons to assist more systematic approaches such as DSE-based approaches. Method-call sequences extracted from code bases can often be helpful in achieving high structural coverage such as branch coverage for two main reasons. First, developers write code such as the MUT in Figure~\ref{fig:mut} in a way that the MUT is invoked by some other code available in code bases. Therefore, some complex branches (that cannot be covered by existing automatically generated method-call sequences) can likely be covered by the method-call sequences collected from these code bases. Second, collecting method-call sequences from these code bases can help test real scenarios of how the MUT is used, providing more chances of finding defects that can incur failures in real scenarios. 

Often, method-call sequences extracted from existing code bases are not directly useful for assisting random or DSE-based approaches. To address this issue, our $\smoot$ approach includes techniques that can efficiently handle large existing code bases and extract method-call sequences suitable for assisting these approaches. $\smoot$ generalizes extracted method-call sequences so that DSE-based approaches can use those sequences as method-call skeletons. $\smoot$ also allows combinations of these method-call sequences to generate new method-call sequences from the extracted sequences. 
}

%In particular, our approach accepts a set of target object types such as the receiver or argument object types of the MUT and mines code bases using those target object types to gather method-call sequences. Our approach addresses three major challenges. (1) Often these code bases are large and analyzing complete code bases can be prohibitively expensive. Therefore, we search for relevant method definitions in these code bases using a text-based keyword search and analyze \emph{only} those relevant method declarations. (2) Furthermore, method-call sequences collected from code bases can often be partial due to missing values for non-primitive or primitive variables. These partial sequences cannot help generate compilable code from collected sequences. We address the issue of missing non-primitive values by collecting new method-call sequences that can produce those non-primitive values, making collected sequences complete. We address the issue of missing primitive values by replacing those primitive variables with symbolic values. (3) Sometimes, constant values in method-call sequences cannot help generate the $\theta$ state. To address the preceding issue, we further generalize the sequences by replacing those constant values with symbolic values, and then use dynamic symbolic execution to generate concrete values based on the constraints collected from the branch points in the MUT.

In summary, this paper makes the following major contributions:
\begin{itemize}
\item The first approach that leverages the information of how method calls are used in practice to address the significant problem of sequence generation in object-oriented unit testing.
\item A technique that mines large code bases by searching for code portions relevant to receiver or argument object types of a MUT. Our approach analyzes \emph{only} those relevant code portions since analyzing complete code bases can be prohibitively expensive. 
%Our approach uses extracted sequences to assist test generation approaches such as random and DSE-based test-generation approaches.
%\item A technique for analyzing large code bases by searching for code portions relevant to receiver or argument object types of a MUT. 
\item A technique for generalizing extracted sequences (i.e., converting sequences into skeletons) to assist DSE-based approaches. Generalization helps address issues where constant values in extracted sequences are different from values required to achieve target states.
\item A technique for generating new sequences by randomly combining extracted sequences. These new sequences try to address the issue where extracted sequences individually are not sufficient to achieve target states.
\item An implementation of the approach and its evaluation upon two state-of-the-art industrial testing tools: Randoop and Pex. Both Pex and Randoop were shown to find serious defects in industrial code bases~\cite{pacheco:feedback, tillman:pexwhite}. Our approach represents a significant, successful step towards addressing complex testing problems in industrial practice, targeting at complex desirable sequences from multiple classes rather than sequences on single classes such as data structures heavily focused by previous approaches~\cite{tonella:etoc,  inkumsah08:improving}.
%Pex is a popular tool with thousands of downloads\footnote{\url{http://research.microsoft.com/pex}}. Moreover, parts of Pex may be integrated into a future version of Microsoft Visual Studio 2010, benefiting an enormous number of developers in industry. Although the success of the current Pex and Randoop has been well witnessed, our intensive experience on applying Pex on various industrial code bases has led us to observe that a significant portion of industrial code bases was not covered due to the current limitation of sequence generation. Such a strong demand from industrial testing practice forms a major motivation of our approach. 
\item Empirical results from two evaluations show that our approach can effectively assist state-of-the-art random and DSE-based approaches in achieving higher branch coverage. Using our approach, we show that a random approach achieves 8.7\% (with a maximum of 20.0\% for one namespace) higher branch coverage and a DSE-based approach achieves 17.4\% (with a maximum of 22.5\% for one namespace) higher branch coverage than without using our approach. Such an improvement is significant since the branches that are not covered by these state-of-the-art approaches are generally quite difficult to cover.
\end{itemize}

The rest of the paper is structured as follows:
Section~\ref{sec:background} presents background on a random and a DSE-based approaches.
Section~\ref{sec:example} explains our approach with an example.
Section~\ref{sec:approach} describes key aspects of our approach.
Section~\ref{sec:eval} presents our evaluation results.
Section~\ref{sec:threats} discusses threats to validity.
Section~\ref{sec:future} discusses limitations of our approach and future work.
Section~\ref{sec:related} presents related work.
Finally, Section~\ref{sec:concl} concludes. 

