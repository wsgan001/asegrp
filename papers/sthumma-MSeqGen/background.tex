\section{Background}
\label{sec:background}

In this paper, we use two existing approaches Randoop~\cite{pacheco:feedback} and Pex~\cite{tillman:pexwhite} as example state-of-the-art approaches for random and DSE-based approaches, respectively. We next briefly describe these two example approaches.

%----------------------------------------------------------------------------------------------------------------
\subsection{Randoop}

Randoop~\cite{pacheco:feedback} is a random testing approach that constructs test inputs (in the form of sequences) incrementally by randomly selecting method calls. For each such randomly selected method call, Randoop finds arguments from previously-constructed inputs or tries to generate new sequences for those arguments. These constructed sequences are considered as test inputs. Unlike a pure random approach, Randoop incorporates feedback obtained from previously constructed test inputs while generating new test inputs. As soon as a test input is constructed, Randoop executes the test input and verifies the output against a set of contracts and filters. 

%----------------------------------------------------------------------------------------------------------------
\subsection{Pex}
\label{sec:pex}

Pex~\cite{tillman:pexwhite} is an automatic unit-test-generation tool from Microsoft. Pex accepts parameterized unit tests (PUT)~\cite{tillmann05:parameterized} as input; PUTs are a new advancement in unit testing and these PUTs accept parameters unlike conventional unit tests, which do not accept any parameters. From these PUTs, Pex generates conventional unit tests that can achieve high structural coverage of the code under test. More specifically, Pex is a DSE-based approach~\cite{godefroid:dart} that initially executes the code under test with arbitrary inputs. During execution, Pex collects symbolic constraints on inputs obtained from predicates in branch statements along the execution. Pex uses a constraint solver to compute variations of the previous inputs to guide future program executions along different paths. As PUTs also accept non-primitive types as arguments, Pex uses a combination of static and dynamic analysis techniques for generating sequences for those non-primitive types. Using static analysis, Pex determines possible constructors and other methods of a class that set values to different fields of that class. Based on constraints collected during execution of the code under test, Pex determines which methods can set values for required fields and tries to cover target branches by combining constructors and method calls.