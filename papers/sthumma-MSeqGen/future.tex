\section{Discussion and Future Work}
\label{sec:future}

Although random and DSE-based approaches show considerable increase in branch coverage with the assistance from our approach, overall coverage achieved by these approaches are still not close to 100\% coverage. The reason is that often code under test includes complex branches that are quite difficult to cover. We next give an example of a difficult branch that is not covered by any of the approaches used in our evaluation. We use the code example shown in Figure~\ref{fig:diffbranch} as an illustrative example. This difficult branch is in the \CodeIn{Visit} method of the \CodeIn{BreadthFirstSearchAlgorithm} class. The receiver-object state to reach Statement 8 requires that the \CodeIn{VisitedGraph} object has a non-empty set of vertices and edges. Reaching Statement 8 also requires a specific object state for the argument \CodeIn{s}. In particular, the vertex represented by the argument \CodeIn{s} should already exist in the \CodeIn{VisitedGraph} object and should have outgoing edges. Although our extracted sequences include a sequence for achieving a desirable receiver-object state, our sequences do not include a necessary sequence for achieving a desirable argument-object state. In future work, we plan to further address these issues by generating new sequences using evolutionary approaches~\cite{tonella:etoc,  inkumsah08:improving}. There, we can use our extracted sequences as an initial set for these evolutionary approaches to evolve. Generation of new sequences using evolutionary approaches can also help reduce the bias in our approach, where our approach gives more preference to verify common usage rather than uncommon usage.

\begin{figure}[t]
\begin{CodeOut}
\begin{alltt}
00:public void Visit(IVertex s) \{
01:\hspace*{0.1in}...
02:\hspace*{0.1in}m\_Q.Push(s);
03:\hspace*{0.1in}while (m\_Q.Count != 0) \{
04:\hspace*{0.2in}IVertex u = (IVertex)m\_Q.Peek(); 
05:\hspace*{0.2in}m\_Q.Pop();
06:\hspace*{0.2in}...
07:\hspace*{0.2in}foreach(IEdge e in VisitedGraph.OutEdges(u)) \{ 
08:\hspace*{0.3in}...		//Difficult branch
09:\hspace*{0.2in}\} 
10:\hspace*{0.1in}\}
11:\}
\end{alltt}
\end{CodeOut}\vspace*{-5ex}
\Caption{\label{fig:diffbranch} An example difficult branch not reached by any approach used in our evaluation.}\vspace*{-5ex}
\end{figure}

In our evaluations, for the \CodeIn{facebook} and \CodeIn{facebook.Utility} namespaces, branch coverage achieved by Randoop (with the assistance of our approach) is lower than branch coverage achieved by the test code (commonly written by  application developers). There are two primary reasons for lower coverage of these namespaces: limitations of the random mechanism of Randoop and our current implementation. Our current implementation does not handle several features such as inheritance or C\# generics. Therefore, our implementation could not capture some sequences due to their use of these features. In future work, we plan to extend our implementation to support these features. 

%In our current approach, gathering sequences is loosely coupled with dynamic symbolic execution. For example, we identify the target classes and gather method-call sequences from code bases. We next verify whether these sequences can help achieve the $\theta$ states. Therefore, some collected sequences can be irrelevant. In future work, we plan to identify a desirable target (such as a branch) that is not achieved by the dynamic symbolic execution and use that information to gather sequences. This additional information can help gather more relevant sequences. 

Our approach extracts sequences from code bases using the receiver or argument object types of a MUT (in a framework under analysis) and generates method bodies to assist test-generation approaches. Sometimes, these sequences may include object types specific to the code bases. For example, these object types can be classes that implement interfaces provided by the framework under analysis. In such scenarios, the method bodies generated by our approach are not compilable. Currently, we fix those compilation errors manually. In future work, we plan to automatically compile and verify extracted sequences to reduce this manual effort.

\Comment{We currently gather \emph{only} one path from a CFG to generate sequences. However, there can be different sequences
across multiple paths in the CFG. In future work, we plan to collect sequences from multiple paths in the CFG. We also plan to develop clustering heuristics to cluster similar sequences.}

