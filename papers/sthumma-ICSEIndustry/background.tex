\section{Background}
\label{sec:background}

TODO: Needs to re-write the sentences.
In this paper, we use Pex as a dynamic symbolic execution tool. Pex~\cite{tillman:pexwhite} 
is an automatic unit-test-generation tool from Microsoft. Pex accepts 
parameterized unit tests (PUT)~\cite{tillmann05:parameterized} as input; 
PUTs are a new advancement in unit testing and these PUTs accept parameters 
unlike conventional unit tests, which do not accept any parameters. 
From these PUTs, Pex generates conventional unit tests that can achieve high 
structural coverage of the code under test. More specifically, Pex is a 
DSE-based approach~\cite{godefroid:dart} that initially executes the code 
under test with arbitrary inputs. During execution, Pex collects symbolic 
constraints on inputs obtained from predicates in branch statements along 
the execution. Pex uses a constraint solver to compute variations of the 
previous inputs to guide future program executions along different paths. 
As PUTs also accept non-primitive types as arguments, Pex uses a combination 
of static and dynamic analysis techniques for generating sequences 
for those non-primitive types. Using static analysis, Pex determines 
possible constructors and other methods of a class that set values 
to different fields of that class. Based on constraints collected 
during execution of the code under test, Pex determines which 
methods can set values for required fields and tries to cover target 
branches by combining constructors and method calls.

TODO: Dynamic code coverage needs to be explained here and why we
use this criteria?