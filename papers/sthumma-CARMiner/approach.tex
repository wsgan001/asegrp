\section{Approach}
\label{sec:approach}
\vspace*{-2ex}
Our CAR-Miner approach accepts an application under analysis and mines exception-handling rules
for all function calls in the application. CAR-Miner detects violations of the mined exception-handling
rules. We next present the details of each phase in our approach.

%--------------------------------------------------------------------------------------------
\subsection{Input Application Analysis}
\label{sec:gatherce}
\vspace*{-2ex}
CAR-Miner accepts an application under analysis and parses the application to collect
each function call, say $FC_a$, in the application from the call sites in the application. 
For example, CAR-Miner collects
the function call \CodeIn{Statement.executeUpdate} as an $FC_a$ from Line 1.9 
in Scenario 1. We denote the set of all function calls as $FCS$. CAR-Miner
mines exception-handling rules for all these function calls.

%--------------------------------------------------------------------------------------------
\vspace*{-1.5ex}
\subsection{Code-Sample Collection}
\label{sec:gatherce}
\vspace*{-2ex}
To mine exception-handling rules for the function call $FC_a$, we need code samples
that already reuse the function. To collect such relevant code samples,
we interact with a code search engine (CSE) such as Google code search~\cite{GCSE} and download
code samples returned by the CSE. For example, we construct
the query ``\CodeIn{lang:java java.sql.Statement executeUpdate}'' to collect
code samples of the $FC_a$ \CodeIn{Statement.executeUpdate}. Often code samples
gathered from a CSE are partial as the CSE returns individual source
files instead of complete projects. We use partial-program analysis 
developed in our previous approach~\cite{thummalapenta07:parseweb} to resolve
object types such as receiver or argument types of function calls in code samples. 
More details of our partial-program analysis are available in our previous paper~\cite{thummalapenta07:parseweb}.
As we collect relevant code samples from other open source projects that already reuse a function, our
approach has an advantage of being able to detect additional rules that do not have enough
supporting samples in the application under analysis.
%-----------------------------------------------------------------
\vspace*{-1ex}
\subsection{Exception-Flow-Graph Construction}
\vspace*{-2ex}
We next analyze the collected code samples and the application to generate traces in the form
of sequence of function calls. Initially, we construct Exception-Flow Graphs (EFG),
which are an extended form of Control-Flow Graphs (CFG).
An EFG provides a graphical representation of all paths
that might be traversed during the execution of a program, including exception paths. 
Construction of an EFG is non-trivial due to the existence of additional paths
that transfer control to exception-handling blocks 
defined in the form of \CodeIn{catch} or \CodeIn{finally} in Java.
We develop an algorithm inspired by Sinha and Harrold~\cite{exception:sinha}
for constructing EFGs with additional paths that describe
exception conditions. Figure~\ref{fig:approchex}a shows the constructed
EFG for Scenario 2, where each node is denoted with the corresponding 
line number of Scenario 2 in Figure~\ref{fig:threescenarios}.

Initially, we build a CFG that represents flow of control during
normal execution and augment the constructed CFG with
additional edges that represent flow of control after exceptions occur. 
We refer to these additional edges as \emph{exception} edges
and all other edges as \emph{normal} edges. In the figure, 
\emph{normal} and \emph{exception} edges are shown in solid and
dotted lines, respectively. For example, an exception
edge is added from Node 5 to Node 13 as the program can follow
this path when \CodeIn{IOException} occurs while creating a \CodeIn{BufferedWriter}
object. As code inside a \CodeIn{catch} or 
a \CodeIn{finally} block gets executed after exceptions occur,
we consider edges between the statements within \CodeIn{catch} and \CodeIn{finally} blocks also
as exception edges. We show nodes related to function calls in normal
paths such as those in a \CodeIn{try} block in white and function calls in exception paths
such as those in a \CodeIn{catch} block in grey. Although function calls in a \CodeIn{finally}
block belong to both normal and exception paths, we consider these paths as exception paths
and show the associated nodes in grey. For simplicity, we ignore the control flow inside
exception blocks.

In the constructed EFG, there is an exception edge from Node 5 to Node 13,
but there is no exception edge from Node 6 to Node 13. The reason is that Node 13
handles a checked exception \CodeIn{IOException}, which is never raised by
function call \CodeIn{DriverManager.getConnection} of Node 6. 
Therefore, we prevent such infeasible control flow through a sound
static analysis tool, called Jex~\cite{Jex:Robillard}.
Jex analyzes source code statically and provides possible exceptions raised by each function call. 
For example, Jex provides that \CodeIn{IOException} can be raised by
\CodeIn{BufferedWriter.Constructor} but not \CodeIn{DriverManager.getConnection}.
While adding \emph{exception} edges, we add only those edges
from a function call to a \CodeIn{catch} block where 
the exception handled by the \CodeIn{catch} block
belongs to the set of possible exceptions thrown by the function call. 
This additional check helps reduce potential false positives by preventing
infeasible exception paths. If the \CodeIn{catch} block
handles \CodeIn{Exception} (the super class of all exception types), we add 
exception edges from each function call to the \CodeIn{catch} block.
We consider a \CodeIn{finally} block as similar to 
a \CodeIn{catch} block that handles \CodeIn{Exception}, and 
add exception edges from each function call to the \CodeIn{finally} block.

As gathered code samples are partial, we use intra-procedural 
analysis for constructing EFGs. Furthermore, before constructing an EFG
for a code sample, we also check whether the code sample includes
any $FC_a$ $\in$ $FCS$. If the code sample does not include any $FC_a$,
we skip the EFG construction for that code sample.
\vspace*{-2ex}
%-----------------------------------------------------------------
\subsection{Static Trace Generation}
\vspace*{-2ex}

We next capture static traces that include
actions that should be taken when exceptions occur while executing
function calls such as $FC_a$ $\in$ $FCS$. For example, consider
the $FC_a$ ``\CodeIn{Connection.createStatement}'' and its corresponding
Node 7 in the EFG. A trace generated for this node is shown in Figure~\ref{fig:approchex}b.
The trace includes three sections: \emph{normal function-call
sequence} ($FC_c^1$...$FC_c^n$), $FC_a$, \emph{exception function-call sequence} ($FC_e^1$...$FC_e^m$).

The $FC_c^1$...$FC_c^n$ sequence starts from the beginning of the 
body of the enclosing function (i.e., caller) of the $FC_a$ function call
to the call site of $FC_a$. The $FC_e^1$...$FC_e^m$ sequence 
includes the longest exception path that starts from the call site of $FC_a$
and terminates either at the end of the enclosing function body 
or at a node in EFG whose outgoing edges are all normal edges. We generate such traces 
from code samples and input application for each $FC_a$ $\in$ $FCS$.

%-----------------------------------------------------------------
\subsection{Trace Post-Processing}
\vspace*{-2ex}

We next identify function calls in $FC_c^1$...$FC_c^n$ or $FC_e^1$...$FC_e^m$
that are not related to $FC_a$ through data-dependency, and 
remove such function calls from each trace. Failing to remove
such unrelated function calls can result in many false positives 
due to frequent occurrences of unrelated function 
calls as shown in the evaluation of PR-Miner~\cite{Zhenmin2005PRMiner}. 
For example, in the trace shown in Figure~\ref{fig:approchex}b,
function calls in the normal function-call sequence related to Nodes 4 and 5
are unrelated to the $FC_a$ of Node 7. Similarly, Node 17 in the exception function-call 
sequence is also unrelated to $FC_a$. 

Figure~\ref{fig:approchex}c shows an example of our data-dependency analysis.
Initially, we generate two kinds of relationships: var dependency of a variable 
and function association of a function call. 
The var dependency of a variable represents the set of variables on which a given variable is dependent upon.
Similarly, a function association of a function call represents the set of variables on which 
a function call is associated with. 

First, we compute the var-dependency relationship information from assignment statements.
For example, in Scenario 2, we identify that the variable \CodeIn{res} is dependent on 
the variable \CodeIn{stmt} from Line 2.8 and is transitively dependent on \CodeIn{conn}
as \CodeIn{stmt} is dependent on \CodeIn{conn} from Line 2.7. 
We compute the function-association relationship based on the var-dependency relationship.
In particular, we identify that a function call is associated with all its variables including the
receiver, arguments, and the return variable, and their transitively dependent variables.
For example, applying the preceding analysis to the function call of Node 7, 
we identify that the associated variables are \CodeIn{conn} and \CodeIn{stmt}.

We use variables associated with each function call to 
identify function calls in the normal function-call sequence $FC_c^1$...$FC_c^n$ or 
the exception function-call sequence $FC_e^1$...$FC_e^m$
that are not related to $FC_a$. 
Starting from $FC_a$, we perform a backward traversal of the trace to filter out function calls
in $FC_c^1$...$FC_c^n$ and a forward traversal to filter out function calls
in $FC_e^1$...$FC_e^m$. 
Assume that variables associated with $FC_a$
are \{$V_a^1$, $V_a^2$,$\ldots$, $V_a^s$\}. Assume that variables associated
with a function call, say $FC_{ce}^k$, in the normal or exception function-call
sequence are \{$V_{ce}^1$, $V_{ce}^2$,$\ldots$, $V_{ce}^t$\}.

In each traversal, we compute an intersection of associated variable sets
of $FC_a$ and $FC_{ce}^k$. If the intersection \{$V_a^1$, $V_a^2$,$\ldots$, $V_a^s$\} $\cap$
\{$V_{ce}^1$, $V_{ce}^2$,$\ldots$, $V_{ce}^t$\} $\neq$ $\phi$, 
we keep the $FC_{ce}^k$ function call (either in the normal or exception function-
call sequence) in the trace; otherwise, we filter out the $FC_{ce}^k$ function call
from the trace. The rationale behind our analysis is that if the intersection
is a non-empty set, it indicates that the $FC_a$ 
is directly or indirectly related to the $FC_{ce}^k$ function call. For example,
the intersection of associated variables for Nodes 6 and 7 is non-empty.
In contrast, the intersection of associated variables for Nodes 5 and 7 is 
empty. Therefore, we keep Node 6 in the trace and filter out Node 5 during backward traversal.
Similarly, during forward traversal, we ignore Node 17 
since the intersection is an empty set. The resulting trace
of ``4,5,6,7,15,16,17'' is ``6,7,15,16'', where

\begin{CodeOut}
\begin{alltt}
\hspace*{0.2in}6 : DriverManager.getConnection
\hspace*{0.2in}7 : Connection.createStatement 
\hspace*{0.2in}15 : Statement.close
\hspace*{0.2in}16 : Connection.close
\end{alltt}
\end{CodeOut}

%-----------------------------------------------------------------
\subsection{Static Trace Mining}
\vspace*{-2ex}
We apply our new mining algorithm described in Section~\ref{sec:condrules} on the set of 
static traces collected for each $FC_a$. We apply mining on the
traces of each $FC_a$ individually. The reason is that if we apply
mining on all traces together, rules related to a $FC_a$
with only a small number of traces can be missed due to rules related to other $FC_a$
with a large number of traces. 

In the phase of static trace mining, we first transform traces suitable for
our mining algorithm. More specifically, as each trace includes 
a normal function-call sequence and an exception function-call sequence,
we build two sequence databases with normal and exception function-call sequences,
respectively, from all the traces of a $FC_a$ function call. 

We next apply our mining algorithm that initially annotates corresponding normal
and exception function-call sequences and combines the annotated 
sequences into a single call sequence. The mining
algorithm produces sequence association rules of the form $FC_c^1$...$FC_c^n$ $\Rightarrow$
$FC_e^1$...$FC_e^m$. As this sequence association rule is specific
to $FC_a$, we add $FC_a$ to the rule as 
($FC_c^1$...$FC_c^n$) $\wedge$ $FC_a$ $\Rightarrow$ ($FC_e^1$...$FC_e^m$).
The preceding sequence association rule describes that
the function call $FC_a$ should be followed by $FC_e^1$...$FC_e^m$
in exception paths only when preceded by $FC_c^1$...$FC_c^n$ in normal paths.
In our approach, we use the frequent closed subsequence mining tool, called
BIDE, developed by Wang and Han~\cite{wang:bide}.
We used the \emph{min\_sup} value as $0.4$, which is set based on our initial empirical experience. 
We repeat the preceding process for each $FC_a$ and rank all final sequence
association rules based on their support values assigned by the frequent subsequence 
miner.


\Comment{
%-----------------------------------------------------------------
\subsubsection{Frequent Itemset Miner}
Let \emph{I} = \{$i_{1}$, $i_{2}$, ..., $i_{n}$\} be the set of all items in 
a given database. Let \emph{min\_sup} is the given user-specified minimum
support threshold. Any sub-set of the given set is referred as an \emph{itemset}.
The support for an \emph{itemset} X, say \emph{sup(X)}, is defined as the 
number of items \emph{Y} in the given database, such that X $\subseteq$ Y.
An \emph{itemset X} is known as frequent is \emph{sup(X)} $\geq$ \emph{min\_sup}.
The \emph{itemset X} is referred as closed frequent itemset if X is frequent and
no superset of X is frequent.

In our approach, the frequent itemset miner gives the frequent sets of function calls 
that often appear in the exception paths. However, as the frequent itemset miner discards
the order of function calls in the traces given as input, some of these frequent itemsets
may be invalid. The primary reason is that each identified static trace starts with
a normal function call $N_i$ followed by method-invocations $E_i$ that appear in the
exception paths. As the itemset miner discards the order, the resulting frequent itemsets
might not contain the $N_i$ associated with the original trace. Therefore, our approach
initially stores all $N_i$ and classifies the frequent itemsets that do not contain
any $N_i$ as invalid.
%-----------------------------------------------------------------
\subsubsection{Frequent Subsequence Miner}
%-----------------------------------------------------------------
\subsubsection{Association Rule Miner}
An association rule represents the association among
data between two different groups. Given two subsets \emph{X} and \emph{Y} of set \emph{I} 
(\emph{X} $\subset$ \emph{I}, \emph{Y} $\subset$ \emph{I}, and \emph{X} $\cap$ \emph{Y} = $\phi$),
an association rule is expressed as \emph{X} $\rightarrow$ \emph{Y} (\emph{c}\%), where \emph{c}
gives the confidence of the rule. The confidence of the association rule \emph{X} $\rightarrow$ \emph{Y} 
is defined as the ratio of transactions including both \emph{X} and \emph{Y} to all the 
transactions that include \emph{X}. The \emph{sup(X)} is defined as the proportion 
of the transactions accessing \emph{X} to the whole transactions. An association rule 
is defined as frequent, if \emph{sup(X)} $\geq$ \emph{min\_sup}. 
}

%---------------------------------------------------------------------
\subsection{Anomaly Detection}

To show the usefulness of our mined exception-handling rules, we apply these
rules on the application under analysis to detect violations. Initially, from each
call site of $FC_a$ in the application, we extract the normal function-call 
sequence, say $C_c^1$$C_c^2$$\ldots$$C_c^a$, from the beginning of 
the body of enclosing function of $FC_a$ to the call site of $FC_a$.
If $FC_c^1$$\dots$$FC_c^n$ $\sqsubseteq$ $C_c^1$$C_c^2$$\ldots$$C_c^a$,
then we extract the exception function-call sequence, say $C_e^1$$C_e^2$$\ldots$$C_e^b$,
from the call site of $FC_a$ to the end of the enclosing function body 
or to a node (in the EFG) whose outgoing edges are all normal edges.
We do not report a violation if $FC_e^1$$\ldots$$FC_e^m$ $\sqsubseteq$ $C_e^1$$C_e^2$$\ldots$$C_e^b$;
otherwise, we report a violation in the application under analysis.
We rank all detected violations based on 
a similar criterion used for ranking exception-handling rules. 

\Comment{We initially capture the context function-call sequence from the call-sites
of $FC_a$ in the input application and find whether the context matches with the context
of any exception-handling rules. If a context match is found,
we check whether the exception path of $FC_a$ includes all function calls
defined by that exception-handling rule. Any missing function call
in the exception function-call sequence is reported as a violation.}
