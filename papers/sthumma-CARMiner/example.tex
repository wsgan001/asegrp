\section{Motivating Example}
\label{sec:example}

We next present example scenarios where simple association
rules of the form ``$FC_a$ $\Rightarrow$ $FC_e$'' are not sufficient to 
characterize exception-handling rules. We use Scenarios 1 and 2 
in Figure~\ref{fig:threescenarios} as illustrative examples. We use
these two scenarios to explain other concepts in the rest of the paper.

Scenarios 1 and 2 show code examples where a \CodeIn{connection} resource is created through 
the function calls \CodeIn{OracleDataSource}.\CodeIn{getConnection} (Line 1.7)
and \CodeIn{DriverManager.getConnection} (Line 2.6), respectively. Scenario 1
attempts to modify the contents of the database through the function call
\CodeIn{Statement.executeUpdate} (Line 1.9), whereas Scenario 2 attempts to read contents
of the database through the function call \CodeIn{Statement.executeQuery} (Line 2.8).

From Scenario 1, an association rule ``\CodeIn{OracleDataSource}.\CodeIn{getConnection}
$\rightarrow$ \CodeIn{Connection.rollback}'' can be inferred.
This rule describes that when ever a connection object is created, 
the function call \CodeIn{rollback} (Line 1.12) should be invoked during exception conditions.
However, this \CodeIn{rollback} function call does not apply to Scenario 2, where there are no modifications
to the contents of the database. One naive approach to handle the preceding issue is to mine
association rule ``\CodeIn{Statement.executeUpdate} $\rightarrow$
\CodeIn{Connection.rollback}''. The rationale is that \CodeIn{Statement.executeUpdate}
is the primary reason to have \CodeIn{Connection.rollback} in the exception path
and the \CodeIn{statement} object is dependent on the \CodeIn{conn} object.
However, this naive approach also results in another association rule 
``\CodeIn{Connection.createStatement} $\rightarrow$ \CodeIn{OracleDataSource.close}''
from Scenario 1. But the preceding rule is not applicable to the 
Scenario 2. The primary reason is that the \CodeIn{connection}
resource is created through different function calls in Scenarios 1 and 2. 

To address these preceding issues, we mine sequence association rules of
the form ``$FC_a$ $\wedge$ ($FC_c^1$...$FC_c^n$) $\Rightarrow$ ($FC_e^1$...$FC_e^m$)''
rather than simple association rules. For example, we identify
a sequence association-rule candidate ``$FC_a$ $\wedge$ $FC_c^1$$FC_c^2$ $\Rightarrow$ $FC_e^1$$FC_e^2$
$FC_e^3$'' from Scenario 1, where

\begin{CodeOut}
$FC_a$ : Connection.createStatement \\
\hspace*{0.3in}$FC_c^1$ : OracleDataSource.Constructor \\
\hspace*{0.3in}$FC_c^2$ : OracleDataSource.getConnection \\
\hspace*{0.3in}$FC_e^1$ : Statement.close \\
\hspace*{0.3in}$FC_e^2$ : Connection.close \\
\hspace*{0.3in}$FC_e^3$ : OracleDataSource.close
\end{CodeOut}

The preceding rule candidate describes that \CodeIn{Connection.createStatement}
should be followed by function calls $FC_e^1$$FC_e^2$$FC_e^3$
in exception paths, only when the \CodeIn{Connection}
object is obtained from \CodeIn{OracleDataSource} through the function-call sequence
$FC_c^1$$FC_c^2$. The same rule does not apply to Scenario 2 as the context
function-call sequence is different in Scenario 2.

\Comment{
We next describe our CAR-Miner approach through the code example shown in Scenario 1
of Figure~\ref{fig:threescenarios}. Initially, CAR-Miner accepts a given input application and gathers
classes and methods from the call sites in the input application.
CAR-Miner interacts with a CSE such as Google code search~\cite{GCSE} 
and gathers related code examples for all classes. 
A code example gathered from Google code search is shown in Scenario 1 of Figure~\ref{fig:threescenarios}.
CAR-Miner analyzes both the input application and gathered code examples statically 
and builds control-flow graphs (CFG). As code examples gathered
from CSE are partial, CAR-Miner uses several heuristics for resolving object types. In contrast,
as the entire source code of the input application is available, CAR-Miner resolves
object types by directly looking up the method signatures. 

After type resolution, CAR-Miner constructs CFGs that represent flow of control 
during \emph{normal} and \emph{exception} cases. The constructed CFG for Scenario 1 is also shown
in Figure~\ref{fig:threescenarios}. We show normal and exception edges in solid
and dotted lines, respectively.	For example, the exception edge ``9 $\rightarrow$ 12''
shows that the method-invocation \CodeIn{Connection}.\CodeIn{rollback} (Line 12) should be executed
when \CodeIn{SQLException} occurs after executing \CodeIn{Statement}.\CodeIn{executeUpdate} (Line 9).
Our approach tries to reduce the number of false positives among extracted rule candidates
by preventing infeasible edges in the constructed CFG. For example,
there are no exception edges from Node 6 as our static analysis determines that the method invocation \CodeIn{setURL}
does not raise any exceptions. We extract several resource-exception models for each 
method invocation in the normal execution path such as 
\CodeIn{Statement.executeUpdate} (being treated as a resource-manipulation action, say \emph{RM}) from the constructed CFG. 
For example, the resource-exception model for the method invocation in Line 9 includes all incoming normal 
paths to Node 9 and all outgoing exception paths from Node 9 in the constructed CFG.
Figure~\ref{fig:resourcemanip} shows the resource-exception model for \CodeIn{Statement.executeUpdate}.

We next pair the \emph{RM} action with each method invocation in the exception paths
(treated as a resource-cleanup action, say \emph{RC}) and verify
whether there is an indirect data dependency between these two actions.
We verify the indirect data dependency by traversing the normal path 
in the reverse direction starting from the \emph{RM} node in the CFG and collect
a method invocation that shares the same receiver object of \emph{RC} and has 
a transitive data dependency with \emph{RM}.
We treat such method invocation as a resource-creation action, say \emph{RT},
and the common receiver object between \emph{RC} and \emph{RT} as a resource.
If such an \emph{RT} method invocation is found,
we create a rule candidate with \emph{RM} and \emph{RC}
actions. We associate the method-invocation sequence that describe the transitive data dependency as
a \emph{trigger} path (\emph{TP}) for the rule candidate. A rule candidate $<$\emph{RM}, \emph{RC}, \emph{TP}$>$
for Statements 9 and 12 is $<$\CodeIn{Statement.executeUpdate}, \CodeIn{Connection.rollback},
\CodeIn{Connection.createStatement}$>$. The rule candidate describes that the cleanup action 
\CodeIn{Connection.rollback} should be executed after exceptions occur while
executing the \CodeIn{Statement.executeUpdate} action. The preceding rule candidate is applicable
only when the \CodeIn{Statement} object is created through \CodeIn{Connection.createStatement}.

CAR-Miner mines extracted rule candidates to identify frequent rules, referred
as exception-handling rules. CAR-Miner uses mined exception-handling rules 
to detect violations in the given application. For example, CAR-Miner detects a violation
when \CodeIn{Connection.rollback} does not appear in an exception path of \CodeIn{Statement.executeUpdate},
where the \CodeIn{Statement} object is created through \CodeIn{Connection.createStatement}.

This example motivates the idea of exploiting a CSE for gathering related code examples 
and applying analysis techniques to compute resource-exception models. 
However, there are many other issues that are
not obvious in this illustrative example, as described next.
(1) How do we analyze partial code examples and build
a CFG that represents both normal and exception cases? (2) How do we 
compute data dependency between resource-manipulation and resource-cleanup actions? 
(3) What heuristics can we use for reducing
the number of false positives that can occur due to intra-procedural
analysis among detected violations? We address these issues in
the subsequent section, where we present key aspects of our CAR-Miner approach.}