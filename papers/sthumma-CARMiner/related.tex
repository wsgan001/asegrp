\section{Related Work}
\label{sec:related}
\vspace*{-1ex}
WN-miner by Weimer and Necula~\cite{WeimerN05} extracts simple association rules of the form
``$FC_a$ $\Rightarrow$ $FC_e$'', when $FC_e$ is found at least once in exception-handling
blocks (i.e., \CodeIn{catch} or \CodeIn{finally} blocks). 
Their approach mines and ranks these rules based on the number of times
$FC_e$ appears after $FC_a$ in normal paths. Due to their ranking criteria,
their approach cannot mine rules that include a $FC_e$ function call such
as \CodeIn{Connection.rollback}, where $FC_e$ can appear \emph{only} in exception
paths. Acharya and Xie~\cite{acharya07techreporterrorhandling} later
proposed a similar approach for detecting API error-handling defects
in C code. Our approach significantly differs and improves upon these previous approaches
as we mine sequence association rules of the form ``($FC_c^1$...$FC_c^n$) $\wedge$ $FC_a$ $\Rightarrow$ ($FC_e^1$...$FC_e^m$)'' that can characterize more exception-handling rules.
Our approach also addresses the problem of lacking enough supporting
samples for these rules in the application under analysis by expanding the data scope to open source repositories
through a code search engine. 

\Comment{Acharya and Xie~\cite{acharya07techreporterrorhandling} 
proposed an approach for detecting API error-handling bugs in C code by approximating
run-time API error behaviors through a push-down model checker. Their approach is
also based on a similar technique used by Weimer and Necula for mining rules. Therefore,
their approach also cannot detect the types of defects where $FC_e$ appears only in
exception paths. Furthermore, their approach is limited to certain types of defects, 
referred as critical defects, because their approach extracts rules from only those paths
that exit the entire application. Our approach can detect more-general 
types of defects in exception paths that violate mined rules.}

CodeWeb~\cite{Michail:mining} mines association rules
from source code as framework reuse patterns. 
CodeWeb mines association rules such as application classes inheriting
from a library class often create objects of another class.
PR-Miner~\cite{Zhenmin2005PRMiner} uses frequent itemset
mining to extract implicit programming rules in large C code bases and detects
violations. DynaMine~\cite{livshits05dynamine} 
uses association rule mining to extract simple rules from version
histories for Java code and detects rule violations. Engler et al.~\cite{Engler2001deviant} 
proposed a general approach for finding defects in C code by applying statistical
analysis to rank deviations from programmer beliefs inferred from source code.
Wasylkowski et al.~\cite{wasylkowski07:detecting} mines rules that include
pairs of API calls and detect violations. 
Perracotta~\cite{Yang06perracotta:mining} mines patterns such as $(ab)^*$ and 
includes techniques for handling imperfect traces.
Sch\"{a}fer et al.~\cite{thorsten:mining} mine 
association rules that describe usage changes in framework evolution.
All these preceding approaches mine simple association rules that are often not
sufficient to characterize complex real rules as shown in our approach. 
In contrast, our approach can mine more complex
rules in the form of sequence association rules.

Our approach is also related to other approaches that analyze exception behavior
of programs. Fu and Ryder~\cite{Fu:exception-chain} proposed an exception-flow analysis
that computes chains of semantically related exception-flow links across procedures.
Our approach uses intra-procedural analysis for constructing exception-flow graphs. 
The Jex~\cite{Jex:Robillard} tool statically analyzes exception flow
in Java code and provides a precise set of exceptions that can be raised by a
function call. We use Jex in our approach to prevent infeasible exception edges
in a constructed EFG. Fu et al.~\cite{Fu:web-service-testing} present a 
\emph{def-use}-based approach that helps gather error-recovery code-coverage information. 
Our approach is different from their approach as our approach detects defects that violate mined
rules rather than focusing on coverage of exception-handling code. 
\Comment{Robillard and Murphy~\cite{robillard:robust} developed an approach
to simplify the exception structure at the design level and thereby improve the robustness
and changeability of the program. In contrast, our approach targets at detecting
exception-handling defects at the implementation level.}

Chang et al.~\cite{chang07:finding} applies frequent subgraph mining on C code
to mine implicit condition rules and detect neglected conditions.
Their approach targets at different types of defects called neglected conditions.
Moreover, their approach does not scale to large code bases as graph mining algorithms
suffer from scalability issues. Finally, DeLine and F\"{a}hndrich~\cite{deline:high-level} proposed an approach 
that allows programmers to manually specify resource management protocols
that can be statically enforced by a compiler. However, their approach requires manual
effort from programmers and also requires the knowledge of the \emph{Vault} specification
language to specify domain-specific protocols. In contrast, our approach 
does not require any manual effort or the knowledge of any specific specification languages.

Javert~\cite{gabel:javert} uses a pattern-based specification
miner to mine smaller patterns such as $(ab)^*$, called \emph{micro patterns},
and then compose these patterns into larger specifications. Their approach does not
require the user to provide any templates. Similar to their approach, our approach
also does not require the user to provide any templates. 
However, their mined patterns cannot characterize exception-handling rules
mined by our approach.

Our previous approaches PARSEWeb~\cite{thummalapenta07:parseweb} 
and SpotWeb~\cite{thummalapenta08:spotweb} also exploit code search engines for gathering related code samples. 
PARSEWeb accepts queries of the form ``\emph{Source} $\rightarrow$ \emph{Destination}'' 
and mines frequent function-call sequences that accept
\emph{Source} and produce \emph{Destination}. SpotWeb accepts an input framework
and detects hotspot classes and functions of the framework. Our new approach CAR-Miner
significantly differs from these previous approaches.
CAR-Miner constructs EFGs and includes new techniques for
collecting and post-processing static traces related to exception handling. Furthermore, CAR-Miner 
incorporates our new mining algorithm for detecting exception-handling rules
as sequence association rules.

 