
% This is "sig-alternate.tex" V1.8 June 2007
% This file should be compiled with V2.3 of "sig-alternate.cls" June 2007
%
% This example file demonstrates the use of the 'sig-alternate.cls'
% V2.3 LaTeX2e document class file. It is for those submitting
% articles to ACM Conference Proceedings WHO DO NOT WISH TO
% STRICTLY ADHERE TO THE SIGS (PUBS-BOARD-ENDORSED) STYLE.
% The 'sig-alternate.cls' file will produce a similar-looking,
% albeit, 'tighter' paper resulting in, invariably, fewer pages.
%
% ----------------------------------------------------------------------------------------------------------------
% This .tex file (and associated .cls V2.3) produces:
%       1) The Permission Statement
%       2) The Conference (location) Info information
%       3) The Copyright Line with ACM data
%       4) NO page numbers
%
% as against the acm_proc_article-sp.cls file which
% DOES NOT produce 1) thru' 3) above.
%
% Using 'sig-alternate.cls' you have control, however, from within
% the source .tex file, over both the CopyrightYear
% (defaulted to 200X) and the ACM Copyright Data
% (defaulted to X-XXXXX-XX-X/XX/XX).
% e.g.
% \CopyrightYear{2007} will cause 2007 to appear in the copyright line.
% \crdata{0-12345-67-8/90/12} will cause 0-12345-67-8/90/12 to appear in the copyright line.
%
% ---------------------------------------------------------------------------------------------------------------
% This .tex source is an example which *does* use
% the .bib file (from which the .bbl file % is produced).
% REMEMBER HOWEVER: After having produced the .bbl file,
% and prior to final submission, you *NEED* to 'insert'
% your .bbl file into your source .tex file so as to provide
% ONE 'self-contained' source file.
%
% ================= IF YOU HAVE QUESTIONS =======================
% Questions regarding the SIGS styles, SIGS policies and
% procedures, Conferences etc. should be sent to
% Adrienne Griscti (griscti@acm.org)
%
% Technical questions _only_ to
% Gerald Murray (murray@acm.org)
% ===============================================================
%
% For tracking purposes - this is V1.8 - June 2007

\documentclass{sig-alternate}

\newcommand{\Comment}[1]{}

\usepackage{url}

\begin{document}
%
% --- Author Metadata here ---
\conferenceinfo{ICSE}{'10 May 2-8, 2010, Cape Town, South Africa}
%\CopyrightYear{2007} % Allows default copyright year (200X) to be over-ridden - IF NEED BE.
%\crdata{0-12345-67-8/90/01}  % Allows default copyright data (0-89791-88-6/97/05) to be over-ridden - IF NEED BE.
% --- End of Author Metadata ---

\title{Is Operator-Based Mutant Selection Superior to Random Mutant Selection? \vspace{-2ex}}

%
% You need the command \numberofauthors to handle the 'placement
% and alignment' of the authors beneath the title.
%
% For aesthetic reasons, we recommend 'three authors at a time'
% i.e. three 'name/affiliation blocks' be placed beneath the title.
%
% NOTE: You are NOT restricted in how many 'rows' of
% "name/affiliations" may appear. We just ask that you restrict
% the number of 'columns' to three.
%
% Because of the available 'opening page real-estate'
% we ask you to refrain from putting more than six authors
% (two rows with three columns) beneath the article title.
% More than six makes the first-page appear very cluttered indeed.
%
% Use the \alignauthor commands to handle the names
% and affiliations for an 'aesthetic maximum' of six authors.
% Add names, affiliations, addresses for
% the seventh etc. author(s) as the argument for the
% \additionalauthors command.
% These 'additional authors' will be output/set for you
% without further effort on your part as the last section in
% the body of your article BEFORE References or any Appendices.

\numberofauthors{1} %  in this sample file, there are a *total*
% of EIGHT authors. SIX appear on the 'first-page' (for formatting
% reasons) and the remaining two appear in the \additionalauthors section.
%
\author{
% 1st. author
\alignauthor Lu Zhang$^{1,2}$, Shan-Shan Hou$^{1,2}$, Jun-Jue
Hu$^{1,2}$, Tao Xie$^3$, Hong Mei$^{1,2}$\\
       \affaddr{\vspace{0.1cm}$^1$Institute of Software, School of Electronics Engineering and Computer Science}\\
       \affaddr{$^2$Key Laboratory of High Confidence Software Technologies (Peking University), Ministry of Education}\\
       \affaddr{Peking University, Beijing, 100871, China}\\
       \affaddr{$^3$Department of Computer Science, North Carolina State University, Raleigh, NC 27695}\\
       \email{\{zhanglu,houss,hujj08,meih\}@sei.pku.edu.cn, xie@csc.ncsu.edu\vspace{-2ex}}\\
}
% There's nothing stopping you putting the seventh, eighth, etc.
% author on the opening page (as the 'third row') but we ask,
% for aesthetic reasons that you place these 'additional authors'
% in the \additional authors block, viz.
% Just remember to make sure that the TOTAL number of authors
% is the number that will appear on the first page PLUS the
% number that will appear in the \additionalauthors section.

\maketitle
\begin{abstract}
Due to the expensiveness of compiling and executing a large number
of mutants, it is usually necessary to select a subset of mutants
to substitute the whole set of generated mutants in mutation
testing and analysis. Most research on mutant selection focused on
operator-based mutant selection, i.e., determining a set of
sufficient mutation operators and selecting mutants generated with
only this set of mutation operators. Recently, researchers began
to leverage statistical analysis to determine sufficient mutation
operators using execution information of mutants. However, whether
mutants selected with these sophisticated techniques is superior
to randomly selected mutants remains an open question. In this
paper, we empirically investigate this open question by comparing
three representative operator-based mutant-selection techniques
with two random techniques. Our empirical results show that
operator-based mutant selection is not superior to random mutant
selection. These results also indicate that random mutant
selection can be a better choice and mutant selection on the basis
of individual mutants is worthy of further investigation.
\end{abstract}


\vspace{-1.5ex}
% A category with the (minimum) three required fields
\category{D.2.5}{Software Engineering}{Testing and
Debugging}\vspace{-2ex} \terms{Measurement,
Experimentation}\vspace{-2ex}
\keywords{Mutation testing, Test-adequacy criterion} % NOT required for Proceedings
\vspace{-2ex}

\section{Introduction}
\label{sec:intro}
Regression test generation aims at generating a test suite that can detect behavioral differences between the original and the new versions of a program. A behavioral difference between two versions of a program can be reflected by the difference between the observable outputs produced by the execution of the same test (referred to as a difference-exposing test) on the two versions. Developers can inspect these behavioral differences to determine whether they are intended or unintended (i.e., regression faults).

Regression test generation can be automated by using Dynamic Symbolic
Execution (DSE)~\cite{dart, cute, exe}, a state-of-the-art test generation
technique, to generate a test suite achieving high
structural coverage. DSE explores paths in a program to
achieve high structural coverage, and exploration of all
these paths can often be expensive. However, if our aim is
to detect behavioral differences between two versions of a
program, we do not need to explore all these paths in the program
since not all these paths are relevant for detecting behavioral
differences.

To formally investigate irrelevant paths for exposing behavioral differences, we adopt the 
Propagation, Infection, and Execution (PIE) model~\cite{voas} of error propagation. According to the PIE model, a fault can be detected by a test if a faulty statement is executed (E), the execution of the faulty statement infects the state (I), and the infected state (i.e., error) propagates to an observable output (P). A change in the new version of a program can be treated as a fault and then the PIE model is applicable for effect propagation of the change. Many paths in a program often cannot help in satisfying any of the conditions P, I, or E of the PIE model. 

In this paper, we present an approach{\footnote{\scriptsize{An earlier version of this work~\cite{taneja09:guided} is described in a four-page paper that appears in the NIER track of ICSE 2009. This
version significantly extends the previous work in the following major ways.
First, in this paper, we develop techniques for efficiently finding irrelevant branches 
that cannot execute any change. 
Second, we develop techniques for exploiting the existing test suite for efficiently generating regression tests.
Third, we automate our approach by developing a tool.
Fourth, we conduct extensive experiments to evaluate our approach.}} \CodeIn{eXpress} and its implementations that uses DSE to detect behavioral differences based on the notion of the PIE model.

Our approach first determines all the branches (in the program under test) that cannot help in achieving any of the conditions E and I of the PIE model in terms of the changes in the program. To make test generation efficient, we develop a new search strategy for DSE to avoid exploring these irrelevant branches (including which can lead to an irrelevant path\footnote{\scriptsize{An irrelevant path is a path that cannot help in achieving P, I, and E of the PIE model.}}). In particular, our approach guides DSE to avoid from flipping branching nodes\footnote{\scriptsize{A branching node in the execution tree of a program is an instance of a conditional statement in the source code. A branching node consists of two sides (or more than two sides for a \CodeIn{switch} statement): the true branch and the false branch. Flipping a branching node is flipping the execution of the program from the true (or false) branch of the branching node to the false (or true) branch. Flipping a branching node representing a switch statement is flipping the execution of the current branch to another unexplored branch.}}, which on flipping execute some irrelevant branch. 

In addition, our approach can exploit the existing test suite (if available) for the original version by seeding the tests in the test suite to the program exploration. Our technique of seeding the exploration with the existing test suite can be used to efficiently augment an existing test suite so that various changed parts of the program (that are not covered by the existing test suite) are covered by the augmented test suite. 

\Comment{In addition, our approach prioritizes the flipping of branching nodes\footnote{A branching node in the execution tree of a program is an instance of a conditional statement in the source code. A branching node consists of two sides (or more than two sides for a \CodeIn{switch} statement): the true branch and the false branch. Flipping a branching node is flipping the execution of the program from the true (or false) branch of the branching node to the false (or true) branch.  Flipping a branching node representing a switch statement is flipping the execution of current branch to some unexplored branch.} in such a manner that behavioral differences are more likely to be detected earlier in path exploration.
}

This paper makes the following major contributions:
\\ \textbf{Path Exploration for Regression Test Generation.} We propose an approach called \CodeIn{eXpress} that uses DSE for efficient generation of regression unit tests. To the best
of our knowledge, ours is the first approach that guides path exploration on the new version specifically for regression test generation.
\\ \textbf{Incremental Exploration.} We develop a technique for exploiting an existing test suite, so that path exploration focuses on covering the changes rather than starting from scratch. To the best of our knowledge, ours is the first technique that leverages an existing test suite for automated regression test generation.
\\ \textbf{Implementation.} We have implemented our \CodeIn{eXpress} approach in a tool as an extension for Pex~\cite{Pex},  an automated structural testing tool for .NET developed at Microsoft Research. Pex has been previously used internally at Microsoft to test core components of the .NET architecture and has found serious
bugs~\cite{Pex}. The current Pex has been downloaded for thousands of times in industry. 
\\ \textbf{Evaluation.} We have conducted experiments on 72 versions (in total) of four programs. Experimental results show that our approach requires about 53\% fewer runs (i.e., explored paths) on average to cause the execution of a changed region and 44\% fewer to cause program-state differences after its execution than exploration without guidance. In addition, our approach requires 71\% fewer runs to cover all the changed regions (i.e, infection) by exploiting an existing test suite than exploration without using the test suite. 
%\end{itemize}
\Comment{
The rest of the paper is organized as follows. Section~\ref{sec:example} 
presents an example to illustrate our approach. Section~\ref{sec:approach} presents 
our approach and the major components involved in the approach. Section~\ref{sec:evaluation}
presents the evaluation results. Section~\ref{sec:validity} discusses the threats to validity of the evaluation results. Section~\ref{sec:related} discusses related
work. Section~\ref{sec:discussion} discusses research issues and future work,
and Section~\ref{sec:conclusion} concludes.
}
\vspace{-2ex}

\section{Experimental Design}
\label{Experiment}

In this section, we first present the research questions in our
study. Then, we describe the experimented techniques, the tool we
used to obtain mutants, the subject programs, and the way we
measure each technique. Finally, we describe the details of our
experimental procedure.

\vspace{-1ex}
\subsection{Research Questions}
\label{Questions}

In our study, we investigate the following research questions:

\begin{itemize}
\vspace{-1ex}

\item \textbf{RQ1}: How does operator-based mutant selection
compare with random mutant selection in terms of average
effectiveness?\vspace{-1ex}

\item \textbf{RQ2}: How does operator-based mutant selection
compare with random mutant selection in terms of
stability?\vspace{-1ex}

\end{itemize}

\subsection{Experimented Techniques}
\label{Techniques}

In our study, we experimented three operator-based
mutant-selection techniques (i.e., Offutt et al.'s 5 mutation
operators~\cite{Offutt:96}, Barbosa et al.'s 10 mutation
operators~\cite{Barbosa:01}, and Siami Namin et al.'s 28 mutation
operators~\cite{SiamiNamin:08})\footnote{As Wong and Mathur's two
mutation operators~\cite{Wong:93,Wong:95} are among Offutt et
al.'s five mutation operators~\cite{Offutt:96} and Offutt et al.
showed that any subset of the five mutation operators is not
sufficient, we did not empirically compare Wong and Mathur's two
mutation operators in our study.} and two random mutant-selection
techniques.

Given a number (denoted as $u$), the first random mutant-selection
technique is to randomly select $u$ mutants. This technique is
basically the $x$\%-random technique studied by Wong and
Mathur~\cite{Wong:93,Wong:95}. The second random mutant-selection
technique employs two steps when selecting each mutant. The first
step randomly selects a mutation operator, and the second step
randomly selects a mutant that is generated with the selected
mutation operator. Using the two steps, the second random
technique continually selects one mutant that has not been
selected previously until $u$ mutants are selected. In this paper,
we refer to the first random technique as the $one$-$round$
$random$ and the second random technique as the $two$-$round$
$random$. For the $one$-$round$ $random$, the probability of
selecting each mutant is equal; but for the $two$-$round$
$random$, the number of selected mutants that are produced by each
mutation operator is about the same.

\vspace{-1ex}
\subsection{Supporting Tool}
\label{Tool}

In our study, we used Proteum~\cite{Delamaro:96}, which is a
mutation system implementing a comprehensive set of mutation
operators for C programs, to generate mutants for each subject.
The version of Proteum used in our study supports 108 mutation
operators, including traditional mutation
operators~\cite{Agrawal:06} and interface mutation
operators~\cite{Delamaro:01}. As the 108 mutation operators
include Offutt et al.'s five mutation operators\footnote{Offutt et
al.'s five mutation operators are defined on programs in
Fortran-77. Agrawal et al.~\cite{Agrawal:06} list the mutation
operators in Proteum that correspond to Offutt et al.'s five
mutation operators.}, Barbosa et al.'s 10 mutation operators, and
Siami Namin et al.'s 28 mutation operators, we are able to use
Proteum to compare random mutant selection with all the three
operator-based mutant-selection techniques.

\vspace{-1ex}
\subsection{Subject Programs}
\label{Subjects}

\begin{table}[t]
\caption{\label{tab:Subjects} Statistics of subjects} \centering
\hspace*{-0.2cm}
\begin{tabular}{|p{5.4em}|p{1.8em}|p{3.5em}|p{1.9em}|p{3.2em}|p{4.1em}|}
  \hline
  % after \\: \hline or \cline{col1-col2} \cline{col3-col4} ...
  ~ & ~ &Net  & Test & ~ & Non-\\
  ~ & ~ & Lines of  &  Pool  & All & Equivalent\\
  Program & Abb. & Code & Size & Mutants & Mutants\\
  \hline
  print\_tokens &PT&343 &4130 &11741 &9326\\
  \hline
  print\_tokens2 &PT2&355 &4115 &10266 &8664\\
  \hline
  replace &RE&513 &5542 &23847 &19861\\
  \hline
  schedule &SC&296 &2650 &4130 &3670 \\
  \hline
  schedule2 &SC2&263 &2710 &6552 &4832 \\
  \hline
  tcas &TC&137 &1608 &4935 &4069\\
  \hline
  tot\_info &TI&281 &1052 &8767 &7876\\
  \hline
\end{tabular}
\vspace{-4ex}
\end{table}

The subjects used in our study are the Siemens programs. The
Siemens programs include seven C programs whose numbers of net
lines of code (not counting whitespace and commenting lines) range
from 137 to 513. Hutchins et al.~\cite{Hutchins:94} first
introduced the Siemens programs in 1994, and after that many
researchers (e.g., Rothermel et
al.~\cite{Rothermel:98,Rothermel:99}, Elbaum et
al.~\cite{Elbaum:00}, Li et al.~\cite{Li:07}, Jones et
al.~\cite{Jones:05}, and Andrews et
al.~\cite{Andrews:05,SiamiNamin:08}) used the Siemens programs as
subjects in testing experiments. In particular, a recent study on
mutant selection by Siami Namin et al.~\cite{SiamiNamin:08} used
only the Siemens programs as subjects. For each of the Siemens
programs, Hutchins et al. provided a test pool, and Rothermel et
al.~\cite{Rothermel:98} augmented the test pool through manually
adding some white-box test cases. After augmentation, the test
pool for each program ensures that ``each executable statement,
edge, and definition-use pair in the base program or its control
flow graph was exercised by at least 30 test
cases"~\cite{Rothermel:98}. Table~\ref{tab:Subjects} depicts the
statistics of the subjects. Note that the second column in
Table~\ref{tab:Subjects} lists the abbreviations of the seven
subjects, and we use these abbreviations to denote the subjects
when presenting our experimental results in Section~\ref{Results}.

Similar to Siami Namin et al.~\cite{SiamiNamin:08}, we considered
the following three reasons when choosing our subjects. First, the
Siemens programs contain typical structures that also appear in
various large programs in C. Thus, findings on these subjects are
very likely to generalize to other programs. Second, there is a
large test pool for each of the Siemens programs. As measuring the
effectiveness of selected mutants relies on the use of different
test suites (see Section~\ref{Measurement} for the details of
measurement in our study), a large test pool allows us to
construct a large number of test suites containing different test
cases. Third, as Proteum generates a large number of mutants for
even a small program, using programs significantly larger than the
Siemens programs as subjects may result in huge computational
cost. Actually, beside Siami Namin et al.~\cite{SiamiNamin:08},
who used only less than one third of the mutants that Proteum
generates for the Siemens programs\footnote{Except for the
smallest subject (i.e., $tcas$), Siami Namin et
al.~\cite{SiamiNamin:08} used 2000 mutants for each other
subject.}, other researchers (i.e., Wong~\cite{Wong:93}, Offutt et
al.~\cite{Offutt:96}, and Barbosa et al.~\cite{Barbosa:01}) used
programs much smaller than the Siemens programs for evaluating
mutant-selection techniques.


\vspace{-1ex}
\subsection{Measurement}
\label{Measurement}

In our study, we adopted a metric that researchers used to
evaluate the effectiveness of mutant-selection techniques in
previous studies on mutant selection for mutation testing (e.g.,
Wong and Mathur~\cite{Wong:93,Wong:95}, Offutt et
al.~\cite{Offutt:96}, and Barbosa et al.~\cite{Barbosa:01}). Given
a program (denoted as $P$) and a set of mutants (denoted as $AM$)
generated for $P$ with all mutation operators, we removed
equivalent mutants from $AM$ and acquired a set of non-equivalent
mutants (denoted as $NEM$). When evaluating a mutant-selection
technique (denoted as $T$), we used $T$ to select mutants from
$NEM$, and denote the set of selected non-equivalent mutants as
$M_T$. To evaluate the effectiveness of $T$, we created a series
of test suites (denoted as $\{ts_1, ts_2, ..., ts_n\}$), each of
which can kill all mutants in $M_T$. We denote the subset of
mutants in $NEM$ that can be killed by $ts_i$
($1\leq$$i$$\leq$$n$) as $Killed_{NEM}(ts_i)$, and then we use
Formula~\ref{form:Measurement} to measure the effectiveness of
$T$.\vspace{-2ex}

\begin{equation}\label{form:Measurement}
    Eff(T)=\frac{\sum_{i=1}^n {\frac{|Killed_{NEM}(ts_i)|}{|NEM|}}}{n}
\end{equation}
\vspace{-2ex}

Intuitively, this metric measures the effectiveness of $T$ as the
representativeness of the set of non-equivalent mutants selected
by $T$ for the whole set of non-equivalent mutants $NEM$. As the
aim of mutation testing is to provide a test-adequacy criterion,
this metric measures the representativeness of $M_T$ for $NEM$ as
the representativeness of the test-adequacy criterion based on
$M_T$ for the test-adequacy criterion based on $NEM$. Thus, the
closer $Eff(T)$ is to 1.0, the more effective $T$ is. When
$Eff(T)$ is equal to 1.0, technique $T$ is able to select a subset
of mutants that fully represent the whole set of non-equivalent
mutants.

As measuring the effectiveness of a subset of mutants in our study
requires a series of test suites, we used a procedure similar to
the procedure used by Offutt et al.~\cite{Offutt:96} to construct
the test suites. That is to say, for a subset of mutants, we
continually selected $k$ test cases from the test pool until the
test suite composed of all the selected test cases is able to kill
all the mutants in the subset. Offutt et al.~\cite{Offutt:96}
selected 200 (i.e., $k$=200) test cases each time when
constructing such a test suite. That is to say, the numbers of
test cases in test suites used by Offutt et al. are multiples of
200 (i.e., 200, 400, 600, etc.). Actually, Offutt et al. used this
way of test-suite construction to simulate the situation of
applying mutation testing as a test-adequacy criterion, and the
number of test cases selected each time represents an increment of
test cases in the process of building up each test suite for
evaluating mutant selection. Considering that testers may use
different incremental numbers to create the test suite, we used
four different incremental numbers (i.e., $k$=25, 50, 100, and
200) including Offutt et al.'s incremental number. Note that one
mutant-selection technique may achieve very different
effectiveness values using test suites created with different
incremental numbers. In our study, given an incremental number, we
constructed 50 test suites when measuring the effectiveness of a
subset of selected mutants.

\begin{table*}[t]
\caption{\label{tab:Ran-Offutt} Offutt et al.'s technique v.s.
random mutant selection} \centering \hspace*{-0.8cm}
\begin{tabular}{|c||c|c||c|c||c|c||c|c||c|c||c|c||c|c||c|c|}
  \hline
  % after \\: \hline or \cline{col1-col2} \cline{col3-col4} ...
  Incr&\multicolumn{2}{|c||}{Program}
  &\multicolumn{2}{|c||}{PT}&\multicolumn{2}{|c||}{PT2}&\multicolumn{2}{|c||}{RE}&\multicolumn{2}{|c||}{SC}&\multicolumn{2}{|c||}{SC2}&\multicolumn{2}{|c||}{TC}&\multicolumn{2}{|c|}{TI}\\
  \cline{2-17}
  ~&\multicolumn{2}{|c||}{Result}
  &Eff&Dev&Eff&Dev&Eff&Dev&Eff&Dev&Eff&Dev&Eff&Dev&Eff&Dev\\
  \hline
  \hline
  ~&\multicolumn{2}{|c||}{Offutt et al.}&99.11&0.27&99.84&0.17&99.09&0.29&99.94&0.07&99.29&0.23&99.54&0.21&99.57&0.32\\
  \cline{2-17}
  ~&one-&50\%&99.09&0.21&99.52&0.14&99.20&0.15&99.11&0.38&99.09&0.29&98.16&0.49&99.76&0.11\\
  \cline{3-17}
  ~&round&75\%&99.35&0.16&99.74&0.10&99.48&0.09&99.44&0.26&99.21&0.25&98.56&0.37&99.84&0.07\\
  \cline{3-17}
  25&random&100\%&99.52&0.12&99.79&0.08&99.57&0.07&99.58&0.18&99.44&0.17&98.81&0.30&99.87&0.05\\
  \cline{2-17}
  ~&two-~&50\%&98.60&0.28&99.40&0.16&99.04&0.19&99.38&0.24&99.03&0.30&98.15&0.59&99.67&0.15\\
  \cline{3-17}
  ~&round&75\%&99.02&0.22&99.58&0.12&99.30&0.14&99.50&0.20&99.32&0.23&98.66&0.34&99.77&0.11\\
  \cline{3-17}
  ~&random&100\%&99.18&0.19&99.70&0.10&99.40&0.11&99.59&0.17&99.52&0.18&98.86&0.30&99.80&0.10\\
  \hline
  \hline
  ~&\multicolumn{2}{|c||}{Offutt et al.}&99.26&0.21&99.91&0.13&99.27&0.20&99.97&0.04&99.40&0.23&99.68&0.11&99.66&0.17\\
  \cline{2-17}
  ~&one-&50\%&99.14&0.21&99.58&0.12&99.42&0.12&99.31&0.34&99.17&0.27&98.62&0.44&99.79&0.11\\
  \cline{3-17}
  ~&round&75\%&99.47&0.15&99.79&0.08&99.57&0.08&99.53&0.25&99.32&0.23&98.95&0.33&99.88&0.06\\
  \cline{3-17}
  50&random&100\%&99.60&0.12&99.89&0.06&99.61&0.08&99.62&0.20&99.58&0.15&99.13&0.25&99.90&0.06\\
  \cline{2-17}
  ~&two-~&50\%&98.68&0.30&99.48&0.14&99.22&0.16&99.48&0.23&99.32&0.25&98.60&0.50&99.73&0.14\\
  \cline{3-17}
  ~&round&75\%&99.18&0.22&99.67&0.10&99.37&0.14&99.64&0.17&99.46&0.20&98.95&0.32&99.80&0.11\\
  \cline{3-17}
  ~&random&100\%&99.28&0.21&99.77&0.08&99.53&0.09&99.68&0.14&99.65&0.14&99.18&0.25&99.90&0.06\\
  \hline
  \hline
  ~&\multicolumn{2}{|c||}{Offutt et al.}&99.34&0.23&99.97&0.05&99.54&0.18&99.98&0.02&99.62&0.21&99.80&0.13&99.74&0.20\\
  \cline{2-17}
  ~&one-&50\%&99.32&0.20&99.66&0.11&99.53&0.10&99.43&0.34&99.36&0.25&98.99&0.38&99.83&0.10\\
  \cline{3-17}
  ~&round&75\%&99.56&0.15&99.77&0.08&99.65&0.08&99.65&0.21&99.52&0.19&99.23&0.28&99.92&0.05\\
  \cline{3-17}
  100&random&100\%&99.62&0.13&99.93&0.04&99.71&0.07&99.73&0.15&99.68&0.14&99.37&0.23&99.94&0.04\\
  \cline{2-17}
  ~&two-~&50\%&99.00&0.27&99.46&0.15&99.36&0.15&99.60&0.20&99.44&0.23&99.05&0.38&99.74&0.17\\
  \cline{3-17}
  ~&round&75\%&99.20&0.23&99.59&0.12&99.50&0.12&99.70&0.16&99.69&0.15&99.32&0.25&99.88&0.08\\
  \cline{3-17}
  ~&random&100\%&99.46&0.19&99.76&0.08&99.61&0.10&99.75&0.12&99.72&0.13&99.45&0.21&99.90&0.06\\
  \hline
  \hline
  ~&\multicolumn{2}{|c||}{Offutt et al.}&99.54&0.26&99.97&0.07&99.65&0.17&99.99&0.01&99.60&0.18&99.89&0.11&99.76&0.20\\
  \cline{2-17}
  ~&one-&50\%&99.46&0.20&99.72&0.08&99.63&0.11&99.64&0.25&99.54&0.22&99.26&0.35&99.93&0.06\\
  \cline{3-17}
  ~&round&75\%&99.59&0.17&99.83&0.07&99.73&0.09&99.75&0.16&99.62&0.17&99.47&0.27&99.94&0.05\\
  \cline{3-17}
  200&random&100\%&99.79&0.09&99.92&0.04&99.83&0.06&99.78&0.13&99.75&0.13&99.60&0.21&99.95&0.04\\
  \cline{2-17}
  ~&two-~&50\%&99.15&0.28&99.61&0.12&99.48&0.14&99.70&0.18&99.65&0.18&99.31&0.33&99.84&0.12\\
  \cline{3-17}
  ~&round&75\%&99.35&0.25&99.73&0.09&99.65&0.10&99.77&0.14&99.78&0.12&99.54&0.23&99.89&0.08\\
  \cline{3-17}
  ~&random&100\%&99.57&0.19&99.85&0.06&99.71&0.09&99.82&0.10&99.81&0.11&99.63&0.20&99.93&0.05\\
  \hline
\end{tabular}
\vspace{-4ex}
\end{table*}

\vspace{-1ex}
\subsection{Experimental Procedure}
\label{Procedure}

For each subject, we used all the 108 mutation operators in
Proteum to generate mutants. The fifth column in
Table~\ref{tab:Subjects} lists the number of all the generated
mutants for each subject.

After acquiring all the mutants, for each subject, we executed
each test case in the test pool of the subject against each mutant
of the subject and the subject in the original form. Thus, we
acquired the information of which mutants are killed by which test
cases for each subject. Similar to Siami Namin et
al.~\cite{SiamiNamin:08}, we deemed mutants that cannot be killed
by any test case as equivalent mutants in our study. The last
column in Table~\ref{tab:Subjects} lists the number of
non-equivalent mutants for each subject.

For a subject, different operator-based mutant-selection
techniques select different numbers of mutants. Thus, it is
difficult for us to compare all the three techniques with random
mutant selection on the same ground. Therefore, we used the
following way to compare an operator-based mutant-selection
technique with random mutant-selection.

When comparing an operator-based mutant-selection technique
(denoted as $T$) with random mutant selection on one subject, we
used $T$ to select a subset of mutants (denoted as $M_T$) from all
the non-equivalent mutants (denoted as $NEM$) of the subject. To
compare $T$ with random mutant selection, we used each random
mutant-selection technique to select a series of subsets of
mutants from $NEM$, each subset containing $50\%*|M_T|$,
$75\%*|M_T|$, and $100\%*|M_T|$ mutants. To reduce accidental
results, for each random technique and each size of subsets, we
randomly selected $m$ subsets of the same size. That is to say,
for each subject and each random technique, we randomly selected
$m$ subsets each containing $50\%*|M_T|$ mutants, $m$ subsets each
containing $75\%*|M_T|$ mutants, and $m$ subsets each containing
$100\%*|M_T|$ mutants. After acquiring the subsets of mutants
selected with $T$ and the random mutant-selection techniques, we
used the metric defined in Section~\ref{Measurement} to measure
the effectiveness of each technique. For each random technique and
each size of subsets (e.g., using a random technique to select
$100\%*|M_T|$ mutants), we used the average effectiveness of the
$m$ subsets as the effectiveness of that technique with that size.
In our study, we set the value of $m$ as 50, which is large enough
to avoid accidental results.

As we used 50 test suites to measure the effectiveness of each
technique and we randomly selected 50 subsets of mutants for each
random technique, we further studied the stability of each
technique in terms of standard deviation of its effectiveness.




\begin{table*}[t]
\caption{\label{tab:Ran-Barbosa} Barbosa et al.'s technique v.s.
random mutant selection} \centering \hspace*{-0.8cm}
\begin{tabular}{|c||c|c||c|c||c|c||c|c||c|c||c|c||c|c||c|c|}
  \hline
  % after \\: \hline or \cline{col1-col2} \cline{col3-col4} ...
  Incr&\multicolumn{2}{|c||}{Program}
  &\multicolumn{2}{|c||}{PT}&\multicolumn{2}{|c||}{PT2}&\multicolumn{2}{|c||}{RE}&\multicolumn{2}{|c||}{SC}&\multicolumn{2}{|c||}{SC2}&\multicolumn{2}{|c||}{TC}&\multicolumn{2}{|c|}{TI}\\
  \cline{2-17}
  ~&\multicolumn{2}{|c||}{Result}
  &Eff&Dev&Eff&Dev&Eff&Dev&Eff&Dev&Eff&Dev&Eff&Dev&Eff&Dev\\
  \hline
  \hline
  ~&\multicolumn{2}{|c||}{Barbosa et al.}&99.20&0.21&99.89&0.13&99.42&0.18&99.97&0.02&99.73&0.13&99.57&0.13&99.62&0.23\\
  \cline{2-17}
  ~&one-&50\%&99.61&0.10&99.91&0.04&99.64&0.06&99.50&0.21&99.45&0.18&98.93&0.26&99.85&0.07\\
  \cline{3-17}
  ~&round&75\%&99.73&0.07&99.93&0.03&99.75&0.05&99.60&0.18&99.61&0.14&99.26&0.19&99.91&0.04\\
  \cline{3-17}
  25&random&100\%&99.80&0.05&99.94&0.03&99.81&0.04&99.76&0.11&99.73&0.10&99.45&0.15&99.92&0.04\\
  \cline{2-17}
  ~&two-~&50\%&99.39&0.16&99.86&0.06&99.54&0.09&99.59&0.16&99.46&0.19&99.01&0.25&99.79&0.10\\
  \cline{3-17}
  ~&round&75\%&99.65&0.11&99.93&0.03&99.66&0.05&99.70&0.11&99.68&0.13&99.30&0.18&99.85&0.07\\
  \cline{3-17}
  ~&random&100\%&99.78&0.08&99.94&0.03&99.75&0.05&99.76&0.09&99.77&0.09&99.47&0.15&99.91&0.04\\
  \hline
  \hline
  ~&\multicolumn{2}{|c||}{Barbosa et al.}&99.31&0.23&99.96&0.05&99.60&0.19&99.98&0.02&99.82&0.11&99.70&0.14&99.73&0.20\\
  \cline{2-17}
  ~&one-&50\%&99.68&0.10&99.94&0.04&99.71&0.06&99.63&0.18&99.46&0.19&99.29&0.22&99.88&0.06\\
  \cline{3-17}
  ~&round&75\%&99.78&0.07&99.96&0.02&99.80&0.04&99.74&0.13&99.72&0.11&99.47&0.17&99.92&0.04\\
  \cline{3-17}
  50&random&100\%&99.83&0.05&99.97&0.02&99.85&0.04&99.75&0.11&99.78&0.09&99.61&0.14&99.95&0.03\\
  \cline{2-17}
  ~&two-~&50\%&99.49&0.17&99.89&0.04&99.57&0.08&99.67&0.15&99.61&0.16&99.27&0.22&99.85&0.08\\
  \cline{3-17}
  ~&round&75\%&99.69&0.10&99.96&0.02&99.73&0.05&99.74&0.12&99.74&0.12&99.53&0.16&99.91&0.05\\
  \cline{3-17}
  ~&random&100\%&99.82&0.08&99.97&0.02&99.79&0.04&99.80&0.09&99.82&0.08&99.67&0.13&99.93&0.04\\
  \hline
  \hline
  ~&\multicolumn{2}{|c||}{Barbosa et al.}&99.45&0.21&99.96&0.07&99.66&0.17&99.99&0.02&99.84&0.12&99.84&0.11&99.77&0.16\\
  \cline{2-17}
  ~&one-&50\%&99.77&0.09&99.96&0.02&99.78&0.06&99.69&0.17&99.62&0.16&99.48&0.20&99.92&0.05\\
  \cline{3-17}
  ~&round&75\%&99.86&0.06&99.97&0.02&99.84&0.04&99.77&0.13&99.77&0.11&99.63&0.16&99.94&0.04\\
  \cline{3-17}
  100&random&100\%&99.88&0.04&99.97&0.02&99.89&0.03&99.84&0.09&99.84&0.09&99.78&0.11&99.96&0.03\\
  \cline{2-17}
  ~&two-~&50\%&99.61&0.16&99.94&0.03&99.70&0.07&99.74&0.12&99.65&0.16&99.54&0.19&99.89&0.07\\
  \cline{3-17}
  ~&round&75\%&99.79&0.09&99.96&0.02&99.79&0.05&99.78&0.10&99.83&0.09&99.71&0.14&99.91&0.06\\
  \cline{3-17}
  ~&random&100\%&99.87&0.06&99.98&0.02&99.84&0.04&99.84&0.08&99.89&0.06&99.80&0.11&99.94&0.04\\
  \hline
  \hline
  ~&\multicolumn{2}{|c||}{Barbosa et al.}&99.61&0.24&99.99&0.01&99.80&0.15&99.99&0.01&99.90&0.06&99.89&0.11&99.79&0.20\\
  \cline{2-17}
  ~&one-&50\%&99.82&0.09&99.96&0.02&99.83&0.06&99.78&0.14&99.70&0.15&99.63&0.20&99.95&0.04\\
  \cline{3-17}
  ~&round&75\%&99.89&0.05&99.99&0.01&99.89&0.04&99.83&0.09&99.87&0.08&99.82&0.13&99.96&0.03\\
  \cline{3-17}
  200&random&100\%&99.92&0.04&99.98&0.01&99.92&0.03&99.89&0.07&99.87&0.08&99.87&0.10&99.97&0.03\\
  \cline{2-17}
  ~&two-~&50\%&99.71&0.15&99.95&0.03&99.77&0.08&99.81&0.11&99.81&0.11&99.67&0.18&99.90&0.07\\
  \cline{3-17}
  ~&round&75\%&99.87&0.07&99.99&0.01&99.85&0.05&99.86&0.08&99.90&0.07&99.82&0.13&99.95&0.04\\
  \cline{3-17}
  ~&random&100\%&99.92&0.05&99.98&0.01&99.89&0.04&99.88&0.08&99.93&0.05&99.88&0.10&99.96&0.03\\
  \hline
\end{tabular}
\vspace{-4ex}
\end{table*}


\begin{table*}[t]
\caption{\label{tab:Ran-SiamiNamin} Siami Namin et al.'s technique
v.s. random mutant selection} \centering \hspace*{-0.8cm}
\begin{tabular}{|c||c|c||c|c||c|c||c|c||c|c||c|c||c|c||c|c|}
  \hline
  % after \\: \hline or \cline{col1-col2} \cline{col3-col4} ...
  Incr&\multicolumn{2}{|c||}{Program}
  &\multicolumn{2}{|c||}{PT}&\multicolumn{2}{|c||}{PT2}&\multicolumn{2}{|c||}{RE}&\multicolumn{2}{|c||}{SC}&\multicolumn{2}{|c||}{SC2}&\multicolumn{2}{|c||}{TC}&\multicolumn{2}{|c|}{TI}\\
  \cline{2-17}
  ~&\multicolumn{2}{|c||}{Result}
  &Eff&Dev&Eff&Dev&Eff&Dev&Eff&Dev&Eff&Dev&Eff&Dev&Eff&Dev\\
  \hline
  \hline
  ~&\multicolumn{2}{|c||}{Siami Namin et al.}&99.71&0.14&99.95&0.02&99.36&0.11&99.60&0.13&99.60&0.19&97.58&0.54&98.94&0.32\\
  \cline{2-17}
  ~&one-&50\%&99.30&0.17&99.70&0.10&99.24&0.14&99.11&0.37&98.99&0.30&98.23&0.48&99.68&0.14\\
  \cline{3-17}
  ~&round&75\%&99.47&0.13&99.80&0.07&99.42&0.12&99.40&0.26&99.17&0.29&98.60&0.38&99.75&0.12\\
  \cline{3-17}
  25&random&100\%&99.57&0.10&99.89&0.05&99.56&0.08&99.49&0.21&99.43&0.18&98.94&0.27&99.85&0.07\\
  \cline{2-17}
  ~&two-~&50\%&98.71&0.26&99.61&0.11&99.01&0.18&99.20&0.31&99.08&0.26&98.30&0.46&99.57&0.19\\
  \cline{3-17}
  ~&round&75\%&99.13&0.21&99.67&0.10&99.26&0.14&99.49&0.20&99.34&0.21&98.70&0.32&99.71&0.13\\
  \cline{3-17}
  ~&random&100\%&99.38&0.16&99.85&0.06&99.37&0.11&99.59&0.16&99.51&0.18&98.97&0.27&99.72&0.14\\
  \hline
  \hline
  ~&\multicolumn{2}{|c||}{Siami Namin et al.}&99.81&0.10&99.97&0.02&99.47&0.11&99.66&0.13&99.64&0.19&98.08&0.5&99.16&0.42\\
  \cline{2-17}
  ~&one-&50\%&99.31&0.17&99.83&0.08&99.38&0.12&99.21&0.41&99.13&0.28&98.72&0.42&99.81&0.09\\
  \cline{3-17}
  ~&round&75\%&99.53&0.13&99.85&0.06&99.55&0.09&99.48&0.25&99.42&0.21&98.99&0.31&99.80&0.10\\
  \cline{3-17}
  50&random&100\%&99.64&0.11&99.91&0.04&99.66&0.07&99.57&0.22&99.52&0.16&99.14&0.25&99.86&0.08\\
  \cline{2-17}
  ~&two-~&50\%&99.03&0.23&99.62&0.11&99.22&0.16&99.40&0.29&99.33&0.25&98.79&0.39&99.64&0.18\\
  \cline{3-17}
  ~&round&75\%&99.32&0.20&99.74&0.09&99.37&0.13&99.60&0.18&99.43&0.21&99.07&0.29&99.69&0.17\\
  \cline{3-17}
  ~&random&100\%&99.51&0.15&99.87&0.05&99.50&0.11&99.64&0.16&99.63&0.15&99.28&0.22&99.79&0.11\\
  \hline
  \hline
  ~&\multicolumn{2}{|c||}{Siami Namin et al.}&99.87&0.10&99.98&0.02&99.61&0.08&99.73&0.07&99.68&0.20&98.56&0.51&99.25&0.40\\
  \cline{2-17}
  ~&one-&50\%&99.52&0.16&99.78&0.08&99.47&0.12&99.46&0.33&99.31&0.27&99.12&0.33&99.79&0.13\\
  \cline{3-17}
  ~&round&75\%&99.63&0.13&99.92&0.04&99.64&0.08&99.61&0.23&99.54&0.18&99.36&0.24&99.86&0.09\\
  \cline{3-17}
  100&random&100\%&99.76&0.09&99.95&0.03&99.73&0.06&99.67&0.19&99.65&0.16&99.45&0.21&99.91&0.06\\
  \cline{2-17}
  ~&two-~&50\%&99.16&0.25&99.64&0.11&99.32&0.15&99.57&0.23&99.43&0.22&99.12&0.32&99.64&0.20\\
  \cline{3-17}
  ~&round&75\%&99.43&0.19&99.83&0.07&99.54&0.11&99.68&0.16&99.62&0.17&99.33&0.25&99.78&0.14\\
  \cline{3-17}
  ~&random&100\%&99.61&0.17&99.90&0.05&99.64&0.09&99.75&0.13&99.71&0.14&99.49&0.20&99.83&0.10\\
  \hline
  \hline
  ~&\multicolumn{2}{|c||}{Siami Namin et al.}&99.94&0.04&99.99&0.01&99.67&0.07&99.75&0.10&99.79&0.18&98.79&0.44&99.44&0.37\\
  \cline{2-17}
  ~&one-&50\%&99.59&0.17&99.83&0.07&99.20&0.11&99.61&0.27&99.58&0.21&99.39&0.29&99.86&0.10\\
  \cline{3-17}
  ~&round&75\%&99.76&0.11&99.96&0.02&99.74&0.08&99.74&0.18&99.68&0.16&99.52&0.23&99.92&0.06\\
  \cline{3-17}
  200&random&100\%&99.80&0.09&99.96&0.02&99.80&0.07&99.78&0.14&99.73&0.14&99.60&0.21&99.91&0.06\\
  \cline{2-17}
  ~&two-~&50\%&99.33&0.24&99.77&0.09&99.50&0.14&99.68&0.20&99.60&0.19&99.33&0.33&99.78&0.16\\
  \cline{3-17}
  ~&round&75\%&99.57&0.17&99.85&0.05&99.64&0.11&99.75&0.15&99.72&0.14&99.53&0.24&99.85&0.10\\
  \cline{3-17}
  ~&random&100\%&99.70&0.15&99.94&0.03&99.71&0.09&99.81&0.11&99.80&0.12&99.69&0.18&99.85&0.11\\
  \hline
\end{tabular}
\vspace{-4ex}
\end{table*}


\vspace{-2ex}
\section{Results and Analysis}
\label{Results}

In this section, we present and analyze the results of comparison
to answer the two research questions. We further analyze the
ability of reduction in mutants for the three operator-based
mutant-selection techniques. The detailed results
of our evaluation are available at \url{https://sites.google.com/site/asergrp/projects/ranmu}.


\vspace{-1ex}
\subsection{Effectiveness}
\label{OffuttEff}

Tables~\ref{tab:Ran-Offutt},~\ref{tab:Ran-Barbosa},
and~\ref{tab:Ran-SiamiNamin} depict the average effectiveness values
and standard-deviation values for comparing Offutt et al.'s
technique, Barbosa et al.'s technique, and Siami Namin et al.'s
technique with random mutant selection, respectively\footnote{Due to
space limit, we use tables instead of figures to present the
experimental results. Tables are more concise but less intuitive
than figures. Also due to space limit, we present results on
effectiveness and standard deviation in each table.}. In the three
tables, we use $Incr$ to represent the four incremental numbers for
creating test suites for evaluating selected mutants; $Eff$ to
represent average effectiveness values measured by the metric
defined in Section~\ref{Measurement}; $Dev$ to represent
standard-deviation values among the corresponding effectiveness
values; and $50\%$, $75\%$, and $100\%$ to represent the use of
random mutant selection to select 50\%, 75\%, and 100\% of mutants
selected by each of the three operator-based mutant-selection
techniques. Both the effectiveness values and the standard-deviation
values are in percentage points. To make the difference clear, we
keep two digits after the decimal point for each value. From the
three tables, we have the following observations related to the
average effectiveness.

First, our results confirm that all the three operator-based
mutant-selection techniques are able to achieve good effectiveness
values using test suites created with all the four incremental
numbers. According to the criterion proposed by Offutt et
al.~\cite{Offutt:96}, a mutant-selection technique is good in
effectiveness if it achieves an average effectiveness value over
99\% using test suites created with incremental number 200. Based
on this criterion, all the three technique are good in
effectiveness, except for Siami Namin et al.'s technique on $tcas$
(i.e., 98.79\%).

Second, despite the effectiveness of the three operator-based
mutant-selection techniques, none of them is superior to random
mutant-selection when selecting the same number of mutants. Both
random techniques outperform Offutt et al.'s technique for four
subjects (i.e., $print\_tokens$, $replace$, $schedule2$, and
$tot\_info$) out of seven. This trend is consistent for all the
four incremental numbers. Both random techniques consistently
outperform Barbosa et al.'s technique for $print\_tokens$,
$replace$, and $tot\_info$ with different incremental numbers. For
$print\_tokens2$ and $schedule2$, the two random techniques
achieve almost the same average effectiveness as Barbosa et al.'s
technique. Both random techniques consistently outperform Siami
Namin et al.'s technique for $replace$, $tcas$, and $tot\_info$
with different incremental numbers. The two-round random technique
also achieves similar effectiveness values as Siami Namin et al.'s
technique for $schedule$ for different incremental numbers.
Overall, for any subject, the difference between one operator-based
mutant-selection technique and its corresponding random
mutant-selection techniques is quite small. This observation
indicates that random mutant selection is still as competitive as
or even better than operator-based mutant-selection in terms of
average effectiveness.

Third, compared with the three operator-based mutant-selection
techniques, random mutant selection is able to achieve competitive
effectiveness when selecting fewer mutants. In general, for any
random technique and any subject, the differences between
selecting 50\%, 75\%, and 100\% mutants are quite small. Thus, the
difference between each operator-based mutant-selection technique
and its corresponding random techniques selecting 50\% or 75\%
mutants are also small. Based on Offutt et al.'s criterion, for
each operator-based mutant-selection technique, using its
corresponding random techniques to select 50\% is also good in
effectiveness, since the average effectiveness values are always over
99\% for all the subjects using test suites created with
incremental number 200. Since there is no sophisticated strategy
in random mutant selection, this observation indicates that it is
likely that mutants selected by either of the three operator-based
mutant-selection techniques are more than necessary.


Fourth, when comparing the two random techniques, the one-round
technique is less effective than the two-round technique for
smaller subjects, but more effective than the two-round technique
for larger subjects. We suspect the reason to be that the
one-round technique may fail to select any mutant for some
important mutation operators when the number of mutants is small.
Without mutants generated with these mutation operators, the
selected set of mutants is not as representative as those
containing mutants generated with all the different mutation
operators. However, for the two-round technique, the distribution
of selected mutants over the mutation operators may be very
different from the distribution of all the non-equivalent mutants
over the mutation operators when the subject becomes large. Thus,
mutants selected by the two-round technique may be less
representative than those selected by the one-round technique for
large subjects.

Finally, although different incremental numbers impact the
effectiveness values for any experimented techniques and any
subjects, the preceding observations are consistent for different
incremental numbers. Typically, with the increase of the
incremental number, the effectiveness value for any technique on
any subject also increases. We suspect the reason to be that
different incremental numbers result in different sizes of created
test suites and the differences in test-suite sizes lead to the
differences in effectiveness values. We further checked the
average sizes of test suites under different incremental numbers
and corroborated this suspicion. Table~\ref{tab:TestSuiteSize}
depicts average test-suite sizes for measuring Offutt et al.'s
technique on the seven subjects under different incremental
numbers. The trends in average test-suite sizes for other
experimented techniques are similar. Due to space limit, we do not
present the average test-suite sizes for them here. It should also
be noted that, due to the nature of the metric used in our study,
the test suites created with any incremental number usually are of
different sizes, and thus the average test-suite sizes are not
multiples of 25, 50, 100, or 200.

\begin{table}[t]
\caption{\label{tab:TestSuiteSize} Average test-suite sizes for
Offutt et al.'s} \centering \hspace*{-0.3cm}
\begin{tabular}{|c|c|c|c|c|}
  \hline
  % after \\: \hline or \cline{col1-col2} \cline{col3-col4} ...
  Program & Increment  &Increment  & Increment &Increment \\
  ~ & 25  &50  & 100 &200 \\
   \hline
  \hline
  PT&309 &479 &725 &1190\\
  \hline
  PT2&196 &307 &471 &725\\
  \hline
  RE&638 &1019 &1609 &2445\\
  \hline
  SC&237 &383 &591 &894\\
  \hline
  SC2&290 &464 &710 &970\\
  \hline
  TC&474 &700 &910 &1175\\
  \hline
  TI&185 &261 &351 &468\\
  \hline
\end{tabular}
\vspace{-4ex}
\end{table}

\vspace{-1.5ex}
\subsection{Stability}
\label{Stability}

Based on the standard-deviation values depicted in
Tables~\ref{tab:Ran-Offutt},~\ref{tab:Ran-Barbosa},
and~\ref{tab:Ran-SiamiNamin}, we have the following observations
on comparing the stability of the three
operator-based mutant-selection techniques and random mutant
selection techniques.

First, all the experimented techniques are quite stable in
effectiveness. For any experimented technique, the
standard-deviation values are typically less than 0.20 percentage
points, and standard-deviation values are rarely
larger than 0.30 percentage points. Most of those exceptionally
large standard-deviation values come from Siami Namin et al.'s
technique and random techniques with 50\% mutants on $tcas$, which
is the smallest subject. Siami Namin et al.'s technique is less
effective and less stable for $tcas$. Random techniques seem to be
less stable for smaller subjects but more stable for larger
subjects.

Second, similar to the observation on average effectiveness,
either of the three operator-based mutant-selection techniques is
not more stable than random mutant selection when selecting the
same number of mutants. Both random techniques are more stable
than Offutt et al.'s technique for $print\_tokens$, $replace$,
$schedule2$, and $tot\_info$; and are as stable as Offutt et al.'s
technique for $print\_tokens2$. Both random techniques are more
stable than Barbosa et al.'s technique for $print\_tokens$,
$replace$, and $tot\_info$; and are as stable as Barbosa et al.'s
technique for $print\_tokens2$,  $schedule2$, and $tcas$. Both
random techniques are more stable than Siami Namin et al.'s
technique for $schedule2$, $tcas$, and $tot\_info$; and are as
stable as Siami Namin et al.'s technique for $replace$. It is
interesting to note that a technique achieving better
effectiveness values is typically also more stable.

Third, when comparing the two random techniques, the one-round
technique seems to be less stable than the two-round technique for
smaller subjects, but more effective than the two-round technique
for larger subjects. This observation is in accordance with the
observation on comparing the effectiveness of the two
random techniques. Furthermore, given a random mutant-selection
technique, selecting more mutants is more stable than selecting
fewer mutants.


\vspace{-1.5ex}
\subsection{Reduction in Mutants}
\label{Reduction}


\begin{table}[t]
\caption{\label{tab:Reduction} Ratio of selected mutants}
\centering \hspace*{-0.1cm}
\begin{tabular}{|c|c|c|c|}
  \hline
  % after \\: \hline or \cline{col1-col2} \cline{col3-col4} ...
  Program & Offutt  &Barbosa  & Siami Namin \\
  ~ & et al. & et al.  &  et al.\\
  \hline
  \hline
  PT&6.00 &14.89 &7.33\\
  \hline
  PT2&5.90 &18.89 &8.64\\
  \hline
  RE&6.84 &16.30 &6.62\\
  \hline
  SC&7.77 &14.11 &7.11\\
  \hline
  SC2&8.09 &14.84 &7.88\\
  \hline
  TC&6.07 &15.31 &7.08\\
  \hline
  TI&10.27 &17.92 &7.29\\
  \hline
  \hline
  Average&7.28 &16.04 &7.42\\
  \hline
\end{tabular}
\vspace{-4ex}
\end{table}

As the preceding results indicate that none of the three
operator-based mutant-selection techniques is superior to random
mutant selection in terms of either effectiveness or stability, we
further examine their ability to reduce the number of mutants.
Table~\ref{tab:Reduction} depicts the ratio of selected mutants to
all the non-equivalent mutants for each subject. From this table,
we have the following observations.

First, all the three operator-based mutant-selection techniques
are able to substantially reduce the number of mutants. Either
Offutt et al.'s technique or Siami Namin et al.'s technique is
able to reduce about 93\% mutants, while Barbosa et al.'s
technique is able to reduce 84\% mutants. Our result for Siami
Namin et al.'s technique is similar to that reported by Siami
Namin et al.~\cite{SiamiNamin:08} (i.e., 92.6\%). Our result for
Offutt et al.'s technique is different from that reported by
Offutt et al.~\cite{Offutt:96} (i.e., 78\%). We suspect the reason
to be that we considered much more mutation operators than Offutt
et al. Our result for Barbosa et al.'s technique is also different
from that reported by Barbosa et al.~\cite{Barbosa:01} (i.e.,
78\%). We suspect the reason to be that Barbosa et al. used 71
mutation operators (which were implemented in Proteum by 2001)
rather than 108 mutation operators in our study. This observation
also indicates that Barbosa et al.'s technique is much more
expensive than either Offutt et al.'s technique or Siami Namin et
al.'s technique.

Second, for each technique, the ratio of selected mutants does not
differ much for a different subject. This observation indicates
that the reduction in mutants is highly predictable for any of the
three operator-based mutant-selection techniques. Furthermore,
when applying random mutant selection in practice, we may use the
average ratio of selected mutants for an operator-based
mutant-selection technique as a guidance to determine the number
of mutants to select. For example, randomly selecting 7\% mutants
from mutants generated with the 108 Proteum mutation operators is
likely to achieve similar effectiveness and stability as Offutt et
al.'s technique.

\vspace{-2ex}
\section{Discussion}
\label{Discussion}

In this section, we discuss the following issues related to our
study: threats to validity, cost of mutant selection, and some
further implications of our experimental results.

\vspace{-1.5ex}
\subsection{Threats to Validity}
\label{Threats} \vspace{-1.5ex}
\subsubsection{Internal Validity}
\label{Internal}

Threats to internal validity are concerned with uncontrolled
factors that may also be responsible for the results. In our
study, the main threat to internal validity is the possible faults
in our experiments and result analysis. To reduce this threat, we
used Proteum for mutant generation. Furthermore, we reviewed all
the code that we produced for our experiments and analysis before
conducting the experiments. Note that the experimented techniques
are all straightforward to implement and their implementation is
thus less likely to threaten the internal validity.

\vspace{-1.5ex}
\subsubsection{External Validity}
\label{External}

Threats to external validity are concerned with whether the
findings in our study are generalizable for other situations. The
main threat to external validity lies in the representativeness of
the subjects. To reduce this threat, we chose seven subjects
written in C and these subjects contain a wide range of data and
control structures commonly used in C (or even C++ and Java)
programs~\cite{SiamiNamin:08}. Conducting more experiments using
more subjects of larger sizes and with structures not contained in
our subjects is one way to further reduce this threat. Note that,
when experimenting with subjects containing object-oriented
structures, mutation operators on these structures~\cite{Ma:05}
should also be considered.

\vspace{-1.5ex}
\subsubsection{Construct Validity}
\label{Construct}

Threats to construct validity are concerned with whether the
measurement in our study reflects real-world situations. The main
threat to construct validity is the way we measured the
effectiveness of selected mutants. To reduce this threat, we used a
metric commonly used by previous studies, such as Offutt et
al.~\cite{Offutt:96} and Barbosa et al.~\cite{Barbosa:01}. Further
reduction of this threat requires the design of better metrics to
assess the effectiveness of selected mutants in mutation testing.
The metric used in our study actually measures to what extent the
selected mutants are representative for mutants generated with all
the mutation operators. Indeed, users of mutant-selection techniques
may be more concerned with the representativeness of the selected
mutants for real bugs. Such concerns may be alleviated by previous
empirical studies~\cite{Andrews:05,Do:06} showing evidence that mutants
generated with mutation operators are similar to real faults in
evaluating test effectiveness. But it is worthwhile of conducting
empirical studies directly on representativeness of the selected
mutants for real bugs in future work.  Furthermore, the metric used in our
study requires a series of randomly created test suites, but test
suites used in practice may not be created randomly.

\vspace{-1.5ex}
\subsection{Cost of Mutant Selection}
\label{Cost}

According to Offutt et al.~\cite{Offutt:96}, the expensiveness of
mutation testing mainly lies in the need to compile and execute a
large number of mutants against each test case. Compared with the
cost of compiling and executing mutants, the cost of mutant
selection is typically very small. In our study, we used Proteum
as the platform for mutant selection. In particular, we ran
Proteum on a PC with a Genuine Intel CPU T1400 at 1.83GHz and 1GB
memory running SUSE Linux (version 2.6.25.5-1.1) with the $tcsh$
shell (version 6.15.00). For each subject, Proteum generated the
selected mutants in a few seconds. For the smallest subject (i.e.,
$tcas$), the time is less than one second; for the largest subject
(i.e., $replace$), the time is less than four seconds; for each
other subject, the time is less than two seconds. But the total
time for us to compile and execute all the mutants for all the
seven subjects against all the test cases under the same hardware
and software environment is about one month CPU time. Note that
Offutt et al.'s, Barbosa et al.'s, and Siami Namin et al.'s
techniques averagely select 7.28\%, 16.04\%, and 7.42\% mutants,
respectively. That is to say, compiling and executing mutants
selected by one of the three operator-based mutant-selection
techniques for all the seven subjects against all the test cases
may take two to five days. As a result, for either operator-based
mutant selection or random mutant selection, there is little
difference in the cost of mutant selection.

One issue that we may need to consider is the way that Barbosa et al.'s
and Siami Namin et al.'s techniques use to determine mutant
operators. In particular, both techniques determine the set of
sufficient mutation operators using training data acquired through
compiling and executing mutants against test cases. That is to
say, both techniques may require some extra cost besides compiling
and executing the selected mutants. However, as their
cross-validation indicates that the set of sufficient mutation
operators determined with some programs may also be applicable for
other programs, the extra cost of either Barbosa et al.'s
technique or Siami Namin et al.'s technique should be considerably
small.

\vspace{-1.5ex}
\subsection{Further Implications}
\label{Implications}

Based on the findings of our study, we have the following
implications.

First, as the three operator-based mutant-selection techniques are
not superior to the two random mutant-selection techniques in
terms of either effectiveness or stability, random
mutant-selection techniques may be a better choice in practice due
to their flexibility in controlling the number of selected
mutants. Note that random mutant-selection techniques can achieve
similar effectiveness and stability even when selecting much fewer
mutants.

Second, the findings in our study also imply that we may need much
fewer mutants in mutation testing than those selected by existing
operator-based mutant-selection techniques. That is to say, it is
very likely to invent new mutant-selection techniques to select
much fewer mutants but with similar or even better effectiveness
and stability. Random mutant selection may be a good starting
point.

Third, considering the nature of random mutant selection, the
following difference between random mutant selection and
operator-based mutant selection may be an explanation of the
surprisingly good results of random mutant selection. Random
mutant selection selects mutants on the basis of individual
mutants, but operator-based mutant selection needs to include or
exclude all the mutants generated with one mutation operator as a
whole. If this difference accounts for the goodness of random
mutant selection, techniques that both consider the difference in
operators and select mutants individually may outperform existing
random and operator-based mutant-selection techniques.

\vspace{-1ex}
\section{Related Work}
\label{RelatedWork}

We next discuss related work on mutation testing and analysis, and
mutant selection.

\vspace{-1ex}
\subsection{Mutation Testing and Analysis}
\label{RelatedMutation}

Mutation testing is a fault-based testing approach, which is first
proposed by Hamlet~\cite{Hamlet:77} and DeMillo et
al.~\cite{DeMillo:78}. In mutation testing, the primary aim is to
provide a rigorous test-adequacy criterion that can help enhance
test suites. For the software under test (SUT), a tester can use
mutation operators to generate a number of mutants. After
identifying equivalent mutants, the tester can use the remaining
non-equivalent mutants to enhance test suites. That is to say, if
a test suite cannot kill all the remaining non-equivalent mutants,
more test cases may be required to enhance the test suite. Another
usage of mutation is mutation analysis, whose aim is not to
enhance test suites but to provide assessment of test
effectiveness to facilitate experiments in testing research. It is
interesting to note that researchers such as Briand et
al.~\cite{Briand:04} have already used mutation faults to measure
test effectiveness even before the empirical confirmation of its
appropriateness by Andrews et al.~\cite{Andrews:05} and Do et
al.~\cite{Do:06}.

For both mutation testing and analysis, a major concern is the
expensiveness of compiling and executing a large number of
mutants. In the literature, there are mainly four categories of
techniques to reduce this cost. The first category is to select a
subset of mutants as substitute. As our research in this paper
falls into this category, we detailedly discuss research in this
category in Section~\ref{RelatedSelection}. The second category is
to use low-cost ways to determine which test case kills which
mutant. Weak mutation~\cite{Howden:82} and firm
mutation~\cite{Woodward:88} are two representative techniques in
this category. The third category is to use efficient ways to
generate, compile and execute mutants. As mutants only slightly
differ from the original program, taking the advantage of the
commonalities of the mutants may accelerate the generation,
compilation and execution of the mutants. Compiler-integrated
mutation~\cite{DeMillo:91} and schema-based
mutation~\cite{Untch:93} are two representative techniques in this
category. The last category is to compile and execute mutants in
parallel. Researchers have investigated parallel compilation and
execution of mutants on different computer
architectures~\cite{Krauser:88,Offutt:92}. Techniques in different
categories are typically complementary to each other, as they can
be combined together to reduce the cost of mutant compilation and
execution.

The second concern in mutation testing and analysis is the cost in
identifying equivalent mutants. In the literature, researchers
proposed several techniques for detecting equivalent mutants
statically~\cite{Offutt:94,Offutt:97,Hierons:99} or
dynamically~\cite{Grun:09,Schuler:09}. For mutation testing, there
is still another concern: the cost of acquiring mutation-adequate
test suites. In the literature, researchers also proposed some
techniques to automatically generate mutation-adequate test
suites~\cite{DeMillo:91b,Offutt:99,Liu:06}. Techniques addressing
both concerns are also complementary to research on mutant
selection. As our research in this paper mainly considers mutant
selection, further discussion related to these two concerns is out
of the scope of this paper.

\vspace{-1ex}
\subsection{Mutant Selection}
\label{RelatedSelection}

As mentioned previously, mutant selection is an important way to
reduce the cost of mutation testing and analysis. Since
Mathur~\cite{Mathur:91} first proposed the idea of excluding some
mutation operators in mutation testing, several researchers have
studied operator-based mutant selection. In operator-based mutant
selection, researchers first determine a set of mutation
operators, and then select only mutants generated with this set of
mutation operators.

Wong and Mathur~\cite{Wong:93,Wong:95} studied 2 mutation
operators among the 22 mutation operators in
Mothra~\cite{DeMillo:87}, and found that mutants generated with these
2 mutation operators can achieve similar results as mutants
generated with all the 22 mutation operators. Offutt et
al.~\cite{Offutt:96} experimentally determined 5 mutation
operators among the 22 mutation operators in Mothra, and found
that the effectiveness values of the 5 mutation operators are
between 99.0\% and 100\% on ten subjects, with the average 99.5\%.
Note that the effectiveness values are based on test suites
created with incremental number 200. Furthermore, Offutt et al.
also found that, without any of the 5 mutation operators, the
effectiveness value on some subject would be lower than 99\%,
which they defined as the minimal requirement of a set of
sufficient mutation operators for mutation testing. Note that Wong
and Mathur's 2 mutation operators are among Offutt et al.'s 5
mutation operators. Barbosa et al.~\cite{Barbosa:01} proposed six
guidelines to determine sufficient mutation operators. The
application of some of the six guidelines requires compilation and
execution of a large number of mutants. Based on the six
guidelines, Barbosa et al. determined 10 mutation operators, and
found that, using the same way as Offutt et al. to measure the
effectiveness, the effectiveness values of the 10 mutation
operators on 27 subjects (which were also used to determine the 10
mutation operators) are between 95.8\% and 100\%, with the average of
99.6\%. Siami Namin et al.~\cite{SiamiNamin:08} leveraged variable
reduction to determine 28 mutation operators using the execution
information of a subset of mutants. As their work aims at mutation
analysis rather than mutation testing, they evaluated the 28
mutation operators on the Siemens programs only in the context of
mutation analysis. They did not provide evidence about whether the
28 operators are sufficient in mutation testing.

Compared with studies on operator-based mutation selection,
studies on random mutant selection, which Acree et
al.~\cite{Acree:79} first proposed in 1979, are limited. Wong and
Mathur~\cite{Wong:93,Wong:95} empirically studied the technique of
randomly selecting $10\%$ to $40\%$ mutants generated with 22
mutation operators in Mothra. Barbosa et al.~\cite{Barbosa:01}
used random mutant selection as a control technique when
evaluating their 10 mutation operators. In their study, Barbosa et
al.'s 10 mutation operators are more effective than random mutant
selection. Our study differs from previous studies on random
mutant selection for mutation testing as follows. First, our study
investigates some operator-based techniques (i.e.,Offutt et al.'s
technique~\cite{Offutt:96} and Siami Namin et al.'s
technique~\cite{SiamiNamin:08}) previously not empirically
compared with random mutant selection, and our results on Barbosa
et al.'s technique~\cite{Barbosa:01} contradict previous results.
Second, our study investigates two random mutant-selection
techniques, while previous studies investigated only one random
mutant-selection technique. Third, our study uses larger subjects
than previous studies on random mutant selection. Finally, our
study investigates both average effectiveness and standard
deviation of effectiveness, while previous studies investigate
only average effectiveness.

\vspace{-1ex}
\section{Conclusion and Future Work}
\label{Conclusion}

In this paper, we report an empirical study attempting to answer
one important open question in the field of mutant selection for
mutation testing. Our experimental results show that none of the
three experimented operator-based mutant-selection techniques is
superior to random mutant selection in terms of either
effectiveness or stability. Furthermore, random mutant selection
can still achieve competitive results when selecting much fewer
mutants than each operator-based mutant-selection technique.

In future work, we plan to investigate three main issues. First,
as Siami Namin et al.'s mutation operators (which are determined
for mutation analysis) are quite effective in mutation testing, we
are thus interested in whether random and operator-based
mutant-selection techniques for mutation testing are also
effective in the context of mutation analysis. Second, we plan to
extend our experiments to other and larger subjects to further
corroborate the findings in our study. Finally, we also plan to
investigate new techniques of mutant selection on the basis of
individual mutants, which our study in this paper has shown to be
a promising direction.


\Comment{ \vspace{-1.5ex}
\section{Acknoledgments}
We thank Gregg Rothermel for providing the Siemens programs and
their test pools through Subject Infrastructure Repository (SIR) at
the University of Nebraska-Lincoln. We thank Auri Vincenzi, Adita
Mathur, Qianxiang Wang, and Jos\'{e} Carlos Maldonado for their help
in accessing Proteum. \vspace{-1.5ex}}
%
% The following two commands are all you need in the
% initial runs of your .tex file to
% produce the bibliography for the citations in your paper.
\bibliographystyle{abbrv}
\vspace{-2ex}
\bibliography{body/sigproc}  % sigproc.bib is the name of the Bibliography in this case
% You must have a proper ".bib" file
%  and remember to run:
% latex bibtex latex latex
% to resolve all references
%
% ACM needs 'a single self-contained file'!
%
%APPENDICES are optional
%\balancecolumns


%\balancecolumns % GM June 2007
% That's all folks!
\end{document}
