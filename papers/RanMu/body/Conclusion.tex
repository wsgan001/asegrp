\vspace{-1ex}
\section{Conclusion and Future Work}
\label{Conclusion}

In this paper, we report an empirical study attempting to answer
one important open question in the field of mutant selection for
mutation testing. Our experimental results show that none of the
three experimented operator-based mutant-selection techniques is
superior to random mutant selection in terms of either
effectiveness or stability. Furthermore, random mutant selection
can still achieve competitive results when selecting much fewer
mutants than each operator-based mutant-selection technique.

In future work, we plan to investigate three main issues. First,
as Siami Namin et al.'s mutation operators (which are determined
for mutation analysis) are quite effective in mutation testing, we
are thus interested in whether random and operator-based
mutant-selection techniques for mutation testing are also
effective in the context of mutation analysis. Second, we plan to
extend our experiments to other and larger subjects to further
corroborate the findings in our study. Finally, we also plan to
investigate new techniques of mutant selection on the basis of
individual mutants, which our study in this paper has shown to be
a promising direction.
