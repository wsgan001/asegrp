\vspace{-2ex}

\section{Experimental Design}
\label{Experiment}

In this section, we first present the research questions in our
study. Then, we describe the experimented techniques, the tool we
used to obtain mutants, the subject programs, and the way we
measure each technique. Finally, we describe the details of our
experimental procedure.

\vspace{-1ex}
\subsection{Research Questions}
\label{Questions}

In our study, we investigate the following research questions:

\begin{itemize}
\vspace{-1ex}

\item \textbf{RQ1}: How does operator-based mutant selection
compare with random mutant selection in terms of average
effectiveness?\vspace{-1ex}

\item \textbf{RQ2}: How does operator-based mutant selection
compare with random mutant selection in terms of
stability?\vspace{-1ex}

\end{itemize}

\subsection{Experimented Techniques}
\label{Techniques}

In our study, we experimented three operator-based
mutant-selection techniques (i.e., Offutt et al.'s 5 mutation
operators~\cite{Offutt:96}, Barbosa et al.'s 10 mutation
operators~\cite{Barbosa:01}, and Siami Namin et al.'s 28 mutation
operators~\cite{SiamiNamin:08})\footnote{As Wong and Mathur's two
mutation operators~\cite{Wong:93,Wong:95} are among Offutt et
al.'s five mutation operators~\cite{Offutt:96} and Offutt et al.
showed that any subset of the five mutation operators is not
sufficient, we did not empirically compare Wong and Mathur's two
mutation operators in our study.} and two random mutant-selection
techniques.

Given a number (denoted as $u$), the first random mutant-selection
technique is to randomly select $u$ mutants. This technique is
basically the $x$\%-random technique studied by Wong and
Mathur~\cite{Wong:93,Wong:95}. The second random mutant-selection
technique employs two steps when selecting each mutant. The first
step randomly selects a mutation operator, and the second step
randomly selects a mutant that is generated with the selected
mutation operator. Using the two steps, the second random
technique continually selects one mutant that has not been
selected previously until $u$ mutants are selected. In this paper,
we refer to the first random technique as the $one$-$round$
$random$ and the second random technique as the $two$-$round$
$random$. For the $one$-$round$ $random$, the probability of
selecting each mutant is equal; but for the $two$-$round$
$random$, the number of selected mutants that are produced by each
mutation operator is about the same.

\vspace{-1ex}
\subsection{Supporting Tool}
\label{Tool}

In our study, we used Proteum~\cite{Delamaro:96}, which is a
mutation system implementing a comprehensive set of mutation
operators for C programs, to generate mutants for each subject.
The version of Proteum used in our study supports 108 mutation
operators, including traditional mutation
operators~\cite{Agrawal:06} and interface mutation
operators~\cite{Delamaro:01}. As the 108 mutation operators
include Offutt et al.'s five mutation operators\footnote{Offutt et
al.'s five mutation operators are defined on programs in
Fortran-77. Agrawal et al.~\cite{Agrawal:06} list the mutation
operators in Proteum that correspond to Offutt et al.'s five
mutation operators.}, Barbosa et al.'s 10 mutation operators, and
Siami Namin et al.'s 28 mutation operators, we are able to use
Proteum to compare random mutant selection with all the three
operator-based mutant-selection techniques.

\vspace{-1ex}
\subsection{Subject Programs}
\label{Subjects}

\begin{table}[t]
\caption{\label{tab:Subjects} Statistics of subjects} \centering
\hspace*{-0.2cm}
\begin{tabular}{|p{5.4em}|p{1.8em}|p{3.5em}|p{1.9em}|p{3.2em}|p{4.1em}|}
  \hline
  % after \\: \hline or \cline{col1-col2} \cline{col3-col4} ...
  ~ & ~ &Net  & Test & ~ & Non-\\
  ~ & ~ & Lines of  &  Pool  & All & Equivalent\\
  Program & Abb. & Code & Size & Mutants & Mutants\\
  \hline
  print\_tokens &PT&343 &4130 &11741 &9326\\
  \hline
  print\_tokens2 &PT2&355 &4115 &10266 &8664\\
  \hline
  replace &RE&513 &5542 &23847 &19861\\
  \hline
  schedule &SC&296 &2650 &4130 &3670 \\
  \hline
  schedule2 &SC2&263 &2710 &6552 &4832 \\
  \hline
  tcas &TC&137 &1608 &4935 &4069\\
  \hline
  tot\_info &TI&281 &1052 &8767 &7876\\
  \hline
\end{tabular}
\vspace{-4ex}
\end{table}

The subjects used in our study are the Siemens programs. The
Siemens programs include seven C programs whose numbers of net
lines of code (not counting whitespace and commenting lines) range
from 137 to 513. Hutchins et al.~\cite{Hutchins:94} first
introduced the Siemens programs in 1994, and after that many
researchers (e.g., Rothermel et
al.~\cite{Rothermel:98,Rothermel:99}, Elbaum et
al.~\cite{Elbaum:00}, Li et al.~\cite{Li:07}, Jones et
al.~\cite{Jones:05}, and Andrews et
al.~\cite{Andrews:05,SiamiNamin:08}) used the Siemens programs as
subjects in testing experiments. In particular, a recent study on
mutant selection by Siami Namin et al.~\cite{SiamiNamin:08} used
only the Siemens programs as subjects. For each of the Siemens
programs, Hutchins et al. provided a test pool, and Rothermel et
al.~\cite{Rothermel:98} augmented the test pool through manually
adding some white-box test cases. After augmentation, the test
pool for each program ensures that ``each executable statement,
edge, and definition-use pair in the base program or its control
flow graph was exercised by at least 30 test
cases"~\cite{Rothermel:98}. Table~\ref{tab:Subjects} depicts the
statistics of the subjects. Note that the second column in
Table~\ref{tab:Subjects} lists the abbreviations of the seven
subjects, and we use these abbreviations to denote the subjects
when presenting our experimental results in Section~\ref{Results}.

Similar to Siami Namin et al.~\cite{SiamiNamin:08}, we considered
the following three reasons when choosing our subjects. First, the
Siemens programs contain typical structures that also appear in
various large programs in C. Thus, findings on these subjects are
very likely to generalize to other programs. Second, there is a
large test pool for each of the Siemens programs. As measuring the
effectiveness of selected mutants relies on the use of different
test suites (see Section~\ref{Measurement} for the details of
measurement in our study), a large test pool allows us to
construct a large number of test suites containing different test
cases. Third, as Proteum generates a large number of mutants for
even a small program, using programs significantly larger than the
Siemens programs as subjects may result in huge computational
cost. Actually, beside Siami Namin et al.~\cite{SiamiNamin:08},
who used only less than one third of the mutants that Proteum
generates for the Siemens programs\footnote{Except for the
smallest subject (i.e., $tcas$), Siami Namin et
al.~\cite{SiamiNamin:08} used 2000 mutants for each other
subject.}, other researchers (i.e., Wong~\cite{Wong:93}, Offutt et
al.~\cite{Offutt:96}, and Barbosa et al.~\cite{Barbosa:01}) used
programs much smaller than the Siemens programs for evaluating
mutant-selection techniques.


\vspace{-1ex}
\subsection{Measurement}
\label{Measurement}

As our study is in the context of mutation testing, we adopted a
metric that researchers used to evaluate the effectiveness of
mutant-selection techniques in previous studies (e.g., Wong and
Mathur~\cite{Wong:93,Wong:95}, Offutt et al.~\cite{Offutt:96}, and
Barbosa et al.~\cite{Barbosa:01}). Given a program (denoted as
$P$) and a set of mutants (denoted as $AM$) generated for $P$ with
all mutation operators, we removed equivalent mutants from $AM$
and acquired a set of non-equivalent mutants (denoted as $NEM$).
When evaluating a mutant-selection technique (denoted as $T$), we
used $T$ to select mutants from $NEM$, and denote the set of
selected non-equivalent mutants as $M_T$. To evaluate the
effectiveness of $T$, we create a series of test suites (denoted
as $\{ts_1, ts_2, ..., ts_n\}$), each of which can kill all
mutants in $M_T$. We denote the subset of mutants in $NEM$ that
can be killed by $ts_i$ ($1\leq$$i$$\leq$$n$) as
$Killed_{NEM}(ts_i)$, and then we use Formula~\ref{form:Measurement} to
measure the effectiveness of $T$.\vspace{-2ex}

\begin{equation}\label{form:Measurement}
    Eff(T)=\frac{\sum_{i=1}^n {\frac{|Killed_{NEM}(ts_i)|}{|NEM|}}}{n}
\end{equation}
\vspace{-2ex}

Intuitively, this metric measures the effectiveness of $T$ as the
representativeness of the set of non-equivalent mutants selected
by $T$ for the whole set of non-equivalent mutants $NEM$. As the
aim of mutation testing is to provide a test-adequacy criterion,
this metric measures the representativeness of $M_T$ for $NEM$ as
the representativeness of the test-adequacy criterion based on
$M_T$ for the test-adequacy criterion based on $NEM$. Thus, the
closer $Eff(T)$ is to 1.0, the more effective $T$ is. When
$Eff(T)$ is equal to 1.0, technique $T$ is able to select a subset
of mutants that fully represent the whole set of non-equivalent
mutants.

As measuring the effectiveness of a subset of mutants in each
experiment requires a series of test suites, we used a procedure
similar to the procedure used by Offutt et al.~\cite{Offutt:96} to
construct the test suites. That is to say, for a subset of
mutants, we continually selected $k$ test cases from the test pool
until the test suite composed of all the selected test cases is
able to kill all the mutants in the subset. Offutt et
al.~\cite{Offutt:96} selected 200 (i.e., $k$=200) test cases each
time when constructing such a test suite. That is to say, the
numbers of test cases in test suites used by Offutt et al. are
multiples of 200 (i.e., 200, 400, 600, etc.). Actually, Offutt et
al. used this way of test-suite construction to simulate the
situation of applying mutation testing as a test-adequacy
criterion, and the number of test cases selected each time
represents an increment of test cases in the process of building
up each test suite for evaluating mutant selection. Considering
that testers may use different incremental numbers to create the
test suite, we used four different incremental numbers (i.e.,
$k$=25, 50, 100, and 200) including Offutt et al.'s incremental
number. Note that one mutant-selection technique may achieve very
different effectiveness values using test suites created with
different incremental numbers. In our study, given an incremental
number, we constructed 50 test suites when measuring the
effectiveness of a subset of selected mutants.

\begin{table*}[t]
\caption{\label{tab:Ran-Offutt} Offutt et al.'s technique v.s.
random mutant selection} \centering \hspace*{-0.8cm}
\begin{tabular}{|c||c|c||c|c||c|c||c|c||c|c||c|c||c|c||c|c|}
  \hline
  % after \\: \hline or \cline{col1-col2} \cline{col3-col4} ...
  Incr&\multicolumn{2}{|c||}{Program}
  &\multicolumn{2}{|c||}{PT}&\multicolumn{2}{|c||}{PT2}&\multicolumn{2}{|c||}{RE}&\multicolumn{2}{|c||}{SC}&\multicolumn{2}{|c||}{SC2}&\multicolumn{2}{|c||}{TC}&\multicolumn{2}{|c|}{TI}\\
  \cline{2-17}
  ~&\multicolumn{2}{|c||}{Result}
  &Eff&Dev&Eff&Dev&Eff&Dev&Eff&Dev&Eff&Dev&Eff&Dev&Eff&Dev\\
  \hline
  \hline
  ~&\multicolumn{2}{|c||}{Offutt et al.}&99.11&0.27&99.84&0.17&99.09&0.29&99.94&0.07&99.29&0.23&99.54&0.21&99.57&0.32\\
  \cline{2-17}
  ~&one-&50\%&99.09&0.21&99.52&0.14&99.20&0.15&99.11&0.38&99.09&0.29&98.16&0.49&99.76&0.11\\
  \cline{3-17}
  ~&round&75\%&99.35&0.16&99.74&0.10&99.48&0.09&99.44&0.26&99.21&0.25&98.56&0.37&99.84&0.07\\
  \cline{3-17}
  25&random&100\%&99.52&0.12&99.79&0.08&99.57&0.07&99.58&0.18&99.44&0.17&98.81&0.30&99.87&0.05\\
  \cline{2-17}
  ~&two-~&50\%&98.60&0.28&99.40&0.16&99.04&0.19&99.38&0.24&99.03&0.30&98.15&0.59&99.67&0.15\\
  \cline{3-17}
  ~&round&75\%&99.02&0.22&99.58&0.12&99.30&0.14&99.50&0.20&99.32&0.23&98.66&0.34&99.77&0.11\\
  \cline{3-17}
  ~&random&100\%&99.18&0.19&99.70&0.10&99.40&0.11&99.59&0.17&99.52&0.18&98.86&0.30&99.80&0.10\\
  \hline
  \hline
  ~&\multicolumn{2}{|c||}{Offutt et al.}&99.26&0.21&99.91&0.13&99.27&0.20&99.97&0.04&99.40&0.23&99.68&0.11&99.66&0.17\\
  \cline{2-17}
  ~&one-&50\%&99.14&0.21&99.58&0.12&99.42&0.12&99.31&0.34&99.17&0.27&98.62&0.44&99.79&0.11\\
  \cline{3-17}
  ~&round&75\%&99.47&0.15&99.79&0.08&99.57&0.08&99.53&0.25&99.32&0.23&98.95&0.33&99.88&0.06\\
  \cline{3-17}
  50&random&100\%&99.60&0.12&99.89&0.06&99.61&0.08&99.62&0.20&99.58&0.15&99.13&0.25&99.90&0.06\\
  \cline{2-17}
  ~&two-~&50\%&98.68&0.30&99.48&0.14&99.22&0.16&99.48&0.23&99.32&0.25&98.60&0.50&99.73&0.14\\
  \cline{3-17}
  ~&round&75\%&99.18&0.22&99.67&0.10&99.37&0.14&99.64&0.17&99.46&0.20&98.95&0.32&99.80&0.11\\
  \cline{3-17}
  ~&random&100\%&99.28&0.21&99.77&0.08&99.53&0.09&99.68&0.14&99.65&0.14&99.18&0.25&99.90&0.06\\
  \hline
  \hline
  ~&\multicolumn{2}{|c||}{Offutt et al.}&99.34&0.23&99.97&0.05&99.54&0.18&99.98&0.02&99.62&0.21&99.80&0.13&99.74&0.20\\
  \cline{2-17}
  ~&one-&50\%&99.32&0.20&99.66&0.11&99.53&0.10&99.43&0.34&99.36&0.25&98.99&0.38&99.83&0.10\\
  \cline{3-17}
  ~&round&75\%&99.56&0.15&99.77&0.08&99.65&0.08&99.65&0.21&99.52&0.19&99.23&0.28&99.92&0.05\\
  \cline{3-17}
  100&random&100\%&99.62&0.13&99.93&0.04&99.71&0.07&99.73&0.15&99.68&0.14&99.37&0.23&99.94&0.04\\
  \cline{2-17}
  ~&two-~&50\%&99.00&0.27&99.46&0.15&99.36&0.15&99.60&0.20&99.44&0.23&99.05&0.38&99.74&0.17\\
  \cline{3-17}
  ~&round&75\%&99.20&0.23&99.59&0.12&99.50&0.12&99.70&0.16&99.69&0.15&99.32&0.25&99.88&0.08\\
  \cline{3-17}
  ~&random&100\%&99.46&0.19&99.76&0.08&99.61&0.10&99.75&0.12&99.72&0.13&99.45&0.21&99.90&0.06\\
  \hline
  \hline
  ~&\multicolumn{2}{|c||}{Offutt et al.}&99.54&0.26&99.97&0.07&99.65&0.17&99.99&0.01&99.60&0.18&99.89&0.11&99.76&0.20\\
  \cline{2-17}
  ~&one-&50\%&99.46&0.20&99.72&0.08&99.63&0.11&99.64&0.25&99.54&0.22&99.26&0.35&99.93&0.06\\
  \cline{3-17}
  ~&round&75\%&99.59&0.17&99.83&0.07&99.73&0.09&99.75&0.16&99.62&0.17&99.47&0.27&99.94&0.05\\
  \cline{3-17}
  200&random&100\%&99.79&0.09&99.92&0.04&99.83&0.06&99.78&0.13&99.75&0.13&99.60&0.21&99.95&0.04\\
  \cline{2-17}
  ~&two-~&50\%&99.15&0.28&99.61&0.12&99.48&0.14&99.70&0.18&99.65&0.18&99.31&0.33&99.84&0.12\\
  \cline{3-17}
  ~&round&75\%&99.35&0.25&99.73&0.09&99.65&0.10&99.77&0.14&99.78&0.12&99.54&0.23&99.89&0.08\\
  \cline{3-17}
  ~&random&100\%&99.57&0.19&99.85&0.06&99.71&0.09&99.82&0.10&99.81&0.11&99.63&0.20&99.93&0.05\\
  \hline
\end{tabular}
\vspace{-4ex}
\end{table*}

\vspace{-1ex}
\subsection{Experimental Procedure}
\label{Procedure}

For each subject, we used all the 108 mutation operators in
Proteum to generate mutants. The fifth column in
Table~\ref{tab:Subjects} lists the number of all the generated
mutants for each subject.

After acquiring all the mutants, for each subject, we executed
each test case in the test pool of the subject against each mutant
of the subject and the subject in the original form. Thus, we
acquired the information of which mutants are killed by which test
cases for each subject. Similar to Siami Namin et
al.~\cite{SiamiNamin:08}, we deemed mutants that cannot be killed
by any test case as equivalent mutants in our study. The last
column in Table~\ref{tab:Subjects} lists the number of
non-equivalent mutants for each subject.

For a subject, different operator-based mutant-selection
techniques select different numbers of mutants. Thus, it is
difficult for us to compare random mutant selection with all the
three techniques on the same ground. Therefore, we used the
following way to compare an operator-based mutant-selection
technique with random mutant-selection.

When comparing an operator-based mutant-selection technique
(denoted as $T$) with random mutant selection on one subject, we
used $T$ to select a subset of mutants (denoted as $M_T$) from all
the non-equivalent mutants (denoted as $NEM$) of the subject. To
compare $T$ with random mutant selection, we used each random
mutant-selection technique to select a series of subsets of
mutants from $NEM$, each subset containing $50\%*|M_T|$,
$75\%*|M_T|$, and $100\%*|M_T|$ mutants. To reduce accidental
results, for each random technique and each size of subsets, we
randomly selected $m$ subsets of the same size. That is to say,
for each subject and each random technique, we randomly selected
$m$ subsets each containing $50\%*|M_T|$ mutants, $m$ subsets each
containing $75\%*|M_T|$ mutants, and $m$ subsets each containing
$100\%*|M_T|$ mutants. After acquiring the subsets of mutants
selected with $T$ and the random mutant-selection techniques, we
used the metric defined in Section~\ref{Measurement} to measure
the effectiveness of each technique. For each random technique and
each size of subsets (e.g., using a random technique to select
$100\%*|M_T|$ mutants), we used the average effectiveness of the
$m$ subsets as the effectiveness of that technique with that size.
In our study, we set the value of $m$ as 50, which is large enough
to avoid accidental results.

As we used 50 test suites to measure the effectiveness of each
technique and we randomly selected 50 subsets of mutants for each
random technique, we further studied the stability of each
technique in terms of standard deviation of its effectiveness.
