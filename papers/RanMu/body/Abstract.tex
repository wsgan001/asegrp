\begin{abstract}
Due to the expensiveness of compiling and executing a large number
of mutants, it is usually necessary to select a subset of mutants
to substitute the whole set of generated mutants in mutation
testing and analysis. Most research on mutant selection focused on
operator-based mutant selection, i.e., determining a set of
sufficient mutation operators and selecting mutants generated with
only this set of mutation operators. Recently, researchers began
to leverage statistical analysis to determine sufficient mutation
operators using execution information of mutants. However, whether
mutants selected with these sophisticated techniques is superior
to randomly selected mutants remains an open question. In this
paper, we empirically investigate this open question by comparing
three representative operator-based mutant-selection techniques
with two random techniques. Our empirical results show that
operator-based mutant selection is not superior to random mutant
selection. These results also indicate that random mutant
selection can be a better choice and mutant selection on the basis
of individual mutants is worthy of further investigation.
\end{abstract}
