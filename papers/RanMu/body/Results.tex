


\begin{table*}[t]
\caption{\label{tab:Ran-Barbosa} Barbosa et al.'s technique v.s.
random mutant selection} \centering \hspace*{-0.8cm}
\begin{tabular}{|c||c|c||c|c||c|c||c|c||c|c||c|c||c|c||c|c|}
  \hline
  % after \\: \hline or \cline{col1-col2} \cline{col3-col4} ...
  Incr&\multicolumn{2}{|c||}{Program}
  &\multicolumn{2}{|c||}{PT}&\multicolumn{2}{|c||}{PT2}&\multicolumn{2}{|c||}{RE}&\multicolumn{2}{|c||}{SC}&\multicolumn{2}{|c||}{SC2}&\multicolumn{2}{|c||}{TC}&\multicolumn{2}{|c|}{TI}\\
  \cline{2-17}
  ~&\multicolumn{2}{|c||}{Result}
  &Eff&Dev&Eff&Dev&Eff&Dev&Eff&Dev&Eff&Dev&Eff&Dev&Eff&Dev\\
  \hline
  \hline
  ~&\multicolumn{2}{|c||}{Barbosa et al.}&99.20&0.21&99.89&0.13&99.42&0.18&99.97&0.02&99.73&0.13&99.57&0.13&99.62&0.23\\
  \cline{2-17}
  ~&one-&50\%&99.61&0.10&99.91&0.04&99.64&0.06&99.50&0.21&99.45&0.18&98.93&0.26&99.85&0.07\\
  \cline{3-17}
  ~&round&75\%&99.73&0.07&99.93&0.03&99.75&0.05&99.60&0.18&99.61&0.14&99.26&0.19&99.91&0.04\\
  \cline{3-17}
  25&random&100\%&99.80&0.05&99.94&0.03&99.81&0.04&99.76&0.11&99.73&0.10&99.45&0.15&99.92&0.04\\
  \cline{2-17}
  ~&two-~&50\%&99.39&0.16&99.86&0.06&99.54&0.09&99.59&0.16&99.46&0.19&99.01&0.25&99.79&0.10\\
  \cline{3-17}
  ~&round&75\%&99.65&0.11&99.93&0.03&99.66&0.05&99.70&0.11&99.68&0.13&99.30&0.18&99.85&0.07\\
  \cline{3-17}
  ~&random&100\%&99.78&0.08&99.94&0.03&99.75&0.05&99.76&0.09&99.77&0.09&99.47&0.15&99.91&0.04\\
  \hline
  \hline
  ~&\multicolumn{2}{|c||}{Barbosa et al.}&99.31&0.23&99.96&0.05&99.60&0.19&99.98&0.02&99.82&0.11&99.70&0.14&99.73&0.20\\
  \cline{2-17}
  ~&one-&50\%&99.68&0.10&99.94&0.04&99.71&0.06&99.63&0.18&99.46&0.19&99.29&0.22&99.88&0.06\\
  \cline{3-17}
  ~&round&75\%&99.78&0.07&99.96&0.02&99.80&0.04&99.74&0.13&99.72&0.11&99.47&0.17&99.92&0.04\\
  \cline{3-17}
  50&random&100\%&99.83&0.05&99.97&0.02&99.85&0.04&99.75&0.11&99.78&0.09&99.61&0.14&99.95&0.03\\
  \cline{2-17}
  ~&two-~&50\%&99.49&0.17&99.89&0.04&99.57&0.08&99.67&0.15&99.61&0.16&99.27&0.22&99.85&0.08\\
  \cline{3-17}
  ~&round&75\%&99.69&0.10&99.96&0.02&99.73&0.05&99.74&0.12&99.74&0.12&99.53&0.16&99.91&0.05\\
  \cline{3-17}
  ~&random&100\%&99.82&0.08&99.97&0.02&99.79&0.04&99.80&0.09&99.82&0.08&99.67&0.13&99.93&0.04\\
  \hline
  \hline
  ~&\multicolumn{2}{|c||}{Barbosa et al.}&99.45&0.21&99.96&0.07&99.66&0.17&99.99&0.02&99.84&0.12&99.84&0.11&99.77&0.16\\
  \cline{2-17}
  ~&one-&50\%&99.77&0.09&99.96&0.02&99.78&0.06&99.69&0.17&99.62&0.16&99.48&0.20&99.92&0.05\\
  \cline{3-17}
  ~&round&75\%&99.86&0.06&99.97&0.02&99.84&0.04&99.77&0.13&99.77&0.11&99.63&0.16&99.94&0.04\\
  \cline{3-17}
  100&random&100\%&99.88&0.04&99.97&0.02&99.89&0.03&99.84&0.09&99.84&0.09&99.78&0.11&99.96&0.03\\
  \cline{2-17}
  ~&two-~&50\%&99.61&0.16&99.94&0.03&99.70&0.07&99.74&0.12&99.65&0.16&99.54&0.19&99.89&0.07\\
  \cline{3-17}
  ~&round&75\%&99.79&0.09&99.96&0.02&99.79&0.05&99.78&0.10&99.83&0.09&99.71&0.14&99.91&0.06\\
  \cline{3-17}
  ~&random&100\%&99.87&0.06&99.98&0.02&99.84&0.04&99.84&0.08&99.89&0.06&99.80&0.11&99.94&0.04\\
  \hline
  \hline
  ~&\multicolumn{2}{|c||}{Barbosa et al.}&99.61&0.24&99.99&0.01&99.80&0.15&99.99&0.01&99.90&0.06&99.89&0.11&99.79&0.20\\
  \cline{2-17}
  ~&one-&50\%&99.82&0.09&99.96&0.02&99.83&0.06&99.78&0.14&99.70&0.15&99.63&0.20&99.95&0.04\\
  \cline{3-17}
  ~&round&75\%&99.89&0.05&99.99&0.01&99.89&0.04&99.83&0.09&99.87&0.08&99.82&0.13&99.96&0.03\\
  \cline{3-17}
  200&random&100\%&99.92&0.04&99.98&0.01&99.92&0.03&99.89&0.07&99.87&0.08&99.87&0.10&99.97&0.03\\
  \cline{2-17}
  ~&two-~&50\%&99.71&0.15&99.95&0.03&99.77&0.08&99.81&0.11&99.81&0.11&99.67&0.18&99.90&0.07\\
  \cline{3-17}
  ~&round&75\%&99.87&0.07&99.99&0.01&99.85&0.05&99.86&0.08&99.90&0.07&99.82&0.13&99.95&0.04\\
  \cline{3-17}
  ~&random&100\%&99.92&0.05&99.98&0.01&99.89&0.04&99.88&0.08&99.93&0.05&99.88&0.10&99.96&0.03\\
  \hline
\end{tabular}
\vspace{-4ex}
\end{table*}


\begin{table*}[t]
\caption{\label{tab:Ran-SiamiNamin} Siami Namin et al.'s technique
v.s. random mutant selection} \centering \hspace*{-0.8cm}
\begin{tabular}{|c||c|c||c|c||c|c||c|c||c|c||c|c||c|c||c|c|}
  \hline
  % after \\: \hline or \cline{col1-col2} \cline{col3-col4} ...
  Incr&\multicolumn{2}{|c||}{Program}
  &\multicolumn{2}{|c||}{PT}&\multicolumn{2}{|c||}{PT2}&\multicolumn{2}{|c||}{RE}&\multicolumn{2}{|c||}{SC}&\multicolumn{2}{|c||}{SC2}&\multicolumn{2}{|c||}{TC}&\multicolumn{2}{|c|}{TI}\\
  \cline{2-17}
  ~&\multicolumn{2}{|c||}{Result}
  &Eff&Dev&Eff&Dev&Eff&Dev&Eff&Dev&Eff&Dev&Eff&Dev&Eff&Dev\\
  \hline
  \hline
  ~&\multicolumn{2}{|c||}{Siami Namin et al.}&99.71&0.14&99.95&0.02&99.36&0.11&99.60&0.13&99.60&0.19&97.58&0.54&98.94&0.32\\
  \cline{2-17}
  ~&one-&50\%&99.30&0.17&99.70&0.10&99.24&0.14&99.11&0.37&98.99&0.30&98.23&0.48&99.68&0.14\\
  \cline{3-17}
  ~&round&75\%&99.47&0.13&99.80&0.07&99.42&0.12&99.40&0.26&99.17&0.29&98.60&0.38&99.75&0.12\\
  \cline{3-17}
  25&random&100\%&99.57&0.10&99.89&0.05&99.56&0.08&99.49&0.21&99.43&0.18&98.94&0.27&99.85&0.07\\
  \cline{2-17}
  ~&two-~&50\%&98.71&0.26&99.61&0.11&99.01&0.18&99.20&0.31&99.08&0.26&98.30&0.46&99.57&0.19\\
  \cline{3-17}
  ~&round&75\%&99.13&0.21&99.67&0.10&99.26&0.14&99.49&0.20&99.34&0.21&98.70&0.32&99.71&0.13\\
  \cline{3-17}
  ~&random&100\%&99.38&0.16&99.85&0.06&99.37&0.11&99.59&0.16&99.51&0.18&98.97&0.27&99.72&0.14\\
  \hline
  \hline
  ~&\multicolumn{2}{|c||}{Siami Namin et al.}&99.81&0.10&99.97&0.02&99.47&0.11&99.66&0.13&99.64&0.19&98.08&0.5&99.16&0.42\\
  \cline{2-17}
  ~&one-&50\%&99.31&0.17&99.83&0.08&99.38&0.12&99.21&0.41&99.13&0.28&98.72&0.42&99.81&0.09\\
  \cline{3-17}
  ~&round&75\%&99.53&0.13&99.85&0.06&99.55&0.09&99.48&0.25&99.42&0.21&98.99&0.31&99.80&0.10\\
  \cline{3-17}
  50&random&100\%&99.64&0.11&99.91&0.04&99.66&0.07&99.57&0.22&99.52&0.16&99.14&0.25&99.86&0.08\\
  \cline{2-17}
  ~&two-~&50\%&99.03&0.23&99.62&0.11&99.22&0.16&99.40&0.29&99.33&0.25&98.79&0.39&99.64&0.18\\
  \cline{3-17}
  ~&round&75\%&99.32&0.20&99.74&0.09&99.37&0.13&99.60&0.18&99.43&0.21&99.07&0.29&99.69&0.17\\
  \cline{3-17}
  ~&random&100\%&99.51&0.15&99.87&0.05&99.50&0.11&99.64&0.16&99.63&0.15&99.28&0.22&99.79&0.11\\
  \hline
  \hline
  ~&\multicolumn{2}{|c||}{Siami Namin et al.}&99.87&0.10&99.98&0.02&99.61&0.08&99.73&0.07&99.68&0.20&98.56&0.51&99.25&0.40\\
  \cline{2-17}
  ~&one-&50\%&99.52&0.16&99.78&0.08&99.47&0.12&99.46&0.33&99.31&0.27&99.12&0.33&99.79&0.13\\
  \cline{3-17}
  ~&round&75\%&99.63&0.13&99.92&0.04&99.64&0.08&99.61&0.23&99.54&0.18&99.36&0.24&99.86&0.09\\
  \cline{3-17}
  100&random&100\%&99.76&0.09&99.95&0.03&99.73&0.06&99.67&0.19&99.65&0.16&99.45&0.21&99.91&0.06\\
  \cline{2-17}
  ~&two-~&50\%&99.16&0.25&99.64&0.11&99.32&0.15&99.57&0.23&99.43&0.22&99.12&0.32&99.64&0.20\\
  \cline{3-17}
  ~&round&75\%&99.43&0.19&99.83&0.07&99.54&0.11&99.68&0.16&99.62&0.17&99.33&0.25&99.78&0.14\\
  \cline{3-17}
  ~&random&100\%&99.61&0.17&99.90&0.05&99.64&0.09&99.75&0.13&99.71&0.14&99.49&0.20&99.83&0.10\\
  \hline
  \hline
  ~&\multicolumn{2}{|c||}{Siami Namin et al.}&99.94&0.04&99.99&0.01&99.67&0.07&99.75&0.10&99.79&0.18&98.79&0.44&99.44&0.37\\
  \cline{2-17}
  ~&one-&50\%&99.59&0.17&99.83&0.07&99.20&0.11&99.61&0.27&99.58&0.21&99.39&0.29&99.86&0.10\\
  \cline{3-17}
  ~&round&75\%&99.76&0.11&99.96&0.02&99.74&0.08&99.74&0.18&99.68&0.16&99.52&0.23&99.92&0.06\\
  \cline{3-17}
  200&random&100\%&99.80&0.09&99.96&0.02&99.80&0.07&99.78&0.14&99.73&0.14&99.60&0.21&99.91&0.06\\
  \cline{2-17}
  ~&two-~&50\%&99.33&0.24&99.77&0.09&99.50&0.14&99.68&0.20&99.60&0.19&99.33&0.33&99.78&0.16\\
  \cline{3-17}
  ~&round&75\%&99.57&0.17&99.85&0.05&99.64&0.11&99.75&0.15&99.72&0.14&99.53&0.24&99.85&0.10\\
  \cline{3-17}
  ~&random&100\%&99.70&0.15&99.94&0.03&99.71&0.09&99.81&0.11&99.80&0.12&99.69&0.18&99.85&0.11\\
  \hline
\end{tabular}
\vspace{-4ex}
\end{table*}


\vspace{-2ex}
\section{Results and Analysis}
\label{Results}

In this section, we present and analyze the results of comparison
to answer the two research questions. We further analyze the
ability of reduction in mutants for the three operator-based
mutant-selection techniques.


\vspace{-1ex}
\subsection{Effectiveness}
\label{OffuttEff}

Tables~\ref{tab:Ran-Offutt},~\ref{tab:Ran-Barbosa},
and~\ref{tab:Ran-SiamiNamin} depict the average effectiveness
values and standard-deviation values for comparing Offutt et al.'s
technique, Barbosa et al.'s technique, and Siami Namin et al.'s
technique with random mutant selection respectively\footnote{Due
to space limit, we use tables instead of figures to present the
experimental results. Tables are more concise but less intuitive
than figures. Also due to space limit, we put results on
effectiveness and standard deviation in each table.}. In the three
tables, we use $Incr$ to represent the four incremental numbers
for creating test suites for evaluating selected mutants; $Eff$ to
represent average effectiveness values measured by the metric
defined in Section~\ref{Measurement}; $Dev$ to represent
standard-deviation values among the corresponding effectiveness
values; and $50\%$, $75\%$, and $100\%$ to represent the use of
random mutant selection to select 50\%, 75\%, and 100\% of mutants
selected by each of the three operator-based mutant-selection
techniques. Both the effectiveness values and the
standard-deviation values are in percentage points. To make the
difference clear, we keep two numbers after the decimal point for
each value. From the three tables, we have the following
observations concerning the average effectiveness.

First, our results confirm that all the three operator-based
mutant-selection techniques are able to achieve good effectiveness
values using test suites created with all the four incremental
numbers. According to the criterion proposed by Offutt et
al.~\cite{Offutt:96}, a mutant-selection technique is good in
effectiveness if it achieves an average effectiveness value over
99\% using test suites created with incremental number 200. Based
on this criterion, all the three technique are good in
effectiveness, except for Siami Namin et al.'s technique on $tcas$
(i.e., 98.79\%).

Second, despite the effectiveness of the three operator-based
mutant-selection techniques, none of them is superior to random
mutant-selection when selecting the same number of mutants. Both
random techniques outperform Offutt et al.'s technique for four
subjects (i.e., $print\_tokens$, $replace$, $schedule2$, and
$tot\_info$) out of seven. This trend is consistent for all the
four incremental numbers. Both random techniques consistently
outperform Barbosa et al.'s technique for $print\_tokens$,
$replace$, and $tot\_info$ with different incremental numbers. For
$print\_tokens2$ and $schedule2$, the two random techniques
achieve almost the same average effectiveness as Barbosa et al.'s
technique. Both random techniques consistently outperform Siami
Namin et al.'s technique for $replace$, $tcas$, and $tot\_info$
with different incremental numbers. The two-round random technique
also achieves similar effectiveness values as Siami Namin et al.'s
technique for $schedule$ for different incremental numbers.
Overall, for any subject the difference between one operator-based
mutant-selection technique and its corresponding random
mutant-selection techniques is quite small. This observation
indicates that random mutant selection is still as competitive as
or even better than operator-based mutant-selection in terms of
average effectiveness.

Third, compared with the three operator-based mutant-selection
techniques, random mutant selection is able to achieve competitive
effectiveness when selecting fewer mutants. In general, for any
random technique and any subject, the differences between
selecting 50\%, 75\%, and 100\% mutants are quite small. Thus, the
difference between each operator-based mutant-selection technique
and its corresponding random techniques selecting 50\% or 75\%
mutants are also small. Based on Offutt et al.'s criterion, for
each operator-based mutant-selection technique, using its
corresponding random techniques to select 50\% is also good in
effectiveness, as the average effectiveness values are always over
99\% for all the subjects using test suites created with
incremental number 200. Since there is no sophisticated strategy
in random mutant selection, this observation indicates that it is
likely that mutants selected by either of the three operator-based
mutant-selection techniques are more than necessary.


Fourth, when comparing the two random techniques, the one-round
technique is less effective than the two-round technique for
smaller subjects, but more effective than the two-round technique
for larger subjects. We suspect the reason to be that the
one-round technique may fail to select any mutant for some
important mutation operators when the number of mutants is small.
Without mutants generated with these mutation operators, the
selected set of mutants is not as representative as those
containing mutants generated with all the different mutation
operators. However, for the two-round technique, the distribution
of selected mutants over the mutation operators may be very
different from the distribution of all the non-equivalent mutants
over the mutation operators when the subject becomes large. Thus,
mutants selected by the two-round technique may be less
representative than those selected by the one-round technique for
large subjects.

Finally, although different incremental numbers impact the
effectiveness values for any experimented techniques and any
subjects, the preceding observations are consistent for different
incremental numbers. Typically, with the increase of the
incremental number, the effectiveness value for any technique on
any subject also increases. We suspect the reason to be that
different incremental numbers result in different sizes of created
test suites and the differences in test-suite sizes lead to the
differences in effectiveness values. We further checked the
average sizes of test suites under different incremental numbers
and corroborated this suspicion. Table~\ref{tab:TestSuiteSize}
depicts average test-suite sizes for measuring Offutt et al.'s
technique on the seven subjects under different incremental
numbers. The trends in average test-suite sizes for other
experimented techniques are similar. Due to space limit, we do not
present the average test-suite sizes for them here. It should also
be noted that, due to the nature of the metric used in our study,
the test suites created with any incremental number usually are of
different sizes, and thus the average test-suite sizes are not
multiples of 25, 50, 100, or 200.

\begin{table}[t]
\caption{\label{tab:TestSuiteSize} Average test-suite sizes for
Offutt et al.'s} \centering \hspace*{-0.3cm}
\begin{tabular}{|c|c|c|c|c|}
  \hline
  % after \\: \hline or \cline{col1-col2} \cline{col3-col4} ...
  Program & Increment  &Increment  & Increment &Increment \\
  ~ & 25  &50  & 100 &200 \\
   \hline
  \hline
  PT&309 &479 &725 &1190\\
  \hline
  PT2&196 &307 &471 &725\\
  \hline
  RE&638 &1019 &1609 &2445\\
  \hline
  SC&237 &383 &591 &894\\
  \hline
  SC2&290 &464 &710 &970\\
  \hline
  TC&474 &700 &910 &1175\\
  \hline
  TI&185 &261 &351 &468\\
  \hline
\end{tabular}
\vspace{-4ex}
\end{table}

\vspace{-1.5ex}
\subsection{Stability}
\label{Stability}

Based on the standard-deviation values depicted in
Tables~\ref{tab:Ran-Offutt},~\ref{tab:Ran-Barbosa},
and~\ref{tab:Ran-SiamiNamin}, we have the following observations
concerning the comparison of stability for the three
operator-based mutant-selection techniques and random mutant
selection.

First, all the experimented techniques are quite stable in
effectiveness. For any experimented technique, the
standard-deviation values are typically less than 0.20 percentage
points, and it is rarely for standard-deviation values to be
larger than 0.30 percentage points. Most of those exceptionally
large standard-deviation values come from Siami Namin et al.'s
technique and random techniques with 50\% mutants on $tcas$, which
is the smallest subject. Siami Namin et al.'s technique is less
effective and less stable for $tcas$. Random techniques seem to be
less stable for smaller subjects but more stable for larger
subjects.

Second, similar to the observation on average effectiveness,
either of the three operator-based mutant-selection techniques is
not more stable than random mutant selection when selecting the
same number of mutants. Both random techniques are more stable
than Offutt et al.'s technique for $print\_tokens$, $replace$,
$schedule2$, and $tot\_info$; and are as stable as Offutt et al.'s
technique for $print\_tokens2$. Both random techniques are more
stable than Barbosa et al.'s technique for $print\_tokens$,
$replace$, and $tot\_info$; and are as stable as Barbosa et al.'s
technique for $print\_tokens2$,  $schedule2$, and $tcas$. Both
random techniques are more stable than Siami Namin et al.'s
technique for $schedule2$, $tcas$, and $tot\_info$; and are as
stable as Siami Namin et al.'s technique for $replace$. It is
interesting to note that a technique achieving better
effectiveness values is typically also more stable.

Third, when comparing the two random techniques, the one-round
technique seems to be less stable than the two-round technique for
smaller subjects, but more effective than the two-round technique
for larger subjects. This observation is in accordance with the
observation concerning the comparison of effectiveness for the two
random techniques. Furthermore, given a random mutant-selection
technique, selecting more mutants is more stable than selecting
fewer mutants.


\vspace{-1.5ex}
\subsection{Reduction in Mutants}
\label{Reduction}


\begin{table}[t]
\caption{\label{tab:Reduction} Ratio of selected mutants}
\centering \hspace*{-0.1cm}
\begin{tabular}{|c|c|c|c|}
  \hline
  % after \\: \hline or \cline{col1-col2} \cline{col3-col4} ...
  Program & Offutt  &Barbosa  & Siami Namin \\
  ~ & et al. & et al.  &  et al.\\
  \hline
  \hline
  PT&6.00 &14.89 &7.33\\
  \hline
  PT2&5.90 &18.89 &8.64\\
  \hline
  RE&6.84 &16.30 &6.62\\
  \hline
  SC&7.77 &14.11 &7.11\\
  \hline
  SC2&8.09 &14.84 &7.88\\
  \hline
  TC&6.07 &15.31 &7.08\\
  \hline
  TI&10.27 &17.92 &7.29\\
  \hline
  \hline
  Average&7.28 &16.04 &7.42\\
  \hline
\end{tabular}
\vspace{-4ex}
\end{table}

As the preceding results indicate that none of the three
operator-based mutant-selection techniques is superior to random
mutant selection in terms of either effectiveness or stability, we
further examine their ability to reduce the number of mutants.
Table~\ref{tab:Reduction} depicts the ratio of selected mutants to
all the non-equivalent mutants for each subject. From this table,
we have the following observations.

First, all the three operator-based mutant-selection techniques
are able to substantially reduce the number of mutants. Either
Offutt et al.'s technique or Siami Namin et al.'s technique is
able to reduce about 93\% mutants, while Barbosa et al.'s
technique is able to reduce 84\% mutants. Our result for Siami
Namin et al.'s technique is similar to that reported by Siami
Namin et al.~\cite{SiamiNamin:08} (i.e., 92.6\%). Our result for
Offutt et al.'s technique is different from that reported by
Offutt et al.~\cite{Offutt:96} (i.e., 78\%). We suspect the reason
to be that we considered much more mutation operators than Offutt
et al. Our result for Barbosa et al.'s technique is also different
from that reported by Barbosa et al.~\cite{Barbosa:01} (i.e.,
78\%). We suspect the reason to be that Barbosa et al. used 71
mutation operators (which were implemented in Proteum by 2001)
rather than 108 mutation operators in our study. This observation
also indicates that Barbosa et al.'s technique is much more
expensive than either Offutt et al.'s technique or Siami Namin et
al.'s technique.

Second, for each technique the ratio of selected mutants does not
differ much for a different subject. This observation indicates
that the reduction in mutants is highly predictable for any of
three operator-based mutant-selection techniques. Furthermore,
when applying random mutant selection in practice, we may use the
average ratio of selected mutants for an operator-based
mutant-selection technique as a guidance to determine the number
of mutants to select. For example, randomly selecting 7\% mutants
from mutants generated with the 108 Proteum mutation operators is
likely to achieve similar effectiveness and stability as Offutt et
al.'s technique.
