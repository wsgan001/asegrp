\vspace{-2ex}
\section{Discussion}
\label{Discussion}

In this section, we discuss the following issues related to our
study: threats to validity, cost of mutant selection, and some
further implications of our experimental results.

\vspace{-1.5ex}
\subsection{Threats to Validity}
\label{Threats} \vspace{-1.5ex}
\subsubsection{Internal Validity}
\label{Internal}

Threats to internal validity are concerned with uncontrolled
factors that may also be responsible for the results. In our
study, the main threat to internal validity is the possible faults
in our experiments and result analysis. To reduce this threat, we
used Proteum for mutant generation. Furthermore, we reviewed all
the code that we produced for our experiments and analysis before
conducting the experiments. Note that the experimented techniques
are all straightforward to implement and their implementation is
thus less likely to threaten the internal validity.

\vspace{-1.5ex}
\subsubsection{External Validity}
\label{External}

Threats to external validity are concerned with whether the
findings in our study are generalizable for other situations. The
main threat to external validity lies in the representativeness of
the subjects. To reduce this threat, we chose seven subjects
written in C and these subjects contain a wide range of data and
control structures commonly used in C (or even C++ and Java)
programs~\cite{SiamiNamin:08}. Conducting more experiments using
more subjects of larger sizes and with structures not contained in
our subjects is one way to further reduce this threat. Note that,
when experimenting with subjects containing object-oriented
structures, mutation operators on these structures~\cite{Ma:05}
should also be considered.

\vspace{-1.5ex}
\subsubsection{Construct Validity}
\label{Construct}

Threats to construct validity are concerned with whether the
measurement in our study reflects real-world situations. The main
threat to construct validity is the way we measured the
effectiveness of selected mutants. To reduce this threat, we used a
metric commonly used by previous studies, such as Offutt et
al.~\cite{Offutt:96} and Barbosa et al.~\cite{Barbosa:01}. Further
reduction of this threat requires the design of better metrics to
assess the effectiveness of selected mutants in mutation testing.
The metric used in our study actually measures to what extent the
selected mutants are representative for mutants generated with all
the mutation operators. Indeed, users of mutant-selection techniques
may be more concerned with the representativeness of the selected
mutants for real bugs. Such concerns may be alleviated by previous
empirical studies~\cite{Andrews:05,Do:06} showing evidence that mutants
generated with mutation operators are similar to real faults in
evaluating test effectiveness. But it is worthwhile of conducting
empirical studies directly on representativeness of the selected
mutants for real bugs in future work.  Furthermore, the metric used in our
study requires a series of randomly created test suites, but test
suites used in practice may not be created randomly.

\vspace{-1.5ex}
\subsection{Cost of Mutant Selection}
\label{Cost}

According to Offutt et al.~\cite{Offutt:96}, the expensiveness of
mutation testing mainly lies in the need to compile and execute a
large number of mutants against each test case. Compared with the
cost of compiling and executing mutants, the cost of mutant
selection is typically very small. In our study, we used Proteum
as the platform for mutant selection. In particular, we ran
Proteum on a PC with a Genuine Intel CPU T1400 at 1.83GHz and 1GB
memory running SUSE Linux (version 2.6.25.5-1.1) with the $tcsh$
shell (version 6.15.00). For each subject, Proteum generated the
selected mutants in a few seconds. For the smallest subject (i.e.,
$tcas$), the time is less than one second; for the largest subject
(i.e., $replace$), the time is less than four seconds; for each
other subject, the time is less than two seconds. But the total
time for us to compile and execute all the mutants for all the
seven subjects against all the test cases under the same hardware
and software environment is about one month CPU time. Note that
Offutt et al.'s, Barbosa et al.'s, and Siami Namin et al.'s
techniques averagely select 7.28\%, 16.04\%, and 7.42\% mutants,
respectively. That is to say, compiling and executing mutants
selected by one of the three operator-based mutant-selection
techniques for all the seven subjects against all the test cases
may take two to five days. As a result, for either operator-based
mutant selection or random mutant selection, there is little
difference in the cost of mutant selection.

One issue that we may need to consider is the way that Barbosa et al.'s
and Siami Namin et al.'s techniques use to determine mutant
operators. In particular, both techniques determine the set of
sufficient mutation operators using training data acquired through
compiling and executing mutants against test cases. That is to
say, both techniques may require some extra cost besides compiling
and executing the selected mutants. However, as their
cross-validation indicates that the set of sufficient mutation
operators determined with some programs may also be applicable for
other programs, the extra cost of either Barbosa et al.'s
technique or Siami Namin et al.'s technique should be considerably
small.

\vspace{-1.5ex}
\subsection{Further Implications}
\label{Implications}

Based on the findings of our study, we have the following
implications.

First, as the three operator-based mutant-selection techniques are
not superior to the two random mutant-selection techniques in
terms of either effectiveness or stability, random
mutant-selection techniques may be a better choice in practice due
to their flexibility in controlling the number of selected
mutants. Note that random mutant-selection techniques can achieve
similar effectiveness and stability even when selecting much fewer
mutants.

Second, the findings in our study also imply that we may need much
fewer mutants in mutation testing than those selected by existing
operator-based mutant-selection techniques. That is to say, it is
very likely to invent new mutant-selection techniques to select
much fewer mutants but with similar or even better effectiveness
and stability. Random mutant selection may be a good starting
point.

Third, considering the nature of random mutant selection, the
following difference between random mutant selection and
operator-based mutant selection may be an explanation of the
surprisingly good results of random mutant selection. Random
mutant selection selects mutants on the basis of individual
mutants, but operator-based mutant selection needs to include or
exclude all the mutants generated with one mutation operator as a
whole. If this difference accounts for the goodness of random
mutant selection, techniques that both consider the difference in
operators and select mutants individually may outperform existing
random and operator-based mutant-selection techniques.
