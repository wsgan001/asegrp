\vspace{-1ex}
\section{Related Work}
\label{RelatedWork}

In this section, we first review the literature on mutation
testing and analysis briefly, and then we discuss research on
mutant selection, which is closely related to our study in this
paper.

\vspace{-1ex}
\subsection{Mutation Testing and Analysis}
\label{RelatedMutation}

Mutation testing is a fault-based testing approach, which is first
proposed by Hamlet~\cite{Hamlet:77} and DeMillo et
al.~\cite{DeMillo:78}. In mutation testing, the primary aim is to
provide a rigorous test-adequacy criterion that can help enhance
test suites. For the software under test (SUT), a tester can use
mutation operators to generate a number of mutants. After
identifying equivalent mutants, the tester can use the remaining
non-equivalent mutants to enhance test suites. That is to say, if
a test suite cannot kill all the remaining non-equivalent mutants,
more test cases may be required to enhance the test suite. Another
usage of mutation is mutation analysis, whose aim is not to
enhance test suites but to provide assessment of test
effectiveness to facilitate experiments in testing research. It is
interesting to note that researchers such as Briand et
al.~\cite{Briand:04} have already used mutation faults to measure
test effectiveness even before the empirical confirmation of its
appropriateness by Andrews et al.~\cite{Andrews:05} and Do et
al.~\cite{Do:06}.

For both mutation testing and analysis, a major concern is the
expensiveness of compiling and executing a large number of
mutants. In the literature, there are mainly four categories of
techniques to reduce this cost. The first category is to select a
subset of mutants as substitute. As our research in this paper
falls into this category, we detailedly discuss research in this
category in Section~\ref{RelatedSelection}. The second category is
to use low-cost ways to determine which test case kills which
mutant. Weak mutation~\cite{Howden:82} and firm
mutation~\cite{Woodward:88} are two representative techniques in
this category. The third category is to use efficient ways to
generate, compile and execute mutants. As mutants only slightly
differ from the original program, taking the advantage of the
commonalities of the mutants may accelerate the generation,
compilation and execution of the mutants. Compiler-integrated
mutation~\cite{DeMillo:91} and schema-based
mutation~\cite{Untch:93} are two representative techniques in this
category. The last category is to compile and execute mutants in
parallel. Researchers have investigated parallel compilation and
execution of mutants on different computer
architectures~\cite{Krauser:88,Offutt:92}. Techniques in different
categories are typically complementary to each other, as they can
be combined together to reduce the cost of mutant compilation and
execution.

The second concern in mutation testing and analysis is the cost in
identifying equivalent mutants. In the literature, researchers
proposed several techniques for detecting equivalent mutants
statically~\cite{Offutt:94,Offutt:97,Hierons:99} or
dynamically~\cite{Grun:09,Schuler:09}. For mutation testing, there
is still another concern: the cost of acquiring mutation-adequate
test suites. In the literature, researchers also proposed some
techniques to automatically generate mutation-adequate test
suites~\cite{DeMillo:91b,Offutt:99,Liu:06}. Techniques addressing
both concerns are also complementary to research on mutant
selection. As our research in this paper mainly considers mutant
selection, further discussion related to these two concerns is out
of the scope of this paper.

\vspace{-1ex}
\subsection{Mutant Selection}
\label{RelatedSelection}

As mentioned previously, mutant selection is an important way to
reduce the cost of mutation testing and analysis. Since
Mathur~\cite{Mathur:91} first proposed the idea of excluding some
mutation operators in mutation testing, several researchers have
studied operator-based mutant selection. In operator-based mutant
selection, researchers first determine a set of mutation
operators, and select only mutants generated with this set of
mutation operators.

Wong and Mathur~\cite{Wong:93,Wong:95} studied two mutation
operators among the 22 mutation operators in
Mothra~\cite{DeMillo:87}, and found mutants generated with these
two mutation operators can achieve similar results as mutants
generated with all the 22 mutation operators. Offutt et
al.~\cite{Offutt:96} experimentally determined five mutation
operators among the 22 mutation operators in Mothra, and found
that the effectiveness values of the five mutation operators are
between 99.0\% and 100\% on ten subjects, with the average 99.5\%.
Note that the effectiveness values are based on test suites
created with incremental number 200. Furthermore, Offutt et al.
also found that, without any of the five mutation operators, the
effectiveness value on some subject would be lower than 99\%,
which they defined as the minimal requirement of a set of
sufficient mutation operators for mutation testing. Note that Wong
and Mathur's two mutation operators are among Offutt et al.'s five
mutation operators. Barbosa et al.~\cite{Barbosa:01} proposed six
guidelines to determine sufficient mutation operators. The
application of some of the six guidelines requires compilation and
execution of a large number of mutants. Based on the six
guidelines, Barbosa et al. determined 10 mutation operators, and
found that, using the same way as Offutt et al. to measure the
effectiveness, the effectiveness values of the 10 mutation
operators on 27 subjects (which were also used to determine the 10
mutation operators) are between 95.8\% and 100\%, with the average
99.6\%. Siami Namin et al.~\cite{SiamiNamin:08} leveraged variable
reduction to determine 28 mutation operators using the execution
information of a subset of mutants. As their work aims at mutation
analysis rather than mutation testing, they evaluated the 28
mutation operators on the Siemens programs only in the context of
mutation analysis. They did not provide evidence about whether the
28 operators are sufficient in mutation testing.

Compared with studies on operator-based mutation selection,
studies on random mutant selection, which Acree et
al.~\cite{Acree:79} first proposed in 1979, are limited. Wong and
Mathur~\cite{Wong:93,Wong:95} empirically studied the technique of
randomly selecting $10\%$ to $40\%$ mutants generated with 22
mutation operators in Mothra. Barbosa et al.~\cite{Barbosa:01}
used random mutant selection as a control technique when
evaluating their 10 mutation operators. In their study, Barbosa et
al.'s 10 mutation operators are more effective than random mutant
selection. Our study differs from previous studies on random
mutant selection for mutation testing as follows. First, our study
investigates some operator-based techniques (i.e.,Offutt et al.'s
technique~\cite{Offutt:96} and Siami Namin et al.'s
technique~\cite{SiamiNamin:08}) previously not empirically
compared with random mutant selection, and our results on Barbosa
et al.'s technique~\cite{Barbosa:01} contradict previous results.
Second, our study investigates two random mutant-selection
techniques, while previous studies investigated only one random
mutant-selection technique. Third, our study uses larger subjects
than previous studies on random mutant selection. Finally, our
study investigates both average effectiveness and standard
deviation of effectiveness, while previous studies investigate
only average effectiveness.
