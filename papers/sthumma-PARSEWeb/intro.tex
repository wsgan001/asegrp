%-------------------------------------------------------------------------
\section{Introduction}
\label{sec:introduction}

The primary goal of software development is to deliver high-quality
software efficiently and in the least amount of time whenever
possible. To achieve the preceding goal, programmers often want to
reuse existing frameworks or libraries instead of developing similar
code artifacts from scratch. The challenging aspect for
programmers in reusing the existing frameworks or libraries is to
understand the usage of Application Programming Interfaces (APIs)
exposed by those frameworks or libraries, because many of the
existing frameworks or libraries are not well-documented. Even when
such documentations exist, they are often outdated~\cite{document:leth}.

In general, the reuse of existing frameworks or libraries involve
instantiation of several object types of those frameworks or
libraries. For example, consider the programming task of parsing
code in a dirty editor (editor whose content is not yet saved) of
the Eclipse IDE framework. As a dirty editor is represented as an
object of the \CodeIn{IEditorPart} type and the programmer needs an
object of \CodeIn{ICompilationUnit} for parsing, the programmer has
to identify a method sequence that takes the \CodeIn{IEditorPart} object
as input and results in an object of \CodeIn{ICompilationUnit}. One
such possible method sequence is shown below:

\begin{CodeOut}
\begin{alltt}
IEditorPart iep = ...
IEditorInput editorInp = iep.getEditorInput();
IWorkingCopyManager wcm = JavaUI.getWorkingCopyManager();
ICompilationUnit icu = wcm.getWorkingCopy(editorInp);
\end{alltt}
\end{CodeOut}

The code sample shown above exhibits the difficulties faced by
programmers in reusing the existing frameworks or libraries. A
programmer unfamiliar to Eclipse may take
long time to identify that an \CodeIn{IWorkingCopyManager} object 
is needed for getting the
\CodeIn{ICompilationUnit} object from an object of the
\CodeIn{IEditorInput} type. Furthermore, it is not trivial to find
an appropriate way of instantiating the
\CodeIn{IWorkingCopyManager} object as the instantiation requires a static
method invocation on the \CodeIn{JavaUI} class.

In many such situations, programmers know what type of object that
they need to instantiate (like \CodeIn{ICompilationUnit}), but do
not know how to write code to get that object from a known object
type (like \CodeIn{IEditorPart}). For simplicity, we refer the known
object type as \emph{Source} and the required object type as
\emph{Destination}. Therefore, the proposed problem can be
translated to a query of the form ``\emph{Source} $\rightarrow$
\emph{Destination}''. There are several existing
approaches~\cite{prospector:jungloid, strathcona:se, xsnippet:saha}
that address the described problem. But the common issue faced by
these existing approaches is that the scope of these approaches is
limited to the information available in a fixed (often small) set of
applications reusing the frameworks or libraries of interest.
%-------------------------------------------------------------------------
\begin{figure*}[t]
\begin{CodeOut}
\begin{alltt}
\hspace*{0.6in}01:\textbf{FileName}:0\_UserBean.java \textbf{MethodName}:ingest  \textbf{Rank}:1 \textbf{NumberOfOccurrences}:6
\hspace*{0.6in}02:QueueConnectionFactory,createQueueConnection() \textbf{ReturnType}:QueueConnection
\hspace*{0.6in}03:QueueConnection,createQueueSession(boolean,Session.AUTO\_ACKNOWLEDGE) \textbf{ReturnType}:QueueSession
\hspace*{0.6in}04:QueueSession,createSender(Queue) \textbf{ReturnType}:QueueSender
\end{alltt}
\end{CodeOut}
\vspace*{-4ex} \Caption{\label{fig:sampleoutput} Method sequence
suggested by PARSEWeb.}
%-------------------------------------------------------------------------
%\begin{figure}[t]
\begin{CodeOut}
\begin{alltt}
\hspace*{0.6in}01:QueueConnectionFactory qcf;
\hspace*{0.6in}02:QueueConnection queueConn = qcf.createQueueConnection();
\hspace*{0.6in}03:QueueSession qs = queueConn.createQueueSession(true,Session.AUTO\_ACKNOWLEDGE);
\hspace*{0.6in}04:QueueSender queueSender = qs.createSender(new Queue());
\end{alltt}
\end{CodeOut}\vspace*{-4ex}
\Caption{\label{fig:javacode} Equivalent Java code for the method sequence suggested by PARSEWeb.} \vspace*{-3ex}
\end{figure*}
%-------------------------------------------------------------------------
Many code search engines (CSE) such as Google~\cite{GCSE} and
Koders~\cite{KODERS}\Comment{, and Krugle~\cite{KRUGLE}} are
available on the web. These CSEs can be used to assist programmers
by providing relevant code examples with usages of the given query
from a large number of publicly accessible source code repositories.
For the preceding example, programmers can issue the query
``\CodeIn{IEditorPart} \CodeIn{ICompilationUnit}'' to gather
relevant code samples with usages of the object types
\CodeIn{IEditorPart} and \CodeIn{ICompilationUnit}. However, these
CSEs are not quite helpful in addressing the described problem
because we observed that Google Code Search Engine (GCSE)
~\cite{GCSE} returns nearly $100$ results for this
query and the desired method sequence shown above is present in the
$25^{th}$ source file among those results.

Our approach addresses the described problem by accepting
queries of the form ``\emph{Source} $\rightarrow$
\emph{Destination}'' and suggests frequently used Method-Invocation
Sequences (MIS) that can transform an object of the \emph{Source}
type to an object of the \emph{Destination} type. Our approach also
suggests relevant code samples that are extracted from a large
number of publicly accessible source code repositories. These
suggested MISs along with the code samples can help programmers in
addressing the described problem and thereby help reduce
programmers' effort in reusing existing frameworks or libraries.

We have implemented the proposed approach with a tool called
PARSEWeb for helping reuse Java code. PARSEWeb interacts with
GCSE~\cite{GCSE} to search for code samples with the usages of the
given \emph{Source} and \emph{Destination} object types, and
downloads the code example results to form a local source code
repository. PARSEWeb analyzes the local source code repository to
extract different MISs and clusters similar MISs using a
sequence postprocessor. These extracted MISs can serve as a solution for
the given query. PARSEWeb also sorts the final set of MISs using
several ranking heuristics. PARSEWeb uses an additional heuristic
called query splitting that helps address the problem where
code samples for the given query are split among different source
files.

This paper makes the following main contributions:\vspace*{-1ex}
\begin{itemize}
\item An approach for reducing programmers' effort while reusing existing frameworks or libraries by providing frequently used MISs and relevant code samples.
\item A technique for collecting relevant code samples dynamically from the web. This contribution has an added advantage of not limiting the scope of the approach to any specific set of frameworks or libraries.
\item A technique for analyzing partial code samples through Abstract Syntax Trees (AST) and Directed Acyclic Graphs (DAG) that can handle control-flow information and method inlining.
\item An Eclipse plugin tool implemented for the proposed approach
and several evaluations to assess the effectiveness of the tool.
\end{itemize}

The rest of the paper is organized as follows. Section
~\ref{sec:example} explains the approach through an example.
Section~\ref{sec:relatedwork} presents related work.
Section~\ref{sec:summary} describes key aspects of the approach.
Section~\ref{sec:implementation} describes implementation details.
Section~\ref{sec:evaluation} discusses evaluation results.
Section~\ref{sec:discussion} discusses limitations and future work.
Finally, Section~\ref{sec:conclusion} concludes.\vspace*{-1ex}
%-------------------------------------------------------------------------
