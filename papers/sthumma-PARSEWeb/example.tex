\section{Example}
\label{sec:example}
We next use an example to illustrate our approach and how our
approach can help in reducing programmers' effort when reusing
existing frameworks or libraries. We use object types
\CodeIn{QueueConnectionFactory} and \CodeIn{QueueSender} from the
OpenJMS library\footnote{\url{http://java.sun.com/products/jms}},
which is an open source implementation of Sun's Java Message Service
API 1.1 Specification. \Comment{This library can be used to send and receive
messages in both synchronous and asynchronous modes. }Consider that a
programmer has an object of type \CodeIn{QueueConnectionFactory} and
does not know how to write code to get a \CodeIn{QueueSender}
object, which is required for sending messages.

To address the problem, the programmer can use our PARSEWeb tool
(implemented for our approach) in the following way. The
programmer first translates the problem into
a ``\CodeIn{QueueConnectionFactory} $\rightarrow$
\CodeIn{QueueSender}'' query. Given this query, PARSEWeb
requests GCSE for relevant code samples with usages of the given
\emph{Source} and \emph{Destination} object types, and downloads the
code samples to form a local source code repository. The downloaded
code samples are often partial and not compilable as GCSE retrieves
(and subsequently PARSEWeb downloads) only source files with usages
of the given object types instead of entire projects. PARSEWeb
analyzes each partial code sample using an Abstract Syntax Tree
(AST) and builds a Directed Acyclic Graph (DAG) that represents each
given code sample in order to capture control-flow information in
the code example. PARSEWeb traverses this DAG to extract MISs that
take \CodeIn{QueueConnectionFactory} as input and result
in an object of \CodeIn{QueueSender}. The output of
PARSEWeb for the given query is shown in Figure~\ref{fig:sampleoutput}.
The sequence starts with the invocation of
the \CodeIn{createQueueConnection} method that results in
an instance of \CodeIn{QueueConnection} type. Similarly, by following
other method invocations, the method sequence finally results in the
\CodeIn{QueueSender} object, which is the desired
destination object of the given query.

The sample output also shows additional details 
such as \CodeIn{FileName}, \CodeIn{MethodName}, \CodeIn{Rank}, and \CodeIn{NumberOfOccurrences}. 
The details \CodeIn{FileName} and
\CodeIn{MethodName} indicate the source file that the programmer can
browse to find a relevant code sample for this MIS. For example, a
code sample of the given query can be found in method
\CodeIn{ingest} of file \CodeIn{0\_UserBean.java}. The prefix of the
file name gives the index of the source file that contained the
suggested solution among the results of GCSE. In this example, the
code sample for the suggested method-invocation sequence can be
found in the first source file returned by GCSE as the prefix value
is zero. Generally, many queries result in more than one possible
solution. The \CodeIn{Rank} attribute gives the rank of the
corresponding MIS among the complete set of results. PARSEWeb
derives the rank of a MIS based on the \CodeIn{NumberOfOccurrences}
attribute and some other heuristics described in
Section~\ref{sec:rankingCriteria}. The suggested MIS contains all
necessary information for the programmer to write code for getting
the \emph{Destination} object from the given \emph{Source} object.
The suggested MIS can be transformed to equivalent Java code by
introducing required intermediate variables. The code sample
suggested along with the MIS can assist programmers in gathering
this additional information regarding the intermediate variables.
Figure~\ref{fig:javacode} shows equivalent Java code for the
suggested MIS.
