%-----------------------------------------------------------------------------
\section{Discussion and Future work}
\label{sec:discussion} In our current implementation, PARSEWeb
interacts with Google Code Search Engine~\cite{GCSE} for gathering
relevant code samples. Therefore, our current results are dependent
on code samples returned by GCSE. We observed that PARSEWeb is not
able to suggest solutions for some of the queries due to lack of
relevant code samples in the results returned by GCSE. We found this
limitation when we evaluated with open source projects
\emph{ASTView}\footnote{\url{http://www.eclipse.org/jdt/ui/astview/index.php}}
and
\emph{Flow4j}\footnote{\url{http://sourceforge.net/projects/flow4jeclipse}},
where Prospector or Strathcona also could not work well. We
alleviated this problem to a certain extent through the query
splitting heuristic. To address the problem further, We plan to
extend our tool to collect code samples from other CSEs such as
Koders~\cite{KODERS}\Comment{ and Krugle~\cite{KRUGLE}}, and analyze
the results to compare different CSEs. We expect that the addition
of new source code repositories to CSEs can also help in alleviating
the current problem.

As our approach deals with code samples that are often partial and not
compilable, there are a few limitations in inferring the information
related to method invocations. We explain the limitation of our
approach through the code sample shown below:

\vspace*{-1ex}
\begin{CodeOut}
\begin{alltt}
QueueConnectionFactory factory = jndiContext.lookup("t");
QueueSession session = factory.createQueueConnection()
\hspace*{0.2in}.createQueueSession(false,Session.AUTOACKNOWLEDGE);
\end{alltt}
\end{CodeOut}
\vspace*{-1ex}

In the second expression of this code sample, our heuristics cannot
infer the receiver type of the \CodeIn{createQueueSession} method
and the return type of the \CodeIn{createQueueConnection} method.
The reason is lack of information regarding the intermediate object
returned by the \CodeIn{createQueueConnection} method. If this intermediate
object is either \emph{Source} or \emph{Destination} of the given
query, we cannot suggest this MIS as a solution. 
Because of this limitation, our approach could not suggest
solutions to some of the queries used in evaluation.

Another issue addressed by our approach is the processing of data
returned by GCSE~\cite{GCSE}. A query such as
``\CodeIn{Enumeration} $\rightarrow$ \CodeIn{Iterator}'' results in
nearly $22,000$ entries. To avoid high runtime costs of downloading and analyzing
the code samples, PARSEWeb downloads only a fixed number of
samples, say $N$. This parameter $N$ is made as a configurable
parameter. The default value of $N$ is set to $200$, which is derived
from our experiments and identified to be large enough to make sure
that only little useful information is lost during downloading.

The current implementation of PARSEWeb can also accept queries
including only the \emph{Destination} object type. Such queries can
help programmers if they are not aware of the \emph{Source} object
type. However, the number of results returned by PARSEWeb for such
queries are often large ($10$ to $50$) due to lack of knowledge of
the \emph{Source} object type. Therefore, to reduce the number of
results and help programmers in quickly identifying the desired MIS,
we plan to extend our tool to infer the \emph{Source} object type
from the given code context. We also plan to extend our tool to
automatically generate compilable source code from the selected MIS.
 \Comment{Our current implementation cannot effectively handle
sequences that involve primitive data types as connector objects
between the given \emph{Source} and \emph{Destination} object types.
We plan to extend our approach to match on variable names also along
with data types to effectively handle primitive data types.}
