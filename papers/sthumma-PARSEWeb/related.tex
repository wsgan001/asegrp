%-------------------------------------------------------------------------
\section{Related Work}
\label{sec:relatedwork}
The problems faced by programmers in reusing existing frameworks or
libraries are addressed by different approaches. Earlier research in
this area mainly concentrated on identifying related samples by
matching keywords~\cite{softwarefactory:mat} or comments~\cite{sfreuse:ye
\Comment{,integrate:ye}}. But solutions suggested based on these approaches
often cannot effectively help programmers in reusing the existing
code samples.

Mandelin et al. developed Prospector~\cite{prospector:jungloid}, a
tool that accepts queries in the form of a pair ($T_{in}, T_{out}$),
where $T_{in}$ and $T_{out}$ are class types,
and suggests solutions by traversing all paths among types of
API signatures between $T_{in}$ and
$T_{out}$. A solution to the query is a synthesized code sample that takes 
an input object of type $T_{in}$ and returns an
output object of type $T_{out}$. Their approach uses API signatures for suggesting
solutions to the given query. As API signatures are used for
addressing the query, Prospector returns many irrelevant examples,
as shown in our evaluation. Our approach is different
from Prospector as Prospector uses API signatures, whereas our
approach uses code samples for solving the given query. This feature
helps PARSEWeb in identifying more relevant code samples by giving
higher preference to code samples that are often used.\Comment{
Another approach by Zaremski and Wing~\cite{signature:zaremski} also
uses signature matching techniques and has similar problems.
Ans: I took this from XSnippet. I am also not sure. I will check in detail,
and will add it later, if it is related.}

Strathcona developed by Holmes and Murphy~\cite{strathcona:se}
maintains an example repository and compares the context of the code
under development with the code samples in the example repository,
and recommends relevant examples. Both PARSEWeb and Strathcona
suggest relevant code samples, but the Strathcona approach is
based on heuristics that are generic and are not tuned for
addressing the described problem. This limitation often results in
irrelevant examples as shown in our evaluation. \Comment{Another related
approach by Sindhgatta~\cite{examples:renuka} also suggests relevant
code examples from a preprocessed repository. One main objective of
this approach is to address the scalability issues in maintaining
the repository. Our approach does not face the scalability issue in
repository maintenance as we do not maintain any repositories.}
XSnippet developed by Sahavechaphan and Claypool~\cite{xsnippet:saha} also tries to
address the described problem by suggesting relevant code snippets
for the object instantiation task at hand. These suggested code snippets are selected
from a sample repository. The major problem with both Strathcona and XSnippet
is the availability of limited code samples stored in the repository.

Coogle developed by Sager et al.~\cite{sager-msr06-clsim} extends
the concept of similarity measures (often used to find similar
documents for a given query) to source code repositories. Their
approach detects similar Java classes in software projects using
tree similarity algorithms. Both PARSEWeb and Coogle use ASTs for
parsing Java code, but a structural similarity at the Java class
level may not effectively address the described problem. Similar to
Strathcona, Coogle may also result in many irrelevant examples.

Another related tool MAPO, developed by Xie and Pei~\cite{mapo:xie},
generates frequent usage patterns of an API by extracting and mining
code samples from open source repositories through CSEs. Both MAPO and PARSEWeb exploit CSEs for
gathering relevant code samples, but MAPO cannot solve queries of
the form ``\emph{Source} $\rightarrow$ \emph{Destination}''.
Programmers need to know the API to be used for using MAPO to
identify usage patterns of that API. Moreover, our approach is more
effective than MAPO as we consider control-flow information while
generating MISs for the given query, whereas MAPO does not consider
the control-flow information.
