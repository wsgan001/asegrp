\section{Evaluations}
\label{sec:evaluation} We implemented a tool named MAM based on our
approach and conducted two evaluations on the tool. Our evaluations
focus on two research questions as follows:
\vspace*{-1.5ex}\begin{enumerate}
\item How effective can our approach mine various mapping relations of APIs
(Section~\ref{sec:evaluation:mining})? \vspace*{-1.5ex}
\item How much benefit can the mined mapping relations of APIs offer in aiding language
migration (Section~\ref{sec:evaluation:migration})?\vspace*{-1.5ex}
\end{enumerate}%\vspace*{-1.5ex}
\begin{table}[t]
\centering
\begin{SmallOut}
\begin {tabular} {|c!{\vrule width 1pt}c!{\vrule width 1pt}c|c!{\vrule width 1pt}c|c|c|c|}
 \hline
\multirow{2}*[-2pt]{\textbf{Project}}&
\multirow{2}*[-2pt]{\textbf{Source}}
& \multicolumn{2}{|c|}{\textbf{Java version}} & \multicolumn{2}{|c|}{\textbf{C\# version}} \\\cline{3-6} &  &  \emph{\textbf{\#C}}& \emph{\textbf{\#M}} &\emph{\textbf{\#C}}& \emph{\textbf{\#M}}\\
\hline \hline
  neodatis &   SourceForge  & 1298 &   9040 & 464  & 3983     \\
\hline
  db4o     &   SourceForge  & 3047 &   17449 & 3051 & 15430     \\
\hline
numerics4j &   SourceForge  & 145 &   973   & 87   & 515     \\
\hline
  fpml     &   SourceForge  & 143  &   879   & 144  & 1103     \\
\hline
  PDFClown &   SourceForge  & 297  &   2239  & 290  & 1393     \\
\hline
  OpenFSM &   SourceForge   & 35   &   179   & 36  & 140    \\
\hline
binaryNotes &   SourceForge  & 178 &   1590  & 197 & 1047    \\
\hline
  lucene   &    Apache      & 1298 &   9040  & 464& 3015    \\
\hline
  logging  &    Apache      & 196 &   1572   & 308 & 1474     \\
\hline
  hibernate &   hibernate  & 3211 &   25798 & 856 & 2538     \\
\hline \hline
  rasp     &   SourceForge  & 320  &   1819  & 557 & 1893     \\
\hline
  llrp     &   SourceForge  & 257 &   3833   & 222 & 978   \\
\hline
 simmetrics &   SourceForge & 107 &   581    & 63 & 325     \\
\hline
  aligner &   SourceForge  & 41    &   232 & 18 & 50     \\
\hline
  fit       &   SourceForge  & 95 &   461    & 43 & 281     \\
\hline\hline
  \multicolumn{2}{|c|}{Total} &   11668 & 75685  &   6900 & 34165   \\
\hline
\end{tabular}\vspace*{-2ex}
\Caption{Subjects} \label{table:subjects}
\end{SmallOut}\vspace*{-3ex}
\end{table}

We choose 15 open source projects that have both Java versions and
C\# versions as the subjects of our evaluations, and
Table~\ref{table:subjects} show these subjects. Column ``Project''
lists names of subjects. Column ``Source'' lists sources of these
subjects. These subjects come from famous open source societies such
as SourgeForge\footnote{\url{http://www.sf.net}},
Apache\footnote{\url{http://www.apache.org/}}, and
hibernate\footnote{\url{http://www.hibernate.org/}}. Columns ``Java
version'' and ``C\# version'' list the two versions from each
subject. All these used versions are the latest versions at the time
of writing. For these two columns, sub-column ``\emph{\#C}'' lists
numbers of classes, and sub-column ``\emph{\#M}'' lists numbers of
methods. We notice that Java versions are much larger than C\#
versions totally. We investigate these projects and find two factors
as follows. One is that Java versions of some projects are more
update-to-date. For example, the latest Java version of
\emph{numericas4j} is 1.3 whereas the latest C\# version is 1.2. The
other factor is that some projects are migrating from Java to C\# in
progress. For example, the
website\footnote{\url{http://wiki.neodatis.org/}} of \emph{neodatis}
states that \emph{neodatis} is a project in Java and is being ported
to C\#. This observation further confirms the usefulness of our
approach as our approach aids migrating from one language to other
languages. Totally, these projects have 18,568 classes and 109,850
methods.

We conducted all the evaluations on a PC with an Intel Qual CPU @
2.83GHz and 1.98M memory running Windows XP.
\subsection{Mining API mapping}
\label{sec:evaluation:mining}

To evaluate the first research question, we use 10 projects from
Table~\ref{table:subjects} as the subjects for mining API mapping.


\textbf{Aligning client code.} We first use our approach to align
client code. The threshold is set to 0.6 based on our initial
experience. We choose a relatively low threshold so that our
approach can take into account as much client code as possible.

Table~\ref{table:analyzingclient} shows the results of this step.
For column ``Aligned'', sub-column ``\emph{\# C}'' lists numbers of
aligned classes, and sub-column ``\emph{\# M}'' lists numbers of
aligned methods. For each project of Column ``C\# version'' and
Column ``Java version'', sub-column ``\emph{\%C}'' lists the
percentage of the aligned classes among total classes of
corresponding versions. Sub-column ``\emph{\%M}'' lists the
percentage of the aligned methods among total methods of
corresponding versions. Row ``Total'' of the two sub-columns lists
the percentage of aligned methods/classes among the total
methods/classes as shown in Table~\ref{table:subjects}. We find that
the results of Table~\ref{table:analyzingclient} fall into three
categories. This first category includes \emph{db4o}, \emph{fpml},
\emph{PDFClown}, \emph{OpenFSM}, and \emph{binaryNotes}. There, our
approach achieves relatively high percentages for both Java versions
and C\# versions. For each of the five project, ``\emph{\%M}'' is
relatively smaller than ``\emph{\%C}'' since methods of those
unaligned classes cannot be aligned and thus are counted as
unaligned\footnote{Another factor lies in that Java versions usually
have many getters and setters and these getters and setters often do
not have corresponding methods in C\# versions.}. The second
category includes \emph{neodatis}, \emph{numerics4j}, and
\emph{lucene}. There, our approach aligns C\# versions well but does
not align Java versions so well. We find that \emph{neodatis} and
\emph{lucene} are migrating from Java to C\# in progress and the
Java version of \emph{numerics4j} is more update-to-date than its
C\# version. As a result, some Java classes or methods do not have
corresponding implementations in C\# versions in these projects and
thus are left unmapped. The third category includes \emph{logging}
and \emph{hibernate}. There, our approach does not align classes and
methods of the two projects well. Although both of the two projects
seem to be migrated from existing Java versions, the programmers of
the two projects often do not refer to names of existing Java
versions for naming entities. For each of the two projects, the
percentage of aligned classes is relatively high, and the percentage
of aligned methods is relatively low. We find that even if our
approach aligns a wrong class pair, our approach does not align
methods within the wrong pair as the method names of a wrong pair
are quite different. The result suggests that we can take method
names into account when aligning classes in future work.

For all these projects, our approach does not align all classes and
all methods. We find another two factors besides the factor of
different entities naming across languages. First, one functionality
may be implemented as a single class in one language version and is
implemented as multiple classes in the other language version.
Second, a Java version and a C\# version sometimes may have quite
different functionalities. We discuss these issues in
Section~\ref{sec:discuss}.
\begin{table}[t]
\centering
\begin{SmallOut}
\begin {tabular} {|c!{\vrule width 1pt}c|c!{\vrule width 1pt}c|c!{\vrule width 1pt}c|c|c|c!{\vrule width 1pt}c|c|c|c|}
 \hline
\multirow{2}*[-2pt]{\textbf{Project}}&
\multicolumn{2}{|c|}{\textbf{Java version}}
& \multicolumn{2}{|c|}{\textbf{C\# version}} & \multicolumn{2}{|c|}{\textbf{Aligned}} \\\cline{2-7} &  \emph{\textbf{\%C}}& \emph{\textbf{\%M}}&  \emph{\textbf{\%C}}& \emph{\textbf{\%M}} &\emph{\textbf{\#C}}& \emph{\textbf{\#M}}\\
\hline\hline
  db4o     &   87.8\%  &  65.5\%&    87.6\%  & 74.1\% & 2674  & 11433   \\
\hline
  fpml       & 93.7\%  &  70.5\% & 93.5\%  & 56.2\% &  134 &  620  \\
\hline
  PDFClown   & 86.5\%  &  51.0\% & 88.6\%  & 82.1\% & 257  & 1143  \\
\hline
  OpenFSM    & 97.1\%  &  72.1\% &  94.4\% &  92.1\%  & 34   & 129   \\
\hline
  binaryNotes & 98.9\%  & 61.1\% &  89.3\% &  92.7\% & 176   & 971  \\
\hline \hline
  neodatis &   44.7\%  &  54.8\% &   100.0\% & 93.6\% & 408  & 3728   \\
\hline
  numerics4j & 57.2\%  &  48.6\% & 95.4\%  & 89.9\% &  75  &  174   \\
\hline
  lucene     &  34.9\%  & 26.6\%&   97.6\% &  79.8\% & 453  & 2406  \\
\hline\hline
  logging    &   91.8\% & 18.1\%&   58.4\% & 19.3\% &  180  & 285   \\
\hline
  hibernate &   26.4\%  & 1.2\%&    99.1\% & 12.6\% &  848  & 319   \\
\hline\hline
  Average &   53.2\% & 30.8\% &   88.8\% & 69.2\% & 524 & 2121 \\
\hline
\end{tabular}\vspace*{-2ex}
\Caption{Results of Aligning client code}
\label{table:analyzingclient}
\end{SmallOut}\vspace*{-4ex}
\end{table}

In summary, as shown by Row ``Average'', our approach aligns most
classes and methods on average. The result confirms that many
programmers refer to existing versions of another language to name
entities of a version under development.

\textbf{Mining API mapping.} We then use our approach to mine
mapping relations of API classes and API methods.

Table~\ref{table:minedresults} shows the results of this step. For
Columns ``Class'' and ``Method'', sub-column ``\emph{Num.}'' lists
numbers of mined mapping relations. The numbers of mined API mapping
are largely proportional to the sizes of projects as shown in
Table~\ref{table:subjects} except \emph{logging} and
\emph{hibernate}. As classes and methods of these two projects are
not quite well aligned, our approach does not mine many mapping
relations of APIs from the two projects. For the remaining projects,
our approach mines many mapping relations of API classes and API
methods. Sub-column ``\emph{Acc.}'' lists accuracies of the top 30
mined API mapping (\emph{i.e.}, percentages of correct mapping
relations). For mined API mapping from each project, we manually
inspect top 30 mined mapping relations of APIs and classify them as
correct or incorrect based on programming experiences. We find that
our approach achieves high accuracies except \emph{hibernate}.
Although our approach does not align \emph{logging} quite well
either, the accuracies of API mapping from \emph{logging} are still
relatively high. To mine API mapping of classes, our approach
requires that names of classes, methods, and variables are all
similar. To mine API mapping of methods, our approach requires that
two built API transformation graphs are similar. The two
requirements are relatively strict. As a result, if the first step
does not align client code quite well, our approach misses some
mapping relations of APIs but does not introduce many false mapping
relations. In other words, our approach is robust to mine accurate
API mapping. Sub-column ``\emph{J2SE}'' lists mined API mapping
between J2SE and NET. We next compare these mapping relations with
manually built mapping relations.

Row ``Total'' lists the total result after we merge all duplicated
mapping relations. In summary, our approach mines a large number of
mapping relations of APIs totally. These mined mapping relations are
accurate and cover various libraries.
%Sub-column ``\emph{1-1}'' list the number of one-to-one relations.
%We find that most mined mapping relations are of this kind. Our
%approach still needs improvements to mine those many-to-many mapping
%relations. We further discuss this issue in
%Section~\ref{sec:discuss}.
\begin{table}[t]
%\flushleft
\centering
\begin{SmallOut}
\begin {tabular} {|c!{\vrule width 1pt}c|c|c!{\vrule width 1pt}c|c|c!{\vrule width 1pt}c|c|c|c|}
 \hline
\multirow{2}*[-2pt]{\textbf{Project}}&
\multicolumn{3}{|c|}{\textbf{Class}}
& \multicolumn{3}{|c|}{\textbf{Method}}  \\\cline{2-7} &  \emph{\textbf{Num.}}&\emph{\textbf{Acc.}}&\emph{\textbf{J2SE}}& \emph{\textbf{Num.}}&  \emph{\textbf{Acc.}}&\emph{\textbf{J2SE}} \\
\hline \hline
  db4o         &   3155  & 83.3\%  &   246 & 10787 & 90.0\%  & 513  \\
\hline
  fpml         &   199   & 83.3\%  &   53  & 508   & 83.3\%  & 229   \\
\hline
  PDFClown     &   539   & 96.7\%  &   102 & 514   & 100.0\% & 126  \\
\hline
  OpenFSM      &   64    & 86.7\%  &   27  & 139   & 73.3\%  &  12  \\
\hline
  binaryNotes  &   287   & 90.0\%  &   56  & 671   & 90.0\%  &  59   \\
\hline \hline
  neodatis     &   526   & 96.7\%  &    86 & 3517  & 100.0\% & 600    \\
\hline
  numerics4j   &   97    & 83.3\%  &   8   & 429   & 83.3\%  &  34  \\
\hline
  lucene       &   718   & 90.0\%  &   147  & 2725  & 90.0\%  & 584   \\
\hline\hline
  logging      &   305   & 73.3\%  &   93  & 56    & 90.0\%  & 21  \\
\hline
  hibernate    &   1126  & 66.7\%  &   164 & 7     & 13.3\%   & 5  \\
\hline\hline
  Total        &   6695   & 86.7\% &   344  & 19110 & 90.0\%  & 1768\\
\hline
\end{tabular}\vspace*{-2ex}
\Caption{Results of mining API mapping} \label{table:minedresults}
\end{SmallOut}\vspace*{-2ex}
\end{table}

\textbf{Comparing with manually built API mapping.} Some translation
tools such as
Java2CSharp\footnote{\url{http://j2cstranslator.wiki.sourceforge.net}}
have manually built files that describe mapping relations of APIs.
For example, one item from the mapping files of Java2CSharp is as
follows:

\begin{CodeOut}\vspace*{-2ex}
\begin{alltt}
package java.math :: System \{
  class java.math.BigDecimal :: System:Decimal \{
    method multiply(BigDecimal)
       \{ pattern =  Decimal.Multiply(@0, @1); \}
  \}
\}
\end{alltt}
\end{CodeOut}\vspace*{-2ex}

This item describes mapping relations between \CodeIn{java.math.Big-
Decimal.multiply()} and \CodeIn{System.Decimal.Multiply()}. The
pattern string describes mapping relations of inputs. In particular,
``\CodeIn{@0}'' denotes the receiver, and ``\CodeIn{@1}'' denotes
the first parameter. Based on this item, Java2CSharp translates the
following code snippet from Java to C\# as follows:

\begin{CodeOut}\vspace*{-2ex}
\begin{alltt}
  BigDecimal m = new BigDecimal(1);
  BigDecimal n = new BigDecimal(2);
  BigDecimal result = m.multiply(n);
  ->
  Decimal m = new Decimal(1);
  Decimal n = new Decimal(2);
  Decimal result = Decimal.Multiply(m,n);
\end{alltt}
\end{CodeOut}\vspace*{-2ex}

To compare with manually built mapping files of Java2CSharp, we
translate our mined API mapping with the following strategy. First,
for each Java class, we translate its mapping relations of classes
with the highest supports into mapping files as relations of
packages and classes. Second, for each Java method, we translate its
mapping relations of methods with the highest supports into mapping
files as relations of methods with pattern strings. For 1-to-1
mapping relations of methods, this step is automatic as mined
mapping relations describe mapping relations of corresponding
methods and inputs. For many-to-many mapping relations of methods,
this step is manual as mined mapping relations do not include
adequate details such as how to deal with multiple outputs. We
further discuss this issue in Section~\ref{sec:discuss}.

The mapping files of Java2CSharp cover 13 packages defined by J2SE
and 2 packages defined by
JUnit\footnote{\url{http://www.junit.org/}}, and we treat these
mapping files as a golden standard. We find 9 packages overlapping
between the mined mapping files and the mapping files of
Java2CSharp. We compare mapping relations of APIs within these
mapping packages, and Table~\ref{table:compare} shows the results.
Columns ``Class'' and `Method'' list results of comparing API
classes and methods, respectively. For their sub-columns, sub-column
``\emph{P}'' denotes precision. Sub-column ``\emph{R}'' denotes
recall. Sub-column ``\emph{F}'' denotes F-score. \emph{Precision},
\emph{Recall}, and \emph{F-score} are defined as
follows:\vspace*{-1ex}
\begin{equation}\label{eq-precison}
Precison=\frac{true\ positives}{true\ positives+false\ positives}
\end{equation}\vspace*{-2ex}
\begin{equation}\label{eq-recall}
Recall=\frac{true\ positives}{true\ positives+false\ negatives}
\end{equation}\vspace*{-2ex}
\begin{equation}\label{eq-fscore}
F\!-\!score=\frac{2\times Precision \times Recall}{Precision+Recall}
\end{equation}\vspace*{-2ex}

\begin{table}[t]
\centering
\begin{SmallOut}
\begin {tabular} {|c!{\vrule width 1pt}c|c|c!{\vrule width 1pt}c|c|c!{\vrule width 1pt}c|c|c|c|}
 \hline
\multirow{2}*[-2pt]{\textbf{Package}}&
\multicolumn{3}{|c|}{\textbf{Class}}
& \multicolumn{3}{|c|}{\textbf{Method}}  \\\cline{2-7} &  \emph{\textbf{P}}& \emph{\textbf{R}}& \emph{\textbf{F}}& \emph{\textbf{P}}& \emph{\textbf{R}}& \emph{\textbf{F}}\\
\hline
  java.io &   78.6\% & 26.8\% &   52.7\% & 93.1\% & 53.2\% & 73.1\%   \\
\hline
  java.lang &   82.6\% & 27.9\% &   55.3\% & 93.8\% & 25.4\% & 59.6\%   \\
\hline
  java.math &   50.0\% & 50.0\% &   50.0\% & 66.7\% & 15.4\% & 41.0\%   \\
\hline
  java.net &   100.0\% & 12.5\% &   56.3\% & 100.0\% & 25.0\% & 62.5\%   \\
\hline
  java.sql &   100.0\% & 33.3\% &   66.7\% & 100.0\% & 15.4\% & 57.7\%   \\
\hline
  java.text &   50.0\% & 10.0\% &   30.0\% & 50.0\% & 16.7\% & 33.3\%   \\
\hline
  java.util &   56.0\% & 25.5\% &   40.7\% & 65.8\% & 12.6\% & 39.2\%   \\
\hline
  junit &   100.0\% & 50.0\% &   75.0\% & 92.3\% & 88.9\% & 90.6\%   \\
\hline
  orw.w3c &   42.9\% & 33.3\% &   38.1\% & 41.2\% & 25.0\% & 33.1\%   \\
\hline
  Average &   68.8\% & 26.4\%  &  47.6\%  & 84.6\% & 28.7\% & 56.7\% \\
\hline
\end{tabular}\vspace*{-2ex}
\Caption{Results of comparing results} \label{table:compare}
\end{SmallOut}\vspace*{-2ex}
\end{table}


In the preceding formulae, true positives represent those mapping
relations that exist in both the mined API mapping and the golden
standard; false positives represent those transitions that exist in
the mined API mapping but not in the golden standard; false
negatives represent those transitions that exist in the golden
standard but not in the mined API mapping. From the results of
Table~\ref{table:compare}, we find that our approach achieves a
relatively high precision and a relatively low recall due to the
three main causing impact factors. First, we find 25 correct mapping
relations of API classes in mined mapping files but not in the
golden standard. For example, the mined mapping files contain a
mapping relation between \CodeIn{org.w3c.dom.Attr} and
\CodeIn{System.Xml.XmlAttribute}, and the mapping relation does not
exist in the mapping files of Java2CSharp. As these items are
counted as false positives, this impact factor reduces the
precisions. Second, although we use 10 large projects as subjects to
mine API mapping, these projects do not cover mapping relations of
all API classes and all API methods. Consequently, our approach does
not mine mapping relations of the entire API classes and the entire
API methods. Although as shown in Table~\ref{table:minedresults} our
approach mines many mapping relations, these mapping relations cover
many libraries. When we limit mapping relations to the packages as
shown in Table~\ref{table:compare}, the mined mapping relations are
actually not so many as expected. On the contrary, the mapping files
of Java2CSharp are more detailed as they are manually built. This
impact factor reduces the recalls. Third, we find that Java2CSharp
maps 51 J2SE classes of these packages to its implemented classes.
For example, Java2CSharp maps \CodeIn{java.util.Set} to
\CodeIn{ILOG.J2CsMapping.Collections.ISet}. Our approach did not
mine these mapping relations since the subjects in
Table~\ref{table:subjects} do not use Java2CSharp's own implemented
classes and methods. This impact factor reduces both the precisions
and the recalls.

In summary, compared with the mapping files of Java2CSharp, our
mined mapping files show a relatively high precisions and relatively
low recalls. The relatively high precisions show that our mined
mapping relations are accurate and contain some mapping relations
that are not covered by Java2CSharp. The relatively low recalls show
that we need improvements such as introducing more subject projects
to cover detailed API mapping.
\begin{table}[t]
\centering
\begin{SmallOut}
\begin {tabular} {|c!{\vrule width 1pt}c|c!{\vrule width 1pt}c|c!{\vrule width 1pt}c|c|c|c|c|c|}
 \hline
\multirow{2}*[-2pt]{\textbf{Projects}}& \multicolumn{2}{|c|}{\textbf{No MF}}  & \multicolumn{2}{|c|}{\textbf{MF}} & \multicolumn{4}{|c|}{\textbf{Ext. MF}}\\
\cline{2-9} &  \emph{\textbf{E}}&  \emph{\textbf{B}}&  \emph{\textbf{E}}&  \emph{\textbf{B}}& \emph{\textbf{E}}&  \emph{\textbf{\%E}}&  \emph{\textbf{B}}&  \emph{\textbf{\%B}}\\
 \hline
  rasp        &  973  &   159 &  708  & 123   &  627  &  11.4\% & 93 &  24.4\%    \\
\hline
  llrp        &  2328 &  122  & 1540  &  114  &  269  &  82.5\% & 42 & 63.2\%   \\
\hline
  simmetrics  &  217  & 13    &  12   & 0     & 6     &  50.0\% &  0  & 0\%   \\
\hline
  aligner     &  368  &  34   &  289  & 0     & 262   &  9.3\%  &  0  & 0\%    \\
\hline
  fit         &  177  & 29    &  27   & 0     & 20    &  25.9\% &  0  & 0\%  \\
\hline Total  & 4063  & 491   & 2576  & 237   & 1174  &  54.4\% & 135 & 43.0\% \\
\hline
\end{tabular}\vspace*{-2ex}
\Caption{\label{table:errors}Compilation errors and bugs}
\end{SmallOut}\vspace*{-3ex}
\end{table}
\subsection{Aiding Language Migration}
\label{sec:evaluation:migration} To evaluate the second research
question, we feed the mined API mapping to the Java2CSharp tool and
investigate whether these mined API mapping can improve the tool's
effectiveness. We choose this tool because this tool is a relatively
mature project at ILOG\footnote{\url{http://www.ilog.com/}} (now
part of IBM) and supports the extension of user-defined mapping
relations of APIs.

We use Java2CSharp to translate five projects listed in
Table~\ref{table:subjects} from Java to C\#, and
Table~\ref{table:errors} shows the results. For each translated C\#
project, Column ``No MF'' lists results without mapping files.
Column ``MF'' lists results with the mapping files of Java2CSharp.
Column ``Ext. MF'' lists results with mapping files that combine
mined API mapping with the existing mapping files of Java2CSharp.
Sub-column ``\emph{E}'' lists numbers of compilation errors.
Sub-column ``\emph{B}'' lists numbers of found bugs. For each
project, we find out those overlapping files between translated
files and existing C\# files and manually inspect the top 5 largest
files by comparing existing C\# files for API related bugs.
Sub-columns ``\emph{\%E}'' and ``\emph{\%B}'' list percentages of
improvements over the mapping files of Java2CSharp. Totally, mined
API mapping helps further reduces 54.4\% compilation errors and
43.0\% found bugs. As the five projects use different libraries, the
numbers of translated projects are different. In particular,
\emph{simmetrics} and \emph{fit} use API classes of J2SE that are
covered by mapping files. Consequently, the translated projects of
\emph{simmetrics} and \emph{fit} have few errors and bugs. The
\emph{aligner} project also mainly uses J2SE, but it uses many API
classes and methods from \CodeIn{java.awt} for its GUI. The mapping
files of Java2CSharp does not cover any classes of
\CodeIn{java.awt}, so the translated project has many errors. As the
existing C\# version of \emph{aligner} does not have GUI, we do not
compare those buggy translated GUI files and we do not find any
bugs. The mined files map \CodeIn{java.awt} to
\CodeIn{System.Windows.Forms} and thus reduce compilation errors.
However, the result is not significant as many classes of the two
packages are still not mapped. For \emph{rasp} and \emph{llrp}, they
both use various libraries besides J2SE. Consequently, the
translated projects have both many errors and bugs. In particular,
\emph{llrp} uses
log4j\footnote{\url{http://logging.apache.org/log4j/}} and
jdom\footnote{\url{http://www.jdom.org/}}, and the mined mapping
files contain mapping relations of the two libraries. As a result,
the mined API mapping helps reduce compilation errors and bugs
significantly. For \emph{rasp}, it uses some libraries such as
Neethi\footnote{\url{http://ws.apache.org/commons/neethi/}} and
WSS4J\footnote{\url{http://ws.apache.org/wss4j/}}. Sine the used
subjects for mining and thus our mined API mapping do not cover the
two libraries, the translated project of \emph{rasp} contains many
``\emph{U}'' and ``\emph{T}'' errors.

In summary, the mined API mapping improves existing language
translation tools such as Java2CSharp. In particular, the mined API
mapping helps effectively reduce ``\emph{U}'' and ``\emph{T}''
errors in the translated projects.


\subsection{Threats to Validity}
\label{sec:evaluation:threats} The threat to external validity
includes the representativeness of the subjects in true practice and
the existing translation tool being compared. Although we applied
our approach on 10 projects for mining API mapping and on additional
5 projects for helping language translation, our approach is
evaluated only on these limited projects. Although Java2CSharp is
the best translation tool as far as we know, other tools may perform
better than the tool. The threat could be reduced by more
evaluations on more subjects and more translation tools in future
work. The threat to internal validity includes human factors for
determining bugs in translated projects. To reduce the threat, we
inspected bugs carefully and refer to existing C\# files for
confirming these bugs. This threat could be further reduced by
running test cases when we improve our approach and produce
translated projects without complication errors.
