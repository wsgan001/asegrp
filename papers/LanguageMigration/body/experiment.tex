
\begin{table}[t]
\centering
\begin{SmallOut}
\begin {tabular} {|c!{\vrule width 1pt}c!{\vrule width 1pt}c|c!{\vrule width 1pt}c|c|c|c|}
 \hline
\multirow{2}*[-2pt]{\textbf{Project}}&
\multirow{2}*[-2pt]{\textbf{Source}}
& \multicolumn{2}{|c|}{\textbf{Java version}} & \multicolumn{2}{|c|}{\textbf{C\# version}} \\\cline{3-6} &  &  \emph{\textbf{\#C}}& \emph{\textbf{\#M}} &\emph{\textbf{\#C}}& \emph{\textbf{\#M}}\\
\hline
  neodatis &   SourceForge  & 1298 &   9040 & 464  & 3983     \\
\hline
  db4o     &   SourceForge  & 3047 &   17449 & 3051 & 15430     \\
\hline
numerics4j &   SourceForge  & 145 &   973   & 87   & 515     \\
\hline
  fpml     &   SourceForge  & 143  &   879   & 144  & 1103     \\
\hline
  PDFClown &   SourceForge  & 297  &   2239  & 290  & 1393     \\
\hline
  OpenFSM &   SourceForge   & 35   &   179   & 36  & 140    \\
\hline
binaryNotes &   SourceForge  & 178 &   1590  & 197 & 1047    \\
\hline
  lucene   &    Apache      & 1298 &   9040  & 464& 3015    \\
\hline
  logging  &    Apache      & 196 &   1572   & 308 & 1474     \\
\hline
  hibernate &   hibernate  & 3211 &   25798 & 856 & 2538     \\
\hline \hline
  rasp     &   SourceForge  & 320  &   1819  & 557 & 1893     \\
\hline
  llrp     &   SourceForge  & 257 &   3833   & 222 & 978   \\
\hline
 simmetrics &   SourceForge & 107 &   581    & 63 & 325     \\
\hline
  aligner &   SourceForge  & 41    &   232 & 18 & 50     \\
\hline
  fit       &   SourceForge  & 95 &   461    & 43 & 281     \\
\hline
  \multicolumn{2}{|c|}{Total} &   11668 & 75685  &   6900 & 34165   \\
\hline
\end{tabular}\vspace*{-2ex}
\Caption{Subjects} \label{table:subjects}
\end{SmallOut}\vspace*{-2ex}
\end{table}


\section{Experiments}
\label{sec:experiment} We implemented a tool named MAM based on our
approach and conducted two experiments on the tool. Our experiments
focus on two research questions as follows:
\vspace*{-1.5ex}\begin{enumerate}
\item How effective can our approach mine various mapping relations of APIs
(Section~\ref{sec:experiment:mining})? \vspace*{-1.5ex}
\item How much benefit can the mined mapping relations of APIs offer in aiding language
migration (Section~\ref{sec:experiment:migration})?\vspace*{-1.5ex}
\end{enumerate}%\vspace*{-1.5ex}

We choose fifteen open source projects that have both Java versions
and C\# versions as the subjects of our experiments, and
Table~\ref{table:subjects} show these subjects. Column ``Project''
lists names of subjects. Column ``Source'' lists sources of these
subjects. These subjects come from famous open source societies such
as SourgeForge\footnote{\url{http://www.sf.net}},
Apache\footnote{\url{http://www.apache.org/}}, and
hibernate\footnote{\url{http://www.hibernate.org/}}. Column ``Java
version'' and Column ``C\# version'' list the two versions from each
subject. All these versions are the latest versions. For the two
columns, sub-column ``\emph{\#C}'' lists numbers of classes.
Sub-column ``\emph{\#M}'' list numbers of methods. We notice that
Java versions are much larger than C\# versions totally. We
investigate these projects and find two factors as follows. One is
that Java versions of some projects are more update-to-data. For
example, the latest Java version of \emph{numericas4j} is 1.3
whereas the latest C\# version is 1.2. The other factor is that some
projects are migrating from Java into C\# in progress. For example,
the website\footnote{\url{http://wiki.neodatis.org/}} of
\emph{neodatis} says that \emph{neodatis} is a project in Java and
is being ported to .Net. This observation further confirms the
usefulness of our approach as our approach aids migrating from one
language to other languages. Totally, these projects have 18568
classes and 109850 methods.

All the experiments were conducted on a PC with an Intel Qual CPU @
2.83GHz and 1.98M memory running the Windows XP operation system.
\subsection{Mining API mapping}
\label{sec:experiment:mining}
\begin{table}[t]
\centering
\begin{SmallOut}
\begin {tabular} {|c!{\vrule width 1pt}c|c!{\vrule width 1pt}c|c!{\vrule width 1pt}c|c|c|c!{\vrule width 1pt}c|c|c|c|}
 \hline
\multirow{2}*[-2pt]{\textbf{Project}}&
\multicolumn{2}{|c|}{\textbf{Java version}}
& \multicolumn{2}{|c|}{\textbf{C\# version}} & \multicolumn{2}{|c|}{\textbf{Aligned}} \\\cline{2-7} &  \emph{\textbf{\%C}}& \emph{\textbf{\%M}}&  \emph{\textbf{\%C}}& \emph{\textbf{\%M}} &\emph{\textbf{\#C}}& \emph{\textbf{\#M}}\\
\hline
  neodatis &   44.7\%  &  54.8\% &   100.0\% & 93.6\% & 408  & 3728   \\
\hline
  db4o     &   87.8\%  &  65.5\%&    87.6\%  & 74.1\% & 2674  & 11433   \\
\hline
  numerics4j & 57.2\%  &  48.6\% & 95.4\%  & 89.9\% &  75  &  174   \\
\hline
  fpml       & 93.7\%  &  70.5\% & 93.5\%  & 56.2\% &  134 &  620  \\
\hline
  PDFClown   & 86.5\%  &  51.0\% & 88.6\%  & 82.1\% & 257  & 1143  \\
\hline
  OpenFSM    & 97.1\%  &  72.1\% &  94.4\% &  92.1\%  & 34   & 129   \\
\hline
  binaryNotes & 98.9\%  & 61.1\% &  89.3\% &  92.7\% & 176   & 971  \\
\hline
  lucene     &  34.9\%  & 26.6\%&   97.6\% &  79.8\% & 453  & 2406  \\
\hline
  logging    &   91.8\% & 18.1\%&   58.4\% & 19.3\% &  180  & 285   \\
\hline
  hibernate &   26.4\%  & 1.2\%&    99.1\% & 12.6\% &  848  & 319   \\
\hline
  Total &   44.9\% & 28.0\%  &   75\% & 62.1\% & 5239 & 21208 \\
\hline
\end{tabular}\vspace*{-2ex}
\Caption{Aligned client code} \label{table:analyzingclient}
\end{SmallOut}\vspace*{-2ex}
\end{table}



To evaluate the first research question, we use ten projects from
Table~\ref{table:subjects} as the subjects for mining API mapping.


\textbf{Aligning client code.} We first use our approach to align
client code based on name similarities. The threshold is set to 0.6.
We choose a relatively low threshold so that our approach does not
loose any useful client code.

Table~\ref{table:analyzingclient} shows the results of this step.
For column ``Aligned'', sub-column ``\emph{\# C}'' lists numbers of
aligned classes. Sub-column ``\emph{\# M}'' lists numbers of aligned
methods. For each project of Column ``C\# version'' and Column
``Java version'', sub-column ``\emph{\%C}'' lists the percent of the
aligned classes to total classes of corresponding versions.
Sub-column ``\emph{\%M}'' lists the percent of the aligned methods
to total methods of corresponding versions. We find that the results
of Table~\ref{table:analyzingclient} fall into three categories.
This first category includes \emph{db4o}, \emph{fpml},
\emph{PDFClown}, \emph{OpenFSM}, and \emph{binaryNotes}. Our
approach achieves relatively high percents for both Java versions
and C\# versions. In each of the five project, ``\emph{\%M}'' is
relatively smaller than ``\emph{\%C}'' for two factors. One is that
methods of those unaligned classes cannot be aligned and thus are
counted as unaligned. The other is that Java versions usually have
many getters and setters and these getters and setters often do not
have corresponding methods in C\# versions. The second category
includes \emph{neodatis}, \emph{numerics4j}, and \emph{lucene}. Our
approach aligns C\# version well but does not aligns Java versions
so well. We find that \emph{neodatis} and \emph{lucene} are
migrating from Java to C\# in progress and the Java version of
\emph{numerics4j} is more update-to-date than its C\# version. As a
result, some Java classes or methods do not have corresponding
implementations in C\# versions in these projects and thus are left
unmapped. The third category includes \emph{logging} and
\emph{hibernate}. Our approach does not align classes and methods of
the two projects well. Although both the two projects seem to be
migrated from existing Java versions, the programmers of the two
projects do not refer to names of existing Java versions for naming
entities. For each of the two projects, the percent of aligned
classes is relatively high, and the percent of aligned methods is
relatively low. We find that even if our approach aligns a wrong
class pair, our approach does not aligns methods within the wrong
pair as the method names of a wrong pair are quite different. The
result suggests that the low threshold may not introduce many
negative impacts on the results of API mapping.

For all these projects, our approach does not align all classes and
all methods. Besides the intuitive factor of naming entities, one
factor lies in that one functionality may be implemented as a single
class in one version and is implemented as several classes in the
other version. Another factor lies in that a Java version and a C\#
version may have quite different functionalities. We further discuss
this issue in Section~\ref{sec:discuss}.

In summary, our approach aligns most classes and methods of eight
projects listed in Table~\ref{table:analyzingclient} using simple
name similarities. The result confirms that many programmers refer
to existing versions to name entities of a version under
development.



\textbf{Mining API mapping.} We then use our approach to mine
mapping relations of API classes and API methods.

Table~\ref{table:minedresults} shows the results of this step. For
Column ``Class'' and Column ``Method'', sub-column ``\emph{Num.}''
lists numbers of mined mapping relations. The numbers of mined API
mapping are largely proportional to the sizes of projects as shown
in Table~\ref{table:subjects} except \emph{logging} and
\emph{hibernate}. As classes and methods of the two projects are not
quite well aligned, our approach does not mine many mapping
relations of APIs from the two projects. For the remaining projects,
our approach mines many mapping relations of API classes and API
methods. Sub-column ``\emph{Acc.}'' lists accuracies of mined API
mapping. For mined API mapping from each project, we manually
inspect top thirty mined mapping relations of APIs, and these
accuracies are percents of correct mapping relations. We find that
our approach achieves high accuracies except \emph{hibernate}.
Although our approach does not align \emph{logging} quite well
either, the accuracies of API mapping from \emph{logging} are still
relatively high. To mine API mapping of classes, our approach
requires names of classes, methods, and variables are all similar.
To mine API mapping of methods, our approach requires two built API
transformation graphs are similar. The two requirements are strict.
As a result, if the first step does not align client code quite
well, our approach loses some mapping relations of APIs but does not
introduce many false mapping relations. In other word, our approach
is robust to mine accurate API mapping. Sub-column ``\emph{1-1}''
list the number of one-to-one relations. We find that most mined
mapping relations are of this kind. Our approach still needs
improvements to mine those n-to-n mapping relations. We further
discuss this issue in Section~\ref{sec:discuss}.


In summary, our approach mines many mapping relations of APIs
totally. These mined mapping relations are accurate and cover
various libraries.
\begin{table}[t] \centering
\begin{SmallOut}
\begin {tabular} {|c!{\vrule width 1pt}c|c!{\vrule width 1pt}c|c|c!{\vrule width 1pt}c|c|c|c|}
 \hline
\multirow{2}*[-2pt]{\textbf{Project}}&
\multicolumn{2}{|c|}{\textbf{Class}}
& \multicolumn{3}{|c|}{\textbf{Method}}  \\\cline{2-6} &  \emph{\textbf{Num.}}&\emph{\textbf{Acc.}}& \emph{\textbf{Num.}}&  \emph{\textbf{Acc.}}&\emph{\textbf{1-1}} \\
\hline
  neodatis   &   526 & 96.7\% &   3517 & 100.0\% & 3506    \\
\hline
  db4o       &   3155  & 83.3\% &   10787 & 90.0\% & 10746    \\
\hline
  numerics4j &   97  & 83.3\%  &   429 & 83.3\% & 427    \\
\hline
  fpml       &   199 & 83.3\% &   508 & 83.3\% & 503    \\
\hline
  PDFClown   &   539 & 96.7\% &   514 & 100.0\% & 504    \\
\hline
  OpenFSM    &   64  & 86.7\%  &   139 &  73.3\% & 139   \\
\hline
  binaryNotes &   287 & 90.0\%  &   671 & 90.0\% & 665     \\
\hline
  lucene      &   718 & 90.0\% &   2725 & 90.0\% & 2707    \\
\hline
  logging     &   305 & 73.3\%  &   56 & 90.0\% & 53    \\
\hline
  hibernate   &   1126 & 66.7\% &   7 & 13.3\% & 7    \\
\hline
  Total       &   7016 & 85.0\%  &  19353  & 81.3\% & 19257  \\
\hline
\end{tabular}\vspace*{-2ex}
\Caption{Mined API mapping} \label{table:minedresults}
\end{SmallOut}\vspace*{-2ex}
\end{table}



\textbf{Comparing with manually built API mapping.} Some translation
tools such as
CSharp2Java\footnote{\url{http://j2cstranslator.wiki.sourceforge.net}}
have manually built files that describe mapping relations of APIs.
For example, one item from the mapping files of CSharp2Java is as
follows:

\begin{CodeOut}\vspace*{-1ex}
\begin{alltt}
package java.math :: System \{
  class java.math.BigDecimal :: System:Decimal \{
    method multiply(BigDecimal)
       \{ pattern =  Decimal.Multiply(@0, @1); \}
  \}
\}
\end{alltt}
\end{CodeOut}\vspace*{-1ex}

This item describes mapping relations between \CodeIn{java.math.Big-
Decimal.multiply()} and \CodeIn{System.Decimal.Multiply()}. The
pattern string describes mapping relations of inputs. In particular,
``\CodeIn{@0}'' denotes the receiver, and ``\CodeIn{@1}'' denotes
the first parameter. Based on this item, CSharp2Java translates the
following code snippet from Java to C\# as follows:

\begin{CodeOut}\vspace*{-1ex}
\begin{alltt}
  BigDecimal m = new BigDecimal(1);
  BigDecimal n = new BigDecimal(2);
  BigDecimal result = m.multiply(n);
  ->
  Decimal m = new Decimal(1);
  Decimal n = new Decimal(2);
  Decimal result = Decimal.Multiply(m,n);
\end{alltt}
\end{CodeOut}\vspace*{-1ex}

To compare with manually built mapping files of CSharp2Java, we
translate the mined API mapping with the following strategy.

(1) For each Java class, we translate its mapping relations of
classes with the highest supports into mapping files as relations of
packages and classes.

(2) For each Java method, we translate its mapping relations of
methods with the highest supports into mapping files as relations of
methods with pattern strings. For 1-to-1 mapping relations of
methods, this step is automatic as mined mapping relations describe
mapping relations of corresponding methods and inputs. For n-to-n
mapping relations of methods, this step is manual as mined mapping
relations do not include adequate details such as how to deal with
multiple outputs. For example, the mapping relation of method as
shown in Figure~\ref{fig:example} has two outputs. We need manually
combine the two outputs with an \emph{or} operator.

After the preceding translation,we find that some packages overlap
in the mined mapping files and the mapping files of CSharp2Java. We
compare mapping relations of APIs within these mapping packages, and
Table~\ref{table:compare} shows the results. For its sub-columns,
sub-column ``\emph{P}'' denotes precision. Sub-column ``\emph{R}''
denotes recall. Sub-column ``\emph{F}'' denotes F-score.
\emph{Precision}, \emph{Recall}, and \emph{F-score} are defined as
follows:
\begin{equation}\label{eq-precison}
Precison=\frac{true\ positives}{true\ positives+false\ positives}
\end{equation}
\begin{equation}\label{eq-recall}
Recall=\frac{true\ positives}{true\ positives+false\ negatives}
\end{equation}
\begin{equation}\label{eq-fscore}
F\!-\!score=\frac{2\times Precision \times Recall}{Precision+Recall}
\end{equation}

\begin{table}[t]
\centering
\begin{SmallOut}
\begin {tabular} {|c!{\vrule width 1pt}c|c|c!{\vrule width 1pt}c|c|c!{\vrule width 1pt}c|c|c|c|}
 \hline
\multirow{2}*[-2pt]{\textbf{Package}}&
\multicolumn{3}{|c|}{\textbf{Class}}
& \multicolumn{3}{|c|}{\textbf{Method}}  \\\cline{2-7} &  \emph{\textbf{P}}& \emph{\textbf{R}}& \emph{\textbf{F}}& \emph{\textbf{P}}& \emph{\textbf{R}}& \emph{\textbf{F}}\\
\hline
  java.io &   78.6\% & 26.8\% &   52.7\% & 93.1\% & 53.2\% & 73.1\%   \\
\hline
  java.lang &   82.6\% & 27.9\% &   55.3\% & 93.8\% & 25.4\% & 59.6\%   \\
\hline
  java.math &   50.0\% & 50.0\% &   50.0\% & 66.7\% & 15.4\% & 41.0\%   \\
\hline
  java.net &   100.0\% & 12.5\% &   56.3\% & 100.0\% & 25.0\% & 62.5\%   \\
\hline
  java.sql &   100.0\% & 33.3\% &   66.7\% & 100.0\% & 15.4\% & 57.7\%   \\
\hline
  java.text &   50.0\% & 10.0\% &   30.0\% & 50.0\% & 16.7\% & 33.3\%   \\
\hline
  java.util &   56.0\% & 25.5\% &   40.7\% & 65.8\% & 12.6\% & 39.2\%   \\
\hline
  junit &   100.0\% & 50.0\% &   75.0\% & 92.3\% & 88.9\% & 90.6\%   \\
\hline
  orw.w3c &   42.9\% & 33.3\% &   38.1\% & 41.2\% & 25.0\% & 33.1\%   \\
\hline
  Total &   68.8\% & 26.4\%  &  47.6\%  & 84.6\% & 28.7\% & 56.7\% \\
\hline
\end{tabular}\vspace*{-2ex}
\Caption{Compared results} \label{table:compare}
\end{SmallOut}\vspace*{-2ex}
\end{table}


In the preceding formulae, true positives represent those mapping
relations that exist in both the mined API mapping and the golden
standard; false positives represent those transitions that exist in
the mined API mapping but not in the golden standard; false
negatives represent those transitions that exist in the golden
standard but not in the mined API mapping. From the results of
Table~\ref{table:compare}, our approach achieves a relatively high
precision and a relatively low recall. We further investigate the
differences, and we find the impact factors as follows.

(1) The mined mapping files contain correct items that do not exist
in the mapping files of CSharp2Java. For example, the mined mapping
files contain a mapping relation between \CodeIn{org.w3c.dom. Attr}
and \CodeIn{System.Xml.XmlAttribute}, and the mapping relation does
not exit in the mapping files of CSharp2Java. As these items are
counted as false positives, this impact factor reduces the
precisions.

(2) Although we use ten large projects as subjects to mine API
mapping, these projects do not cover mapping relations of all API
classes and all API methods. Consequently, our approach does not
mine mapping relations of the entire API classes and the entire API
methods. Although as shown in Table~\ref{table:minedresults} our
approach mines many mapping relations, these mapping relations cover
many libraries. When we limit mapping relations to the packages as
shown in Table~\ref{table:compare}, the mined mapping relations are
actually not so many as expected. On the contrary, the mapping files
of CSharp2Java are more detailed as they are manually built. This
impact factor reduces the recalls.

(3) Some API classes and API methods between Java and C\# have
different behaviors. To hide these behaviors from client code,
CSharp2Java maps these classes and methods to its implemented
classes and methods. For example, CSharp2Java maps
\CodeIn{java.util. Set} to
\CodeIn{ILOG.J2CsMapping.Collections.ISet}. Our approach did not
mine these mapping relations as the subjects in
Table~\ref{table:subjects} do not show such mapping relations. This
impact factor reduces both the precisions and the recalls.

In summary, compared with the mapping files of CSharp2Java, our
mined mapping files show a relatively high precisions and relatively
low recalls. The relatively high precisions show that our mined
mapping relations are accurate and contain some mapping relations
that are not covered by CSharp2Java. The relatively low recalls show
that we need improvements such as introducing more subject projects
to cover detailed API mapping.

\subsection{Aiding Language Migration}
\label{sec:experiment:migration} To evaluate the second research
question, we feed the mined API mapping to the CSharp2Java tool and
investigate whether these mined API mapping can improve the tool. We
choose this tool because this tool is a relatively mature project at
ILOG\footnote{\url{http://www.ilog.com/}} (now IBM) and supports the
extension of user-defined mapping relations of APIs.

We use CSharp2Java to translate five projects listed in
Table~\ref{table:subjects} from Java to C\#, and
Table~\ref{table:errors} shows the results. For each translated C\#
project, Column ``No MF'' lists the number of compilation errors
without mapping files. Column ``MF'' lists the number of compilation
errors with the mapping files of CSharp2Java. Column ``Ext. MF''
lists the number of compilation errors with extended mapping files.
We produce these mapping files by combing mined API mapping with the
existing mapping files of CSharp2Java. Totally, the mapping files of
CSharp2Java helps reduce 36.6\% compilation errors, and the extended
mapping files helps further reduces 54.4\% compilation errors. For
the three columns, sub-column ``\emph{U}'' lists numbers of
compilation errors that are caused by wrong using statements.
Sub-column ``\emph{T}'' lists numbers of compilation errors that are
caused by unsolved classes. Sub-column ``\emph{O}'' lists the
numbers of compilation errors that are caused by other factors. The
three types of compilation errors are not independent. For example,
if a mapping relation maps a class in Java to a wrong class in C\#,
both ``\emph{U}'' and ``\emph{T}'' decrease as CSharp2Java can
translate a variable in Java to a variable in C\# whose type is
resolvable. However, as the translated variable may not provide
desirable methods, `\emph{O}'' may increase when code use some
methods of the variable. From the results of
Table~\ref{table:errors}, we find that API mapping helps reduces
``\emph{U}'' and ``\emph{T}'' without significant increases of
``\emph{O}''. In other word, both manually built mapping files and
our mined mapping files are useful and accurate. As the five
projects use different libraries, the numbers of translated projects
are different. In particular, \emph{simmetrics} and \emph{fit} use
API classes of J2SE that are covered by mapping files. Consequently,
the translate projects of \emph{simmetrics} and \emph{fit} have no
``\emph{U}'' and ``\emph{T}'' errors. The \emph{aligner} project
also mainly uses J2SE, but it uses many API classes and methods from
\CodeIn{java.awt} for its GUI. The mapping files of CSharp2Java do
not cover \CodeIn{java.awt}, so the translated project has many
``\emph{U}'' and ``\emph{T}'' errors. The mined files map
\CodeIn{java.awt} to \CodeIn{System.Windows.Forms} and thus reduce
``\emph{U}'' errors. However, the mined files also introduce many
``\emph{T}'' errors as many classes of the two packages are still
not mapped. For \emph{rasp} and \emph{llrp}, they both use various
libraries besides J2SE. Consequently, the translated projects both
have many ``\emph{U}'' and ``\emph{T}'' errors. In particular,
\emph{llrp} use
log4j\footnote{\url{http://logging.apache.org/log4j/}} and
jdom\footnote{\url{http://www.jdom.org/}}, and the mined mapping
files contain mapping relations of the two libraries. As a result,
the mined API mapping helps reduce compilation errors significantly.
For \emph{rasp}, it uses some libraries such as
Neethi\footnote{\url{http://ws.apache.org/commons/neethi/}} and
WSS4J\footnote{\url{http://ws.apache.org/wss4j/}}. Even mined API
mapping does not cover the two libraries. As a result, the
translated project of \emph{rasp} contains many ``\emph{U}'' and
``\emph{T}'' errors.

In summary, the mined API mapping improves existing language
translation tools such as CSharp2Java. In particular, the mined API
mapping helps reduce ``\emph{U}'' and ``\emph{T}'' errors in the
translated projects.
\begin{table}[t]
\centering
\begin{SmallOut}
\begin {tabular} {|c!{\vrule width 1pt}c|c|c!{\vrule width 1pt}c|c|c!{\vrule width 1pt}c|c|c|c|c|c|}
 \hline
\multirow{2}*[-2pt]{\textbf{Projects}}& \multicolumn{3}{|c|}{\textbf{No MF}}  & \multicolumn{3}{|c|}{\textbf{MF}} & \multicolumn{3}{|c|}{\textbf{Ext. MF}}\\
\cline{2-10} &  \emph{\textbf{U}}&  \emph{\textbf{T}}&  \emph{\textbf{O}}&  \emph{\textbf{U}}&  \emph{\textbf{T}}&  \emph{\textbf{O}}&  \emph{\textbf{U}}&  \emph{\textbf{T}}&  \emph{\textbf{O}}\\
 \hline
  rasp &  405 &   518 &  50  & 173  &  480  &  55  & 103  & 455  &69     \\
\hline
  llrp &  1033 &  1292  & 3  &  463  &  1076 &  1  &  6   & 262  & 1    \\
\hline
  simmetrics &  164 & 32 &  21 & 0   & 12    & 0   & 0    &  6  & 0    \\
\hline
  aligner &  283 &  77 &  8  & 223 & 57 & 9 & 140 & 113 & 9     \\
\hline
  fit &  63 & 94 &  17 & 0  & 20 & 7 &  0 & 3 & 7  \\
\hline
\multirow{2}*[-2pt]{Total} &1948  & 2016 & 99 & 859 & 1645 & 72 & 249 & 839 & 86\\
\cline{2-10} &\multicolumn{3}{|c|}{4063} &\multicolumn{3}{|c|}{2576}&\multicolumn{3}{|c|}{1174}\\
\hline
\end{tabular}\vspace*{-2ex}
\Caption{\label{table:errors}Compilation errors}
\end{SmallOut}\vspace*{-2ex}
\end{table}

\subsection{Threats to Validity}
\label{sec:experiment:threats} The threat to external validity
includes the representativeness of the subjects in true practice.
bla bla...

The threat to external validity includes the representativeness of
the subjects in true practice. bla bla...

The threat to external validity includes the representativeness of
the subjects in true practice. bla bla...

The threat to external validity includes the representativeness of
the subjects in true practice. bla bla...
