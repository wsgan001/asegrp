\section{Introduction}
\label{sec:introduction} 

A programming language serves as a means for instructing computers to achieve a programming task at hand.
Since their inception, various programming languages came into existence due to  
reasons such as existence of many platforms or requirements for different programming styles.
The HOPL\footnote{\url{http://hopl.murdoch.edu.au}} website lists 
8,512 different programming languages. 
To address business requirements and to survive in competing markets, companies
often have to develop different versions of their projects in different languages.
For example, many well-known projects such as Lucene\footnote{\url{http://lucene.apache.org/}},
Db4o\footnote{\url{http://www.db4o.com/}}, and WordNet\footnote{\url{http://wordnet.princeton.edu/}} provide
multiple versions in different languages. There are instances where
companies incurred huge losses due to lack of multiple versions of their projects.
For example, Terekhov and Verhoef~\cite{terekhov2000realities} stated that at least three
companies went bankrupt and another company lost 50 million dollars due to failed language migration projects.

For some open source projects, although companies do not offically provide multiple versions, external programmers often
create their versions in different languages. For example, WordNet does not
provide a C\# version. However, Simpson and Crowe developed
a C\# version of WordNet.Net\footnote{\url{http://opensource.ebswift.com/WordNet.Net/}}. 
Another example is iText\footnote{\url{http://www.lowagie.com/iText/}}, which
provides Java version only. Kazuya developed a C\# version of 
iText.Net\footnote{\url{http://www.ujihara.jp/iTextdotNET/en/}}. As described by Jones~\cite{jones1998estimating}, 
about one-third of the existing projects have multiple versions in different languages.

Migrating projects from one language to another language (such as from Java to C\#) manually
is a tedious and error-prone task. A natural way to address this issue is to develop
a translation tool that can automatically translate projects from one language to another.
However, it is challenging to develop such a translation tool as the translation tool should
have knowledge of how one programming language is mapped to the other language. In literature,
there exist approaches~\cite{mossienko2003automated, yasumatsu1995spice, hainaut2008migration}
that address the problem of language migration partially. These approaches address 
the problem of language migration partially, because these approaches expect programmers
to describe how one language is mapped to another language. Based on the mappings provided
as input, these existing approaches translate projects from one language to another. 
As programming languages provide a large number of Application Programming Interfaces (API), writing these rules manually 
for all APIs is tedious and error-prone. As a result, existing
approaches~\cite{mossienko2003automated,yasumatsu1995spice,hainaut2008migration}
support only a subset of APIs or even ignore the mapping relations
of APIs. Such a limitation causes many compilation errors in migrated projects and limits their usage in practice. 

In this paper, we propose a novel approach that automatically captures
how APIs of one language are mapped to the APIs of another language. We refer this
mapping as \emph{mapping relations of APIs}. There are two possible alternatives
for capturing mapping relations of APIs. First, capture mapping relations based
on API implementations. Second, mine mapping relations based on API usages in the client code.
In our approach, we use the second alternative rather than the first alternative 
for three major reasons: (1) Often API implementations such as implementations of C\# base class 
libraries are not available. (2) Capturing relations based on API implementations
often can have relatively low confidence than mining mapping relations based on API usages.
The reason is that API implementations have only one call site for the analysis, whereas
API usages can have many call sites for mining; providing relatively high confidence on mapped
relations. (3) Mapping relations of APIs are more complex and cannot be captured solely
based on the information available in the API implementations. We next
show why it is not possible to capture mapping relations
based on the information available in API implementations using two illustrative examples.

Consider the following two API methods in Java and C\#:

\begin{CodeOut}
$m_1$ in Java: BigDecimal java.math.BigDecimal.multiply (BigDecimal $p_1^1$)\\
\hspace*{0.12in}$m_2$ in C\#:\ \ \ \  Decimal
System.Decimal.Multiply (Decimal $p_1^2$, Decimal $p_2^2$)
\end{CodeOut}

Here, $m_1$ has a receiver, say $v_1^1$, of type \CodeIn{BigDecimal}
and has one parameter $p_1^1$, and $m_2$ has two parameters $p_1^2$
and $p_2^2$. Based on the definitions of these inputs, $v_1^1$ is
mapped to $p_1^2$, and $p_1^1$ is mapped to $p_2^2$. This preceding
example shows the complexities involved in mapping parameters of an API
method in one language with an API method in the other language. 
We next provide a more complex example where an API method of one 
language is mapped to more than one API method in the other language. Consider the following two API methods:

\begin{CodeOut}
$m_3$ in Java: E java.util.LinkedList.removeLast()\\
\hspace*{0.12in}$m_4$ in C\#: void System.Collections.Generic.LinkedList.RemoveLast()
\end{CodeOut}

Although the method names of $m_3$ and $m_4$ are the same, $m_3$ in Java
cannot be directly mapped with $m_4$ in C\#. The reason is that $m_3$ in Java
returns the last element removed from the list, whereas $m_4$ does not return any
element. Therefore, $m_3$ is mapped to two API methods $m_4$ and $m_5$ (shown below) in C\#.
The API method $m_5$ returns the last element and should be called before calling $m_4$.

\begin{CodeOut}
$m_5$ in C\#: void System.Collections.Generic.LinkedList.Last()
\end{CodeOut}

This example shows that an API method of one language is mapped
to multiple API methods of the other language. Therefore, capturing API mapping relationships
based on API implementations is often not possible. We next describe our approach
that mines mapping relations using the API client code.

Our approach accepts existing projects such as Lucene that have both Java and C\# versions,
and mines mapping relations of APIs. We refer these existing projects as client code
using the APIs of two languages. First, our approach aligns classes and methods of the 
two versions by using a matching algorithm based on similarities in the names
of classes and methods. Aligning client code based on names of classes and methods is
based on our observation of how existing projects such as rasp\footnote{\url{http://sourceforge.net/projects/r-asp/}} are
migrated from one language to another. We observed that while migrating rasp project
from Java to C\#, programmers first rename source files from Java to C\# and systematically
address the compilation errors by replacing Java APIs with C\# APIs. During this procedure,
names of classes, methods, fields of classes, or local variables in methods often remain
the same between the two versions. Therefore, we use name similarities for aligning client
code of the two versions. Second, our approach maps API classes of one language with
the other language by matching the names of fields of classes and local variables of methods in the client code.
Finally, our approach maps API methods of one language with the other language. Mapping API methods
is challenging as one API method of one language can be mapped to multiple API
methods of the other language (as shown in our preceding example). To address this challenge, 
we construct a graph, referred as \emph{API transformation graph} (ATG), for aligned methods
of the client code in both languages. These ATGs precisely capture inputs and outputs
of API methods, and help mine relationships API methods. (@Hao, could you please revise this preceding sentence and put more details)

This paper makes the following major contributions:

\begin{itemize}\vspace*{-1.5ex}
\item A first approach that mines mapping relations of APIs between
different languages using API client code. Our
approach addresses an important and yet challenging problem that is not
addressed by previous work on language migration.\vspace*{-1.5ex}
\item A technique to build and compare API
transformation graphs (ATG). ATGs describe data dependencies among
inputs and outputs of API methods. Our approach uses these ATGs to
mine complex many-to-many mapping relations between API methods.\vspace*{-1.5ex}
\item A tool named MAM based on our approach and two
evaluations on 15 projects with both Java and C\# versions.
These projects include 18,568 classes and 109,850 methods. The
results show that our approach mines 26,369 mapping relations of
APIs with 83.2\% accuracies and the mined API mapping relations reduce 55.4\% of
compilation errors during translation projects from Java to C\#.
\end{itemize}\vspace*{-1.5ex}

The remainder of this paper is organized as follows. Section~\ref{sec:example}
illustrates our approach using an example. Section~\ref{sec:mapping}
presents definitions. Section~\ref{sec:approach} presents our
approach. Section~\ref{sec:evaluation} presents our evaluations.
Section~\ref{sec:discuss} discusses issues of our approach.
Section~\ref{sec:related} presents related work. Finally,
Section~\ref{sec:colcusion} concludes.
