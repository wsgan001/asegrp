\section{Related Work}
\label{sec:related}

Our approach is related to previous work on
language translation and library migration.

\textbf{Language translation.} To reduce manual effort of language
translation~\cite{samet1981experience}, researchers propose various
approaches to automate the
process~\cite{hassan2005lightweight,van1999identifying,waters1988program,mossienko2003automated,yasumatsu1995spice}.
Most of these approaches focus on the syntax differences between
languages. For example, Deursen \emph{et
al.}~\cite{van1999identifying} propose an approach to identify
objects in legacy code, and the results are useful to deal with
differences between object-oriented and procedural languages. As
shown by El-Ramly \emph{et al.}~\cite{el2006experiment}'s experience
report, existing approaches and tools support only a subset of APIs,
and consequently it becomes an important and yet challenging task to
automate API transformation. Our approach mines API mapping between
languages to aid language translation, addressing a significant
problem not addressed by the previous approaches and complementing
these approaches.

\textbf{Library migration.} With evolution of libraries, some APIs
may become incompatible across library versions. To deal with the
problem, some approaches have been proposed. In particular, Henkel
and Diwan~\cite{henkel2005catchup} propose an approach that captures
and replays API refactoring actions to keep client code updated.
Xing and Stroulia~\cite{xing2007api} propose an approach that
recognizes the changes of APIs by comparing the differences of two
versions of libraries. Balaban \emph{et
al.}~\cite{balaban2005refactoring} propose an approach to help
translate client code when mapping relations of libraries are
available. Different from these approaches, our approach focuses on
mapping relations of APIs across different languages. In addition, since
our approach uses API transformation graphs to mine mapping
relations of APIs, our approach helps mine mapping relations for
those API methods whose input orders are changed or whose
functionalities are split into several methods if our approach is
applied in library migration.
