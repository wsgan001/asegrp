\section{Related Work}
\label{sec:related}

Our approach is related to previous work on two areas:
language translation and library migration.

\textbf{Language translation.} To reduce manual efforts of language
translation~\cite{samet1981experience}, researchers proposed various
approaches~\cite{hassan2005lightweight,van1999identifying,waters1988program,mossienko2003automated,yasumatsu1995spice, kim2009discovering} to automate the process.
However, all these approaches focus on the syntax or structural differences between
languages. For example, Kim and Notkin~\cite{kim2009discovering} proposed LSdiff
approach that infers systematic structural differences as logic rules.
Deursen \emph{et al.}~\cite{van1999identifying} proposed an approach to identify
objects in legacy code. Their approach uses these objects to deal with the
differences between object-oriented and procedural languages. As
shown in El-Ramly \emph{et al.}~\cite{el2006experiment}'s experience
report, existing approaches support only a subset of APIs for language translation,
making the task of language translation as a challenging problem.
In contrast to previous approaches, our approach automatically mines API mapping between
languages to aid language translation, addressing a significant
problem not addressed by the previous approaches and complementing
these approaches.

\textbf{Library migration.} With evolution of libraries, some APIs
may become incompatible across library versions. To address this
problem, Henkel and Diwan~\cite{henkel2005catchup} proposed an approach that captures
and replays API refactoring actions to update the client code.
Xing and Stroulia~\cite{xing2007api} proposed an approach that
recognizes the changes of APIs by comparing the differences between two
versions of libraries. Balaban \emph{et al.}~\cite{balaban2005refactoring} proposed
an approach to translate client code when mapping relations of libraries are
available. In contrast to these approaches, our approach focuses on
mapping relations of APIs across different languages. In addition, since
our approach uses ATGs to mine API mapping relations, our approach can also
mine mapping relations between API methods with different parameters or between
API methods whose functionalities are split among several API methods in the other language.
