\section{Related Work}
\label{sec:related} In this section, we introduce related work and
discuss our contributions.

\textbf{Language migration.} It is a research topic with a long
history to migrate projects of one language into other
languages~\cite{samet1981experience}. To reduce the human effort of
language migration, researchers propose various approaches to
automate the
process~\cite{van1999identifying,waters1988program,mossienko2003automated,yasumatsu1995spice,hainaut2008migration}.
Most of these approaches focus the syntax differences among
languages. For example, Deursen \emph{et
al.}~\cite{van1999identifying} propose an approach to identify
objects in legacy code, and the results are useful to deal with the
difference  between object-oriented languages and procedural
languages. As shown by El-Ramly \emph{et
al.}~\cite{el2006experiment}'s experience report, existing
approaches and tools support only a subset of APIs, and consequently
it becomes an important to automate API transformation. Our approach
mines API mapping among languages to aid language migration,
complementing the preceding approaches.

\textbf{Library migration.} With the evaluation of libraries, some
APIs may become incompatible. To deal with the problem, some
approaches have been proposed. In particular, Henkel and
Diwan~\cite{henkel2005catchup} propose an approach that captures and
replay API refracturing actions to keep client code updated. Xing
and Stroulia~\cite{xing2007api} propose an approach that recognizes
the changes of APIs by comparing the differences of two versions of
libraries. Balaban \emph{et al.}~\cite{balaban2005refactoring}
propose an approach to help translate client code when mapping
relations of libraries are available. Different from these
approaches, our approach focuses on mapping relations of APIs among
different languages. In addition, as our approach uses ATGs to mine
mapping relations of APIs, our approach helps mine mapping relations
for those API methods whose input orders is changed or whose
functionalities are split into several methods if our approach is
applied in library migration.
