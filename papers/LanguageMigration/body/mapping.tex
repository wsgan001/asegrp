\section{Definitions}
\label{sec:mapping}

We next present definitions of terms used in the rest of the paper.

\textbf{API.} An Application Programming Interface (API)~\cite{orenstein2000quickstudy}
is a set of classes and methods provided by frameworks or libraries.

\textbf{API library.} An API library is a framework
or library that provides reusable API classes and methods.

\textbf{Client code.} Client code is application code
that reuses or extends API classes and methods provided by API
libraries.

The definitions of API library and client code are
relative to each other. For example, Lucene uses classes and methods provided by
J2SE\footnote{\url{http://java.sun.com/j2se/1.5.0/}}. Therefore, we consider Lucene as client code and J2SE as an API library. At the same time, Nutch\footnote{\url{http://lucene.apache.org/nutch/}} uses classes and methods provided by Lucene. Therefore, we consider Nutch as client code and Lucene as an API library. In
general, for programmers of client code, source files of API libraries may not be available.

\textbf{Mapping relation.} For entities $E_1$ (such as API classes and methods) in a
language $L_1$ and entities $E_2$ in another
language $L_2$, a mapping relation is a triple $\langle E_1, E_2,
b \rangle$ where translating between $E_1$ and $E_2$ maintains the $b$ behavior.
The $b$ behavior is specific to the type of the entity.

%\textbf{1-to-1 mapping relation of API classes.} For an API class
%$c_1$ defined by $L_1$ and an API class $c_2$ defined by $L_2$, a
%1-to-1 mapping relation of API classes is a triple $\langle c_1,
%c_2, s \rangle$, where $s$ denotes the $s$ behavior.
%
%One API class defined by $L_1$ can have more than one 1-to-1 mapping
%relation with API classes defined by $L_2$. For example, data in
%\CodeIn{java.util.ArrayList} of Java can be stored in either
%\CodeIn{System. Collections.ArrayList} \textbf{or}
%\CodeIn{System.Collections.Generic. List} of C\#, so the Java class
%has two 1-to-1 mapping relations with these two C\# classes.

\textbf{Mapping relation of API classes.} For data entities whose type set is $C_1$ in $L_1$ and data entities whose type set is $C_2$ in
$L_2$, a mapping relation of API classes is a triple
$\langle C_1, C_2, s \rangle$, where translating between $C_1$ and $C_2$ maintains the $s$ behavior.

Since we use mapping relations of API classes for translating data entities such as
variables, parameters, and constants, we require that two mapped
API classes have the same program behavior to store data, referred to as
the $s$ behavior. For example, the current time in \CodeIn{java.lang.System} of Java
is stored in \CodeIn{System.DateTime} of C\#, whereas the
environment settings in \CodeIn{java.lang.System} of Java is stored in
\CodeIn{System.Environment} of C\#. Therefore, the Java class has a
one-to-many mapping relation with two C\# classes.



%\textbf{1-to-1 mapping relation of API methods.} For an API method
%$m_1$ defined by $L_1$ and an API method $m_2$ defined by
%$L_2$, a mapping relation of API methods is a triple $\langle m_1,
%m_2, t \rangle$, where $t$ denotes the $t$ behavior.
%
%As one API class defined by $L_1$ can have more than one 1-to-1 mapping
%relation with API classes defined by $L_2$, one API method defined by $L_1$ can have more than one 1-to-1 mapping relation with API methods defined by $L_2$.

\textbf{Mapping relation of API methods.} For invoked API methods
$M_1$ in $L_1$ and invoked API methods $M_2$ in
$L_2$, a mapping relation of API methods is a triple
$\langle M_1, M_2, t \rangle$, where translating between $M_1$ and $M_2$ maintains the $t$ behavior.

Since we use mapping relations of API methods for translating API methods
that accept inputs to produce desirable outputs, we require two
mapped API methods have the same program behavior of inputs, outputs, and
functionalities. We refer to this behavior as the \emph{t} behavior. For example, Section~\ref{sec:introduction} shows a one-to-many mapping relation between $\{m_3\}$ of Java and $\{m_4, m_5\}$ of C\#. 