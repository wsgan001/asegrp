\section{Definitions}
\label{sec:mapping}

We next present definitions of the terms used in the rest of the paper.

\textbf{API:} An Application Programming Interface (API)~\cite{orenstein2000quickstudy}
is a set of classes and methods provided by frameworks or libraries.

\textbf{API library:} An API library refers to a framework
or library that provides reusable API classes and methods.

\textbf{Client code:} Client code refers to the application code
that reuses or extends API classes and methods provided by API libraries.
The definitions of API library and Client code are relative to each other. 
For example, Lucene\footnote{\url{???}} uses J2SE\footnote{\url{http://java.sun.com/j2se/1.5.0/}} 
as an API library, whereas Nutch\footnote{\url{http://lucene.apache.org/nutch/}} uses
Lucene as an API library. Therefore, we consider Lucene as Client code and API library
for the J2SE API library and Nutch, respectively. In general, for Client code, 
source files of API libraries are often not available.

\textbf{Mapping relation:} A mapping relation refers to a replaceable
relation among entities such as API classes or methods defined by two different languages. For example,
consider two languages $L_1$ and $L_2$, and two entities $e_1$ and $e_2$
in languages $L_1$ and $L_2$, respectively. We define a mapping relation
between the entities $e_1$ and $e_2$, if $e_1$ of the Language $L_1$ can
be translated to $e_2$ of the Language $L_2$ without introducing new defects
in the translated entity. (@Hao: Should this be something like both entities
should have the same behavior).

\textbf{Mapping relation of API classes:} We define a mapping relation between 
two API classes $c_1$ and $c_2$ of languages $L_1$ and $L_2$, respectively, 
if the API class $c_1$ of $L_1$ is translated to the API class $c_2$ of $L_2$ 
without introducing new defects in the translated code. Our mapping
relation of API classes is many-to-many (@Hao: Please check whether it is many-to-many). For example, 
the \CodeIn{java.util.ArrayList} class of Java is mapped to either 
\CodeIn{System.Collections.ArrayList} \textbf{or} \CodeIn{System.Collections.Generic.List} of C\#.
On the other hand, the \CodeIn{java.lang.System} class of Java is mapped
to \CodeIn{System.DataTime} \textbf{and} \CodeIn{System.Environment} of C\#
based on how Client code uses these classes. For example,
\CodeIn{java.lang.System} is mapped with \CodeIn{System.DataTime}
when client code uses the two classes to get the current time and is
mapped with \CodeIn{System.Environment} when client code uses the
two classes to get environment settings. (@Hao: Could you please revise
this sentence. I haven't got what this means?).

Furthermore, mapped API classes can have different functionalities in different
languages. For example, \CodeIn{java.lang.String} of Java is mapped to \CodeIn{System.String} 
of C\#. However, \CodeIn{System.String} has an API method 
\CodeIn{insert}, which does not exist in \CodeIn{java.lang.String}.

\textbf{Mapping relation of API methods:} We define a mapping relation
between two API methods $m_1$ and $m_2$ of languages $L_1$ and $L_2$,
respectively, if $m_1$ is translated to $m_2$ without introducing defects 
in the translated code.

Both the mapping relations of API classes and methods are required for
achieving language translation. In particular, mapping relation of API classes is
required to translate variables such as \CodeIn{file} in Figure~\ref{}.
Similarly, mapping relation of API methods is
required to translate API methods such as \CodeIn{exist()} in Figure~\ref{}.
When an API method is translated from one language to another, the translated
method accepts the same parameters (both variables and constants) and implement
the same functionality as the original method.

\textbf{Merged API method:} A merged API method of $L_1$ refers to an API method
that is created by merging two other API methods of $L_1$. (@Hao: From my
understanding, i think that merged API method is within the language. Is that correct?)
For example, consider two API methods $m_1$ and $m_2$ defined in classes
$C_1$ and $C_2$ of $L_1$, respectively, with the
following signatures:

\CodeIn{$m_1$ signature: $o_1$ $C_1$.$m_1$($inp_1^1$, $inp_2^1$, ..., $inp_m^1$)}

\CodeIn{$m_2$ signature: $o_2$ $C_2$.$m_2$($inp_1^2$, $inp_2^2$, ..., $inp_n^2$)}

We merge methods $m_1$ and $m_2$ to create a new merged API method $m_{new}$ if the
output $o_1$ of $m_1$ is used either as a receiver object or a 
parameter for $m_2$ (i.e., $o_1$ == $C_2$ or $o_1$ == $inp_i^2$)
in Client code. The signature of the new merged API method $m_{new}$ is 
shown below:

\CodeIn{$m_{new}$ signature: $o_2$ $m_{new}$($inp_1^1$, $inp_2^1$, ..., $inp_m^1$,$inp_1^2$, $inp_2^2$, ..., $inp_n^2$)}

We next present an example for a merged API method using the
illustrative code example shown in Section~\ref{sec:example}. (@Suresh:
Insert a figure for the code sample and refer it from here).
Consider the \CodeIn{file} variable, which is a return variable
for the constructor and a receiver object for the \CodeIn{exist()} method.
As the output of one API method is passed as receiver object of
another API method, we can combine these two methods to create a new
merged API method $m_{new}$. Figure~\ref{fig:example} shows 
the $m_{new}$ method \CodeIn{boolean File.exists(string)}. The $m_{new}$
method accepts a \CodeIn{string} parameter that represents a file name
and returns a boolean value that describes whether a file exists or not. 

A merged API method can be further merged with other API methods or 
other merged API methods. For simplicity, we use API method to refer
to both API method and merged API method in the rest of the paper.