\section{Definitions}
\label{sec:mapping}

We next present definitions of terms used in the rest of the paper.

\textbf{API:} An Application Programming Interface (API)~\cite{orenstein2000quickstudy}
is a set of classes and methods provided by frameworks or libraries.

\textbf{API library:} An API library refers to a framework
or library that provides reusable API classes and methods.

\textbf{Client code:} Client code refers to the application code
that reuses or extends API classes and methods provided by API
libraries. The definitions of API library and client code are
relative to each other. For example, Lucene uses
J2SE\footnote{\url{http://java.sun.com/j2se/1.5.0/}} as an API
library, whereas Nutch\footnote{\url{http://lucene.apache.org/nutch/}} uses Lucene as
an API library. Therefore, we consider Lucene as client code and API
library for the J2SE API library and Nutch, respectively. In
general, for client code, source files of API libraries are often
not available.

\textbf{Mapping relation:} A mapping relation refers to a
replaceable relation among entities such as API classes or methods
defined by two different languages. For example, consider two
languages $L_1$ and $L_2$, and two entities $e_1$ and $e_2$ in
$L_1$ and $L_2$, respectively. We define a mapping
relation between $e_1$ and $e_2$, if $e_1$ of $L_1$ can be translated to $e_2$ of $L_2$
without introducing new defects in the translated code.

\textbf{Mapping relation of API classes:} We define a mapping
relation between two API classes $c_1$ and $c_2$ of $L_1$ and $L_2$,
respectively, if $c_1$ of $L_1$ is translated to $c_2$ of $L_2$
without introducing new defects in the translated code.

Our mapping relation of API classes is many-to-many. For example,
\CodeIn{java.util.ArrayList} of Java is mapped to either
\CodeIn{System. Collections.ArrayList} \textbf{or}
\CodeIn{System.Collections.Generic. List} of C\#, whereas
\CodeIn{java.lang.System} of Java is mapped to
\CodeIn{System.DataTime} \textbf{and} \CodeIn{System.Environment} of
C\# based on how client code uses these classes. In particular, when
client code uses APIs to get the current time,
\CodeIn{java.lang.System} is mapped to \CodeIn{System.DataTime}. In
contrast, when client code uses APIs to get environment settings,
\CodeIn{java.lang.System} is mapped with
\CodeIn{System.Environment}.

Furthermore, mapped API classes may have different behaviors. For
example, \CodeIn{java.lang.String} of Java is mapped to
\CodeIn{System. String} of C\#. However, \CodeIn{System.String} has
an API method \CodeIn{insert}, which does not exist in
\CodeIn{java.lang.String}.

\textbf{Mapping relation of API methods:} We define a mapping relation
between two API methods $m_1$ and $m_2$ of languages $L_1$ and $L_2$,
respectively, if $m_1$ is translated to $m_2$ without introducing defects
in the translated code.

\textbf{Merged API method:} A merged API method of $L_1$ refers to
an API method that is created by merging two other API methods of
$L_1$. For example, consider two API methods $m_1$ and $m_2$ defined
in classes $C_1$ and $C_2$ of $L_1$, respectively, with the
following signatures:

\CodeIn{$m_1$ signature: $o_1$ $C_1$.$m_1$($inp_1^1$, $inp_2^1$, ..., $inp_m^1$)}

\CodeIn{$m_2$ signature: $o_2$ $C_2$.$m_2$($inp_1^2$, $inp_2^2$, ..., $inp_n^2$)}

We merge methods $m_1$ and $m_2$ to create a new merged API method $m_{new}$ if the
output $o_1$ of $m_1$ is used either as a receiver or as a
parameter for $m_2$ (i.e., $o_1$ == $C_2$ or $o_1$ == $inp_i^2$)
in client code. The signature of the new merged API method $m_{new}$ is
shown below:

\CodeIn{$m_{new}$ signature: $o_2$ $m_{new}$($inp_1^1$, $inp_2^1$, ..., $inp_m^1$, \\ $inp_1^2$, $inp_2^2$, ..., $inp_n^2$)}

We next present an example for a merged API method using the
illustrative code example shown in Section~\ref{sec:example}. For
the code example shown in Figure~\ref{fig:challenge}, consider the
\CodeIn{file} variable, which is a return variable for the
constructor and a receiver object for the \CodeIn{exists} method. As
the output of one API method is passed as receiver object of another
API method, we can combine these two methods to create a new merged
API method $m_{new}$. Figure~\ref{fig:example} shows the $m_{new}$
method \CodeIn{boolean File.exists(string)}. The $m_{new}$ method
accepts a \CodeIn{string} parameter that represents a file name and
returns a boolean value that describes whether a file exists or not.

A merged API method can be further merged with other API methods or
other merged API methods. For simplicity, we use API method to refer
to both API method and merged API method in the rest of the paper.

Both the mapping relations of API classes and methods are required
for achieving language translation. In particular, mapping relation
of API classes is required to translate variables such as
\CodeIn{file} in Figure~\ref{fig:challenge}. Similarly, mapping
relation of API methods is required to translate API methods such as
\CodeIn{exist} in Figure~\ref{fig:challenge}. When an API method is
translated from one language to another, the translated method
accepts the same parameters (both variables and constants) and
implement the same functionality as the original method.
