\section{Definitions}
\label{sec:mapping}

We next present definitions of terms used in the rest of the paper.

\textbf{API.} An Application Programming Interface (API)~\cite{orenstein2000quickstudy}
is a set of classes and methods provided by frameworks or libraries.

\textbf{API library.} An API library is a framework
or library that provides reusable API classes and methods.

\textbf{Client code.} Client code is application code
that reuses or extends API classes and methods provided by API
libraries.

The definitions of API library and client code are relative to each other. For example, Lucene uses
J2SE\footnote{\url{http://java.sun.com/j2se/1.5.0/}} as an API
library, whereas Nutch\footnote{\url{http://lucene.apache.org/nutch/}} uses Lucene as
an API library. Therefore, we consider Lucene as client code and API
library for the J2SE API library and Nutch, respectively. In
general, for programmers of client code, source files of API libraries are often
not available.

\textbf{Mapping relation.} For a set of entities $E_1$ defined by a
language $L_1$ and another set of entities $E_2$ defined by another
language $L_2$, a mapping relation is a triple $\langle E_1, E_2,
b_i \rangle$ where $E_1$ and $E_1$ have the same behavior $b_i$.

We use mapping relations of API classes for translating data entities such as
variables, parameters, and constants, so we require that two mapped
API classes have the same program behavior, referred to as
the $s$ behavior.

\textbf{1-to-1 mapping relation of API classes.} For an API class
$c_1$ defined by $L_1$ and an API class $c_2$ defined by $L_2$, an
1-to-1 mapping relation of API classes is a triple $\langle c_1,
c_2, s \rangle$, where $s$ denotes the $s$ behavior.

One API class defined by $L_1$ can have more than one 1-to-1 mapping
relation with API classes defined by $L_2$. For example, data in
\CodeIn{java.util.ArrayList} of Java can be stored in either
\CodeIn{System. Collections.ArrayList} \textbf{or}
\CodeIn{System.Collections.Generic. List} of C\#, so the Java class
has two 1-to-1 mapping relations with the two C\# classes.

\textbf{1-to-many mapping relation of API classes.} For an API class
$c_1$ defined by $L_1$ and a set of API classes $C_2$ defined by
$L_2$, a 1-to-many mapping relation of API classes is a triple
$\langle c_1, C_2, s \rangle$, where $s$ denotes the $s$ behavior.

For example, the current time in \CodeIn{java.lang.System} of Java
is stored in \CodeIn{System.Environment} of C\#, whereas the
environment settings in \CodeIn{java.lang.System} is stored in
\CodeIn{System.Environment} of C\#, so the Java class has a
1-to-many mapping relation with the two C\# classes.

We use mapping relations of API methods for translating API methods
that accept inputs to produce desirable outputs, so we require that two
mapped API methods have the same program behavior of inputs, outputs, and
functionalities, referred to as the ``\emph{t}'' behavior.

\textbf{Merged API method.} A merged API method $cm$ is a sequence of API
methods $M$ combined by inputs and outputs, so the $t$ behavior of
$cm$ is the combination of $M$. Consider two API methods $m_1$ and
$m_2$ defined in classes $C_1$ and $C_2$ of $L_1$, respectively,
with the following signatures:

\CodeIn{$m_1$ signature: $o_1$ $C_1$.$m_1$($inp_1^1$, $inp_2^1$,
..., $inp_m^1$)}

\CodeIn{$m_2$ signature: $o_2$ $C_2$.$m_2$($inp_1^2$, $inp_2^2$,
..., $inp_n^2$)}

We merge methods $m_1$ and $m_2$ to create a new \emph{merged API
method} $m_{new}$ if the output $o_1$ of $m_1$ is used either as a
receiver or as a parameter for $m_2$ (i.e., $o_1$ == $C_2$ or $o_1$
== $inp_i^2$) in client code. The signature of the new merged API
method $m_{new}$ is shown below:

\CodeIn{$m_{new}$ signature: $o_2$ $m_{new}$($inp_1^1$, $inp_2^1$,
..., $inp_m^1$, \\ $inp_1^2$, $inp_2^2$, ..., $inp_n^2$)}

For the Java code example shown in Figure~\ref{fig:challenge},
consider the \CodeIn{file} variable, which is a return for
the constructor and a receiver for the \CodeIn{exists}
method. Since the output of one API method is passed as a receiver 
of another API method, we can combine these two methods to create a
new merged API method \CodeIn{boolean File.exists(string)} as shown
in Figure~\ref{fig:example} (b). The merged API method accepts a
\CodeIn{string} parameter that represents a file name and returns a
boolean value that describes whether a file exists or not. For
simplicity, we consider each API method as a merged API method when
we define mapping relations of API methods.

\textbf{Mapping relation of API methods.} For a merged API method
$cm_1$ defined by $L_1$ and a merged API method $cm_2$ defined by
$L_2$, a mapping relation of API methods is a triple $\langle cm_1,
cm_2, t \rangle$, where $t$ denotes the $t$ behavior.
