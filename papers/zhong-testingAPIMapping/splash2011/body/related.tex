\section{Related Work}
\label{sec:related}

Our approach is related to previous work on areas as follows.

\subsection{API Translation} To reduce efforts of language translation, researchers proposed various
approaches to automate the process (\emph{e.g.}, JSP to ASP~\citep{hassan2005lightweight}, Cobol to Hibol~\citep{waters1988program}, Fickle to Java~\citep{ancona2007provenly}, Cobol to Java~\citep{mossienko2003automated}, Smalltalk to C~\citep{yasumatsu1995spice}, and Java to C\#~\citep{el2006experiment}). Song and Tilevich~\citep{song2009enhancing} provided an enhanced specification to improve source-to-source translation approaches. El-Ramly \emph{et al.}~\citep{el2006experiment} point out that API translation is an important part of language translation, and our previous work~\citep{zhong2010mining} mines API mapping relations from existing applications in different languages to improve the process. Besides language migration, other processes also involve API translation. For example, programmers often need to update applications with the latest version of API libraries, and a new version may contain breaking changes. Henkel and Diwan~\citep{henkel2005catchup} proposed an approach that captures and replays API refactoring actions to update the client code. Xing and Stroulia~\citep{xing2007api} proposed an approach that recognizes the changes of APIs by comparing the differences between two versions of libraries. Balaban \emph{et al.}~\citep{balaban2005refactoring} proposed an approach to migrate code when mapping relations of libraries are available. As another example, programmers may translate applications to use alternative APIs. Dig \emph{et al.}~\citep{dig2009refactoring} propose \emph{CONCURRENCER} that translates sequential API elements to concurrent API elements in Java. Nita and Notkin~\citep{nita2010using} propose twinning to automate the process given that API mapping is specified. Kapur \emph{et al.}~\citep{kapur2010refactoring} proposed Trident that allows programmers to refactor references for library migration.
Our approach detects behavioral differences between mapped API elements, and the results help the preceding approaches translate applications with fewer defects.

\subsection{Language Comparison} To reveal differences between languages, researchers conducted various empirical comparisons on languages. Garcia \emph{et al.}~\citep{Garcia2003} present a comparison study on six languages to reveal their differences of supporting generic programming. Cabral and Marques~\citep{cabral2007exception} compare exception handling mechanisms between Java and .NET programs. Appeltauer \emph{et al.}~\citep{appeltauer2009comparison} compare 11 context-oriented programming languages (\emph{e.g.}, Lisp) for their designs and performance. Rodriguez \emph{et al.}~\citep{yakushevcomparing} compared various generic libraries in Haskell. Trent \emph{et al.}~\citep{trent2008performance} compared PHP with JSP for their performance on Apache and Lighttpd.
To the best of our knowledge, no previous work systematically compares behavioral differences of API elements from different languages. Our approach enables us to produce such a comparison study, complementing the preceding empirical comparisons.

%\textbf{Regression testing and differential testing.} Our approach uses a similar test oracle with regressing testing~\citep{yoo2009tr,taneja08diffgen} and differential testing~\citep{mckeeman1998differential}

%\textbf{Mining specifications.} Some of our previous approaches~\citep{zhong09:inferring,zhong09:mapo,thummalapenta09:mining,thummalapenta09:mseqgen,acharya09:mining} focus on mining specifications. MAM mines API mapping relations across different languages for language migration, whereas the previous approaches mine API properties of a single language to detect defects or to assist programming.
