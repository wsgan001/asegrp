\section{Introduction}
\label{sec:introduction}

Since the inception of computer science, many programming languages (\emph{e.g.}, Cobol, Fortran, or Java) have been introduced to serve specific requirements\footnote{\url{http://hopl.murdoch.edu.au}}. For example, Cobol is introduced specifically for developing business applications. In general, software companies or open source organizations often release their applications in different languages to survive in competing markets and to address various business requirements such as platform independence. An empirical study~\cite{jones1998estimating} shows that nearly one third applications have multiple versions in different languages. A natural way to implement an application in a different language is to translate from an existing application. For example, Lucene.Net was translated from Java Lucene according to its website\footnote{\url{http://lucene.apache.org/lucene.net/}}. As another example, the NeoDatis object database was also translated form Java to C\# according to its website\footnote{\url{http://wiki.neodatis.org/}}. During translation, one primary goal is to ensure that both applications exhibit the same behavior.

Since existing applications typically use API libraries, it is essential to understand API mapping relations of one programming language, referred to as $L_1$, to another language, referred to as $L_2$ when translating applications from $L_1$ to $L_2$. Robillard~\cite{robillard2009makes} points out that it is hard to use API elements, and our previous work~\cite{zhong2010mining} shows that the mapping relations between mapped API elements of different languages can also be complicated. In some cases, programmers may fail to find an existing class that has the same behavior in the other language. For example, Figures~\ref{fig:db4ojava} and \ref{fig:db40net} show two methods implemented in db4o\footnote{\url{http://www.db4o.com}} of its Java version and its C\# version, respectively. When translating the Java code shown in Figure~\ref{fig:db4ojava} to C\#, programmers of db4o may fail to find an existing C\# class that has the same behaviors with the \CodeIn{ByteArrayInputStream} class in Java, so they implement a C\# class with the same name to fix the behavioral difference. Behavioral differences of mapped API elements (\emph{i.e}., classes, methods, and fields) may occur in many places. To reduce translation effort, programmers of db4o developed their own translation tool, called sharpen\footnote{\url{http://developer.db4o.com/Blogs/News/tabid/171/entryid/653/Default.aspx}}, for translating db4o from Java to C\#. For API translation, sharpen systematically replaces all API elements in Java with equivalent elements in C\# to ensure that translated C\# applications have the same behaviors with the original Java ones.

\begin{figure}[t]
\begin{CodeOut}%\vspace*{-2ex}
\begin{alltt}
01: private long readLong(ByteArrayInputStream is)\{
02:  ...
03:  l += ((long) (is.read())) << i;
04: ...\}
\end{alltt}
\end{CodeOut}\vspace*{-4ex}
\caption{A method in the Java version of db4o.}\vspace*{-2ex}
\label{fig:db4ojava}
\begin{CodeOut}%\vspace*{-2ex}
\begin{alltt}
05: private long ReadLong(ByteArrayInputStream @is)\{
06:  ...
07:  l += ((long)(@is.Read())) << i;
08:  ...\}
\end{alltt}
\end{CodeOut}\vspace*{-4ex}
\caption{A method in the C\# version of db4o.}\vspace*{-2ex}
\label{fig:db40net}
\end{figure}

In practice, as pointed out by Keyvan Nayyeri\footnote{\url{http://dotnet.dzone.com/print/26587}}, one of the most common problems is that translated code does not return expected outputs, partially because behavioral differences of mapped API elements are not fully fixed. It is desirable to detect such differences, but existing approaches~\cite{orso1using,jin2010automated} cannot detect such differences effectively since these existing approaches require that both the versions under consideration belong to the same language, but in our context, the versions belong to different languages, making these existing approaches inapplicable.
%jin2010automated
\begin{figure}[t]
\begin{CodeOut}%\vspace*{-2ex}
\begin{alltt}
09: DatagramSocket socket = ...;
10: DatagramPacket package = ...;
11: socket.receive(package);
\end{alltt}
\end{CodeOut}\vspace*{-5ex}
\caption{Sample code in Java.}\vspace*{-2ex}
\label{fig:javacode}
\begin{CodeOut}%\vspace*{-2ex}
\begin{alltt}
12: UdpClient socket = ...;
13: IPEndPoint remoteIpEndPoint = ...;
14: try\{
15:  byte[] data_in = socket.Receive(ref remoteIpEndPoint);
16:  PacketSupport tempPacket =
          new PacketSupport(data_in, data_in.Length);
17:   tempPacket.IPEndPoint = remoteIpEndPoint;
18: \} catch (System.Exception e)\{...\}
19: PacketSupport package = tempPacket;
\end{alltt}
\end{CodeOut}\vspace*{-5ex}
\caption{Translated C\# code by JLCA.}\vspace*{-2ex}
\label{fig:codeJLCA}
\end{figure}

To address the preceding issue, we propose a novel approach, called TeMAPI (\textbf{Te}sting \textbf{Ma}pping relations of \textbf{API}s), that generates test cases to detect behavioral differences among API mapping relations automatically. Given a translation tool from one language $L_1$ to the other language $L_2$, TeMAPI generates various test cases to detect behavioral differences among the tool's API mapping relations. TeMAPI next executes translated test cases to detect behavioral differences. In this paper, we primarily focus on behavioral differences that can be observed via return values of API methods or exceptions thrown by API methods.

TeMAPI addresses three major technical challenges in effectively detecting behavioral differences. (1) It is challenging to directly extract API mapping relations from translation tools for testing since they may follow different formats to describe such relations. For example, Java2CSharp\footnote{\url{http://j2cstranslator.sourceforge.net/}} uses mapping files, sharpen hardcodes relations in source files, and closed source translation tools such as JLCA typically hide mapping relations in binary files. Besides the format problem, the interfaces of two mapped API elements can be different, and one API element can be mapped to multiple API elements. For example, JLCA\footnote{\url{http://msdn.microsoft.com/en-us/magazine/cc163422.aspx}} translates the \CodeIn{java.net.DatagramSocket.receive(Data- gramPacket)} method in Java as shown in Figure~\ref{fig:javacode} to multiple C\# elements as shown in Figure~\ref{fig:codeJLCA}. In addition, detecting behavioral differences between existing applications and their translated applications addresses the problem only partially since applications typically use only a small portion of API elements. To address this issue, TeMAPI synthesizes a wrapper method for each API element at the finest level. After we translate synthesized wrappers using a translation tool, TeMAPI analyzes translated code for behavioral differences between mapped API elements. (2) Using a basic technique such as generating test cases with \CodeIn{null} values may not be significant in detecting behavioral differences among API mapping relations. Since we focus on object-oriented languages such as Java or C\#, to detect behavioral differences, generated test cases need to exercise various object states, which can be achieved using method-call sequences. To address this issue, TeMAPI leverages two existing state-of-the-art test generation techniques: random~\cite{pacheco2007feedback} and dynamic-symbolic-execution-based~\cite{koushik:cute, godefroid:dart, tillmann2008pex} ones. (3) Generating test cases on $App_1$ and applying those test cases on $App_2$ may not exercise many behaviors of API methods in $App_2$, thereby not sufficient to detect related behavioral differences. To address this issue, TeMAPI uses a round-trip technique that also generates test cases on $App_2$ and applies them back on $App_1$. We describe more details of our approach to address these challenges in subsequent sections.

This paper makes the following major contributions:

\begin{itemize}\vspace*{-1.5ex}
\item A novel approach, called TeMAPI, that automatically generates test cases to detect behavioral differences among API mapping relations. Given a translation tool, TeMAPI detects behavioral differences of all its API mapping relations automatically. It is important to detect such differences, since they can introduce defects in translated applications silently.%\vspace*{-1.5ex}
\item A tool implemented for TeMAPI and three evaluations on five popular translation tools. The results show the effectiveness of our approach in detecting behavioral differences of mapped API elements between different languages.
%Our tool can handle both mapping relations from one API element to another API element and also from many API elements to many other API elements.
%\vspace*{-1.5ex}
\item The first empirical comparison on behavioral differences of mapped API elements between the J2SE and .NET frameworks. TeMAPI enables us to produce such a comparison. The results show that various factors such as \CodeIn{null} inputs, \CodeIn{string} values, input ranges, different understanding, exception handling, static values, inheritance relations, and invocation sequences can lead to behavioral differences of mapped API elements. Based on the results, we further analyze their implications from various perspectives.
\end{itemize}\vspace*{-1.5ex}

The rest of this paper is organized as follows.
%Section~\ref{sec:mapping} presents our test adequacy criteria.
Section~\ref{sec:example} presents an illustrative example.
Section~\ref{sec:approach} presents our approach.
Section~\ref{sec:evaluation} presents our evaluation.
Section~\ref{sec:discuss} discusses issues of our approach.
Section~\ref{sec:related} presents related work.
Section~\ref{sec:conclusion} concludes.


%As stated by Sebesta~\cite{sebesta2002concepts}, modern programming languages start around 1958 to 1960 with the development of Algol, Cobol, Fortran and Lisp. Ever since then, thousands of programming languages came to existence as shown by HOPL website\footnote{\url{http://hopl.murdoch.edu.au}}. For various considerations, programmers often need to translate projects from one language to another language. For example, as stated by , to provide the language and platform independence, he translates Compose*~\cite{garcia-compose} in C\# to Compose*/J in Java. To relief the efforts of translating, programmers may use existing translation tools or even implement their own translation tools. For example, Salem \emph{et al.}~\cite{AgtashAEMBS06} report their experience of translating the BLUE financial system of the ICT company from Java to C\# by the JLCA\footnote{\url{http://tinyurl.com/2c4coln}} tool. For another example, to translate db4o\footnote{\url{http://www.db4o.com}} from Java to C\#, its programmers develop their own translation tool named sharpen\footnote{\url{http://tinyurl.com/22rsnsk}}.

%To translate a source file, a translation tool needs to its structures and its used API elements. As a project typically use thousands of API elements, it is often more difficult to translate used API elements especially for those languages whose structures are similar. In particular, El-Ramly \emph{et al.}~\cite{el2006experiment} conduct an experiment to translate Java programs to C\#. One of their learnt lessons is ``it
%becomes very important to develop methods for
%automatic API transformation''. Barry compares C\# with Java\footnote{\url{http://tinyurl.com/26d8xcp}}, and claims ``although coding in C\# is easy for a Java programmer..., the biggest challenge in moving from Java to the .NET Framework is learning the details of another set of class libraries''. If not knowing API mapping relations, a translation tool or a programmer cannot translate used API elements correctly. Even when such mapping relations are available, a migration process may introduce defects in translated code since mapped API elements can have behavioral differences. As reported by Panesar\footnote{\url{http://tinyurl.com/3xpsdtx}}, even most common methods such as \CodeIn{String.subString(int, int)} can have behavioral differences between Java and C\#. We investigate the mapping relations of existing translation tools, and we confirm that the behavioral differences of mapped API elements can cause defects in translated code. In particular, in Java2CSharp\footnote{\url{http://j2cstranslator.sourceforge.net/}}, one item of mapping files is described in its mapping files as follows:
%
%\begin{CodeOut}%\vspace*{-2ex}
%\begin{alltt}
%1 package java.lang::System\{
%2  class java.lang.String :: System:String\{
%3   method valueOf(Object) { pattern = @1.ToString(); }
%4   ...\}\}
%\end{alltt}
%\end{CodeOut}
%
%Line 2 of this item describes that the \CodeIn{java.lang.String} class in Java is mapped to the \CodeIn{System.String} class in C\#. Line 3 of this item describes that the \CodeIn{java.lang.String.valueOf(Object)} method is mapped to the \CodeIn{System.String.ToString()} method in C\#, and \CodeIn{@1} denotes the first parameter of the \CodeIn{valueOf(Object)} method. Based on the preceding mapping relation, Java2CSharp translates a Java code snippet (Lines 5 and 6) to a C\# code snippet (Lines 7 and 8) as follows.
%
%\begin{CodeOut}%\vspace*{-2ex}
%\begin{alltt}
%\textbf{  Java Code}
%5 Object obj = ...
%6 String value = java.lang.String.valueOf(obj);
%\textbf{  C# Code translated by Java2CSharp}
%7 Object obj = ...
%8 String value = obj.ToString();
%\end{alltt}
%\end{CodeOut}
%
%The translated code snippet compile well, but it has behavioral differences with the original Java code snippet. For example, if Line 5 assigns null to \CodeIn{obj}, \CodeIn{value} of Line 6 will be ``null''. If Line 7 assigns null to \CodeIn{obj}, \CodeIn{value} of Line 8 will not be set to ``null'' since it throws \CodeIn{NullReferenceException}.
%
%As it throw exceptions, the preceding difference of API mapping relation is relatively easy to detect since programmers often use extreme inputs such as null values as test cases. In particular, sharpen is aware of the differences, and the mapping relation in sharpen is defined as follows:
%
%\begin{CodeOut}
%\begin{alltt}
%9 public abstract class Configuration \{
%10 protected void setUpStringMappings() \{
%11   mapMethod("java.lang.String.valueOf",
%              runtimeMethod("getStringValueOf"));
%12  ...\} \}
%\end{alltt}
%\end{CodeOut}
%
%Based on Line 11 of the preceding mapping relation, sharpen translates the Java code snippet (Lines 5 and 6) to a C\# code snippet (Lines 7 and 8) as follows.
%
%\begin{CodeOut}%\vspace*{-2ex}
%\begin{alltt}
%\textbf{  C# Code translated by sharpen}
%13 Object obj = ...
%14 String value = getStringValueOf(obj);
%\end{alltt}
%\end{CodeOut}
%
%In sharpen, the \CodeIn{getStringValueOf(object)} method is implemented as follows:
%
%\begin{CodeOut}%\vspace*{-2ex}
%\begin{alltt}
%15 public static string GetStringValueOf(object value)\{
%16  return null == value? "null": value.ToString();
%17	\}
%\end{alltt}
%\end{CodeOut}
%
%If Line 13 assigns a null value to \CodeIn{obj}, \CodeIn{value} in Line 14 will also be ``null'' as expected. By implementing its own mapped C\# method, sharpen hides the preceding difference, but it still fails to hide all differences. For example, we find that if Line 5 assigns a \CodeIn{false} boolean value to \CodeIn{obj}, \CodeIn{value} in Line 6 will be ``false'', but if Lines 7 and 13 assign a \CodeIn{false} boolean value to \CodeIn{obj}, \CodeIn{value} of Line 8 and \CodeIn{value} in Line 14 will both be ``False''. This difference is relatively difficult to detect, since a programmer typically does not know the internal logic of the method to construct appropriate test cases.
%
%
%It is desirable to detect differences of API mapping relations since the differences will potentially introduce defects to client codek, the same inputs, but it is challenging to detect behavioral differences of API mapping relations via testing for three factors: (1) API elements are typically quite large in size, so it takes great efforts to write test cases manually for API elements and their mapping relations; (2) Other types of migrations such as library migrations~\cite{nita2010using} can use existing test cases to ensure the quality of migrated code, but for language migration, translated test cases may also have defects at the first place; (3) It requires many test cases to reveal all behaviors of API elements, and simply generating extreme values such as null values are not sufficient to reveal all API behaviors.
%
%In this paper, we propose an approach, called TeMAPI (\textbf{Te}sting \textbf{Ma}pping relations of \textbf{API}s), that detects behavioral differences of API mapping relations via testing. TeMAPI generates various test cases and compares testing results of mapped API elements for their behavioral differences. This paper makes the following major contributions:
%
%\begin{itemize}\vspace*{-1.5ex}
%\item A novel approach, called TeMAPI, that detect behavioral differences of mapped API elements via testing. Given a translation tool, TeMAPI detects behavioral differences of its all API mapping relations automatically. It is important to detect these behavioral differences since they can introduce defects in translated code silently.\vspace*{-1.5ex}
%\item Test adequacy criteria proposed for generating sufficient test cases to test API mapping. TeMAPI targets at generating adequate test cases that can reveal all behaviors of API elements to test their mapping relations.\vspace*{-1.5ex}
%\item A tool implemented for TeMAPI and two
%evaluations on ?? projects that include ?? mapping relations from Java to C\#, and ?? mapping relations from C\# to Java. The results show that our tool detects ?? unique defects of mapping relations...
%\end{itemize}\vspace*{-1.5ex}
%
%The rest of this paper is organized as follows.
%Section~\ref{sec:mapping} presents our test adequacy criteria.
%Section~\ref{sec:example} illustrates our approach using an example.
%Section~\ref{sec:approach} presents our approach.
%Section~\ref{sec:evaluation} presents our evaluation results.
%Section~\ref{sec:discuss} discusses issues of our approach.
%Section~\ref{sec:related} presents related work.
%Finally, Section~\ref{sec:conclusion} concludes.
