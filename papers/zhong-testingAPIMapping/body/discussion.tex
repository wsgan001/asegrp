
%\begin{figure}[t]
%\centering
%\includegraphics[scale=1,clip]{figure/n2n.eps}\vspace*{-3ex}
% \caption{Merging technique}\vspace*{-3.5ex}
% \label{fig:n2n}
%\end{figure}

\section{Discussion and Future Work}
\label{sec:discuss}

We next discuss issues in our approach and describe how we address
these issues in our future work.

\textbf{Detecting more behavior difference.} As shown in our evaluations, TeMAPI does not cover all feasible paths, so it may fail to reveal some behaviors. To detect more behavior differences, some directions seem promising. (1) We can test side effects or  mock objects to test methods without return values. (2) To test API methods that return random values, we can check the distribution of their returned values. (3) To test methods that need to read files, we can generate test cases based on Java provides the Compatibility Kit (JCK)\footnote{\url{http://jck.dev.java.net}} where standard call sequences and files are prepared. (4) Other tools such as jCute~\cite{sen2006scalable} and JPF~\cite{visser2003mcp} may help generate more test case. We plan to explore these directions in future work.

\textbf{Testing translation of code structures.} As shown in our evaluations, translation tools may fail to translate if code structures are complicated. We notice that other translation tools encounter with similar problems. For example, Daniel \emph{et al.}~\cite{daniel2007automated} propose an approach that tests refactory engines by comparing their refactored results given the same generated abstract syntax trees. The idea inspires our future work to testing code structures for translation tools by comparing the translation results given the same code structures.

%\textbf{Testing API mapping of single language.} We find that many existing approaches translate applications within single languages. For example, twinning~\cite{nita2010using} translates applications based on mapping relations of API invocations from different API libraries, and CatchUp!~\cite{henkel2005catchup} translates applications based on mapping relations of API invocations from different versions. In future work, we plan to adapt our approach to test mapping relations of API invocations within single languages. 