\section{Conclusion}
\label{sec:conclusion}

Translated applications can exhibit behavioral differences from the original applications due to inconsistencies among API mapping relations. In this paper, we proposed an approach, called TeMAPI, that detects behavioral differences of mapped API elements via testing. For our approach, we implemented a tool and conducted three evaluations on five translation tools to show the effectiveness of our approach. The results show that our approach detects various behavioral differences between mapped API elements. We further analyze these differences and their implications. Our approach enables such findings that can help improve existing translation tools and help programmers better understand differences between different languages such as Java and C\#.

%Given a translation tool, TeMAPI combines random testing with dynamic-symbolic-execution-based testing to generate test cases that reveal behavioral differences between mapped API elements defined by the translation tool.
%Mapping relations of APIs are quite useful for the translation tools, but these mapping relations also can introduce defects to translated code since mapped API methods may have behavioral differences. In this paper, we propose an approach, called TeMAPI, that detects behavioral differences of mapped API methods via testing. TeMAPI targets at generating test cases that covers all feasible paths and  sequences to reveal behavioral differences of both single invocations and invocation sequences. We implemented a tool and conducted three evaluations on five translation tools to show the effectiveness of our approach. The results show that our approach detects various differences between mapped API invocations. We further analyze these differences and their implications. The results can help improve existing translation tools and help programmers better understand differences of Java and C\#.
