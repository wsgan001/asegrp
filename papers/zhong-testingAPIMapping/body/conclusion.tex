\section{Conclusion}
\label{sec:conclusion}

API Mapping relations serve as a basis for automatic translation tools to translate applications from one language to another. However, original and translated applications can exhibit different behaviors due to inconsistencies among mapping relations. In this paper, we proposed an approach, called TeMaAPI, that detects different behaviors of mapped API methods via testing. TeMaAPI targets at generating test cases that covers all feasible paths and  sequences to reveal different behaviors of both single methods and method sequences. We implemented a tool and conducted three evaluations on five translation tools to show the effectiveness of our approach. The results show that our approach detects various differences between mapped API methods. We further analyze these differences and their implications. We expect that our results can help improve existing translation tools and help programmers better understand differences of Java and C\#.

%Mapping relations of APIs are quite useful for the migration tools, but these mapping relations also can introduce defects to translated code since mapped API methods may have different behaviors. In this paper, we propose an approach, called TeMaAPI, that detects different behaviors of mapped API methods via testing. TeMaAPI targets at generating test cases that covers all feasible paths and  sequences to reveal different behaviors of both single invocations and invocation sequences. We implemented a tool and conducted three evaluations on five migration tools to show the effectiveness of our approach. The results show that our approach detects various differences between mapped API invocations. We further analyze these differences and their implications. The results can help improve existing migration tools and help programmers better understand differences of Java and C\#.
