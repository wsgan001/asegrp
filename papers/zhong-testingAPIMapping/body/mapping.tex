\section{Test Adequacy Criteria}
\label{sec:mapping}
As stated by Andrews \emph{et al.}~\cite{andrews2003test}, a test adequacy criterion is a predict, and a test suite is adequate with respect to a criterion only if all defined properties of the criterion are satisfied by the test suite. In this paper, we define two test adequacy criteria for API mapping relations as follows.

\textbf{Path criterion.} Given an API invocation $inv$ in a language $L_1$, and its mapped API invocation $\psi(inv)$ in a language $L_2$, a adequate test suite should cover all internal paths of $inv$. This criterion ensures that  $\psi(inv)$ in $L_2$ return the same values with $inv$ in $L_1$ with respect to all paths of $inv$.

\textbf{Sequence criterion.} Given API invocations $inv_{1},\ldots,inv_{m}$ in a language $L_1$ and its mapped API invocations $\psi(inv_{1},$ $\ldots,inv_{m})$ in a language $L_2$, a adequate test suite should cover all states of $obj$ with respect to the field criterion. This criterion ensures that  $\psi(inv_{1},$ $\ldots,inv_{m})$ in $L_2$ return the same values with $inv_{1},\ldots,inv_{m}$ in $L_1$ with respect to all call sequences of $inv_{1},\ldots,inv_{m}$.

TeMaAPI targets at generating test cases that satisfy both the path criterion and the sequence criterion for testing API mapping relations.
