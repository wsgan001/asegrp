\section{Evaluations}
\label{sec:evaluation}

We implemented a tool for TeMaAPI and
conducted evaluations using our tool to address the following
research questions:

%\vspace*{-1.5ex}
\begin{enumerate}
\item How effectively can existing migration tool translate API invocations (Section~\ref{sec:evaluation:invocation})? \vspace*{-1.8ex}
\item How effectively can our approach detect different behaviors of single API invocations (Section~\ref{sec:evaluation:single})? \vspace*{-1.8ex}
\item How effectively can our approach detect different behaviors of API invocation sequences (Section~\ref{sec:evaluation:sequence})?
\end{enumerate}%\vspace*{-1.5ex}

\begin{table}[t]
\centering
\begin{SmallOut}
\begin {tabular} {|c|l|c|c|c|c|c|c|}
 \hline
\textbf{Name}& \textbf{Version}& \textbf{Source} &\textbf{Description}\\
\hline
Java2CSharp  &  1.3.4 & IBM (ILOG) & Java to C\# \\
\hline
JLCA         &  3.0   & Microsoft  & Java to C\# \\
\hline
sharpen      &  1.4.6 & db4o       & Java to C\# \\
\hline
Net2Java     &  1.0   & NetBean    &  C\# to Java\\
\hline
VB \& C\# to Java converter    &  1.6   & Tangible   &  C\# to Java\\
\hline
\end{tabular}\vspace*{-2ex}
\Caption{Subject tools} \label{table:subjects}
\end{SmallOut}\vspace*{-2ex}
\end{table}

Table~\ref{table:subjects} shows the subject tools in our evaluation. Column ``Name'' lists the names of subject tools. In the rest of the paper, we call ``VB \& C\# to Java converter'' as converter for short. Column ``Source'' lists their companies. Although all these tools are developed by commercial companies, Java2CSharp, sharpen, and Net2Java are all open source. Column ``Description'' lists the main functionalities of subject tools. We choose these tools as subjects since we find that many programmer recommend these tools in various forums.

All evaluations were conducted on a PC with Intel Qual CPU @
2.83GHz and 1.98M memory running Windows XP.

\subsection{Translating Generated Client Code}
\label{sec:evaluation:invocation}
For Java to C\# tools, we use TeMaAPI to generate client-code methods for J2SE 6.0\footnote{\url{http://java.sun.com/javase/6/docs/api/}}. As described in Section~\ref{sec:approach:generating}, TeMaAPI ignores all generic API methods. Table~\ref{table:java2csharp} shows the translation results. Column ``Type'' lists the types of generated methods. In particular, ``sfg'' denotes getters of static fields; ``sfs'' denotes setters of static fields; ``nfg'' denotes getters of non-static fields; ``nfs'' denotes setters of non-static fields; ``sm'' denotes static methods; ``nm'' denotes non-static methods; and ``Total'' denotes the sums of all methods. Column ``Number'' lists numbers of corresponding types of methods. Columns ``Java2CSharp'', ``JLCA'', and ``sharpen'' list the translation results of corresponding migration tools. For these columns, sub-column ``M'' lists the number of translated methods without compilation errors, and sub-column ``\%'' lists the percentage from translated method without compilation errors to corresponding generated methods.
\begin{table}[t]
\centering
\begin{SmallOut}
\begin {tabular} {|c|r|r|r|r|r|c|c|}
 \hline
\multirow{2}*[-2pt]{\textbf{Type}}&
\multicolumn{1}{|c}{\multirow{2}*[-2pt]{\textbf{No}}}
& \multicolumn{2}{|c|}{\textbf{Java2CSharp}} & \multicolumn{2}{|c|}{\textbf{JLCA}}& \multicolumn{2}{|c|}{\textbf{sharpen}} \\\cline{3-8} &  &  \textbf{M}& \textbf{\%} &  \textbf{M}& \textbf{\%}&  \textbf{M}& {\%}\\
\hline
sfg  &  16962 & 237 & 1.4\% & 3744 & 22.1\% & 47 & 0.3\%\\
\hline
sfs  &  0    & 0    & n/a   & 0    & n/a    & 0  & n/a  \\
\hline
nfg  &  832  & 0    & 0.0\% & 121  & 14.5\% & 26 & 2.2\%\\
\hline
nfs  &  823  & 0    & 0.0\% & 79   & 9.6\%  & 0   & 0.0\%\\
\hline
sm   &1175   & 97   & 8.3\% & 198  & 16.9\% & 26  & 2.2\%\\
\hline
nm   &175400 & 3589 & 2.0\% & 39536& 22.5\% & 1112& 0.6\%  \\
\hline
Total &195192& 3923 &  2.0\% & 43678 & 22.4\% & 1185 & 0.6\%\\
\hline
\end{tabular}\vspace*{-2ex}
\Caption{Translation results of Java to C\# tools} \label{table:java2csharp}
\end{SmallOut}\vspace*{-2ex}
\end{table}

From the results of Table~\ref{table:java2csharp}, we find that it is quite challenging for a migration tool to cover all API invocations. One challenge lies in that APIs are quite large in size. Although JLCA can translate 43678 generated methods, it covers only 22.4\% of total generated methods. The other challenge lies in that many API invocations of different languages cannot be accurately mapped. For example, as pointed out by our previous work~\cite{zhong2010mining}, one API method in one language may be mapped to several API methods in another language. We find that all the three migration tools have techniques to deal with the differences. In particular, Java2CSharp and sharpen develop their own assemblies and map some API invocations to their implemented assemblies instead of standard C\# API invocations. For example, Java2CSharp maps the \CodeIn{java.lang.Class.forName (String)} method in Java to the \CodeIn{ILOG.J2CsMapping.Reflect. Helper.GetNativeType(String)} method in C\#, and the latter method is provided by Java2CSharp. JLCA does not implement additional assemblies, but generate additional source code to hide different behaviors. For example, TeMaAPI generates one method for \CodeIn{java.awt.Dialog.add(Component m0,int m1,} \CodeIn{Frame c0)}:

\begin{CodeOut}%\vspace*{-2ex}
\begin{alltt}
public Component testadd79nm(Component m0,int m1,Frame c0)\{
  java.awt.Dialog obj = new java.awt.Dialog(c0);
  return obj.add(m0,m1);
\}
\end{alltt}
\end{CodeOut}

JLCA translates the preceding method as follows:

\begin{CodeOut}%\vspace*{-2ex}
\begin{alltt}
@Test
public virtual Control testadd79nm(Control m0, int m1,
                                              Form c0)\{
  Form obj = SupportClass.DialogSupport.CreateDialog(c0);
  Control temp_Control;
  temp_Control = m0;
  obj.Controls.Add(temp_Control);
  if (m1 != -1)
    obj.Controls.SetChildIndex(temp_Control, m1);
  return temp_Control;
\}
\end{alltt}
\end{CodeOut}

JLCA maps the API method in Java into many API methods in C\#. Besides adding a temporary variable named \CodeIn{tmp\_Control}, JLCA generates a class named \CodeIn{SupprtClass} where more API invocations are introduced.

Although all the three migration tools take the different behaviors of mapped APIs seriously, they do not cover all differences of API mapping relations. For example, when JLCA translates generated code, it generates a report with 1265 warning messages for different behaviors of API mapping relations. One warning message is ``Method 'java.lang.String.indexOf' was converted to 'System.String.IndexOf' which may throw an exception''. Still, the report does not tell programmers when such an exceptions will be thrown, and the rest of our evaluation targets at the problem.


\begin{table}[t]
\centering
\begin{SmallOut}
\begin {tabular} {|c|r|r|r|r|r|c|c|}
 \hline
\multirow{2}*[-2pt]{\textbf{Type}}&
\multicolumn{1}{|c}{\multirow{2}*[-2pt]{\textbf{No}}}
& \multicolumn{2}{|c|}{\textbf{Net2Java}} & \multicolumn{2}{|c|}{\textbf{converter}}\\\cline{3-6} &  &  \textbf{M}& \textbf{\%} &  \textbf{M}& \textbf{\%}\\
\hline
sfg  &  3223 & 1    & 0.0\% & 3    & 0.1\% \\
\hline
sfs  &  8    & 0    & 0.0\% & 0    & 0.0\%   \\
\hline
nfg  &  117  & 0    & 0.0\% & 0    & 0.0\%\\
\hline
nfs  &  115  & 0    & 0.0\% & 0    & 0.0\%\\
\hline
sm   &996    & 22   & 2.2\% & 387  & 38.9\% \\
\hline
nm   &190376 & 4    & 0.0\% & 6    & 0.0\% \\
\hline
Total &194835& 27   &  0.0\% & 396 & 0.2\%\\
\hline
\end{tabular}\vspace*{-2ex}
\Caption{Translation results of C\# to Java tools} \label{table:csharp2java}
\end{SmallOut}\vspace*{-2ex}
\end{table}

For C\# to Java migration tools, we use TeMaAPI to generate client-code methods for the .Net framework client profile\footnote{\url{http://tinyurl.com/252t2ax}}. As described in Section~\ref{sec:approach:generating}, besides generic methods, TeMaAPI also ignores methods whose parameters are marked with \CodeIn{out} or \CodeIn{ref}. Table~\ref{table:csharp2java} shows the translation results. Columns of Table~\ref{table:csharp2java} are of the same meanings with the columns of Table~\ref{table:java2csharp}. From the results of Table~\ref{table:java2csharp}, we find that TeMaAPI generates almost the same size of methods as it generates for J2SE 6.0, and both the two tool translate only quite a small size of API invocations. As described in the wikipedia\footnote{\url{http://tinyurl.com/yj4v2m2}}, C\# provides many features that Java does not have (\emph{e.g.}, partial class, reference parameters, output parameters, and named arguments). We suspect that a C\# to Java migration tool needs to deal with these differences, so many mapping relations of APIs are not addressed yet.
\begin{table}[t]
\centering
\begin{SmallOut}
\begin {tabular} {|p{3.6em}|r|r|r|r|r|c|c|}
 \hline
\multicolumn{1}{|c}{\multirow{2}*[-2pt]{\textbf{Name}}}&
\multicolumn{1}{|c|}{\multirow{2}*[-2pt]{\textbf{No}}}
& \multicolumn{2}{|c|}{\textbf{Java2CSharp}} & \multicolumn{2}{|c|}{\textbf{JLCA}}& \multicolumn{2}{|c|}{\textbf{sharpen}} \\\cline{3-8} &  &  \textbf{M}& \textbf{\%} &  \textbf{M}& \textbf{\%}&  \textbf{M}& {\%}\\
\hline
awt  &  29199  & 0     &  0.0\%  &  8637  &  29.6\%  &  0   & 0.0\%\\
\hline
bean &  \hfill 1768   & 20    &  1.1\%  &  14    &  0.8\%   &  0   & 0.0\% \\
\hline
io   &  \hfill 3109   & 592   &  19.0\% & 1642   &  52.8\%  & 43   & 1.4\%\\
\hline
lang &  \hfill 5221   & 1494  &  28.6\% & 2377   &  45.5\%  & 791  & 15.2\%\\
\hline
math &  \hfill 1584   & 101   &  6.4\%  & 232    &  14.6\%  & 0    & 0.0\%\\
\hline
java.net  &  \hfill 1990   & 52    &  2.6\%  & 482    &  24.2\%  & 10   & 0.5\%  \\
\hline
nio  &  \hfill 536    & 30    &  5.6\%  & 0      &  0.0\%  &  0    & 0.0\%  \\
\hline
java.rmi  &  \hfill 1252   & 0     &  0.0\%  &  707   &  56.5\%  &  0   & 0.0\%\\
\hline
java.sec. &  \hfill 2797   & 50    &  1.8\%  &  702    &  25.1\%   &  0   & 0.0\% \\
\hline
java.sql   &  \hfill 3495   & 20   &  0.6\% & 183   &  5.2\%  & 0   & 0.0\%\\
\hline
text  &  \hfill 1068   & 96   &  9.0\% & 321   &  30.1\%  & 0  & 0.0\%\\
\hline
util  &  \hfill 9586   & 1372   &  14.3\%  & 1879    &  19.6\%  & 341    & 3.6\%\\
\hline
acc.  &  \hfill 237   & 1    &  0.4\%  & 25    &  10.5\%  & 0   & 0.0\%  \\
\hline
activation     &  \hfill 538   & 0    &  0.0\%  & 165   &  30.7\%  & 0   & 0.0\%  \\
\hline
activity      &   \hfill 252   & 0    &  0.0\%  &  0    &  0.0\%   &  0   & 0.0\%\\
\hline
annotation    &  \hfill 19    &  0    &  0.0\%  &  0    &  0.0\%   &  0   & 0.0\% \\
\hline
crypto        &  \hfill 625   &  0    &  0.0\%  &  263  &  42.1\%  &  0    & 0.0\%\\
\hline
imageio      &  \hfill 1261   &  0    &  0.0\%  &  0    &  0.0\%  & 0      & 0.0\%\\
\hline
model       &  \hfill 103    & 0     &  0.0\%  & 0     &  0.0\%  & 0    & 0.0\%\\
\hline
man.   &  \hfill 5380   & 2    &   0.0\%  & 0     &  0.0\%  & 0    & 0.0\%  \\
\hline
naming       &  \hfill 3565   & 0    &   0.0\%  & 1365   &  38.3\%  &  0    & 0.0\%  \\
\hline
javax.net           &  \hfill 285    & 0     &  0.0\%  &  0     &  0.0\%   &  0   & 0.0\%\\
\hline
print         &  \hfill 1534   & 0     &  0.0\%  &  0    &  0.0\%   &  0   & 0.0\% \\
\hline
javax.rmi           &  \hfill 45     &  0    &  0.0\%  & 0     &  0.0\%   & 0   & 0.0\%\\
\hline
script         &  \hfill 195   & 0     &  0.0\% &  0     &  0.0\%  & 0  & 0.0\%\\
\hline
javax.sec.       &  \hfill 1435  & 0     &  0.0\%  & 619     &  43.1\%  & 0    & 0.0\%\\
\hline
sound          &  \hfill 515   & 0    &  0.0\%  & 56    &  10.9\%  & 0   & 0.0\%  \\
\hline
javax.sql            &  \hfill 640   & 0    &  0.0\%  & 0     &  0.0\%   & 0   & 0.0\%  \\
\hline
swing          &  102389& 10   &  0.0\%  &  21364 &  20.9\%   &  0   & 0.0\%\\
\hline
tools          &  \hfill 11    & 0     &  0.0\%  &  0    &  0.0\%   &  0   & 0.0\% \\
\hline
transaction    &  \hfill 264   &  0    &  0.0\%  & 0     &  0.0\%   & 0   & 0.0\%\\
\hline
javax.xml            &  \hfill 4188  &  34   &  0.8\% &  580   &  13.8\%  & 0  & 0.0\%\\
\hline
ietf        & \hfill  189   & 0     &  0.0\%  & 0     &  0.0\%  & 0    & 0.0\%\\
\hline
org.omg              &  \hfill 8937   & 0    &  0.0\%  & 1578  &  17.7\%  & 0   & 0.0\%  \\
\hline
w3c.dom          &  \hfill 83     & 0    &  0.0\%  & 14     &  16.9\%   & 0   & 0.0\%  \\
\hline
org.xml             &   \hfill 897    & 49   &  5.5\%  & 473    & 52.7\%    & 0   & 0.0\%\\
\hline
\end{tabular}\vspace*{-2ex}
\Caption{Translation results of package level} \label{table:package}
\end{SmallOut}\vspace*{-2ex}
\end{table}

Comparing the translation results between Java-to-C\# tools and C\#-to-Java tools, we find that Java-to-C\# tools cover much more API invocations. To fully explore the translation results of Java-to-C\# tools, we present the results of the package level in Table~\ref{table:package}. Column ``Name'' lists the names of Java packages. To save space, we omit the prefixes such as ``java.'', ``javax.'', and ``org.'' if it does not introduce ambiguity. We also use short names for some packages. In particular, we use ``acc.'' to denote the \CodeIn{javax.accessibility} package, ``man.'' to denote the \CodeIn{javax.management} package, ``java. sec.'' to denote the \CodeIn{java.security} package, and ``javax.sec.'' to denote the \CodeIn{javax.security} package. Other columns of Table~\ref{table:package} are of the same meanings with the columns of Table~\ref{table:java2csharp}. From the results of Table~\ref{table:package}, we find that all the three migration tools cover the \CodeIn{java.io} package, the \CodeIn{java.lang} package, and the \CodeIn{java.util} package. The three packages may be quite important for most Java programs. Almost for all packages, JLCA covers more API invocations. In particular, JLCA covers GUI-related packages such as the \CodeIn{java.awt} package and the \CodeIn{javax.swing} package. As a result, JLCA can translate some Java programs with GUI interfaces whereas the other two tools cannot. Still, we find that all the three tools do not cover 15 packages. As most of these packages are related to networks, all the three tools will fail to translate a program that use network APIs.


In summary, we find that APIs are quite large in size, and migration tools typically cover only a small set of API invocations. Existing migration tools already notice the differences between mapped APIs, but many differences are left unsolved.
\subsection{Testing Single Invocations}
\label{sec:evaluation:single}


To test different behaviors of single invocations, TeMaAPI leverages Pex to search internal paths for C\# client-code methods. These methods include the translated C\# methods without compilation errors as shown in Table~\ref{table:java2csharp} and the generated C\# methods that can be translated into Java without compilation errors as shown in Table~\ref{table:csharp2java}. When Pex searches those paths, TeMaAPI records the inputs and output of each iteration. Based on these inputs and outputs, TeMaAPI generates JUnit test cases to ensure that the generated methods and the translated methods produce the same outputs given the same inputs. As it requires human interactions to test GUI related APIs, we filter out GUI related APIs such as awt and swing although some migration tools such as JLCA are able to translate some GUI related APIs. In addition, when Pex searches methods without return values, we ignore those paths that do not throw any exceptions since we cannot generate JUnit test cases for them.


We run the generated JUnit test cases, and Table~\ref{table:singleinvoc} shows the results. Column ``Name'' lists the name of migration tools. Columns ``Pass'', ``Error'', and ``Failure'' list passes, errors, and failures after running JUnit test cases, respectively. Sub-column ``M'' lists numbers of test cases. Sub-column ``\%'' lists percentages from the numbers of corresponding test cases to the numbers of total generated test cases. From the results of Table~\ref{table:singleinvoc}, we find that totally only about half the generated JUnit test cases get passed. It turns out that TeMaAPI is quite effective to detect different behaviors of mapped API invocations since it searches every paths of methods under test. The results also reflect that the API mapping relations defined in JLCA and sharpen are more reliable than other tools since more test cases of the two tools get passed than other four tools.

\begin{table}[t]
\centering
\begin{SmallOut}
\begin {tabular} {|c|r|r|r|r|r|c|c|}
 \hline
\multirow{2}*[-2pt]{\textbf{Name}}
& \multicolumn{2}{|c|}{\textbf{Pass}} & \multicolumn{2}{|c|}{\textbf{Error}}& \multicolumn{2}{|c|}{\textbf{Failure}} \\\cline{2-7}  &  \textbf{M}& \textbf{\%} &  \textbf{M}& \textbf{\%}&  \textbf{M}& {\%}\\
\hline
Java2CSharp  &   6949  & 45.9\% & 5248 & 34.0\% & 3261 & 21.1\% \\
\hline
JLCA         &   17189 & 52.0\% & 8901 & 26.9\% & 6944 & 21.0\% \\
\hline
sharpen      &  1617  & 59.2\% & 662  & 24.2\% & 451  & 16.5\%\\
\hline
net2java     &   51    & 14.5\% & 40   & 11.4\%  & 261   & 74.1\%\\
\hline
converter    &  278    &  36.4\% & 302  & 39.6\% & 182   & 23.9\%\\
\hline
Total        &  26084  & 49.8\%  & 15153 & 29.0\% &11099 & 21.2\%  \\
\hline
\end{tabular}\vspace*{-2ex}
\Caption{Results of testing single invocations} \label{table:singleinvoc}
\end{SmallOut}\vspace*{-2ex}
\end{table}

For tools such as Java2CSharp, JLCA, and sharpen, we further present their testing results of the package level in Table~\ref{table:packagetest}. Column ``Name'' lists names of J2SE packages. For columns ``Java2CSharp'', ``JLCA'', and ``sharpen'', sub-column ``R'' lists numbers of generated JUnit test cases, and sub-column ``\%'' lists percentages from the test cases end with errors or failures to the total test cases. From the results of Table~\ref{table:packagetest}, we find that for the \CodeIn{java.sql} package and the \CodeIn{java.util} package, all the tools suffer relatively high error/failure percentages, and for the \CodeIn{java.lang} package and the \CodeIn{java.math} package, all the tools achieve relatively low error/failure percentages. The results may reflect that some packages between Java and C\# are more similar than others, so that they can more easily mapped. We also find that for package the \CodeIn{java.text} package, the \CodeIn{javax.xml} package, and the \CodeIn{org.xml} package, JLCA achieve lower error/failure percentages than other tools. The results indicate that a migration tool can achieve better translation results if they carefully prepare the mapping relations of APIs.


Table~\ref{table:singleinvoc} and Table~\ref{table:packagetest} show that many generated JUnit tests do not get passed. Still, we notice that even if a JUnit test case does not get passed, we do not have fully confidence that it is caused by different behaviors of mapped APIs. For example, some API methods such as \CodeIn{java.util.Random.nextDouble()} generate random values. As a result, we find that its corresponding JUnit test case fails, but we cannot conclude that these methods have different behaviors.

To better understand different behaviors of mapped API invocations, we manually inspected 3759 JUnit test cases that end with errors or failures. In particular, for tools such as Java2CSharp, JLCA, and sharpen, we investigate the test cases for the \CodeIn{java.lang} package since TeMaAPI generates many test cases for the package as shown in Table~\ref{table:package}. For tools such as Net2Java and converter, we inspect all their test cases since TeMaAPI test cases in small sizes for them. Our findings are as follows:

\textbf{Finding 1:} 37.2\% test cases show the different behaviors caused by null inputs or outputs.

We find that Java API invocations and their mapped C\# API invocations can have different behaviors when inputs are null values. In some cases, a Java API method can accept null values, but its mapped C\# API method will throw exceptions given a null value. One such example is shown in Section~\ref{sec:introduction}. In other cases, a Java API method will throw exceptions given a null value, but its mapped C\# API method can accept null values. For example, JLCA maps the \CodeIn{java.lang.Integer.parseInt(String,int)} method in Java to the \CodeIn{System.Convert.ToInt32(string,int)} in C\#. TeMaAPI detects that when the inputs of the C\# method are null and 10, its output is 0. Given the same inputs, the Java method throws a \CodeIn{NumberFormatException}. We also find that given the same inputs, a method may produce null outputs, whereas its mapped method will not. For example, converter maps the \CodeIn{Sys- tem.Collections.Queue.ToArray()} in C\# to the \CodeIn{java.util. LinkedList.toArray()} method in Java. Given an empty list, the C\# method produce a null value, whereas the Java method produce an empty array.


\textbf{Implication 1:} Although programmers may come to agreements on functionalities of API invocations, the behaviors for null values are typically controversial. Programmers or migration tool should deal with null values carefully across Java and C\#.


\begin{table}[t]
\centering
\begin{SmallOut}
\begin {tabular} {|p{3.4em}|r|r|r|r|r|r|r|r|r|r|}
 \hline
\multicolumn{1}{|c}{\multirow{2}*[-2pt]{\textbf{Name}}}
& \multicolumn{2}{|c|}{\textbf{Java2CSharp}} & \multicolumn{2}{|c|}{\textbf{JLCA}}& \multicolumn{2}{|c|}{\textbf{sharpen}} \\\cline{2-7} &  \textbf{R}&  \textbf{\%} &   \textbf{R}& \textbf{\%} & \textbf{R}&   \textbf{\%}\\
\hline
bean &  \hfill 17     &    82.4\%  &  18        &  33.3\%   &  0      & n/a \\
\hline
io   &  \hfill 4155   &  67.8\%  &  6981       &  58.0\%   &   33    & 1.4\%\\
\hline
lang &  \hfill 3480   &   37.5\%  &  4431      &  26.1\%   &   1753 & 29.3\%\\
\hline
math &  \hfill 561    &   4.3\%  &   1629     &   1.5\%   &  0      & n/a\\
\hline
java.net  &   438     &   25.1\% &   3941     &   47.8\%  & 9       & 44.4\%  \\
\hline
nio  &  \hfill 27     &  48.1\% &    0        &   n/a     &  0     &  n/a \\
\hline
java.rmi  &  \hfill 0   &   n/a   &   884     &   32.6\%  &  0     & n/a\\
\hline
java.sec. &  \hfill 45  &   55.6\%  &  828    &  35.6\%   &  0    & n/a \\
\hline
java.sql   &  \hfill 260&   88.1\%  & 1465    &  91.0\%   &   0     & n/a\\
\hline
text  &  \hfill 566   &   61.5\%  & 374      &  18.2\%   & 0      & n/a\\
\hline
util  &  \hfill 5519  &   60.8\%  & 6177     & 70.2\%  & 935      & 62.4\%\\
\hline
acc.  &  \hfill 1    &   0.0\%   & 0         & n/a    & 0          & n/a \\
\hline
activation  &  0     &    n/a    & 694      & 53.9\% & 0           & n/a  \\
\hline
crypto      &  0     &     n/a    & 298     & 24.2\% &  0        & n/a\\
\hline
man.        &  2     &    0.0\%  & 0        & n/a    &  0          & n/a  \\
\hline
naming      &  0     &    n/a     & 1569    & 40.6\%  &  0         & n/a  \\
\hline
javax.sec.  &  0     &   n/a     & 683     & 45.9\%  &  0        & n/a\\
\hline
sound       &  0     &   n/a     & 66       & 36.4\%  &   0        &n/a  \\
\hline
javax.xml   &  110   &    71.8\%  &  628    & 45.9\%  &   0         & n/a\\
\hline
org.omg     &  0     &   n/a     & 1842    & 45.9\%  & 0           & n/a  \\
\hline
w3c.dom     &  0     &   n/a     & 18      & 33.3\%  &  0         & n/a  \\
\hline
org.xml     &   277  &   70.0\%  & 483     & 27.3\%  & 0         & n/a\\
\hline
\end{tabular}\vspace*{-2ex}
\Caption{Testing results of package level } \label{table:packagetest}
\end{SmallOut}\vspace*{-2ex}
\end{table}

\textbf{Finding 2:} 22.2\% test cases show the different behaviors caused by stored string values.

We find that string values between Java APIs and their mapped C\# APIs are typically different. For example, each Java class has a \CodeIn{toString()} method inherited from the \CodeIn{java.lang.Object} class, and each C\# class also has a \CodeIn{ToString()} method inherited from the \CodeIn{System.Object} class. Many migration tools map the two API methods, and the return values of the two methods are quite different in many cases. For another example, many API classes declare methods like \CodeIn{getName} or \CodeIn{getMessage}. These methods also return string values that can be quite different. In particular, we find that the \CodeIn{Message} fields of exceptions in C\# often return detailed string messages. One such message is ``Index was outside the bounds of the array'' provided by the \CodeIn{System.IndexOutOfRan- geException.Message} field. On the other hand, exceptions in Java often provide only null messages.

\textbf{Implication 2:} String values including names are typically different between Java APIs and their mapped C\# APIs. Programmers should not rely on these string values unless migration tools can hide the differences.

\textbf{Finding 3:} 8.1\% test cases show the different behaviors caused by exception handling.

We find that two mapped API methods can throw exceptions that are not mapped. For example, when indexes are out of bounds, the \CodeIn{java.lang.StringBuffer.insert(int,char)} method in Java throws \CodeIn{ArrayIndexOutofBoundsException}. Java2CSharp maps the methods to the \CodeIn{StringBuilder.Insert(int,char)} method in C\# that throws \CodeIn{ArgumentOutOfRangeException} when indexes are out of bounds. As Java2CSharp maps \CodeIn{ArrayIndexOut- ofBoundsException} in Java to \CodeIn{IndexOutOfRangeException} in C\#, the mapped C\# method may fail to catch exceptions when indexes are out of bounds.

\textbf{Implication 3:} Even if two methods are of the same functionality, they may produce exceptions that are not mapped. Programmers should be careful to deal with exception handling, unless migrations tools can hide the differences.

\textbf{Finding 4:} 3.9\% test cases show the different behaviors caused by ranges of parameters.

We find that Java methods and their mapped C\# methods may have different ranges for parameters. In particular, we find that Java API methods often do not check whether values are out of ranges. For example, the \CodeIn{java.lang.Double.shortValue()} method in Java accepts values that are larger than short values, but its mapped C\# methods such as the \CodeIn{Convert.ToInt16(double)} method throw \CodeIn{OverflowException} when values are larger than short values.

\textbf{Implication 4:} Programmers should be aware of the different input ranges of API methods between Java and C\#. As C\# API methods typically check ranges of input, C\# programmers may not check ranges of inputs themselves, and thus introduce potential defects in translated Java code.

\textbf{Finding 5:} 3.5\% test cases show the different behaviors caused by static values.

We find that mapped static fields may have different values. For example, the \CodeIn{java.lang.reflect.Modifier} class has many static fields to represent modifiers (\emph{e.g.}, FINAL, PRIVATE and PROTECTED). Java2CSharp maps these fields to the fields of the \CodeIn{ILOG. J2CsMapping.Reflect} class. Although many mapped fields of the two class are of the same values, we find that fields such as VOLATILE and TRANSIENT are of different values. In addition, we find that different values sometimes reveal different ranges of data types. For example, \CodeIn{java.lang.Double.MAX\_VALUE} in Java is 1.7976931348623157E+308, and \CodeIn{System.Double.MaxValue} in C\# is 1.79769313486232E+308. Although the difference is not quite large, it can cause serious defects if a program needs highly accurate calculation results.

\textbf{Implication 5:} Programmers should be aware of that static fields may have different values even if they has the same names. As these differences reveal that Java and C\# may define different bounds for data types, programmers should also be aware of these different bounds if they need highly accurate results of extremely large or small calculation results. 

\textbf{Finding 6:} 17.1\% test cases show the different behaviors caused by different understanding or implementation.

We find that Java developers and C\# developers may have different understanding or implementation for mapped API methods. For example, according to their documents, the \CodeIn{java.lang.String- Buffer.capacity()} method in Java returns ``the current capacity of the String buffer'', and the \CodeIn{System.Text.StringBuilder. Capacity} field in C\# can return ``the maximum number of characters that can be contained in the memory allocated by the current instance''. JLCA maps the method in Java to the field in C\#, and we find that in many cases they are of different values. For sample, given a string whose value is ``0'', the \CodeIn{capacity()} in Java returns 0, but the \CodeIn{Capacity} field in C\# is 16. We notice that some such differences may indicate defects in mapped methods. For example, Java2CSharp maps the \CodeIn{java.lang.Integer.toHexString(int)} method in Java to the \CodeIn{ILOG.J2CsMapping.Util.IlNumber.To- String(int,16)} method in C\#. Given a integer whose value is -2147483648, the Java method returns ``80000000'', but the C\# method returns ``\textbackslash080000000''. We suspect that the mapped C\# method may have some defects to deal with the value.

\textbf{Implication 5:} Although programmers can come to agreement on functionalities of many API methods, they may have different understanding on some methods. In some cases, such differences may indicate defects in mapped API methods.

\textbf{Finding 6:} 8.0\% test cases end with failures since corresponding API methods produce random values. 

Besides the preceding mentioned methods whose functionalities are to generate random values, we find that other methods can also generate values. For example, each Java class has a \CodeIn{hashCode()} method inherited from the \CodeIn{java.lang.Object} class, and each C\# class has a \CodeIn{GetHashCode()} inherited from the \CodeIn{System.Object} class. Both the two methods return a hash code for the current object, so migration tools such as JLCA map the two methods. As such code is generated randomly, we cannot conclude they have different behaviors although corresponding JUnit test cases all fail. To test such mapped API methods, we plan to introduce other test oracles in future work. For example, we can generate many test cases and check whether mapped methods generate random values of the same ranges.

In summary, TeMaAPI effectively detects many different behaviors of mapped single API invocations between Java and C\#. We inspect some detected differences, and we find that various factors such as null values, string values, exception handling, ranges of parameters, static values, and different understanding can cause differences of mapped single API invocations. 
\subsection{Testing Invocation Sequences}
\label{sec:evaluation:sequence}


