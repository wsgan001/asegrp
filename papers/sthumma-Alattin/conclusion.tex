\section{conclusion}
\label{sec:conclusion}

To reduce false positives in static defect detection based on code mining, we have developed a novel approach, called Alattin, that includes a new mining algorithm and a technique for detecting neglected conditions based on the mining algorithm. Our new mining algorithm mines alternative patterns classified into two categories: balanced and imbalanced. 
In balanced patterns, all alternatives are frequent, whereas in imbalanced patterns, some alternatives are infrequent.
We conduct two evaluations to show the effectiveness of our Alattin approach. Our evaluation results show that (1) alternative patterns reach more than 40\% of all mined patterns for APIs provided by six open source libraries; (2) the mining of alternative patterns helps reduce nearly 28\% of false positives among detected violations.

In this paper, we follow a problem-driven methodology in advancing the field of mining software engineering data. Our current approach and previous approach~\cite{thummalapenta09:mining} serve as examples in this direction. More specifically, in our approaches, we empirically investigate problems in the software engineering domain and identify required types of patterns for addressing those problems. We further develop new mining algorithms for mining these required types of patterns, rather than being constrained by available mining algorithms from the data mining community. Our approaches primarily target at reducing false negatives and false positives among detected violations. Our previous approach~\cite{thummalapenta09:mining}, which mines programming rules as sequence association rules, focuses on reducing
false negatives by detecting new kinds of defects. In contrast, our current approach focuses on a new sub-direction of reducing false positives among detected violations. In future work, we plan to further expand our research by investigating broader types of problems, patterns, mining algorithms, and defects.