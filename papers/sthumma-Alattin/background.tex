\section{Rule Templates}
\label{sec:ruletmpl}

\begin{figure}[t]
\begin{CodeOut}
\begin{alltt}
01:public static void verifyBCEL(String cName) \{ 
02:\hspace*{0.1in}VerificationResult vr0, vr1, vr2, vr3; 
03:\hspace*{0.1in}int mId = 0;
04:\hspace*{0.1in}Verifier verf = VerifierFactory.getVerifier(cName);
05:\hspace*{0.1in}if(verf != null) \{ 	
06:\hspace*{0.2in}vr0 = verf.doPass1();
07:\hspace*{0.2in}if(vr0 != VerificationResult.VR\_OK)
08:\hspace*{0.4in}return;
09:\hspace*{0.2in}vr1 = verf.doPass2();	
10:\hspace*{0.2in}if (vr1 == VerificationResult.VR\_OK) \{ 
11:\hspace*{0.4in}JavaClass jc = Repository.lookupClass(cName);
12:\hspace*{0.4in}for(mId=0; mId<jc.getMethods().length; mId++)\{ 
13:\hspace*{0.5in}vr2 = verf.doPass3a(mId);
14:\hspace*{0.5in}vr3 = verf.doPass3b(mId);
15:\hspace*{0.5in}if(Pass3aVerifier.do\_verify(verf)) \{ ... \}
16:\hspace*{0.1in}\} \hspace*{0.2in}\} \hspace*{0.1in}\} \} 
\end{alltt}
\end{CodeOut}\vspace{-5ex}
\Caption{\label{fig:patternex} Code sample gathered from a code search engine.}\vspace{-4ex}
\end{figure}

We next present the rule templates defined by our approach for
capturing programming rules around individual API calls (for generality,
we refer methods in an API as individual API calls). We use the code example
shown in Figure~\ref{fig:patternex} as an illustrative example for
describing our rule templates. 

In general, a method invocation
in Java consists of four elements: the receiver object, method name, arguments, and return object.
For example, the method invocation in Statement 13 of the code example has the receiver object \CodeIn{verf},
argument \CodeIn{mId}, and the return object \CodeIn{vr2}.
The condition checks for a method invocation can appear before the call site (say, preceding conditions),
or can appear after the call site (say, succeeding conditions). More specifically,
we are interested in condition checks on the receiver object and arguments 
before a call site and condition checks on the receiver object and return object after
the call site. To capture such condition checks as programming rules, 
we define a rule template and a programming rule as follows:\\

\textbf{Definition 1:} \emph{A rule template is a five-tuple ($T_{type}$, $O_{type}$, $MI$, $AI$, $POS$), where\\
\hspace*{0.3in}$T_{type}$: template type\\
\hspace*{0.3in}$O_{type}$: object type such as a receiver or return\\
\hspace*{0.3in}$MI$: method invocation\\
\hspace*{0.3in}$AI$: optional additional information\\
\hspace*{0.3in}$POS$: location with respect to a call site}

\textbf{Definition 2:} \emph{A programming rule is an instance of a rule template.}\\

The element $T_{type}$ (template type) of a rule template describes
the type of the condition checks in the programming rule, whereas the element $O_{type}$ (object type)
represents the participating object of the individual API call such as the receiver,
argument, or return object. The element $MI$ (method invocation) stores the individual API call.
The element $AI$ is optional additional information associated with each programming
rule. The information stored in the $AI$ element is dependent on $T_{type}$.
As we capture both preceding and succeeding programming rules, the element
$POS$ describes the location of the programming rule with respect to the call site of an 
individual API call. We next describe each element in detail.

\setlength{\tabcolsep}{1pt}
\begin{table*}[t]
\begin{SmallOut}
\begin{CodeOut}
\begin{center}
\centering \caption {\label{tab:conditiontypes} Possible $T_{type}$ values of a rule template and associated
additional information}
\begin {tabular} {|l|l|l|c|}
\hline
Name&\CenterCell{Description}&\CenterCell{Additional Info}&\CenterCell{Coverage}\\
\hline
\hline direct-null-check&direct null check. e.g., if(var != null) \{ .. \}&operator involved, e.g., != & 97/497\\
\hline indirect-null-check&if variable is an argument of a &operator involved, e.g., == & 2/497 \\
		   &method invocation. e.g., if(MI(var) == null) \{ .. \}&method invocation, MI&\\
\hline direct-boolean-check&if the variable type is boolean. e.g., if(var) \{ .. \}& &65/497\\
\hline indirect-boolean-check&indirect boolean check. e.g., if(MI(var)) \{ .. \}&method invocation, e.g., MI& 25/497\\
\hline direct-const-check&if the variable is compared with a constant.&operator involved, e.g., ==&110/497\\
			 & e.g., if(var == SUCCESS)&constant value, e.g., SUCCESS&\\
\hline indirect-const-check&indirect constant equality check.&operator involved, e.g., ==&1/497\\
		 	 & e.g., if(MI(var) == FAILURE)&constant value, e.g., FAILURE&\\
		 	 & &method invocation, e.g., MI&\\
\hline direct-retval-check&if the variable is compared with the&operator involved, e.g., <&0/497\\
			 &return value of a method invocation.&method invocation, e.g., $MI_{other}()$&\\
			 &e.g., if(var < $MI_{other}()$)&&\\ 
\hline indirect-retval-check&if the variable is compared indirectly with the&operator involved, e.g., >&0/497\\
			 &return value of a method invocation.&method invocation, e.g., MI&\\
			 &e.g., if(MI(var) > $MI_{other}()$)&method invocation, e.g., $MI_{other}()$&\\ 
\hline instance-check&if the conditional check involves \CodeIn{instanceof} operator&type-name, e.g., Integer&106/497\\
			 &e.g., if(var instanceof Integer)&&\\
\hline direct-expr-check&if the variable is compared with an &operator involved, e.g., <&35/497\\
			 &expression such as another variable. e.g., if(var < expr) \{ ... \}&other expression, e.g., expr&\\		 			 			 
\hline
\end{tabular}
\end{center}
\end{CodeOut}
\end{SmallOut}
\vspace*{-4ex}
\end{table*}

Table~\ref{tab:conditiontypes} shows several possible values for the $T_{type}$ element
of the rule template. For each template type, we present the name, description with an example, 
and additional information ($AI$) that is associated with the programming rule. 
The additional information is dependent on the $T_{type}$ element.
For example, the template type \CodeIn{direct-null-check} 
captures programming rules such as \CodeIn{null} checks done on the receiver, arguments, or return objects. 
In the description of each template type, we use notation \CodeIn{var} 
to denote a receiver, argument, or return object of the individual API call. The additional
information for the template type \CodeIn{direct-null-check} stores the operator involved in the 
conditional expression such as ``=='' or ``!=''. Column ``Coverage'' shows the number
of rules that belong to each template type divided by the total number of 
rules that we found in our preliminary investigation with BCEL. All template
types described in the table include 88\% of the rules we found during our investigation.

Our template types are broadly classified into two different
categories: direct and indirect. The rationale behind these two categories 
is that a condition check can be performed directly on the variable or
can be done indirectly through another method invocation, say MI, where the current
variable is an argument of that method invocation. The template
type \CodeIn{indirect-null-check} in the table is an example for the indirect template type. 
For indirect template types, the optional additional information 
also stores the associated method invocation. This gathered additional information 
for each programming rule is later used while detecting violations.

Apart from template types shown in the table, we define two more template types
that are specific to the receiver object of a method invocation. 

\textbf{before.} this template type represents condition checks 
on other method invocations of the same receiver object preceding the call site of 
an individual API call. For example, Statements 6 to 9 in the code example
describe that before invoking the \CodeIn{doPass2} method, the method \CodeIn{doPass1} must
be invoked and a condition check must be performed on the return value of the method \CodeIn{doPass1}.
The corresponding programming rule can be captured by this template type as ``\CodeIn{direct-const-check} on \CodeIn{return} of \CodeIn{doPass1} \CodeIn{before} \CodeIn{doPass2}''.

\textbf{after.} this template type captures condition checks on other method invocations
of the same receiver object succeeding the call site of an individual API call. In general, 
this template type is useful for methods such as \CodeIn{Iterator.next()} and 
\CodeIn{Iterator.hasNext()} that are often used in loops, where
one of the methods such as \CodeIn{hasNext} is involved in the conditional expression of the loop. If not,
this template type can result in many false positives based on our
empirical investigation. Therefore, in the NCDetector framework,
we limit this template type to those APIs (such as Java Util APIs) that are often
suggested to be used in loops.

Definition 2 describes programming rules as instances of a rule template. However,
to provide better readability, we present programming rules in a textual form in this paper. 
For example, the programming rule ``\CodeIn{direct-null-check} on \CodeIn{receiver before doPass1}''
indicates that a \CodeIn{null} check should be done on the receiver object of 
\CodeIn{doPass1} before its call site. In this instance, $T_{type}$ is \CodeIn{direct-null-check},
$O_{type}$ is \CodeIn{receiver}, $MI$ is \CodeIn{doPass1}, and $POS$ is \CodeIn{before}.