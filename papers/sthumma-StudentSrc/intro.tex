%-------------------------------------------------------------------------------
\section{Introduction}
\label{sec:intro}

Various Code Search Engines (CSEs)\footnote{\url{http://gonzui.sourceforge.net/links.html}}
are available on the web that can search in billions of lines of available open source code. 
\Comment{Some of these CSEs like Google, Koders, DocJar, Codease, and Krugle are non-academic. SPARS-J and 
Sourcerer are two of the few academic code search engines.} Given a query, CSEs 
can suggest relevant code samples with usages of keywords in the given query. But these CSEs
are not quite helpful in practice, as they often produce a large number of code
examples for a given query and the desired code sample is often not available 
among the first several results. For example, the Google Code Search Engine
(GCSE)\footnote{\url{http://www.google.com/codesearch}} 
returns near 3000 samples for the query \CodeIn{lang:java
java.sql.Statement executeUpdate}.\Comment{; this query can be used to
extract information regarding the usage of Application Programming Interface (API) 
\CodeIn{executeUpdate} of \CodeIn{java.sql.Statement} class of JDBC.}

Along with assisting programmers in reusing code samples
from existing frameworks or libraries, the code samples 
returned by CSEs can also be used for other tasks 
like identifying framework hotspots and finding bugs in projects 
that reuse open source frameworks or libraries. 
Exploiting CSEs for these tasks requires analysis of 
code samples returned by CSEs. This analysis of code samples 
is non-trivial because the code samples returned by CSEs are often partial. 
The reason for partial code samples is that CSEs retrieve 
only source files with usages of the given query instead of entire projects. 
Due to the preceding limitations, traditional code analysis cannot be applied 
for analyzing the gathered code samples. Therefore, our framework 
uses several heuristics (Section ~\ref{sec:framework}) 
for analyzing these code samples and transforms them into 
an intermediate form. We used the Directed Acyclic Graph
(DAG) as an intermediate form because the DAG represents control-flow information
through its branches and joins, and provides effective traversal mechanisms. 

We developed two tools by extending our framework: PARSEWeb
and Hotspotter. We showed the effectiveness of our framework by comparing the results 
of developed tools with the results of related tools.
